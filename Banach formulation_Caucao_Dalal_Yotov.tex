\documentclass[11pt]{article}
\textwidth 17.5cm
\textheight 23cm
\oddsidemargin 0.25cm
\addtolength{\voffset}{-2.4cm}
\addtolength{\hoffset}{-0.7cm}
%
\setlength{\parindent}{12pt}
\setlength{\parskip}{3pt}
\usepackage{amssymb,amsmath,epsfig}
%\usepackage[colorlinks=false,breaklinks=true,linkcolor=blue]{hyperref}
\usepackage{graphics,psfrag,graphicx,color}
\usepackage{float,hhline}
\usepackage{multicol}
\usepackage{comment}
\usepackage[dvipsnames]{xcolor}
\usepackage{overpic}
\usepackage{caption}

\usepackage{hyperref}
\hypersetup{
	colorlinks=true,
	linkcolor=blue,
        citecolor=blue,
	filecolor=magenta,      
	urlcolor=cyan,
}

\pdfminorversion=7
\numberwithin{equation}{section}
%\usepackage{refcheck}
\allowdisplaybreaks

% OUR DEFINITIONS %%%%%%%%%%%%%%%%%%%%%%%%%%%%%%%%%%%
%
\newcommand{\punto}{\hspace{2pt}\cdot\hspace{2pt}}
\newcommand{\ds}{\displaystyle}
\newcommand{\ts}{\textstyle}
\newcommand{\smallfrac}[2]{{\textstyle\frac{#1}{#2}}}
%%%% greek bold letters
\newcommand{\bgamma}{{\boldsymbol\gamma}}
\newcommand{\bkappa}{{\boldsymbol\kappa}}
\newcommand{\blambda}{{\boldsymbol\lambda}}
\newcommand{\bLambda}{{\boldsymbol\Lambda}}
\newcommand{\bbeta}{{\boldsymbol\eta}}
\newcommand{\bdelta}{{\boldsymbol\delta}}
\newcommand{\bomega}{{\boldsymbol\omega}}
\newcommand{\bsi}{{\boldsymbol\sigma}}
\newcommand{\bSigma}{{\boldsymbol\Sigma}}
\newcommand{\bphi}{{\boldsymbol\phi}}
\newcommand{\bPhi}{{\boldsymbol\Phi}}
\newcommand{\bvarphi}{{\boldsymbol\varphi}}
\newcommand{\btheta}{{\boldsymbol\theta}}
\newcommand{\bpsi}{{\boldsymbol\psi}}
\newcommand{\bPi}{{\boldsymbol\Pi}}
\newcommand{\btau}{{\boldsymbol\tau}}
\newcommand{\bzeta}{{\boldsymbol\zeta}}
\newcommand{\bps}{{\boldsymbol\psi}}
\def\bnu{\boldsymbol\nu}
\newcommand{\bchi}{{\boldsymbol\chi}}
\newcommand{\bxi}{{\boldsymbol\xi}}
\newcommand{\brho}{{\boldsymbol\rho}}
\newcommand{\ubsi}{\underline{\bsi}}
\newcommand{\ubtau}{\underline{\btau}}
\newcommand{\ubu}{\underline{\bu}}
\newcommand{\ubv}{\underline{\bv}}
%
\newcommand{\ubphi}{\ul{\boldsymbol\phi}}
\newcommand{\ubvarphi}{\ul{\boldsymbol\varphi}}
\newcommand{\ubpsi}{\ul{\boldsymbol\psi}}
\newcommand{\ublambda}{\ul{\boldsymbol\lambda}}
\newcommand{\ubxi}{\ul{\boldsymbol\xi}}
\newcommand{\ubchi}{\ul{\boldsymbol\chi}}
%%%% bold letters
\newcommand{\bva}{{\boldsymbol\varphi}}
\newcommand{\bv}{{\mathbf{v}}}
\newcommand{\bw}{{\mathbf{w}}}
\newcommand{\f}{\mathbf{f}}
\newcommand{\g}{\mathbf{g}}
\newcommand{\ba}{\mathbf{a}}
\newcommand{\bb}{\mathbf{b}}
\newcommand{\bc}{\mathbf{c}}
\newcommand{\bi}{\mathbf{i}}
\newcommand{\bj}{\mathbf{j}}
\newcommand{\bp}{\mathbf{p}}
\newcommand{\bq}{\mathbf{q}}
\newcommand{\br}{\mathbf{r}}
\newcommand{\bu}{\mathbf{u}}
\newcommand{\bU}{\mathbf{U}}
\newcommand{\bz}{{\mathbf{z}}}
\newcommand{\bt}{{\mathbf{t}}}
\newcommand{\bn}{{\mathbf{n}}}
\newcommand{\be}{{\mathbf{e}}}
\def\bs{\mathbf{s}}
\newcommand{\0}{{\mathbf{0}}}
\def\bA{\mathbf{A}}
\def\bB{\mathbf{B}}
\def\bC{\mathbf{C}}
\def\bD{\mathbf{D}}
\def\bE{\mathbf{E}}
\def\bF{\mathbf{F}}
\def\bG{\mathbf{G}}
\def\bK{\mathbf{K}}
\def\bI{\mathbf{I}}
\def\bN{\mathbf{N}}
\def\bX{\mathbf{X}}
\def\bY{\mathbf{Y}}
\def\bZ{\mathbf{Z}}
\def\bV{\mathbf{V}}
\def\bW{\mathbf{W}}
\def\bM{\mathbf{M}}
\def\bT{\mathbf{T}}
\def\bO{\mathbf{O}}
\def\bP{\mathbf{P}}
\def\bQ{\mathbf{Q}}
\def\bS{\mathbf{S}}
\def\bRT{\mathbf{RT}}
\def\bAP{\mathbf{AP}}
\def\bx{\mathbf{x}}
\def\by{\mathbf{y}}
\def\bz{\mathbf{z}}
\def\bsgn{\mathbf{sgn}}
\newcommand{\bL}{\mathbf{L}}
\newcommand\bH{\mathbf{H}}

%%%%%%  leters in mathbb style
\newcommand\bbA{\mathbb{A}}
\newcommand\bbB{\mathbb{B}}
\newcommand\bbF{\mathbb{F}}
\newcommand\bbG{\mathbb{G}}
\newcommand\bbN{\mathbb{N}}
\newcommand\bbM{\mathbb{M}}
\newcommand\bbI{\mathbb{I}}
\newcommand\bbP{\mathbb{P}}
\newcommand\bbQ{\mathbb{Q}}
\newcommand\bbR{\mathbb{R}}
\newcommand\bbH{\mathbb{H}}
\newcommand\bbW{\mathbb{W}}
\newcommand\bbX{\mathbb{X}}
\newcommand\bbT{\mathbb{T}}
\newcommand\bbL{\mathbb{L}}
\newcommand{\bR}{\mathcal{\mathbf{R}}}
\newcommand{\bbRT}{\mathbb{RT}}
\newcommand{\bbC}{\mathbb{C}}
\newcommand{\bbV}{\mathbb{V}}
\newcommand{\bbK}{\mathbb{K}}
%%%%%%  leters in mathcal style
\newcommand{\cA}{\mathcal{A}}
\newcommand{\cB}{\mathcal{B}}
\newcommand{\cC}{\mathcal{C}}
\newcommand{\cl}{\mathcal{l}}
\newcommand{\cj}{\mathcal{j}}
\newcommand{\cE}{\mathcal{E}}
\newcommand{\cF}{\mathcal{F}}
\newcommand{\cG}{\mathcal{G}}
\newcommand{\cH}{\mathcal{H}}
\newcommand{\cJ}{\mathcal{J}}
\newcommand{\cN}{\mathcal{N}}
\newcommand{\cM}{\mathcal{M}}
\newcommand{\cT}{\mathcal{T}}
\newcommand{\cK}{\mathcal{K}}
\newcommand{\cL}{\mathcal{L}}
\newcommand{\cD}{\mathcal{D}}
\newcommand{\cO}{\mathcal{O}}
\newcommand{\cP}{\mathcal{P}}
\newcommand{\cQ}{\mathcal{Q}}
\newcommand{\cR}{\mathcal{R}}
%%%%%%%%   math operators
\def\P{\mathrm{P}}
\def\R{\mathrm{R}}
\def\Q{\mathrm{Q}}
\def\S{\mathrm{S}}
\def\T{\mathrm{T}}
\def\H{\mathrm{H}}
\def\L{\mathrm{L}}
\def\M{\mathrm{M}}
\def\U{\mathrm{U}}
\def\V{\mathrm{V}}
\def\W{\mathrm{W}}
\def\X{\mathrm{X}}
\def\rd{\mathrm{d}}
\def\re{\mathrm{e}}
\def\rD{\mathrm{D}}
\def\rN{\mathrm{N}}
\def\rP{\mathrm{P}}
\def\rQ{\mathrm{Q}}
\def\rp{\mathrm{p}}
\def\rq{\mathrm{q}}
\def\rt{\mathrm{t}}
\def\ttd{\mathtt{d}}
\def\tD{\mathtt{D}}
\def\tF{\mathtt{F}}
\def\tR{\mathtt{R}}
\def\ttr{\mathtt{r}}
\def\tS{\mathtt{S}}
\def\tT{\mathtt{T}}
\def\ttg{\mathtt{g}}
\def\tts{\mathtt{s}}
\def\ttr{\mathtt{r}}
\def\esssup{\mathrm{ess\,sup}}
\def\bBDM{\mathbf{BDM}}
\def\bbBDM{\mathbb{BDM}}
\def\BJS{\mathtt{BJS}}
\def\dc{\mathrm{dc}}
\def\skew{\mathrm{skew}}
\def\sym{\mathrm{sym}}
\def\bdiv{\mathbf{div}}
\def\bcurl{\mathbf{curl}\,}
\def\curl{\mathrm{curl}\,}
\def\tr{\mathrm{tr}}
\def\brot{\mathbf{rot}\,}
\def\rot{\mathrm{rot}\,}
\def\div{\mathrm{div}}
\def\dist{\mathrm{dist}\,}
\def\pil{\left<}
\def\pir{\right>}
\def\rl{\\[0.5ex][1.5ex]}
\def\sk{\mathrm{sk}}
\def\coeff{\mathbf{coeff}}
%%%% math operators
\def\RT{\mathrm{RT}}
\def\rC{\mathrm{C}}
\def\rS{\mathrm{S}}
\def\id{\mathrm I}
\def\as{\mathrm J}
\def\lip{\mathrm{lip}}
\def\ST{\mathrm{ST}}
\def\perm{\mathrm K}
\def\norm{\mathbf n}
\def\with{{\quad\hbox{with}\quad}}
\def\qin{{\quad\hbox{in}\quad}}
\def\qon{{\quad\hbox{on}\quad}}
\def\qan{{\quad\hbox{and}\quad}}
\def\llbrace{\{\hspace{-0.3em}\{}
\def\rrbrace{\}\hspace{-0.3em}\}}
\def\ov{\overline}
\def\ul{\underline}
\def\wt{\widetilde}
\def\wh{\widehat}
%%%% definition of constants
\newcommand{\CbK}{C_{\bK}}
\newcommand{\CKo}{C_{\rm Ko}}
\newcommand{\Cdiv}{C_{\div}}
\newcommand{\Ccea}{C_{\rm cea}}
\newcommand{\Clc}{C_{\rm LC}}
\newcommand{\Cgi}{C_{\rm GI}}
\newcommand{\Cst}{C_{\rm ST}}
%%%% functional spaces
\newcommand{\bbLtrh}{\bbL^2_{\tr, h}(\Omega_\rS)}
\newcommand{\bbLskewh}{\bbL^2_{\sk, h}(\Omega_\rS)}
\newcommand{\Lambdash}{\Lambda^{\rS}_h(\Sigma)}
\newcommand{\bLambdash}{\bLambda^{\rS}_h(\Sigma)}
\newcommand{\tbLambdash}{\tilde{\bLambda}^{\rS}_h(\Sigma)}
\newcommand{\Lambdadh}{\Lambda^{\rD}_h(\Sigma)}
\newcommand{\Ldh}{L^2_h(\Omega_\rD)}
\newcommand{\Ldho}{L^2_{h,0}(\Omega_\rD)}
\newcommand{\bbXsh}{\bbX_h(\Omega_\rS)}
\newcommand{\bbXsho}{\bbX_{h,0}(\Omega_\rS)}
\newcommand{\bHsh}{\bH_h(\Omega_\rS)}
\newcommand{\bHlsh}{\bH^1_h(\Omega_\rS)}
\newcommand{\bHdh}{\bH_h(\Omega_\rD)}
\newcommand{\bHdho}{\bH_{h,0}(\Omega_\rD)}
\newcommand{\tbHdho}{\tilde{\bH}_{h,0}(\Omega_\rD)}
\newcommand{\Ltwosh}{L^2_h(\Omega_\rS)}
% END OF OUR DEFINITIONS
%

\newcommand{\cred}[1]{{\color{red}#1}}   %barva
\newtheorem{thm}{Theorem}[section]
\newtheorem{rem}{Remark}[section]
\newtheorem{lem}[thm]{Lemma}
\newtheorem{cor}[thm]{Corollary}
\newtheorem{defi}{Definition}[section]
\newtheorem{prop}{Proposition}[section]
\newenvironment{proof}{\noindent{\it Proof.}}{\hfill$\square$}

%
\numberwithin{equation}{section}
%\numberwithin{figure}{section}
%\numberwithin{table}{section}

% END OF OUR DEFINITIONS %%%%%%%%%%%%%%%%%%%%%%%%%%%%%

\allowdisplaybreaks

%**********************************************************************
%**********************************************************************

\title{A Banach space mixed formulation for the fully dynamic \\ Navier--Stokes--Biot coupled problem}

\author{{\sc Sergio Caucao}\thanks{Grupo de Investigaci\'on en An\'alisis Num\'erico y C\'alculo Cient\'ifico (GIANuC$^2$) and Departamento de Matem\'atica y F\'isica Aplicadas, 
Universidad Cat\'olica de la Sant\'isima Concepci\'on, Casilla 297, Concepci\'on, Chile, 
email: {\tt scaucao@ucsc.cl}. Supported in part by ANID-Chile through the projects {\sc Centro de Mode\-lamiento Matem\'atico} (FB210005) and Fondecyt 1250937.}
\quad
{\sc Aashi Dalal}\thanks{Department of Mathematics, University of Pittsburgh, Pittsburgh, PA 15260, USA, email: {\tt aad100@pitt.edu}.}
\quad
{\sc Ivan Yotov}\thanks{Department of Mathematics, University of Pittsburgh, Pittsburgh, PA 15260, USA, email: {\tt yotov@math.pitt.edu}. Supported in part by NSF grant DMS 1418947}}

\date{\today}

\begin{document}
	
\maketitle

\begin{abstract}
\noindent 
We introduce and analyse a mixed formulation for the coupled problem
arising in the interaction between a free fluid and a poroelastic medium. 
The flows in the free fluid and poroelastic regions are governed by the Navier--Stokes and Biot equations, respectively, and the transmission conditions are given by mass conservation, balance of stresses, and the Beavers--Joseph--Saffman law. 
We apply dual-mixed formulations in both Navier--Stokes and Darcy equations, where the symmetry of the Navier--Stokes pseudostress tensor is imposed in a weak sense and a displacement-based formulation for elasticity equation. 
In turn, since the transmission conditions are essential in the mixed formulation, they are imposed weakly by introducing the traces of the fluid velocity and the poroelastic medium pressure on the interface as the associated Lagrange multipliers. 
The existence and uniqueness of a solution are established for the continuous weak formulation and the semi-discrete continuous-in-time formulation with nonmatching grids within a Banach space framework. 
The proof relies on classical results for nonlinear monotone operators, complemented by a regularization technique and the Banach fixed-point method, under small data assumptions.
We then present error analysis with corresponding rates of convergence for semidiscrete continuous-in-time formulation. 
Numerical experiments are presented to verify the theoretical rates of convergence and to illustrate the method's performance in applications involving flow through a filter.
\end{abstract}
	
\noindent
{\bf Key words}: Navier--Stokes--Biot; fluid-poroelastic structure interaction; Banach space formulation; mixed finite elements; pseudostress-velocity-vorticity formulation
%	
%\smallskip\noindent
%{\bf Mathematics subject classifications (2000)}: 65N30, 65N12, 65N15, 35Q79, 80A20, 76R05, 76D07
	
\maketitle
	
%**********************************************************************
%**********************************************************************
	
\section{Introduction}

The interaction between a free fluid and a deformable porous medium is a challenging multiphysics problem with diverse applications. 
For instance, in petroleum engineering, it is essential for predicting and optimizing the processes involved in efficient gas and oil extraction from hydraulically fractured reservoirs. 
In hydrology, it aids in tracking how pollutants discharged into ground and surface water enter the water supply and in designing strategies for cleaning water contaminants. 
In biomedicine, it plays a crucial role in modeling blood-vessel interactions, simulating low-density lipoprotein (LDL) transport, and optimizing drug delivery.
The free fluid region is typically modeled by the Stokes (or Navier--Stokes) equations, while the flow through the deformable porous medium is governed by the Biot system of poroelasticity. 
The latter combines the equations for the deformation of the elastic structure with Darcy's equation, which describes fluid flow through the porous medium. 
This coupling accounts for the poroelastic interaction.
The two regions are coupled through dynamic and kinematic interface conditions, which include the balance of forces, continuity of normal velocity, and a no-slip or slip-with-friction condition for tangential velocity. 
The fluid-porous structure interaction (FPSI) problem exhibits characteristics of both coupled Stokes--Darcy flows and fluid-structure interaction.

The literature on related coupled problems is extensive, and here we provide only a partial list of relevant publications. 
To the authors' knowledge, one of the first works to analyze the Stokes--Biot coupled problem is \cite{s2005}, where variational formulations for the Stokes--Biot system were developed using semigroup methods. 
The Navier--Stokes--Biot system was investigated numerically in \cite{bqq2009}, where the efficiency of monolithic solvers and heterogeneous domain decomposition strategies was compared using a test problem with a significant added-mass effect.
The Stokes-Biot system was further studied in \cite{byz2015} using a non-iterative operator-splitting scheme and in \cite{clqwy2016} employing an optimization-based decoupling strategy. 
Mixed Darcy formulations, where the continuity of flux is an essential condition, were considered in \cite{byzz2015}, leveraging Nitsche's interior penalty method. 
The resulting method was loosely coupled, non-iterative, and conditionally stable. 
The well-posedness of the fully dynamic incompressible Navier--Stokes equations coupled with the Biot system was established in \cite{Cesmelioglu2017}. 
The fully dynamic Navier--Stokes--Biot system with a mixed Darcy formulation was analyzed in \cite{wy2022} using a Lagrange multiplier method.
All the above-mentioned works are based on displacement formulations for the elasticity equation. 
In a recent contribution \cite{cly2022}, a fully mixed formulation of the quasi-static Stokes--Biot model was developed, relying on dual mixed formulations for Darcy, elasticity, and Stokes. 
Specifically, the study employed a velocity-pressure Darcy formulation, a weakly symmetric stress-displacement-rotation elasticity formulation, and a weakly symmetric stress-velocity-vorticity Stokes formulation. 
These features make the approach particularly suitable for multipoint stress mixed finite element approximations, reducible to a positive-definite cell-centered pressure-velocities-traces system.
More recently, a fully-mixed formulation of the quasi-static Navier--Stokes--Biot model was studied in \cite{lcy2022}. 
This formulation was augmented with suitable redundant Galerkin-type terms derived from the equilibrium and constitutive equations.

The goal of the present paper is to develop and analyze a new fully dynamic Banach space mixed formulation of the Navier--Stokes--Biot model and to study a suitable conforming numerical discretization. 
The Navier--Stokes--Biot model is particularly well-suited for fast flows, such as blood flow or flows through industrial filters. 
However, the problem is significantly more challenging due to the presence of the nonlinear convective term in the fluid momentum equation.
Previous works, such as \cite{Cesmelioglu2017,wy2022}, focus on the analysis of the weak formulation of the fully dynamic Navier--Stokes--Biot system. 
In this paper, the authors employ a classical dual mixed velocity-pressure formulation for the Darcy flow, a displacement-based formulation for the elasticity equation, and a weakly symmetric nonlinear pseudostress-velocity-vorticity formulation for the Navier--Stokes equations.
This new formulation offers several advantages, including local conservation of mass for the Darcy flow, local momentum conservation for the Navier--Stokes equations, and accurate approximations with continuous normal components across element edges or faces for the Darcy velocity and the free fluid stress.
More precisely, similarly to \cite{cot2017,cgo2021,lcy2022}, we introduce a nonlinear pseudostress tensor that combines the fluid stress tensor with the convective term. 
Subsequently, we eliminate the pressure unknown by utilizing the deviatoric tensor. 
To impose the weak symmetry of the Navier--Stokes pseudostress, vorticity is introduced as an additional unknown.
The approximation of the fluid stress in the $\bbH(\bdiv)$ space ensures the compatible enforcement of momentum conservation. 
The transmission conditions, including mass conservation, momentum conservation, and the Beavers--Joseph--Saffman slip with friction condition, are imposed weakly through the incorporation of two Lagrange multipliers: the traces of the fluid velocity and the Darcy pressure on the interface.
We note that the Banach space formulation leverages the natural function spaces that arise from applying the Cauchy--Schwarz and H\"older inequalities to the terms obtained by testing and integrating by parts the equations of the model.

The main contributions of this paper are as follows. The resulting weak formulation is a nonlinear time-dependent system, which is challenging to analyze due to the presence of the time derivative of the displacement in certain non-coercive terms. 
To address this difficulty, we consider an alternative mixed elasticity formulation as in \cite{s2005,aeny2019}, with the structural velocity and elastic stress as the primary variables.
This approach results in a system characterized by a degenerate evolution operator in time and a nonlinear saddle-point spatial operator. 
The structure of the problem is similar to that analyzed in \cite{s2010,aeny2019} (see also \cite{bg2004} for the linear case). 
However, the analysis in \cite{s2010} is restricted to the Hilbert space framework, while \cite{aeny2019} focuses on the Sobolev space setting. 
In this work, we conduct the analysis within a Banach space framework.
By employing techniques from \cite{Showalter} and \cite{cmo2018}, we combine classical monotone operator theory with a suitable regularization technique in Banach spaces to establish the well-posedness of the alternative formulation and demonstrate that its solution satisfies the original formulation. 
Furthermore, we prove the uniqueness of the solution to the original formulation and provide a stability bound.
We also introduce a semidiscrete continuous-in-time formulation based on stable mixed finite element spaces for the Navier--Stokes, Darcy, and elasticity equations, along with conforming Lagrange multiplier discretizations. 
Well-posedness and stability results are established using arguments analogous to those applied in the continuous case. 
Additionally, we perform an error analysis and establish rates of convergence for all variables.
Finally, we present numerical experiments employing a fully discrete finite element method with backward Euler time discretization to verify the theoretical rates of convergence and illustrate the method's behavior in modeling air flow through a filter with real-world parameters.

The rest of this work is organized as follows. 
The remainder of this section introduces the standard notation and functional spaces employed throughout the paper. 
In Section~\ref{sec:Navier-Stokes--Biot model problem}, we present the mathematical model along with the corresponding interface, boundary, and initial conditions. 
Section~\ref{sec:variational-formulation} focuses on the continuous weak formulation, which forms the basis of the numerical method, as well as an alternative formulation required for the analysis. 
Stability properties of the associated operators are also established. 
In Section~\ref{sec:well-posedness-model}, we prove the well-posedness of both the alternative and original formulations, employing a suitable fixed-point approach to establish the existence, uniqueness, and stability of the solution. 
Section~\ref{sec:Semidiscrete continuous-in-time approximation} introduces and analyzes the semi-discrete continuous-in-time approximation, including its well-posedness, stability, and error analysis. 
Finally, Section \ref{sec:numerical} presents the fully discrete scheme, which is then used to carry out the numerical experiments.

%**********************************************************************
%**********************************************************************

\subsection*{Notations}

Let $\cO\subseteq \R^n$, $n\in \{2,3\}$, denote a domain with Lipschitz boundary $\Gamma$.
For $s\geq 0$ and $p\in [1,+\infty]$, we denote by $\L^p(\cO)$ and $\W^{s,p}(\cO)$ the usual Lebesgue and Sobolev spaces endowed with the norms $\|\cdot\|_{\L^p(\cO)}$ and $\|\cdot\|_{\W^{s,p}(\cO)}$, respectively.
Note that $\W^{0,p}(\cO) = \L^p(\cO)$.
If $p=2$ we write $\H^s(\cO)$ in place of $\W^{s,2}(\cO)$, and denote the corresponding norm by $\|\cdot\|_{\H^s(\cO)}$.
In addition, we denote by $\H^{1/2}(\Gamma)$ the trace space of $\H^1(\cO)$, and let $\H^{-1/2}(\Gamma)$ be the dual space of $\H^{1/2}(\Gamma)$ endowed with the norms $\|\cdot\|_{\H^{1/2}(\Gamma)}$ and $\|\cdot\|_{\H^{-1/2}(\Gamma)}$, respectively.
By $\bM$ and $\bbM$ we will denote the corresponding vectorial and tensorial counterparts of the generic scalar functional space $\M$.
The $\L^2(\cO)$ inner product for scalar, vector, or tensor valued functions is denoted by $(\cdot,\cdot)_{\cO}$. For a section of the boundary $\Gamma$, the $\L^2(\Gamma)$ inner product or duality pairing is denoted by $\pil\cdot,\cdot\pir_\Gamma$. For a Banach space $\V$, we denote its dual space by $\V'$. 
For an operator $\cA:\V \to \U'$, its adjoint operator is denoted by $\cA':\U \to \V'$. 
For any vector fields $\bv=(v_i)_{i=1,n}$ and $\bw=(w_i)_{i=1,n}$, 
we set the gradient, the symmetric part of the gradient, divergence,
and tensor product operators, as
\begin{equation*}
\nabla\bv := \left(\frac{\partial v_i}{\partial x_j}\right)_{i,j=1,n},\quad 
\be(\bv) := \frac{1}{2}\left\{ \nabla\bv + (\nabla\bv)^\rt \right\},\quad
\div(\bv) := \sum_{j=1}^n \frac{\partial v_j}{\partial x_j},\qan 
\bv\otimes\bw := (v_i w_j)_{i,j=1,n} \,.
\end{equation*}
Furthermore, for any tensor field $\btau:=(\tau_{ij})_{i,j=1,n}$ and $\bzeta:=(\zeta_{ij})_{i,j=1,n}$, we let $\bdiv(\btau)$ be the divergence operator $\div$ acting along the rows of $\btau$, and define the transpose,
the trace, the tensor inner product, and the deviatoric tensor, respectively, as
\begin{equation}\label{eq:transpose-trace-tensor-deviatoric}
\btau^\rt := (\tau_{ji})_{i,j=1,n},\quad \tr(\btau):=\sum_{i=1}^n \tau_{ii},\quad \btau:\bzeta:=\sum_{i,j=1}^n \tau_{ij}\zeta_{ij},\qan \btau^\rd:=\btau-\frac{1}{n}\,\tr(\btau)\,\bI\,,
\end{equation}
where $\bI$ is the identity matrix in $\R^{n\times n}$.
In addition, we recall that
%\begin{equation*}
$\bH(\div;\cO):=\Big\{ \bw\in\bL^2(\cO) :\quad \div(\bw)\in \L^2(\cO) \Big\}$,
%\end{equation*}
equipped with the norm $\|\bw\|^2_{\bH(\div;\cO)} := \|\bw\|^2_{\bL^2(\cO)} + \|\div(\bw)\|^2_{\L^2(\cO)}$, is a standard Hilbert space in the realm of mixed problems.
The space of matrix valued functions whose rows belong to $\bH(\div;\cO)$ will be denoted by $\bbH(\bdiv;\cO)$ and endowed with the norm $\|\btau\|^2_{\bbH(\bdiv;\cO)} := \|\btau\|^2_{\bbL^2(\cO)} + \|\bdiv(\btau)\|^2_{\bL^2(\cO)}$. 
Moreover, given a separable Banach space $\V$ endowed with the norm $\| \cdot \|_{\V}$, we introduce
the Bochner spaces $\L^2(0,T;\V)$, $\H^s(0,T;\V)$, with integer $s \ge 1$,
and $\L^{\infty}(0,T;\V)$ and $\W^{1,\infty}(0,T;\V)$, endowed with the norms
\begin{equation*}
\begin{array}{c}
\ds\|f\|^{2}_{\L^{2}(0,T;\V)} \,:=\, \int^T_0 \|f(t)\|^{2}_{\V} \,dt \,,\quad
\ds\|f\|^2_{\H^s(0,T;\V)} \,:=\, \int^T_0 \sum^{s}_{i=0} \|\partial^{i}_t f(t)\|^2_{\V}\,dt\,, \\[3ex]
\ds\|f\|_{\L^\infty(0,T;\V)} \,:=\, \mathop{\esssup}\limits_{t\in [0,T]} \|f(t)\|_{\V}\, ,\quad \|f\|_{\W^{1,\infty}(0,T;\V)} \,:=\, \mathop{\esssup}\limits_{t\in [0,T]} \{\|f(t)\|_{\V} + \|\partial_t f(t)\|_{\V}\}.
\end{array}
\end{equation*}

%**********************************************************************
%**********************************************************************

\section{Navier--Stokes--Biot model problem}\label{sec:Navier-Stokes--Biot model problem}

Let $\Omega\subset \R^n$, $n\in\{2,3\}$ be a Lipschitz domain, which is subdivided into two polygonal non-overlapping and possibly non-connected regions: fluid region $\Omega_f$ and poroelastic region $\Omega_p$.
Let $\Gamma_{fp} := \partial\Omega_f\cap\partial\Omega_p$ denote the (nonempty) interface between these regions and let $\Gamma_f := \partial\Omega_f\setminus\Gamma_{fp}$ and $\Gamma_p := \partial\Omega_p\setminus\Gamma_{fp}$ denote the external parts of the boundary $\partial\Omega$.
In turn, given $\star\in\{f,p\}$, let $\Gamma_\star = \Gamma^\rD_\star\cup \Gamma^\rN_\star$, with $\Gamma^\rD_\star\cap \Gamma^\rN_\star = \emptyset$ and $|\Gamma^\rD_\star|, |\Gamma^\rN_\star| > 0$. We denote by $\bn_f$ and $\bn_p$ the unit normal vectors which point outward from $\partial\Omega_f$ and $\partial\Omega_p$, respectively, noting that $\bn_f = - \bn_p$ on $\Gamma_{fp}$. Figure \ref{fig:domain_sketch} gives a schematic representation of the geometry. 
%
\begin{figure}[t]
\centering\includegraphics[scale=1]{domain-sketch.png}
	
\vspace{-0.2cm}
\caption{Schematic representation of a 2D computational domain.}
\label{fig:domain_sketch}
\end{figure}
%
Let $(\bu_\star,p_\star)$ be the veloci\-ty-pressure pair in $\Omega_\star$, and let $\bbeta_p$ be the displacement in $\Omega_p$.
Let $\mu>0$ be the fluid viscosity, let $\rho_f$ be the density, let $\f_\star$ be 
the body force terms, and let $q_p$ be external source or sink term. The flow in $\Omega_f$ is governed by the Navier--Stokes equations:
%
\begin{align}
& \rho_f\,\left(\frac{\partial\,\bu_f}{\partial\,t} + (\nabla\bu_f)\,\bu_f\right) - \bdiv(\bT_f) \,=\, \f_f,\quad \div(\bu_f) \,=\, 0 \qin \Omega_f\times (0,T], \nonumber\\ 
&\left(\bT_f - \rho_f\,(\bu_f\otimes\bu_f) \right)\bn_f \,=\, \0 \qon \Gamma^\rN_f\times (0,T],\quad 
\bu_f \,=\, \0 \qon \Gamma^\rD_f\times (0,T],\label{eq:Navier-Stokes-1}
\end{align}
where $\bT_f := -p_f\,\bI + 2\,\mu\,\be(\bu_f)$ denotes the stress tensor.
While the standard Navier--Stokes equations are presented above to describe the behaviour of the fluid in $\Omega_f$, in this work we make use of an equivalent version of \eqref{eq:Navier-Stokes-1} based on the introduction of a pseudostress tensor relating the stress tensor $\bT_f$ with the convective term.
More precisely, analogously to \cite{cgo2021}, we introduce the nonlinear-pseudostress tensor
\begin{equation}\label{eq:pseudostress-tensor-formulae-1}
\bsi_f := \bT_f - \rho_f(\bu_f\otimes\bu_f) = -p_f\,\bI + 2\,\mu\,\be(\bu_f) - \rho_f(\bu_f\otimes \bu_f) \qin \Omega_f\times (0,T]\,.
\end{equation}
Now, in order to derive our weak formulation, similarly to \cite{cgot2016,cgos2017}, we first observe, owing to the fact that $\tr(\be(\bu_f)) = \div(\bu_f) = 0$ there hold
\begin{equation}\label{eq:div-tr-identities}
\bdiv(\bu_f\otimes\bu_f) \,=\, (\nabla\bu_f)\,\bu_f,\quad
\tr(\bsi_f) \,=\, -\,n\,p_f - \rho_f\,\tr(\bu_f\otimes\bu_f)\,.
\end{equation}
In particular, from the second equation in \eqref{eq:div-tr-identities} we observe that the pressure $p_f$ can be written in terms of the velocity $\bu_f$, and the nonlinear pseudostress tensor $\bsi_f$, as
\begin{equation}\label{eq:pseudostress-pressure-formulae}
p_f \,=\, -\frac{1}{n}\,\Big( \tr(\bsi_f) + \rho_f\,\tr(\bu_f\otimes\bu_f) \Big)  \qin \Omega_f\times (0,T] \,.
\end{equation}
%
Hence, replacing back \eqref{eq:pseudostress-pressure-formulae} into \eqref{eq:pseudostress-tensor-formulae-1}, and using the definition of the deviatoric operator \eqref{eq:transpose-trace-tensor-deviatoric}, we obtain
$\bsi^\rd_f \,=\, 2\,\mu\,\be(\bu_f) - \rho_f\,(\bu_f\otimes\bu_f)^\rd$.
In addition, to weakly impose the symmetry of the pseudostress tensor and apply the integration by parts formula, we introduce the additional unknown
\begin{equation*}
\bgamma_f \,:=\, \frac{1}{2}\,\left\{ \nabla\bu_f - (\nabla\bu_f)^\rt \right\}\,,
\end{equation*}
which represents the vorticity (or skew-symmetric part of the velocity gradient). 
Thus, employing the identities \eqref{eq:div-tr-identities}, the Navier--Stokes problem \eqref{eq:Navier-Stokes-1} can be rewritten as the set of equations with unknowns $\bsi_f, \bgamma_f$ and $\bu_f$, given by
\begin{equation}\label{eq:Navier-Stokes-3}
\begin{array}{c}
\ds \frac{1}{2\,\mu}\,\bsi^\rd_f = \nabla\bu_f - \bgamma_f - \frac{\rho_f}{2\,\mu}\,(\bu_f\otimes\bu_f)^\rd \,,\quad 
\rho_f\,\frac{\partial\,\bu_f}{\partial\,t} - \bdiv(\bsi_f) = \f_f \qin \Omega_f\times (0,T] \,, \\[2ex]
\ds \bsi_f\bn_f \,=\, \0 \qon \Gamma^\rN_f\times (0,T] \,,\quad 
\bu_f \,=\, \0 \qon \Gamma^\rD_f\times (0,T]\,,
\end{array}
\end{equation}
where $\bsi_f$ is a symmetric tensor in $\Omega_f\times (0,T]$.
Notice that, as suggested by \eqref{eq:pseudostress-pressure-formulae}, $p_f$ is eliminated from the present formulation and can be computed afterwards in terms of $\bsi_f$ and $\bu_f$.
In addition, the fluid stress $\bT_f$ can be recovered from \eqref{eq:pseudostress-tensor-formulae-1}.
For simplicity, we assume that $|\Gamma^\rN_f| > 0$, which allows us to control $\bsi_f$ by $\bsi^\rd_f$, cf. \eqref{eq:tau-H0div-Xf-inequality}. The case $|\Gamma^\rN_f| = 0$ can be handled as in \cite{gmor2014} and \cite{gov2020} by introducing an additional
variable corresponding to the mean value of $ \tr(\bsi_f)$. In addition, in order to simplify the characterization of the normal trace space $\bsi_f\bn_f|_{\Gamma_{fp}}$, we assume that $\Gamma^\rD_f$ is not adjacent to $\Gamma_{fp}$, i.e., $\dist(\Gamma^\rD_f,\Gamma_{fp}) \geq s >0$.

In turn, in $\Omega_p$ we consider the fully dynamic Biot system \cite{b1941}:
\begin{equation}\label{eq:Biot-model}
\begin{array}{c}
\ds \rho_p\,\frac{\partial^2\bbeta_p}{\partial\,t^2} - \bdiv(\bsi_p) = \f_p,\quad 
\mu\,\bK^{-1}\bu_p + \nabla\,p_p = \0 \qin \Omega_p\times(0,T] \,, \\[2ex]
\ds \frac{\partial}{\partial\,t}\left( s_0\,p_p + \alpha_p\,\div(\bbeta_p) \right) + \div(\bu_p) = q_p \qin \Omega_p\times(0,T] \,, \\[2ex]
\ds \bu_p\cdot\bn_p \,=\, 0 \qon \Gamma^\rN_p\times (0,T],\quad 
p_p \,=\, 0 \qon \Gamma^\rD_p\times (0,T],\quad
\bbeta_p \,=\, \0 \qon \Gamma_p\times (0,T]\,,
\end{array}
\end{equation}
where $\rho_p$ represents the fluid density in the poroelastic region, 
$s_0 > 0$ is a storage coefficient, $0 < \alpha_p \leq 1$ is the Biot--Willis constant and $\bK$ the symmetric and uniformly positive definite permeability tensor, satisfying, for some constants $0< k_{\min}\leq k_{\max}$,
\begin{equation}\label{eq:permeability-bounds}
\forall\, \bw\in\R^n \quad k_{\min}\,|\bw|^2 \,\leq\, \bw^\rt\,\bK^{-1}(\bx)\bw \,\leq\, k_{\max}\,|\bw|^2 \quad \forall\, \bx\in\Omega_p \,.
\end{equation}
In addition, $\bsi_e$ and $\bsi_p$ denote the elastic and poroelastic stress tensors, respectively, and both satisfy
\begin{equation}\label{eq:bsie-bsip-definitions}
A(\bsi_e) \,= \be(\bbeta_p)  \qan
\bsi_p \,:=\, \bsi_e - \alpha_p\,p_p\,\bI \,,
\end{equation}
%
where $A$ is a symmetric and positive definite compliance tensor,
satisfying, for some $0 < a_{\min} \leq a_{\max} < \infty$,
%
\begin{equation}\label{eq:A-bounds}
\forall\, \btau\in\R^{n\times n}, \quad a_{\min}
\, \btau : \btau \, \leq \, A(\btau):\btau \, \leq \, a_{\max} \,
\btau : \btau \quad \forall\, \bx\in\Omega_p \,.
\end{equation}
%
In this paper we will consider an isotropic material, with $A$ given as
%
\begin{equation}\label{eq:elasticity-stress-isotropic}
A(\bsi_e) \,=\, \frac{1}{2\,\mu_p} \left(\bsi_e - \frac{\lambda_p}{2\,\mu_p + n\,\lambda_p}\,\tr(\bsi_e)\,\bI\right),\quad\mbox{ with }\quad 
A^{-1}(\bsi_e) \,=\, 2\,\mu_p\,\bsi_e + \lambda_p\,\tr(\bsi_e)\,\bI \,,
\end{equation}
%
where $\ds 0<\lambda_{\min}\leq \lambda_p(\bx)\leq \lambda_{\max}$ and 
$\ds 0<\mu_{\min} \leq\mu_p(\bx) \leq\mu_{\max}$ are the Lam\'e parameters.
In this case, 
$\bsi_e \,:=\, \lambda_p\,\div(\bbeta_p)\,\bI + 2\,\mu_p\,\be(\bbeta_p)$,  $\ds a_{\min}=1/(2 \mu_{\max} + n \, \lambda_{\max})$, and 
$\ds a_{\max}=1/(2 \mu_{\min})$.
%
In order to avoid the issue with restricting the mean value of the pressure, we assume that $|\Gamma^\rD_p| > 0$. Also, in order to simplify the characterization of the normal trace space $\bu_p\cdot\bn_p|_{\Gamma_{fp}}$, we assume that $\Gamma^\rD_p$ is not adjacent to $\Gamma_{fp}$, i.e., $\dist(\Gamma^\rD_p,\Gamma_{fp}) \geq s >0$.

Next, we introduce the transmission conditions on the interface $\Gamma_{fp}$:
\begin{equation}\label{eq:interface-conditions}
\begin{array}{c}
\ds \bu_f\cdot\bn_f + \left(\frac{\partial\,\bbeta_p}{\partial\,t} + \bu_p\right)\cdot\bn_p \,=\, 0,\quad 
\bT_f\bn_f + \bsi_p\bn_p \,=\, \0 \qon \Gamma_{fp}\times (0,T] \,, \\[2ex]
\ds \bT_f\bn_f + \mu\,\alpha_{\BJS}\sum^{n-1}_{j=1}\,\sqrt{\bK^{-1}_j}\left\{\left(\bu_f - \frac{\partial\,\bbeta_p}{\partial\,t}\right)\cdot\bt_{f,j}\right\}\,\bt_{f,j} \,=\, -\,p_p\bn_f \qon \Gamma_{fp}\times (0,T] \,,
\end{array}	
\end{equation}
where $\bt_{f,j}$, $1\leq j\leq n-1$, is an orthogonal system of unit tangent vectors on $\Gamma_{fp}$, $\bK_j = (\bK\,\bt_{f,j})\cdot\bt_{f,j}$, and $\alpha_{\BJS} \geq 0$ is an experimentally determined friction coefficient.
The first and second equations in \eqref{eq:interface-conditions} corresponds to mass conservation and conservation of momentum on $\Gamma_{fp}$, respectively, whereas the third one can be decomposed into its normal and tangential components, as follows:
\begin{equation}\label{eq:BJS-condition-split}
(\bT_f\bn_f)\cdot\bn_f \,=\, -\,p_p,\quad
(\bT_f\bn_f)\cdot\bt_{f,j} \,=\, -\,\mu\,\alpha_{\BJS}\,\sqrt{\bK^{-1}_j}\left(\bu_f - \frac{\partial\,\bbeta_p}{\partial\,t}\right)\cdot\bt_{f,j} \qon \Gamma_{fp}\times (0,T]\,.
\end{equation}
The first condition in \eqref{eq:BJS-condition-split} corresponds to the balance of normal stress, whereas the second one is known as the Beavers--Joseph--Saffman slip with friction condition.
Notice that the second and third equations in \eqref{eq:interface-conditions} can be rewritten in terms of tensor $\bsi_f$ as follows:
\begin{equation}\label{eq:BJS-condition-2}
\begin{array}{c}
\ds \bsi_f\bn_f + \rho_f(\bu_f\otimes \bu_f)\bn_f + \bsi_p\bn_p = \0 \,, \\[2ex]
\ds \bsi_f\bn_f + \rho_f(\bu_f\otimes \bu_f)\bn_f
+ \mu\,\alpha_{\BJS}\sum^{n-1}_{j=1}\,\sqrt{\bK^{-1}_j}\left\{\left(\bu_f - \frac{\partial\,\bbeta_p}{\partial\,t}\right)\cdot\bt_{f,j}\right\}\,\bt_{f,j} \,=\, -\,p_p\bn_f\,.
\end{array}	
\end{equation}

 
Finally, the above system of equations is complemented by a set of initial conditions:
\begin{equation*}
\begin{array}{c}
\ds \bu_f(\bx,0) \,=\, \bu_{f,0}(\bx) \qin \Omega_f\,,\quad	
p_p(\bx,0) \,=\, p_{p,0}(\bx) \qin \Omega_p \,, \\[1ex]
\ds \bbeta_p(\bx,0) = \bbeta_{p,0}(\bx),\qan
\partial_t\,\bbeta_p(\bx,0) = \bu_{s,0}(\bx) \qin \Omega_p,
\end{array}
\end{equation*}
where $\bu_{f,0}, p_{p,0}, \bbeta_{p,0}$, and $\bu_{s,0}$ are suitable initial data.
Conditions on these initial data are discussed later on in Lemma \ref{lem:sol0-in-M-operator}.

%**********************************************************************
%**********************************************************************

\section{The variational formulation}\label{sec:variational-formulation}

In this section we proceed analogously to \cite[Section~3]{aeny2019} (see also \cite{cgos2017,cly2022,gmor2014}) and derive a weak formulation of the coupled problem given by \eqref{eq:Navier-Stokes-3}, \eqref{eq:Biot-model}, and \eqref{eq:interface-conditions}--\eqref{eq:BJS-condition-2}.
 
 
\subsection{Preliminaries}

We first introduce further notations and definitions. In what follows, given $\star\in\big\{ f, p \big\}$, we set
\begin{equation*}
(p,q)_{\Omega_\star} \,:=\, \int_{\Omega_\star} p\,q,\quad 
(\bu,\bv)_{\Omega_\star} \,:=\, \int_{\Omega_\star} \bu\cdot\bv \qan
(\bsi,\btau)_{\Omega_\star} \,:=\, \int_{\Omega_\star} \bsi:\btau.
\end{equation*}
In addition, similarly to \cite{cgo2021} and \cite{cgos2017}, in the sequel we will employ suitable Banach spaces to deal with the nonlinear stress tensor and velocity of the Navier--Stokes equation, together with the subspace of skew-symmetric tensors of $\bbL^2(\Omega_f)$ for the vorticity, that is,
\begin{equation*}
\begin{array}{l}
\bbX_f \,:=\, \Big\{ \btau_f\in \bbL^2(\Omega_f) :\quad \bdiv(\btau_f)\in \bL^{4/3}(\Omega_f) \qan \btau_f\bn_f = \0 \qon \Gamma^\rN_f \Big\}, \\ [2ex] 
\bV_f \,:=\, \bL^4(\Omega_f)\,,\quad
\bbQ_f \,:=\, \Big\{ \bchi_f\in \bbL^2(\Omega_f) :\quad \bchi^\rt_f = - \bchi_f \Big\}\,,
\end{array}
\end{equation*}
endowed with the corresponding norms
\begin{equation*}
\|\btau_f\|^2_{\bbX_f} \,:=\, \|\btau_f\|^2_{\bbL^2(\Omega_f)} + \|\bdiv(\btau_f)\|^2_{\bL^{4/3}(\Omega_f)},\quad
\|\bv_f\|_{\bV_f} \,:=\, \|\bv_f\|_{\bL^4(\Omega_f)},\quad
\|\bchi_f\|_{\bbQ_f} \,:=\, \|\bchi_f\|_{\bbL^2(\Omega_f)}.
\end{equation*}
In turn, in order to deal with the velocity, pressure and displacement on the poroelastic region $\Omega_p$, we will use standard Hilbert spaces, respectively,
\begin{equation*}
\begin{array}{l}
\bX_p \,:=\, \Big\{ \bv_p\in \bH(\div;\Omega_p) :\quad \bv_p\cdot\bn = 0 \qon \Gamma^\rN_p \Big\}, \\ [2ex]
\W_p \,:=\, \L^2(\Omega_p)\,,\quad
\bV_p \,:=\, \Big\{ \bxi_p\in \bH^1(\Omega_p) :\quad \bxi_p = \0 \qon \Gamma_p \Big\}, 
\end{array}
\end{equation*}
endowed with the standard norms
\begin{equation*}
\|\bv_p\|_{\bX_p} \,:=\, \|\bv_p\|_{\bH(\div;\Omega_p)},\quad
\|w_p\|_{\W_p} \,:=\, \|w_p\|_{\L^2(\Omega_p)},\quad
\|\bxi_p\|_{\bV_p} \,:=\, \|\bxi_p\|_{\bH^1(\Omega_p)}.
\end{equation*}
Finally, analogously to \cite{gmor2014,cgos2017} we need to introduce the spaces of traces $\Lambda_p := (\bX_p\cdot\bn_p|_{\Gamma_{fp}})'$ and $\bLambda_f := (\bbX_f\bn_f|_{\Gamma_{fp}})'$.
According to the normal trace theorem, since $\bv_p\in \bX_p\subset \bH(\div;\Omega_p)$, then $\bv_p\cdot\bn_p\in \H^{-1/2}(\partial\,\Omega_p)$.
It is shown in \cite{akyz2018} that, since $\bv_p\cdot\bn_p = 0$ on $\Gamma^\rN_p$ and $\dist(\Gamma^\rD_p,\Gamma_{fp}) \geq s> 0$, it holds that
$\bv_p\cdot\bn_p\in \H^{-1/2}(\Gamma_{fp})$ and
%
\begin{equation}\label{eq:trace-inequality-1}
\pil \bv_p \cdot \bn_p, \xi \pir_{\Gamma_{fp}} 
\,\leq\, C\,\| \bv_p \|_{\bH(\div; \Omega_p)} \| \xi \|_{\H^{1/2}(\Gamma_{fp})} \quad \forall\,\bv_p \in \bX_p, \, \xi \in \H^{1/2}(\Gamma_{fp}).
\end{equation}
%
Since $\dist(\Gamma^\rD_f,\Gamma_{fp}) \geq s> 0$, 
using similar arguments, in combination with \cite[Lemma~3.5]{cgo2021}, we have,
%
\begin{equation}\label{eq:trace-inequality-2}
\langle \btau_f \, \bn_f,\bpsi \rangle_{\Gamma_{fp}}
\le C \|\btau_f\|_{\bbH(\bdiv_{4/3};\Omega_f)}\|\bpsi\|_{\bH^{1/2}(\Gamma_{fp})}
\quad \forall \, \btau_f \in \bbX_f, \,
\bpsi \in \bH^{1/2}(\Gamma_{fp}).
\end{equation}
%
Therefore we take
%
$\Lambda_p = \H^{1/2}(\Gamma_{fp})$ and $\bLambda_f = \bH^{1/2}(\Gamma_{fp})$,
%
endowed with the norms
%
\begin{equation}\label{eq:H1/2-norms}
\|\xi\|_{\Lambda_p} \,:=\, \|\xi\|_{\H^{1/2}(\Gamma_{fp})} \quad \text{and} \quad
\|\bpsi\|_{\bLambda_f} \,:=\, \|\bpsi\|_{\bH^{1/2}(\Gamma_{fp})}\,.
\end{equation} 

%**********************************************************************
%**********************************************************************

\subsection{Lagrange multiplier formulation}

We now proceed with the derivation of our Lagrange multiplier variational formulation for the coupling of the Navier--Stokes and Biot problems.
To this end, we begin by introducing two Lagrange multiplier which has a meaning of Navier--Stokes velocity and Darcy pressure on the interface, respectively,
\begin{equation*}
\bvarphi := \bu_f|_{\Gamma_{fp}} \in \bLambda_f \qan 
\lambda := p_p|_{\Gamma_{fp}}\in \Lambda_p \,.
\end{equation*}
Then, similarly to \cite{cgos2017,aeny2019}, we test the first and second equation of \eqref{eq:Navier-Stokes-3}, and the first, second and third equations of \eqref{eq:Biot-model} with arbitrary $\btau_f\in \bbX_f$, $\bv_f\in \bV_f$, $\bxi_p\in \bV_p,\bv_p\in \bX_p$ and $w_p\in\W_p,$  integrate by parts, utilize the fact that $\bsi^\rd_f:\btau_f = \bsi^\rd_f:\btau^\rd_f$, and impose the remaining equations weakly, as well as the symmetry of $\bsi_f$ and the transmission conditions (cf. first equation of \eqref{eq:interface-conditions} and \eqref{eq:BJS-condition-2}) to obtain the variational problem: Given 
$\f_f : [0,T]\to \bL^2(\Omega_f)$, $\f_p : [0,T]\to \bL^2(\Omega_p)$, $q_p : [0,T]\to \L^2(\Omega_p)$
and $(\bu_{f,0},p_{p,0},\bbeta_{p,0},\bu_{s,0})\in \bV_f\times \W_p\times \bV_p\times \bV_p$\,,
find $(\bsi_f,\bu_p, \bbeta_p, \bu_f, p_p, \bgamma_f, \bvarphi, \lambda) : [0,T]\to \bbX_f\times \bX_p\times \bV_p\times \bV_f\times \W_p\times \bbQ_f\times \bLambda_f\times \Lambda_p$, such that $\bu_f(0)=\bu_{f,0}$, $p_p(0)=p_{p,0}$, $\bbeta_p(0)=\bbeta_{p,0}$ and $\partial_t\,\bbeta_p(0) = \bu_{s,0}$, and for a.e. $t\in (0,T)$:
\begin{subequations}\label{eq:continuous-weak-formulation-1}
\begin{align}
& \ds \frac{1}{2\,\mu}\,(\bsi^\rd_f,\btau^\rd_f)_{\Omega_f} - \pil\btau_f\bn_f,\bvarphi \pir_{\Gamma_{fp}} + (\bu_f,\bdiv(\btau_f))_{\Omega_f} + (\bgamma_f,\btau_f)_{\Omega_f} 
+ \frac{\rho_f}{2\,\mu}\,((\bu_f\otimes\bu_f)^\rd,\btau_f)_{\Omega_f} = 0, \label{eq:continuous-weak-formulation-1a}\\[1ex]
& \ds \rho_f\, (\partial_t\,\bu_f,\bv_f)_{\Omega_f} - (\bv_f,\bdiv(\bsi_f))_{\Omega_f} = (\f_f,\bv_f)_{\Omega_f}, \label{eq:continuous-weak-formulation-1b} \\[1ex]
& \ds -\,\,(\bsi_f,\bchi_f)_{\Omega_f} = 0, \label{eq:continuous-weak-formulation-1c} \\[1ex]
& \ds \rho_p (\partial_{tt}\bbeta_p,\bxi_p)_{\Omega_p} + 2\,\mu_p\,(\be(\bbeta_p),\be(\bxi_p))_{\Omega_p} + \lambda_p\,(\div(\bbeta_p),\div(\bxi_p))_{\Omega_p} -\,\,\alpha_p\,(p_p,\div(\bxi_p))_{\Omega_p} \nonumber \\[0.5ex] 
& \ds\quad +\pil\bxi_p\cdot\bn_p,\lambda\pir_{\Gamma_{fp}} - \mu\,\alpha_{\BJS}\,\sum_{j=1}^{n-1} \pil\sqrt{\bK^{-1}_j}\left( \bvarphi - \partial_t\,\bbeta_p \right)\cdot\bt_{f,j},\bxi_p\cdot\bt_{f,j} \pir_{\Gamma_{fp}} = (\f_p,\bxi_p)_{\Omega_p}, \label{eq:continuous-weak-formulation-1d} \\[1ex]
& \ds \mu\,(\bK^{-1}\bu_p,\bv_p)_{\Omega_p} - (p_p,\div(\bv_p))_{\Omega_p} + \pil\bv_p\cdot\bn_p,\lambda\pir_{\Gamma_{fp}} = 0, \label{eq:continuous-weak-formulation-1e} \\[1ex]
& \ds s_0\,(\partial_t\, p_p,w_p)_{\Omega_p} + \alpha_p\,(\div(\partial_t\,\bbeta_p), w_p)_{\Omega_p} + (w_p,\div(\bu_p))_{\Omega_p} = (q_p,w_p)_{\Omega_p},\label{eq:continuous-weak-formulation-1f} \\[1ex]
& \ds - \pil\bvarphi\cdot\bn_f + \left(\partial_t\,\bbeta_p + \bu_p\right)\cdot\bn_p,\xi\pir_{\Gamma_{fp}} = 0, \label{eq:continuous-weak-formulation-1g} \\[1ex]
& \ds \pil\bsi_f\bn_f,\bpsi\pir_{\Gamma_{fp}} + \mu\,\alpha_{\BJS}\,\sum_{j=1}^{n-1} \pil\sqrt{\bK^{-1}_j}\left( \bvarphi - \partial_t\,\bbeta_p \right)\cdot\bt_{f,j},\bpsi\cdot\bt_{f,j} \pir_{\Gamma_{fp}} \nonumber \\[0.5ex]
 & \ds\quad +\,\, \rho_f\pil \bvarphi\cdot\bn_f, \bvarphi\cdot\bpsi \pir_{\Gamma_{fp}} + \pil\bpsi\cdot\bn_f,\lambda\pir_{\Gamma_{fp}} = 0,\label{eq:continuous-weak-formulation-1h}
\end{align}
\end{subequations}
for all $(\btau_f, \bv_p, \bxi_p, \bv_f, w_p, \bchi_f, \bpsi, \xi)\in \bbX_f\times \bX_p\times \bV_p\times \bV_f\times \W_p\times \bbQ_f\times \bLambda_f\times \Lambda_p$.

We note that the fifth and sixth terms on the left-hand side of \eqref{eq:continuous-weak-formulation-1d} are obtained using the equation
\begin{equation}\label{eq:elasticity-stress-tensor-identity}
\pil\bsi_p\bn_p,\bxi_p\pir_{\Gamma_{fp}} 
= - \pil\bxi_p\cdot\bn_p,\lambda\pir_{\Gamma_{fp}} 
+ \mu\,\alpha_{\BJS}\,\sum_{j=1}^{n-1} \pil\sqrt{\bK^{-1}_j}\left( \bvarphi - \partial_t\,\bbeta_p \right)\cdot\bt_{f,j},\bxi_p\cdot\bt_{f,j} \pir_{\Gamma_{fp}},
\end{equation}
which follows from combining the second and third equations in \eqref{eq:interface-conditions} (or \eqref{eq:BJS-condition-2}). 
Note that \eqref{eq:continuous-weak-formulation-1a}--\eqref{eq:continuous-weak-formulation-1c} correspond to the Navier--Stokes equations, \eqref{eq:continuous-weak-formulation-1d} is the elasticity equation, \eqref{eq:continuous-weak-formulation-1e}--\eqref{eq:continuous-weak-formulation-1f} are the Darcy equations, whereas \eqref{eq:continuous-weak-formulation-1g}--\eqref{eq:continuous-weak-formulation-1h}, together with the interface terms in \eqref{eq:continuous-weak-formulation-1d}, enforce weakly the transmission conditions \eqref{eq:interface-conditions}. 
In particular, \eqref{eq:continuous-weak-formulation-1g} imposes the mass conservation, \eqref{eq:continuous-weak-formulation-1h} imposes the last equation in \eqref{eq:interface-conditions} which is a combination of balance of normal stress and the BJS condition, while the interface terms in \eqref{eq:continuous-weak-formulation-1d} imposes the conservation of momentum.

We emphasize that directly analyzing \eqref{eq:continuous-weak-formulation-1} is challenging due to the presence of $\partial_t\bbeta_p$ in several non-coercive terms (in addition to the nonlinear nature of the model). 
Motivated by \cite{ejs2009}, \cite{s2005}, and \cite{aeny2019}, we analyze an alternative formulation, which will then be used to establish the well-posedness of \eqref{eq:continuous-weak-formulation-1}.

%**********************************************************************
%**********************************************************************
 
\subsection{Alternative formulation}

We proceed analogously to \cite{aeny2019} and derive a system of evolutionary saddle point type, which fits the general framework studied in \cite{s2010}.
Following the approach from \cite{s2005}, we do this by considering a mixed elasticity formulation with the structure velocity $\bu_s := \partial_t\,\bbeta_p\in \bV_p$ and elastic stress as primary variables.
We begin recalling that the elasticity stress tensor $\bsi_e$ is connected to the displacement $\bbeta_p$ through the relation \eqref{eq:bsie-bsip-definitions}. The space and norm for the elastic stress are respectively
%
\begin{equation*}
\bSigma_e \,:=\, \Big\{ \btau_e\in \bbL^2(\Omega_p) :\quad \btau^\rt_e \,=\, \btau_e \Big\} \qan
\|\btau_e\|_{\bSigma_e} \,:=\, \|\btau_e\|_{\bbL^2(\Omega_p)} \,.
\end{equation*}
%
The derivation of the alternative variational formulation differs from the original one in the way the equilibrium equation $\rho_p\,\partial_{tt}\bbeta_p - \bdiv(\bsi_p) = \f_p$ (cf. \eqref{eq:Biot-model}) is handled.
As before, we multiply it by a test function $\bv_s\in \bV_p$ and integrate by parts.
However, instead of using the constitutive relation from the first equation in \eqref{eq:bsie-bsip-definitions}, we use only the second equation in \eqref{eq:bsie-bsip-definitions}, which results in
\begin{equation*}
\rho_p\,(\partial_t\bu_s,\bv_s)_{\Omega_p} + (\bsi_e, \be(\bv_s))_{\Omega_p} - \alpha_p\,(p_p,\div(\bv_s))_{\Omega_p} - \pil\bsi_p\bn_p, \bv_s\pir_{\Gamma_{fp}} \,=\, (\f_p, \bv_s)_{\Omega_p} \,,
\end{equation*}
where the term $\partial_{tt} \bbeta_p$ has been replaced by $\partial_t \bu_s$, and the last term on the left-hand side is treated as in \eqref{eq:continuous-weak-formulation-1d} by employing \eqref{eq:elasticity-stress-tensor-identity}.
Furthermore, we eliminate the displacement $\bbeta_p$ from the system by differentiating in time the first equation in \eqref{eq:bsie-bsip-definitions}. 
Multiplying by a test function $\btau_e\in \bSigma_e$ gives
%
\begin{equation*}
(A(\partial_t\,\bsi_e), \btau_e)_{\Omega_p} - (\be(\bu_s), \btau_e)_{\Omega_p} \,=\, 0 \,.
\end{equation*}
The rest of the equations are handled in the same way as in the original weak formulation \eqref{eq:continuous-weak-formulation-1}.

Next, given $\bw_f\in \bV_f$ and $\bzeta \in \bLambda_f$, we define the bilinear forms $a_f : \bbX_f\times \bbX_f\to \R, a^s_p : \bSigma_e\times \bSigma_e\to \R, \kappa_{\bw_f} : \bV_f\times \bbX_f\to \R, a^d_p : \bX_p\times \bX_p\to \R, a^e_p : \bV_p\times \bV_p\to \R, b_f : \bV_f\times \bbX_f\to \R$, $b_p : \W_p\times \bX_p\to \R, b_s : \bSigma_e \times \bV_p\to \R$, and $b_\sk : \bbQ_f\times \bbX_f\to \R$, as:
\begin{subequations}\label{eq:bilinear-forms}
\begin{align}
& a_f(\bsi_f,\btau_f) := \frac{1}{2\,\mu} (\bsi^\rd_f,\btau^\rd_f)_{\Omega_f} ,\quad a^s_p(\bsi_e,\btau_e) \,:=\, (A(\bsi_e),\btau_e)_{\Omega_p}, \label{bilinear-form-1} \\[0.5ex]
& \ds \kappa_{\bw_f}(\bu_f,\btau_f) := \frac{\rho_f}{2\,\mu} ((\bw_f\otimes\bu_f)^\rd, \btau_f)_{\Omega_f}, \quad a^d_p(\bu_p,\bv_p) := \mu (\bK^{-1}\bu_p,\bv_p)_{\Omega_p} , \label{bilinear-form-2}\\[0.5ex]
& \ds a^e_p(\bbeta_p,\bxi_p) := 2 \mu_p (\be(\bbeta_p),\be(\bxi_p))_{\Omega_p} + \lambda_p (\div(\bbeta_p),\div(\bxi_p))_{\Omega_p}, \label{bilinear-form-3}\\[0.5ex]
& \ds b_f(\btau_f,\bv_f) := (\bdiv(\btau_f),\bv_f)_{\Omega_f} ,\quad
b_p(\bv_p,w_p) := - (\div(\bv_p),w_p)_{\Omega_p} , \label{bilinear-form-4}\\[0.5ex]
& \ds b_s(\btau_e,\bv_s) := -\,(\btau_e,\be(\bv_s))_{\Omega_p},\quad
b_\sk(\btau_f,\bchi_f) := (\btau_f,\bchi_f)_{\Omega_f} , \label{bilinear-form-5}
\end{align}
\end{subequations}
and the forms on the interface $c_\BJS : (\bV_p\times \bLambda_f)\times (\bV_p\times \bLambda_f)\to \R$, $c_{\Gamma} : (\bV_p\times \bLambda_f)\times \Lambda_p\to \R$, $l_{\bzeta} : \bLambda_f \times \bLambda_f\to \R$, $b_{\bn_f} : \bbX_f\times \bLambda_f\to \R$, and $b_{\bn_p} : \bX_p\times \Lambda_p\to \R$
\begin{subequations}\label{eq:forms-on-interface-1}
\begin{align}
& c_{\BJS}(\bbeta_p, \bvarphi;\bxi_p, \bpsi) 
\,:=\, \mu\,\alpha_{\BJS}\,\sum^{n-1}_{j=1} \pil\sqrt{\bK^{-1}_j}( \bvarphi-\bbeta_p)\cdot\bt_{f,j},(\bpsi-\bxi_p)\cdot\bt_{f,j}\pir_{\Gamma_{fp}} \,, \label{interface-bilinear-form-1}\\[0.5ex]
& \ds c_{\Gamma}(\bxi_p,\bpsi;\xi) \,:=\, -\,\pil\bxi_p\cdot\bn_p,\xi\pir_{\Gamma_{fp}} - \pil\bpsi\cdot\bn_f,\xi\pir_{\Gamma_{fp}}\,,\quad
l_{\bzeta}(\bvarphi,\bpsi):=\rho_f\pil \bzeta\cdot\bn_f, \bvarphi\cdot\bpsi \pir_{\Gamma_{fp}} \,, \label{interface-bilinear-form-2}\\[0.5ex]
& \ds  b_{\bn_f}(\btau_f,\bpsi) \,:=-\, \pil\btau_f\bn_f,\bpsi\pir_{\Gamma_{fp}} \,,\quad
b_{\bn_p}(\bv_p,\xi) \,:=\, \pil\bv_p\cdot\bn_p, \xi\pir_{\Gamma_{fp}} 
\label{interface-bilinear-form-3} \,.
\end{align}
\end{subequations}
Hence the Lagrange variational formulation alternative to \eqref{eq:continuous-weak-formulation-1}, reads:
given $\f_f:[0,T]\to \bL^2(\Omega_f),\, \f_p : [0,T]\to \bL^2(\Omega_p),\, q_p:[0,T]\to \L^2(\Omega_p)$ and $(\bu_{f,0},p_{p,0},\bu_{s,0})\in \bV_f\times \W_p\times \bV_p$, find $(\bsi_f, \bu_p, \bsi_e, \bu_f, p_p, \bgamma_f, \newline \bu_s, \bvarphi, \lambda) : [0,T]\to \bbX_f\times \bX_p\times \bSigma_e\times \bV_f\times \W_p\times \bbQ_f\times \bV_p\times \bLambda_f\times \Lambda_p$, such that $(\bu_f(0),p_p(0),\bu_s(0)) = (\bu_{f,0},p_{p,0},\bu_{s,0})$ and for a.e. $t\in (0,T)$:
\begin{subequations}\label{eq:continuous-alternative-weak-formulation-1}
\begin{align}
& \ds a_f(\bsi_f,\btau_f) + b_{\bn_f}(\btau_f,\bvarphi) + b_f(\btau_f,\bu_f) + b_\sk(\bgamma_f,\btau_f)+\kappa_{\bu_f}(\bu_f, \btau_f) = 0, \label{eq:continuous-alternative-weak-formulation-1a}\\[1ex]
& \ds \rho_f\, (\partial_t\,\bu_f,\bv_f)_{\Omega_f} -b_f(\bsi_f,\bv_f) = (\f_f,\bv_f)_{\Omega_f}, \label{eq:continuous-alternative-weak-formulation-1b} \\[1ex]
& \ds - b_\sk(\bsi_f,\bchi_f) = 0, \label{eq:continuous-alternative-weak-formulation-1c} \\[1ex]
& \ds \rho_p\,(\partial_t\bu_s,\bv_s)_{\Omega_p}  - b_s(\bsi_e,\bv_s)  + \alpha_p\,b_p(\bv_s,p_p) - c_{\Gamma}(\bv_s,\0;\lambda)+\,\, c_{\BJS}(\bu_s,\bvarphi;\bv_s,\0)  \nonumber \\[0.5ex] 
& \ds = (\f_p,\bv_s)_{\Omega_p},\label{eq:continuous-alternative-weak-formulation-1d1} \\[1ex]
& \ds a^s_p(\partial_t\,\bsi_e,\btau_e) + b_s(\btau_e,\bu_s)= 0, \label{eq:continuous-alternative-weak-formulation-1d2} \\[1ex]
& \ds a^d_p(\bu_p,\bv_p) + b_p(\bv_p,p_p) + b_{\bn_p}(\bv_p,\lambda) = 0, \label{eq:continuous-alternative-weak-formulation-1e} \\[1ex]
& \ds s_0\,(\partial_t\, p_p,w_p)_{\Omega_p}  - \alpha_p\,b_p(\bu_s,w_p)  - b_p(\bu_p,w_p) = (q_p,w_p)_{\Omega_p},\label{eq:continuous-alternative-weak-formulation-1f} \\[1ex]
& \ds c_{\Gamma}(\bu_s,\bvarphi;\xi) - b_{\bn_p}(\bu_p,\xi) = 0, \label{eq:continuous-alternative-weak-formulation-1g} \\[1ex]
& \ds - b_{\bn_f}(\bsi_f,\bpsi) + \,\, c_{\BJS}(\bu_s,\bvarphi;\0,\bpsi) + l_{\bvarphi}(\bvarphi,\bpsi) - c_{\Gamma}(\0,\bpsi;\lambda) = 0,\label{eq:continuous-alternative-weak-formulation-1h}
\end{align}
\end{subequations}
for all $(\btau_f, \bv_p, \btau_e, \bv_f, w_p, \bchi_f, \bv_s, \bpsi, \xi)\in \bbX_f\times \bX_p\times \bSigma_e\times \bV_f\times \W_p\times \bbQ_f\times \bV_p\times \bLambda_f\times \Lambda_p$. We note that \eqref{eq:continuous-alternative-weak-formulation-1} also requires initial data for $\bsi_e(0)$, which will be constructed in Lemma \ref{lem:sol0-in-M-operator}.

There are many different way of ordering the Lagrange multiplier formulation described above, but for the sake of the subsequent analysis, we proceed as in \cite{aeny2019}, and adopt one leading to an evolution problem in a mixed form. Hence, \eqref{eq:continuous-alternative-weak-formulation-1} results in:
given $\f_f:[0,T]\to \bL^2(\Omega_f),\, \f_p : [0,T]\to \bL^2(\Omega_p),\, q_p:[0,T]\to \L^2(\Omega_p)$ and $(\bu_{f,0},p_{p,0},\bu_{s,0})\in \bV_f\times \W_p\times \bV_p$,
find $(\bsi_f, \bu_p, \bsi_e, \bu_f, p_p, \bgamma_f, \bu_s, \bvarphi, \lambda) : [0,T]\to \bbX_f\times \bX_p\times \bSigma_e\times \bV_f\times \W_p\times \bbQ_f\times \bV_p\times \bLambda_f\times \Lambda_p$, such that $(\bu_f(0),p_p(0),\bu_s(0)) = (\bu_{f,0},p_{p,0},\bu_{s,0})$ and for a.e. $t\in (0,T)$:
\begin{align}
& a_f(\bsi_f,\btau_f) + \kappa_{\bu_f}(\bu_f, \btau_f) + a^d_p(\bu_p,\bv_p) + a^s_p(\partial_t\,\bsi_e,\btau_e) \nonumber \\[0.5ex] 
&\quad +\, b_f(\btau_f,\bu_f) + b_p(\bv_p,p_p) + b_\sk(\btau_f,\bgamma_f)
+ b_s(\btau_e,\bu_s) + b_{\bn_f}(\btau_f,\bvarphi) + b_{\bn_p}(\bv_p,\lambda)  \,=\, 0 \,,  \nonumber \\[0.5ex] 
& \rho_f\,(\partial_t\,\bu_f,\bv_f)_{\Omega_f} + s_0\,(\partial_t\,p_p,w_p)_{\Omega_p}
+ \rho_p\,(\partial_t\bu_s,\bv_s)_{\Omega_p}  \nonumber \\[0.5ex] 
&\quad +\, c_{\BJS}(\bu_s,\bvarphi;\bv_s,\bpsi) + c_{\Gamma}(\bu_s,\bvarphi;\xi) - c_{\Gamma}(\bv_s,\bpsi;\lambda) 
+ \alpha_p\,b_p(\bv_s,p_p) - \alpha_p\,b_p(\bu_s,w_p)  \nonumber \\[0.5ex] 
&\quad -\, b_f(\bsi_f,\bv_f) - b_p(\bu_p,w_p) - b_\sk(\bsi_f,\bchi_f)  
- b_s(\bsi_e,\bv_s) - b_{\bn_f}(\bsi_f,\bpsi) - b_{\bn_p}(\bu_p,\xi) \nonumber \\[0.5ex] 
&\quad +\, l_{\bvarphi}(\bvarphi,\bpsi) =\, (\f_f,\bv_f)_{\Omega_f} +(q_p,w_p)_{\Omega_p} + (\f_p,\bv_s)_{\Omega_p}\,, \label{eq:NS-Biot-formulation-2}
\end{align}
for all $(\btau_f, \bv_p, \btau_e, \bv_f, w_p, \bchi_f, \bv_s, \bpsi, \xi)\in \bbX_f\times \bX_p\times \bSigma_e\times \bV_f\times \W_p\times \bbQ_f\times \bV_p\times \bLambda_f\times \Lambda_p$. 


\medskip

Now, we group the spaces, unknowns, and test functions as follows:
\begin{equation*}
\begin{array}{c}
\ds \bQ := \bbX_f\times \bX_p\times \bSigma_e,\quad
\bS := \bV_f\times \W_p\times \bbQ_f\times \bV_p\times \bLambda_f\times \Lambda_p, \\ [1.5ex]
\ds \ubsi := (\bsi_f, \bu_p, \bsi_e)\in \bQ,\quad 
\ubu := (\bu_f, p_p, \bgamma_f, \bu_s, \bvarphi, \lambda)\in \bS, \\[1ex]
\ds \ubtau := (\btau_f, \bv_p, \btau_e)\in \bQ,\quad 
\ubv := (\bv_f, w_p, \bchi_f, \bv_s, \bpsi, \xi)\in \bS,
\end{array}
\end{equation*}
where the spaces $\bQ$ and $\bS$ are respectively endowed with the norms
\begin{align*}
\|\ubtau\|^2_{\bQ} & = \|\btau_f\|^2_{\bbX_f} + \|\bv_p\|^2_{\bX_p} + \|\btau_e\|^2_{\bSigma_e}\,, \\[1ex]
\|\ubv\|^2_{\bS} & = \|\bv_f\|^2_{\bV_f} + \|w_p\|^2_{\W_p} + \|\bchi_f\|^2_{\bbQ_f} + \|\bv_s\|^2_{\bV_p} + \|\bpsi\|^2_{\bLambda_f} + \|\xi\|^2_{\Lambda_p}\,.
\end{align*}
Hence, we can express \eqref{eq:NS-Biot-formulation-2} in operator notation as a nonlinear perturbation of a degenerate evolution problem in mixed form:
%
\begin{align}
& \frac{\partial}{\partial\,t}\,\cE_1(\ubsi(t)) + \cA(\ubsi(t)) + \cB'(\ubu(t)) + \cK_{\bu_f(t)}(\ubu(t))  =  \bF(t) \qin \bQ', \nonumber\\ 
& \frac{\partial}{\partial\,t}\,\cE_2(\ubu(t)) - \cB(\ubsi(t)) + \cC(\ubu(t))+\cL_{\bvarphi(t)}(\ubu(t))  =  \bG(t) \qin \bS',\label{eq:alternative-formulation-operator-form}
\end{align}
where, given $\bu_f\in \bV_f$ and $\bvarphi \in \bLambda_f$, the operators
$\cE_1 : \bQ\to \bQ'$, $\cE_2 : \bS\to \bS'$, 
$\cA\,:\,\bQ\to \bQ'$, $\cB\,:\, \bQ\to \bS'$, $\cC\,:\,\bS\to \bS'$,  $\cK_{\bu_f}\,:\,\bS\to \bQ'$, and $\cL_{\bvarphi}\,:\,\bS\to \bS'$ are defined as follows:
\begin{subequations}\label{operators-1}
\begin{align}
& \cE_1(\ubsi)(\ubtau) \,:=\, a^s_p(\bsi_e,\btau_e)\,,\quad 
\cE_2(\ubu)(\ubv) \,:=\, \rho_f\,(\bu_f,\bv_f)_{\Omega_f} + s_0\,(p_p,w_p)_{\Omega_p}
+ \rho_p\,(\bu_s,\bv_s)_{\Omega_p}, \label{defn-E1-E2}  \\[1ex]  
&\ds \cA(\ubsi)(\ubtau) := a_f(\bsi_f,\btau_f) + a^d_p(\bu_p,\bv_p)\,, \label{defn-A} \\[1ex]
&\ds \cB(\ubtau)(\ubv) \,:=\, b_f(\btau_f,\bv_f) + b_p(\bv_p,w_p) + b_\sk(\btau_f,\bchi_f) + b_s(\btau_e,\bv_s) + b_{\bn_f}(\btau_f,\bpsi) + b_{\bn_p}(\bv_p,\xi)\,, \label{defn-B} \\[1ex]
&\ds \cC(\ubu)(\ubv) \,:=\, c_\BJS(\bu_s,\bvarphi;\bv_s,\bpsi) + c_{\Gamma}(\bu_s,\bvarphi;\xi) - c_{\Gamma}(\bv_s,\bpsi;\lambda) + \alpha_p\,b_p(p_p,\bv_s) - \alpha_p\,b_p(w_p,\bu_s)\,, \label{defn-C} \\[1ex]
&\ds \cK_{\bu_f}(\ubu)(\ubtau) := \kappa_{\bu_f}(\bu_f,\btau_f)\,,\quad
\cL_{\bvarphi}(\ubu)(\ubv) := l_{\bvarphi}(\bvarphi,\bpsi),\label{defn-K-L}
\end{align}
\end{subequations}
%
whereas the functionals $\bF\in \bQ'$ and $\bG\in \bS'$ are defined respectively as follows:
\begin{equation}\label{operators-3}
\bF(\ubtau) := 0 \qan
\bG(\ubv) := (\f_f,\bv_f)_{\Omega_f} + (q_p,w_p)_{\Omega_p} + (\f_p,\bv_s) \,.
\end{equation}

%**********************************************************************
%**********************************************************************

\subsection{Operator properties}

We next discuss boundness, continuity, monotonicity, and inf-sup stability properties of the operators defined in \eqref{operators-1} (cf. \eqref{eq:bilinear-forms}--\eqref{eq:forms-on-interface-1}). 
We first note that for given $\bw_f$, the term $\kappa_{\bw_f}(\bu_f,\btau_f)$ requires $\bu_f$ to live in a space smaller than $\bL^2(\Omega_f)$. In particular, using Cauchy--Schwarz inequality twice gives
%
\begin{equation}\label{kappa-cont}
|\kappa_{\bw_f}(\bu_f,\btau_f)| 
\,\le\, \frac{\rho_f\,n^{1/2}}{2\,\mu} \|\bw_f\|_{\bL^4(\Omega_f)}\|\bu_f\|_{\bL^4(\Omega_f)}\|\btau^\rd_f\|_{\bbL^2(\Omega_f)}.
\end{equation}
%
Accordingly, we look for both $\bu_f, \bw_f \in \bL^4(\Omega_f)$. Furthermore, applying H\"older's inequality, we obtain
%
\begin{equation}\label{bf-cont}
|b_f(\btau_f,\bv_f)| \leq \|\bdiv(\btau_f)\|_{\bL^{4/3}(\Omega_f)} \|\bv_f\|_{\bL^4(\Omega_f)}.
\end{equation}
%
For the interface nonlinear term $l_{\bzeta}(\bvarphi,\bpsi)$, applying H\"older's inequality and using the continuous injection $\bi_\Gamma$ of $\bH^{1/2}(\partial \Omega_f)$ into $\bL^3(\partial \Omega_f)$, we have
%
\begin{equation}\label{eq:injection-H1/2-into-L3-interface}
\big|l_{\bzeta}(\bvarphi,\bpsi)\big|\leq\ \|\bi_{\Gamma}\|^3 \|\bzeta\|_{\bH^{1/2}(\Gamma_{fp})} \|\bvarphi\|_{\bH^{1/2}(\Gamma_{fp})} \|\bpsi\|_{\bH^{1/2}(\Gamma_{fp})}.
\end{equation}
%
In addition, using the continuous trace operator
$\gamma_0: \bH^1(\Omega_p) \to \bL^2(\Gamma_{fp})$, there hold
%
\begin{align}
  &  |\pil\bpsi\cdot\bt_{f,j},\bv_s\cdot\bt_{f,j}\pir_{\Gamma_{fp}}| \leq \|\gamma_0\|\|\bv_s\|_{\bH^1(\Omega_p)} \|\bpsi\|_{\bL^2(\Gamma_{fp})}, \label{bjs-cont} \\
  & |\pil\bv_s\cdot\bn_p,\xi\pir_{\Gamma_{fp}}| \leq \|\gamma_0\|\|\bv_s\|_{\bH^1(\Omega_p)} \|\xi\|_{\L^2(\Gamma_{fp})}. \label{cgamma-cont-1} 
\end{align}
Now, we establish stability properties for  $\cE_1, \cE_2, \cA, \cB, \cC, \cK_{\bw_f}, \cL_{\bzeta}$, and $\bG(t)$. 
\begin{lem}\label{lem:cont}
The linear operators $\cE_1, \cE_2, \cA, \cB$ and $\cC$ are continuous. In particular, there exist positive constants $C_{\cE_1}$, $C_{\cE_2}$, $C_{\cA}$, $C_{\cB}$, and $C_f$ such that
 %
\begin{gather}
 \big|\cE_1(\ubsi)(\ubtau)\big| \,\leq\, C_{\cE_1}\,\|\ubsi\|_{\bQ} \|\ubtau\|_{\bQ},\quad
 \big|\cE_2(\ubu)(\ubv)\big| \,\leq\, C_{\cE_2}\,\|\ubu\|_{\bS}\,\|\ubv\|_{\bS},
 \label{eq:continuity-cE1-cE2}\\[1ex]
    \big|\cA(\ubsi)(\ubtau)\big| \,\leq\, C_{\cA}\,\|\ubsi\|_{\bQ}\|\ubtau\|_{\bQ},\quad
\big|\cB(\ubtau)(\ubv)\big| \,\leq\, C_{\cB}\,\|\ubtau\|_{\bQ}\,\|\ubv\|_{\bS},
\quad \big|\cC(\ubu)(\ubv)\big| 
\,\leq\, C_f\,\|\ubu\|_{\bS}\,\|\ubv\|_{\bS}. \label{eq:continuity-cA-cB-cC}
  \end{gather}
%
Furthermore, given $\bw_f \in \bL^4(\Omega_f)$ and $\bzeta \in \bH^{1/2}(\Gamma_{fp})$, the operators $\cK_{\bw_f}$ and $\cL_{\bzeta}$ are bounded:
%
\begin{gather}
\big|\cK_{\bw_f}(\ubu)(\ubtau)\big|
\,\leq\, \frac{\rho_f\,n^{1/2}}{2\,\mu}\,\|\bw_f\|_{\bV_f}\|\ubu\|_{\bS}\,\|\ubtau\|_{\bQ},
\label{eq:continuity-cK-wf}\\[1ex]
\big|\cL_{\bzeta}(\ubu)(\ubv)\big|\leq C_{\cL}\|\bzeta\|_{\bLambda_f}\|\ubu\|_{\bS}\|\ubv\|_{\bS},
 \label{eq:continuity-cL-zeta}   
\end{gather}
%
and the linear functional $\bG(t) \in \bS'$ is continuous:
%
\begin{equation}\label{cont-F-G}  
  \big|\bG(t)(\ubv)\big| \leq C_{\bG}(t) \|\ubv\|_{\bS}.
\end{equation}
%
\end{lem}

\begin{proof}
We recall that the operators and functionals are defined in \eqref{operators-1} and \eqref{operators-3} with the associated bilinear forms defined in \eqref{eq:bilinear-forms} and \eqref{eq:forms-on-interface-1}. The continuity of $\cE_1$ follows from \eqref{eq:A-bounds} with $C_{\cE_1}:= a_{\max}$  . For the continuity of $\cE_2$, applying the Cauchy--Schwarz and H\"older's inequalities, we obtain
\begin{equation*}%\label{eq:continuity-cE_2-1}
\big|\cE_2(\ubu)(\ubv)\big| 
\,\leq\, \rho_f |\Omega_f|^{1/2}\|\bu_f\|_{\bV_f}\|\bv_f\|_{\bV_f} 
+ s_0\|p_p\|_{\W_p}\|w_p\|_{\W_p} 
+ \rho_p \|\bu_s\|_{\bV_p}\|\bv_s\|_{\bV_p}
\,\leq\, C_{\cE_2}\,\|\ubu\|_\bS\,\|\ubv\|_\bS \,,
\end{equation*}
with $C_{\cE_2}:=\max\{\rho_f|\Omega_f|^{1/2},s_0,\rho_p\}$. The continuity of $\cA$ follows from \eqref{eq:permeability-bounds} with $C_{\cA}:=\max\{ \frac{1}{2\mu}, \mu\,k_{\max}\}$.
The continuity of $\cB$ follows from \eqref{eq:trace-inequality-1}, \eqref{eq:trace-inequality-2}, and \eqref{bf-cont}. The continuity of $\cC$ follows from \eqref{bjs-cont} and \eqref{cgamma-cont-1} with $C_f$ depending on $\mu$, $k_{\min}^{-1}$, $\alpha_{\BJS}$, and $\alpha_p$. Finally, the bounds for $\cK_{\bw_f}$ and $\cL_{\bzeta}$,  \eqref{eq:continuity-cK-wf} and \eqref{eq:continuity-cL-zeta}, follow from \eqref{kappa-cont} and \eqref{eq:injection-H1/2-into-L3-interface}, respectively, with $C_{\cL}:= \|\bi_{\Gamma}\|^3$, whereas the continuity of $\bG$ \eqref{cont-F-G} follows easily from its definition with $C_{\bG}(t) := \big(|\Omega_f|^{1/2}\|\f_f(t)\|^2_{\bL^2(\Omega_f)} + \|q_p(t)\|_{\L^2(\Omega_p)}^2 + \|\f_p(t)\|^2_{\bL^2(\Omega_p)}\big)^{1/2}$.
\end{proof}

Next, we establish the non-negativity of the forms $a_f$ and $a^d_p$, the positive semi-definiteness of $c_\BJS$, and the monotonicity of the operators $\cA, \cE_1, \cE_2$ and $\cC$, respectively.
\begin{lem}\label{lem:coercivity-properties-A-E2}
There hold
\begin{equation}\label{eq:coercivity-af}
a_f(\btau_f, \btau_f) 
\,\geq\, \frac{1}{2\,\mu}\,\|\btau^\rd_f\|^2_{\bbL^2(\Omega_f)} \quad \forall\,\btau_f\in \bbX_f \,,
\end{equation}
%
\begin{equation}\label{eq:coercivity-adp}
a^d_p(\bv_p, \bv_p) 
\,\geq\, \mu\,k_{\min}\,\|\bv_p\|^2_{\bL^2(\Omega_p)} \quad \forall\, \bv_p\in \bX_p \,,
\end{equation}
and there exist a constant $c_I > 0$, such that
\begin{equation}\label{eq:positivity-aBJS}
c_\BJS(\bv_s,\bpsi;\bv_s,\bpsi) 
\,\geq c_I\,\sum^{n-1}_{j=1} \|( \bpsi-\bv_s)\cdot\bt_{f,j}\|^2_{\L^2(\Gamma_{fp})} \quad \forall\, (\bv_s, \bpsi)\in \bV_p\times \bLambda_f \,.
\end{equation}
In addition, the operators $\cA, \cE_1, \cE_2$ and $\cC$ are monotone. 
\end{lem}
%
\begin{proof}
We begin by noting that the non-negativity properties \eqref{eq:coercivity-af} and \eqref{eq:coercivity-adp} follows straightforward from \eqref{eq:permeability-bounds}, and the definition of $a_f$ and $a^d_p$ (cf. \eqref{bilinear-form-1}, \eqref{bilinear-form-2}), respectively.
In turn, from the definition of $c_\BJS$ (cf. \eqref{interface-bilinear-form-1}), \eqref{eq:positivity-aBJS} follows with a positive constant $c_I:= \mu\,\alpha_{\BJS}/\sqrt{ k_{\max}}$.
On the other hand, combining \eqref{eq:coercivity-af} and \eqref{eq:coercivity-adp} we deduce that 
\begin{equation}\label{eq: operator A-monotone}
\cA(\ubtau)(\ubtau) 
\,\geq\, \frac{1}{2\mu}\,\|\btau^\rd_f\|^2_{\bbL^2(\Omega_f)} + \, \mu\,k_{\min}\,\|\bv_p\|^2_{\bL^2(\Omega_p)} \,,
\end{equation}
which implies the monotonicity of $\cA$. Finally, from the definition of the operators $\cE_1, \cE_2$ and $\cC$ (cf. \eqref{defn-E1-E2}, \eqref{defn-C}) we have that for all $\ubtau\in \bQ$ and for all $\ubv\in \bS$ there exist positive constants $C_A, C_{fps}>0$, such that
\begin{equation}\label{eq: operator E_1-monotone}
\cE_1(\ubtau)(\ubtau) 
\,\geq\, C_A\,\|\btau_e\|^2_{\bSigma_e} \,,
\end{equation}
%
\begin{equation}\label{eq: operator E_2-monotone}
\cE_2(\ubv)(\ubv) \,\geq\, C_{fps}\,\Big( \|\bv_f\|^2_{\bL^2(\Omega_f)} + \|w_p\|^2_{\W_p}+\|\bv_s\|^2_{\bL^2(\Omega_p)}  \Big)\,,
\end{equation}
with $C_A:= a_{\min}$ and $C_{fps}:=\min\{\rho_f,s_0,\rho_p\}$, and
\begin{equation}\label{eq: operator C-monotone}
\cC(\ubv)(\ubv) 
\,\geq c_I\,\sum^{n-1}_{j=1} \|( \bpsi-\bv_s)\cdot\bt_{f,j}\|^2_{\L^2(\Gamma_{fp})} \,,
\end{equation}
which implies the monotonicity of $\cE_1, \cE_2$ and $\cC$, completing the proof.
\end{proof}

We end this section by establishing the inf-sup conditions associated to the forms $b_\star$, with $\star\in \{f, p, s, \sk, \bn_f, \bn_p \}$.
\begin{lem}\label{lem:inf-sup-conditions}
There exist constants $\beta_1, \beta_2, \beta_3 > 0$ such that
\begin{equation}\label{eq:inf-sup-vs}
\sup_{\0\neq \btau_e\in \bSigma_e} \frac{b_s(\btau_e,\bv_s)}{\|\btau_e\|_{\bSigma_e}} 
\,\geq\, \beta_1\,\|\bv_s\|_{\bV_p} \quad \forall\,\bv_s\in \bV_p \,,
\end{equation}
%
\begin{equation}\label{eq:inf-sup-qp-xi}
\sup_{\0\neq \bv_p\in \bX_p} \frac{b_p(\bv_p,w_p) + b_{\bn_p}(\bv_p,\xi)}{\|\bv_p\|_{\bX_p}} 
\,\geq\, \beta_2\,\|(w_p,\xi)\|_{\W_p\times \Lambda_p} \quad \forall\,(w_p,\xi)\in \W_p\times \Lambda_p\,,
\end{equation}
and
\begin{equation}\label{eq:inf-sup-vf-chif}
\sup_{\0\neq \btau_f\in \bbX_f} \frac{B_f(\btau_f,(\bv_f,\bchi_f,\bpsi))}{\|\btau_f\|_{\bbX_f}} 
\,\geq\, \beta_3\,\|(\bv_f, \bchi_f, \bpsi)\|_{\bV_f\times \bbQ_f\times \bLambda_f} \quad \forall\,(\bv_f,\bchi_f,\bpsi)\in \bV_f\times \bbQ_f\times \bLambda_f\,,
\end{equation}
where 
$B_f(\btau_f,(\bv_f,\bchi_f,\bpsi)) \,:=\, b_f(\btau_f,\bv_f) + b_\sk(\btau_f,\bchi_f) + b_{\bn_f}(\btau_f,\bpsi)$.
%\begin{equation*}
%B_f(\btau_f,(\bv_f,\bchi_f,\bpsi)) \,:=\, b_f(\btau_f,\bv_f) + b_\sk(\btau_f,\bchi_f) + b_{\bn_f}(\btau_f,\bpsi).
%\end{equation*}
\end{lem}
%
\begin{proof}
For the proof of \eqref{eq:inf-sup-vs} we refer the reader to \cite[Lemma ~4.1, eq. (4.1)]{aeny2019}, whereas combining a slight adaptation of \cite[Lemma ~3.6, eq. (3.5)]{gos2011} and \cite[Lemma ~3.8, eq. (3.10)]{gos2011} we deduce \eqref{eq:inf-sup-qp-xi}.
Next, to prove \eqref{eq:inf-sup-vf-chif}, we need to establish individual inf-sup conditions for $b_f$, $b_\sk$, and $b_{\bn_f}$. For the inf-sup conditions of $b_f$ and $b_\sk$, we refer to \cite[Lemma 3.5, eqs. (3.57) and (3.61)]{gobs2021}. To derive the inf-sup condition for $b_{\bn_f}$, we define $\btau_1 := \be(\bz_1)$ in $\Omega_f$, where $\bz_1 \in \bH^1_{\Gamma^{\rD}_f}(\Omega_f)$ is the unique weak solution of the boundary value problem
\begin{equation*}
-\bdiv(\be(\bz_1)) = \0 \qin \Omega_f,\quad 
\bz_1 = \0 \qon \Gamma^{\rD}_f,\quad 
\be(\bz_1)\bn_f = \left\{\begin{array}{l}
\0 \qon \Gamma^{\rN}_f\,, \\ 
\bpsi \qon \Gamma_{fp}\,.
\end{array} \right.
\end{equation*}
It is clear that $\btau_1\in \bbL^2(\Omega_f)$, $\bdiv(\btau_1) = \0\in \bL^{4/3}(\Omega_f)$, $\btau_1\bn = \0$ on $\Gamma^{\rN}_f$ so then $\btau_1\in \bbX_f$, and using the fact that $\be(\bz_1):\bchi_f = 0$ in $\Omega_f$, yields $B_f(\btau_1,(\bv_f,\bchi_f,\bpsi)) = \|\bpsi\|_{\bLambda_f}^2$. 
Thus, proceeding as in \cite[Section~2.4.3.2]{Gatica}, there holds $\|\btau_1\|_{\bbX_f} \leq C\,\|\bpsi\|_{\bLambda_f}$, and we deduce that
\begin{equation*}%\label{eq:inf-sup-L4-L2sk-Hhalf-2}
\sup_{\0 \neq \btau_f\in \bbX_f} \frac{B_f(\btau_f,(\bv_f,\bchi_f,\bpsi))}{\|\btau_f\|_{\bbX_f}} 
\,\geq\, \frac{\|\bpsi\|_{\bLambda_f}^2}{C\,\|\bpsi\|_{\bLambda_f}} = \frac{1}{C}\,\|\bpsi\|_{\bLambda_f}.
\end{equation*}
Finally, combining the inf-sup conditions of $b_f, b_\sk$ and $b_{\bn_f}$ we conclude \eqref{eq:inf-sup-vf-chif}.
\end{proof}

%**********************************************************************
%**********************************************************************

\section{Well-posedness of the model}\label{sec:well-posedness-model}

In this section we establish the solvability of \eqref{eq:continuous-weak-formulation-1}. To that end, we start with the analysis of the alternative formulation \eqref{eq:NS-Biot-formulation-2} (cf. \eqref{eq:alternative-formulation-operator-form}).

\subsection{Preliminary results}

A key result that we use to establish the existence of a solution to \eqref{eq:alternative-formulation-operator-form} is the following theorem (see \cite[Theorem~IV.6.1(b)]{Showalter}) for details). In what follows, $Rg(\cA) $ denotes the range of $\cA$.
\begin{thm}\label{thm:auxiliary-theorem}
Let the linear, symmetric and monotone operator $\cN$ be given for the real vector space $E$ to its algebraic dual $E^*$, and let $E'_b$ be the Hilbert space which is the dual of $E$ with the seminorm
\begin{equation*}
|x|_b = \big(\cN\,(x)(x)\big)^{1/2} \quad x\in E.
\end{equation*}
Let $\cM\subset E\times E'_b$ be a relation with domain $\cD = \Big\{ x\in E \,:\, \cM(x) \neq \emptyset \Big\}$.

Assume $\cM$ is monotone and $Rg(\cN + \cM) = E'_b$.
Then, for each $u_0\in \cD$ and for each $\cF\in \W^{1,1}(0,T;E'_b)$, there is a solution $u$ of
\begin{equation}\label{eq: 3.1}
\frac{d}{dt}\big(\cN(u(t))\big) + \cM\big(u(t)\big) \ni \cF(t) \quad a.e. \, \ 0 < t < T\,,
\end{equation}
with
\begin{equation*}
\cN(u)\in \W^{1,\infty}(0,T;E'_b),\quad u(t)\in \cD,\quad \mbox{ for all }\, 0\leq t\leq T,\qan \cN(u(0)) = \cN(u_0)\,.
\end{equation*}
\end{thm}
Recalling the definition of the operators $\cE_1, \cE_2, \cA, \cB, \cC, \cK_{\bw_f}$ and $\cL_{\bzeta}$ (cf. \eqref{operators-1}), problem \eqref{eq:alternative-formulation-operator-form} can be written in the form of \eqref{eq: 3.1} with
%
\begin{equation}\label{eq:defn-E-N-M}
E:= \bQ\times \bS, \quad u:= 
\left(\begin{array}{c}
\ubsi \\ \ubu
\end{array}\right),\quad 
\cN:= \begin{pmatrix}
\cE_1 & \0 \\
\0 & \cE_2
\end{pmatrix},\quad 
\cM:= \begin{pmatrix}
\cA & \cB' + \cK_{\bu_f} \\
-\cB  & \cC + \cL_{\bvarphi}
\end{pmatrix}.
\end{equation}
%
In particular, we observe from \eqref{defn-E1-E2} that
%
\begin{equation*}%\label{defn-N}
\cN(u)(v) := (A(\bsi_e),\btau_{e})_{\Omega_p} + \rho_f\,(\bu_{f},\bv_{f})_{\Omega_f} + s_0\,(p_{p},w_{p})_{\Omega_p} + \rho_p\,(\bu_{s},\bv_{s})_{\Omega_p}\,,
\end{equation*}
%  
which gives the semi-norm:
\begin{equation*}
\|v\|_{\cN} := \big( \cN(v)(v) \big)^{1/2} 
= \big( (A(\btau_e),\btau_{e})_{\Omega_p} + \rho_f\,\|\bv_{f}\|^2_{\bL^2(\Omega_f)} + s_0\,\|w_{p}\|^2_{\L^2(\Omega_p)} + \rho_p\,\|\bv_{s}\|^2_{\bL^2(\Omega_p)} \big)^{1/2},
\end{equation*}
%
which is equivalent to $\|v\|_{\cN,2} := \big( \|\btau_{e}\|^2_{\bbL^2(\Omega_p)} + \|\bv_{f}\|^2_{\bL^2(\Omega_f)} + \|w_{p}\|^2_{\L^2(\Omega_p)} + \|\bv_{s}\|^2_{\bL^2(\Omega_p)} \big)^{1/2}$. Therefore we define the space $E'_b:= \bQ_2'\times \bS_2'$ where,
%
\begin{equation}\label{eq:defn-E'_b-D}
\bQ_2':=\{\0\}\times \{\0\} \times\bbL^2(\Omega_p) \quad \text{and}\quad \bS_2':=\bL^2(\Omega_f) \times \L^2(\Omega_p) \times  \{\0\}\times \bL^2(\Omega_p) \times \{\0\} \times \{0\}\,.
\end{equation}
%
In order to establish the range condition $Rg(\cN + \cM) = E'_b$,
we consider the following resolvent system: find $(\ubsi,\ubu)\in \bQ\times \bS$, such that
\begin{align}
(\cE_1 + \cA)(\ubsi) + (\cB' + \cK_{\bu_f})(\ubu) &= \wh{\bF} \qin \bQ', \nonumber\\ 
-\,\cB(\ubsi) + (\cE_2 + \cC +\cL_{\bvarphi})(\ubu) &= \wh{\bG} \qin \bS',\label{eq:T-auxiliary-problem-operator-A}
\end{align}
%
for all $(\ubtau,\ubv)\in \bQ\times \bS$ where $\wh{\bF} \in \bQ_2'$ and $\wh{\bG} \in \bS_2'$ are such that
\begin{align*}
\wh{\bF}(\ubtau) &\,:=\, (\wh{\f}_e,\btau_e)_{\Omega_p} \quad \forall\,\ubtau \in \bQ\,, \nonumber\\ 
\wh{\bG}(\ubv) &\,:=\, (\wh{\f}_f,\bv_f)_{\Omega_f} + (\wh{q}_p,w_p)_{\Omega_p} + (\wh{\f}_p,\bv_s)_{\Omega_p} \quad \forall\,\ubv \in \bS\,. \nonumber
\end{align*}
Using the definition of operators $\cE_1, \cE_2, \cA, \cB, \cC, \cK_{\bu_f}$ and $\cL_{\bvarphi}$, \eqref{eq:T-auxiliary-problem-operator-A} can be rewritten as: for some functionals $\wh{\f}_e \in \bL^2(\Omega_p), \wh{\f}_f \in \bL^2(\Omega_f), \wh{q}_p \in \L^2(\Omega_p), \wh{\f}_p \in \bL^2(\Omega_p)$, find $(\ubsi,\ubu)\in \bQ\times \bS$, such that for all $(\ubtau,\ubv)\in \bQ\times \bS$,
%
\begin{align}
&  a_f(\bsi_f,\btau_f) + \kappa_{\bu_f}(\bu_f, \btau_f) + a^d_p(\bu_p,\bv_p) + a^s_p(\bsi_e,\btau_e) +\,\, b_f(\btau_f,\bu_f)+ b_p(\bv_p,p_p) + b_\sk(\btau_f,\bgamma_f)  \nonumber \\[0.5ex] 
&\quad +\, b_s(\btau_e,\bu_s) + b_{\bn_f}(\btau_f,\bvarphi) + b_{\bn_p}(\bv_p,\lambda)  \,=\, (\wh{\f}_e,\btau_e)_{\Omega_p} \,,  \nonumber \\[0.5ex]  
& \rho_f\,(\bu_f,\bv_f)_{\Omega_f} + s_0\,(p_p,w_p)_{\Omega_p}
+ \rho_p\,(\bu_s,\bv_s)_{\Omega_p}+\,\, c_{\BJS}(\bu_s,\bvarphi;\bv_s,\bpsi) + c_{\Gamma}(\bu_s,\bvarphi;\xi)- c_{\Gamma}(\bv_s,\bpsi;\lambda)    \nonumber \\[0.5ex]  
&\quad +\, \alpha_p\,b_p(\bv_s,p_p) - \alpha_p\,b_p(\bu_s,w_p) -\,\, b_f(\bsi_f,\bv_f) - b_p(\bu_p,w_p) - b_\sk(\bsi_f,\bchi_f) - b_s(\bsi_e,\bv_s)  \nonumber \\[0.5ex]  
&\quad -\, b_{\bn_f}(\bsi_f,\bpsi) - b_{\bn_p}(\bu_p,\xi)+l_{\bvarphi}(\bvarphi,\bpsi) =\, (\wh{\f}_f,\bv_f)_{\Omega_f} + (\wh{q}_p,w_p)_{\Omega_p} + (\wh{\f}_p,\bv_s)_{\Omega_p} \,.
\label{eq:T-auxiliary-problem-operator-B}
\end{align}

Using Theorem~\ref{thm:auxiliary-theorem}, we can show that the problem \eqref{eq:alternative-formulation-operator-form} (cf. \eqref{eq:NS-Biot-formulation-2}) is well-posed.
To that end, we proceed in the following manner.

\medskip

\noindent{\bf Step 1.} Introduce a fixed-point operator $\bT$ associated to problem \eqref{eq:T-auxiliary-problem-operator-A}.

\noindent{\bf Step 2.} Prove $\bT$ is a contraction mapping and  \eqref{eq:T-auxiliary-problem-operator-A} has unique solution.

\noindent{\bf Step 3.} Show that the alternative formulation \eqref{eq:alternative-formulation-operator-form} is well posed.

\medskip

\noindent Each step will be covered in detail in the corresponding subsection.

%**********************************************************************
%**********************************************************************

\subsubsection{Step 1: A fixed-point approach}\label{sec:fixed-point-approach}
We begin the well-posedness of \eqref{eq:T-auxiliary-problem-operator-A} by defining the operator $\bT : \bV_f \times \bLambda_f\to \bV_f\times \bLambda_f$ by
\begin{equation}\label{eq:definition-operator-T}
\bT(\bw_f,\bzeta) \,:=\, (\bu_f,\bvarphi) \quad \forall\,(\bw_f,\bzeta)\in \bV_f\times \bLambda_f,
\end{equation}
where $\ubu := (\bu_f, p_p, \bgamma_f, \bu_s, \bvarphi, \lambda)\in \bQ$ is the second component of the unique solution (to be confirmed below) of the problem:
Find $(\ubsi,\ubu)\in \bQ\times \bS$, such that
\begin{align}
(\cE_1 + \cA)(\ubsi) + (\cB' + \cK_{\bw_f})(\ubu) &= \wh{\bF} \qin \bQ'\,, \nonumber\\ 
-\,\cB(\ubsi) + (\cE_2 + \cC +\cL_{\bzeta})(\ubu) &= \wh{\bG} \qin \bS'\,. \label{eq:T-auxiliary-problem-operator}
\end{align}
Hence, it is not difficult to see that $(\ubsi,\ubu)\in \bQ\times \bS$ is a solution of \eqref{eq:T-auxiliary-problem-operator-A} if and only if $(\bu_f,\bvarphi)\in \bV_f\times \bLambda_f$ is a fixed-point of $\bT$, that is
\begin{equation*}%\label{eq:fixed-point-uf}
\bT(\bu_f,\bvarphi) \,=\, (\bu_f,\bvarphi).
\end{equation*}
In what follows we focus on proving that $\bT$ possesses a unique fixed-point.
However, we remark in advance that the definition of $\bT$ will make sense only in a closed ball of $\bV_f\times \bLambda_f$.

Before continuing with the solvability analysis of \eqref{eq:T-auxiliary-problem-operator}, we provide the hypotheses under which $\bT$ is well defined.
To that end, we introduce operators that will be used to regularize the problem \eqref{eq:T-auxiliary-problem-operator}.
Let $R_f\,:\,\bbX_f\to \bbX'_f$ and $R_p\,:\,\bX_p\to \bX'_p$ be defined by
\begin{subequations}\label{eq:Rf-Rp-definition}
\begin{align}
& \ds R_f(\bsi_f)(\btau_f) \,:=\, r_f(\bsi_f, \btau_f) 
\,=\, (|\bdiv(\bsi_f)|^{-2/3}\,\bdiv(\bsi_f),\bdiv(\btau_f))_{\Omega_f}, \label{eq: R_f defn} \\[1ex]
& \ds R_p(\bu_p)(\bv_p) \,:=\, r_p(\bu_p, \bv_p)
\,=\, (\div(\bu_p),\div(\bv_p))_{\Omega_p}. \label{eq:R_p defn} 
\end{align}
\end{subequations}

\begin{lem}\label{lem:Rf-Rp-properties}
The operators $R_f$ and $R_p$ (cf. \eqref{eq: R_f defn}, \eqref{eq:R_p defn}) are bounded, continuous, coercive and monotone, that is, they satisfy the following bounds:
\begin{subequations}\label{eq:R_f R_p bounded-coerc-cont-monotone}
\begin{align}
& \ds R_f(\bsi_f)(\btau_f) \,\leq\, \|\bdiv(\bsi_f)\|^{1/3}_{\bL^{4/3}(\Omega_f)} \|\bdiv(\btau_f)\|_{\bL^{4/3}(\Omega_f)}\,, 
\quad  R_f(\btau_f)(\btau_f) \,\geq\, \|\bdiv(\btau_f)\|^{4/3}_{\bL^{4/3}(\Omega_f)}\,, \label{eq: R_f bounded-coerc} \\[1ex] 
& \ds \|\cR_f(\btau_1) - \cR_f(\btau_2)\|_{\bbX'_f} 
\,\leq\, C\,\|\btau_1 - \btau_2\|^{1/3}_{\bbX_f}\,, \label{eq:R_f cont} \\[1ex]
& \ds (R_f(\btau_1) - R_f(\btau_2))(\btau_1 - \btau_2)
\,\geq\, C\,\frac{\|\bdiv(\btau_1) - \bdiv(\btau_2)\|^2_{\bL^{4/3}(\Omega_f)}}{\|\btau_1\|^{2/3}_{\bbX_f} + \|\btau_2\|^{2/3}_{\bbX_f}}\,, \label{eq:R_f monotone} \\[1ex]
& \ds R_p(\bu_p)(\bv_p) \,\leq\, \|\div(\bu_p)\|_{\bL^2(\Omega_p)} \|\div(\bv_p)\|_{\bL^2(\Omega_p)}\,, 
\quad  R_p(\bv_p)(\bv_p) \,\geq\, \|\div(\bv_p)\|^2_{\bL^2(\Omega_p)}\,, \label{eq: R_p bounded-coerc} \\[1ex]
& \ds \|R_p(\bu_1) - R_p(\bu_2)\|_{\bX'_p} 
\,\leq\, C\,\|\bu_1 - \bu_2\|_{\bX_p}\,, \label{eq:R_p cont} \\[1ex]
& \ds (R_p(\bu_1) - R_p(\bu_2))(\bu_1 - \bu_2)
\,\geq\, \|\bdiv(\bu_1) - \bdiv(\bu_2)\|^2_{\bL^{2}(\Omega_p)}\,. \label{eq:R_p monotone}
\end{align}
\end{subequations}
\end{lem}
%
\begin{proof}
The proof of \eqref{eq: R_f bounded-coerc}--\eqref{eq:R_f monotone} follow from similar arguments to \cite[Lemma 3.4]{cy2021} for $q=4/3$ and $p=4$. 
Inequalities \eqref{eq: R_p bounded-coerc}--\eqref{eq:R_p monotone} follow immediately from the definition \eqref{eq:R_p defn}.
\end{proof}

\medskip
According to the above, we define the operator $\cR : \bQ \to \bQ'$ as
\begin{equation*}
\cR(\ubsi)(\ubtau) \,:=\, R_f(\bsi_f)(\btau_f) + R_p(\bu_p)(\bv_p)\,,
\end{equation*}
and notice from \eqref{eq: R_f bounded-coerc} and \eqref{eq: R_p bounded-coerc} that
\begin{subequations}\label{eq: R-continuous-coercive}
\begin{align}
& \ds \cR(\ubsi)(\ubtau) 
\,\leq\, (\,\|\bdiv(\bsi_{f})\|^{1/3}_{\bL^{4/3}(\Omega_f)}+\, \|\div(\bu_{p})\|_{\bL^2(\Omega_p)})\|\ubtau\|_{\bQ}\,, \label{eq: R-continuous} \\[1ex]
& \ds \cR(\ubtau)(\ubtau) 
\,\geq\, \|\bdiv(\btau_{f})\|^{4/3}_{\bL^{4/3}(\Omega_f)}+\, \|\div(\bv_{p})\|^2_{\bL^2(\Omega_p)}\,. \label{eq: R-coercive}
\end{align}
\end{subequations}
On the other hand, given $q \in (1,2]$, we recall that there exist constants $c_1(\Omega_f), c_2(\Omega_f) > 0$, such that
\begin{equation}\label{eq:tau-d-H0div-inequality}
c_1(\Omega_f)\,\|\btau_{f,0}\|^q_{\bbL^2(\Omega_f)} \,\leq\, \|\btau^\rd_f\|^q_{\bbL^2(\Omega_f)} + \|\bdiv(\btau_f)\|^q_{\bL^{4/3}(\Omega_f)} \quad \forall\,\btau_f = \btau_{f,0} + c\,\bI\in \bbH(\bdiv_{4/3};\Omega_f)
\end{equation}
%
\begin{equation}\label{eq:tau-H0div-Xf-inequality}
\mbox{and}\quad c_2(\Omega_f)\,\|\btau_f\|^q_{\bbX_f} \,\leq\, \|\btau_{f,0}\|^q_{\bbX_f} \quad \forall\,\btau_f = \btau_{f,0} + c\,\bI\in \bbX_f,
\end{equation}
where $\btau_{f,0}\in \bbH_0(\bdiv_{4/3};\Omega_f) := \Big\{ \btau_f\in \bbH(\bdiv_{4/3};\Omega_f) :\quad (\tr(\btau_f),1)_{\Omega_f} = 0 \Big\}$ and $c\in \R$.
For the proof of \eqref{eq:tau-d-H0div-inequality} we refer to \cite[Lemma 3.2]{cgo2021}, whereas \eqref{eq:tau-H0div-Xf-inequality} follows after a slight adaptation of \cite[Lemma 2.5]{Gatica} and the fact that $\btau_f\bn_f\in \bH^{-1/2}(\partial\Omega_f)$ for all $\btau_f\in \bbH(\bdiv_{4/3};\Omega_f)$ (cf. \cite[eq. (2.5)]{cmo2018}). We emphasize that the proof of \eqref{eq:tau-H0div-Xf-inequality} utilizes the boundary condition $\btau_f\bn_f = \0$ on $\Gamma^\rN_f$, with $|\Gamma^\rN_f|>0$ for $\btau_f \in \bbX_f.$

Thus, for $\epsilon > 0$, we consider a regularization of \eqref{eq:T-auxiliary-problem-operator}  defined by: Find $(\ubsi_\epsilon,\ubu_\epsilon)\in \bQ\times \bS$ satisfying
%
\begin{align}
(\cE_1 + \cA + \epsilon\,\cR)(\ubsi_\epsilon) + (\cB' + \cK_{\bw_f})(\ubu_\epsilon) &= \wh{\bF} \qin \bQ'\,, \nonumber\\ 
-\,\cB(\ubsi_\epsilon) + (\cE_2 + \cC+\cL_{\bzeta})(\ubu_\epsilon) &= \wh{\bG} \qin \bS'\,.\label{eq:regularization-T-auxiliary-problem-operator}
\end{align}
We will prove that \eqref{eq:regularization-T-auxiliary-problem-operator} is well-posed.
To than end, we first state the following preliminary lemma.
%
\begin{lem}\label{lem:auxiliary-problem-first-fow}
Given $\bw_f\in \bV_f$ and $\ubv\in \bS$, for each $\epsilon>0$, there exists a unique $\ubsi_\epsilon(\ubv)\in \bQ$, such that
\begin{equation}\label{eq:auxiliary-problem-first-row}
(\cE_1 + \cA + \epsilon\,\cR)(\ubsi_\epsilon(\ubv))(\ubtau) \,=\, \left(\wh{\bF} - (\cB' + \cK_{\bw_f})(\ubv)\right)(\ubtau) \quad \forall\,\ubtau\in \bQ.
\end{equation}
Moreover, there exists a constant $C>0$, such that
\begin{equation}\label{eq:auxiliary-bound-1}
\|\ubsi_\epsilon(\ubv)\|_\bQ \,\leq\, C \left(\|\wh{\bF}\|_{\bQ'} + (1 + \|\bw_f\|_{\bV_f})\,\|\ubv\|_\bS\right)^3.
\end{equation}
Let $\beta:= \min\left\{ \beta_1, \beta_2, \beta_3\right\}$, where $\beta_i$ are the inf-sup constants in \eqref{eq:inf-sup-vs}--\eqref{eq:inf-sup-vf-chif}. In addition, let
\begin{equation}\label{eq:r_1^0 defn}
r_1^0:= \frac{\mu\,\beta}{3\,\rho_f\,n^{1/2}}.
\end{equation}
Then for each $\bw_f\in \bV_f$ satisfying $\|\bw_f\|_{\bV_f} \leq r_1^0$, there holds
\begin{equation}\label{eq:auxiliary-bound-2}
\|\ubv\|_\bS \,\leq\, C\,\Big( \|\wh{\bF}\|_{\bQ'} + \|\ubsi_\epsilon(\ubv)\|_\bQ + \,\|\bdiv(\bsi_{f,\epsilon}(\ubv))\|^{1/3}_{\bL^{4/3}(\Omega_f)}+\, \|\div(\bu_{p,\epsilon}(\ubv))\|_{\bL^2(\Omega_p)}\Big),
\end{equation}
and, given $\ubv_1, \ubv_2\in \bS$ for which $\ubsi_\epsilon(\ubv_1)$ and $\ubsi_\epsilon(\ubv_2)$ satisfy \eqref{eq:auxiliary-problem-first-row}, there hold
\begin{align}
&\|\ubv_1 - \ubv_2\|_\bS \,\leq\, C\,\Big( \|\ubsi_\epsilon(\ubv_1) - \ubsi_\epsilon(\ubv_2)\|_\bQ + \,\|\bdiv(\bsi_{f,\epsilon}(\ubv_1) - \bsi_{f,\epsilon}(\ubv_2))\|^{1/3}_{\bL^{4/3}(\Omega_f)} \nonumber\\
& \qquad \qquad \qquad\qquad+\, \|\div(\bu_{p,\epsilon}(\ubv_1)-\bu_{p,\epsilon}(\ubv_2))\|_{\bL^2(\Omega_p)}\Big),\label{eq:auxiliary-bound-3}
\end{align}
%and
\begin{equation}\label{eq:auxiliary-bound-4}
\mbox{and}\quad \|\ubsi_\epsilon(\ubv_1) - \ubsi_\epsilon(\ubv_2)\|_\bQ 
\,\leq\, C_{\ubsi_{\epsilon}}\,(1 + \|\bw_f\|_{\bV_f})\,\|\ubv_1 - \ubv_2\|_\bS\,,
\end{equation}
with $C_{\ubsi_{\epsilon}}$ depending on $\bsi_{f,\epsilon}(\ubv_1)$ and $\bsi_{f,\epsilon}(\ubv_2)$, which are fixed.
\end{lem}
%
\begin{proof}
We begin by noting that given $\bw_f\in \bV_f$ and $\ubv\in \bS$ the functional $\wh{\bF} - (\cB' + \cK_{\bw_f})(\ubv)$ is bounded and continuous (cf. \eqref{eq:continuity-cA-cB-cC}, \eqref{eq:continuity-cK-wf}). In addition, combining \eqref{eq:continuity-cE1-cE2}, \eqref{eq:continuity-cA-cB-cC}, and Lemmas \ref{lem:coercivity-properties-A-E2} and \ref{lem:Rf-Rp-properties}, we deduce that the operator $\cE_1 + \cA + \epsilon\,\cR$ is bounded, continuous and monotone. 
Moreover, given $\ubtau = (\btau_f, \bv_p, \btau_e)\in \bQ$, employing the non-negativity bounds of $\cA, \cE_1$ and $ \cR$ (cf. \eqref{eq: operator A-monotone}, \eqref{eq: operator E_1-monotone}, \eqref{eq: R-coercive}), it follows that
\begin{equation}\label{eq:E1-A-epsR coercive}
(\cE_1 + \cA + \epsilon\,\cR)(\ubtau)(\ubtau) \geq \, C_1\,\Big( \|\btau^\rd_f\|^2_{\bbL^2(\Omega_f)} + \|\bdiv(\btau_f)\|^{4/3}_{\bL^{4/3}(\Omega_f)} \Big)
\,+\, C_2\,\Big(\|\bv_p\|^2_{\bX_p} + \|\btau_e\|^2_{\bSigma_e}\Big) \,,
\end{equation}
with $C_1:= \min\big\{ 1/(2\,\mu), \epsilon \big\}$ and $C_2:=\min\big\{\mu\,k_{\min},\epsilon, C_A \big\}$.
Notice that for the terms $\|\btau^\rd_f\|_{\bbL^2(\Omega_f)}$ and $\|\bdiv(\btau_f)\|_{\bL^{4/3}(\Omega_f)}$, there are only two possibilities: either $\|\btau^\rd_f\|_{\bbL^2(\Omega_f)}$ is less than or greater than $\|\bdiv(\btau_f)\|_{\bL^{4/3}(\Omega_f)}$. Using this fact in conjunction with some algebraic computations and the inequalities \eqref{eq:tau-d-H0div-inequality}--\eqref{eq:tau-H0div-Xf-inequality} with exponent $q=4/3$, we can deduce that
\begin{equation}\label{eq:coercivity-E1-A-R}
(\cE_1 + \cA + \epsilon\,\cR)(\ubtau)(\ubtau)
\geq C_3\big( \|\btau_{f}\|^{4/3}_{\bbX_f} 
+ \|\bv_p\|^2_{\bX_p} 
+ \|\btau_e\|^2_{\bSigma_e} \big)\,,
\end{equation}
implying the coercivity of $\cE_1 + \cA + \epsilon\,\cR$, with $C_3=C_\rd\min\{C_1,C_2\}\min\big\{ \|\btau_{f}\|^{2/3}_{\bbX_f},1 \big\}$ and $C_\rd$ depending on $c_1(\Omega_f), c_2(\Omega_f)$. 
Note that although $C_3$ depends on $\|\btau_{f}\|_{\bbX_f}$, it is always bounded since it is a minimum.
Then, as a direct application of the Browder--Minty theorem \cite[Theorem 10.49]{Renardy-Rogers}, we obtain that problem \eqref{eq:auxiliary-problem-first-row} has a unique solution.

We now proceed with deriving \eqref{eq:auxiliary-bound-1}, \eqref{eq:auxiliary-bound-2}, and \eqref{eq:auxiliary-bound-3}.
Taking $\ubtau=\ubsi_\epsilon(\ubv)$ in \eqref{eq:auxiliary-problem-first-row}, proceeding as in \eqref{eq:coercivity-E1-A-R}, considering that: either $\|\bsi_{f,\epsilon}(\ubv)\|_{\bbX_f}$ is less than or greater than $\|\bu_p(\ubv)\|_{\bX_p} 
+ \|\bsi_e(\ubv)\|_{\bSigma_e}$, and using \eqref{eq:continuity-cA-cB-cC} and \eqref{eq:continuity-cK-wf}, we get
\begin{equation*}
\wh{C}_3\,\|\bsi_{\epsilon}(\ubv)\|^{4/3}_{\bQ} 
\leq \left(\|\wh{\bF}\|_{\bQ} + \left(C_{\cB} + \frac{\rho_f\,n^{1/2}}{2\,\mu}\,\|\bw_f\|_{\bV_f}\right)\|\ubv\|_{\bS}\right) \|\bsi_{\epsilon}(\ubv)\|_{\bQ}\,,
\end{equation*}
with $\wh{C}_3=C_3\min\big\{ \big(\|\bu_p(\ubv)\|_{\bX_p} 
+ \|\bsi_e(\ubv)\|_{\bSigma_e}\big)^{2/3},1 \big\}/2$,
which implies \eqref{eq:auxiliary-bound-1}.
In turn, we have from the inf-sup condition of $\cB$ (cf. Lemma \ref{lem:inf-sup-conditions}) and the continuity of $\cK_{\bw_f}$ (cf. \eqref{eq:continuity-cK-wf}), that
\begin{equation*}%\label{eq:inf-sup-B}
\sup_{\underline{\0} \neq\ubtau\in \bQ} \frac{(\cB' + \cK_{\bw_f})(\ubv)(\ubtau)}{\|\ubtau\|_\bQ} 
\,\geq\, \Big( \beta - \frac{\rho_f\,n^{1/2}}{2\,\mu}\,\|\bw_f\|_{\bV_f} \Big)\,\|\ubv\|_\bS,
\end{equation*}
where $\beta := \min\big\{\beta_1, \beta_2, \beta_3\big\}$.
Consequently, by requiring that $\|\bw_f\|_{\bV_f} \leq r_1^0$,
and using \eqref{eq:auxiliary-problem-first-row} and the continuity of $\wh{\bF}, \cE_1, \cA$ and $\cR$ (cf. \eqref{eq:continuity-cE1-cE2}, \eqref{eq:continuity-cA-cB-cC}, \eqref{eq: R-continuous}), we obtain
\begin{align}
& \frac{5\beta}{6}\,\|\ubv\|_\bS 
\,\leq\, \sup_{\underline{\0} \neq\ubtau\in \bQ} \frac{\wh{\bF}(\ubtau) - (\cE_1 + \cA + \epsilon\,\cR)(\ubsi_\epsilon(\ubv))(\ubtau)}{\|\ubtau\|_\bQ} \nonumber\\ 
&\quad \leq \, \|\wh{\bF}\|_{\bQ'} + (C_{\cE_1} + C_{\cA})\,\|\ubsi_\epsilon(\ubv)\|_\bQ + \epsilon\,\|\bdiv(\bsi_{f,\epsilon}(\ubv))\|^{1/3}_{\bL^{4/3}(\Omega_f)}+\epsilon\, \|\div(\bu_{p,\epsilon}(\ubv))\|_{\bL^2(\Omega_p)},\label{eq:inf-sup-B-1}
\end{align}
concluding \eqref{eq:auxiliary-bound-2}.
Next, given $\ubv_1, \ubv_2\in \bS$ for which $\ubsi_\epsilon(\ubv_1)$ and $\ubsi_\epsilon(\ubv_2)$ satisfy \eqref{eq:auxiliary-problem-first-row}, we deduce that
for all $\ubtau \in \bQ$ there holds
\begin{equation}\label{eq:identity-E1-A-R-2}
(\cB' + \cK_{\bw_f})(\ubv_2 - \ubv_1)(\ubtau) \,=\, (\cE_1 + \cA)(\ubsi_\epsilon(\ubv_1) - \ubsi_\epsilon(\ubv_2))(\ubtau) + \epsilon\,\big(\cR(\ubsi_\epsilon(\ubv_1)) - \cR(\ubsi_\epsilon(\ubv_2))\big)(\ubtau)\,.
\end{equation}
Proceeding analogously to \eqref{eq:auxiliary-bound-2} we derive \eqref{eq:auxiliary-bound-3}.
We notice that if we assume that $\bsi_\epsilon(\ubv_1) = \bsi_\epsilon(\ubv_2)$, \eqref{eq:auxiliary-bound-3} implies that $\ubv_1 = \ubv_2$.
Equivalently, this shows that given $\ubv_1, \ubv_2\in \bS$ with $\ubv_1 \neq \ubv_2$ the solutions $\bsi_\epsilon(\ubv_1)$ and $\bsi_\epsilon(\ubv_2)$ of \eqref{eq:auxiliary-problem-first-row} are in fact different. 

On the other hand, given $\ubv_i\in \bS$ for which $\ubsi_\epsilon(\ubv_i) := (\bsi_{f,\epsilon}(\ubv_i),\bu_{p,\epsilon}(\ubv_i),\bsi_{e,\epsilon}(\ubv_i))$, with $i\in \{1,2\}$, satisfying \eqref{eq:auxiliary-problem-first-row}, we subtract the problems and then test with $\ubtau = (\0,\bu_{p,\epsilon}(\ubv_1) - \bu_{p,\epsilon}(\ubv_2), \bsi_{e,\epsilon}(\ubv_1) - \bsi_{e,\epsilon}(\ubv_2))$, 
%we obtain
%\begin{align*}
%&(\cE_1 + \cA + \epsilon\,\cR)(\bsi_{f,\epsilon}(\ubv_1)-\bsi_{f,\epsilon}(\ubv_2),\bu_{p,\epsilon}(\ubv_1)-\bu_{p,\epsilon}(\ubv_2),\bsi_{e,\epsilon}(\ubv_1)-\bsi_{e,\epsilon}(\ubv_2)) \nonumber \\
%&(\0,\bu_{p,\epsilon}(\ubv_1) - \bu_{p,\epsilon}(\ubv_2),\bsi_{e,\epsilon}(\ubv_1) - \bsi_{e,\epsilon}(\ubv_2)) \nonumber \\
%&\,=\, (\cB' + \cK_{\bw_f})(\ubv_2-\ubv_1)(\0,\bu_{p,\epsilon}(\ubv_1) - \bu_{p,\epsilon}(\ubv_2),\bsi_{e,\epsilon}(\ubv_1) - \bsi_{e,\epsilon}(\ubv_2))\,, %\label{eq:coerc-E1-A-R/cont-B}
%\end{align*} 
and use non-negativity bounds of $a^d_p, \cE_1$ and $\cR_p$ (cf. \eqref{eq:coercivity-adp}, \eqref{eq: operator E_1-monotone}, \eqref{eq: R_p bounded-coerc}), the continuity of $\cB$ (cf. \eqref{eq:continuity-cA-cB-cC}), and the Cauchy--Schwarz inequality, to obtain
\begin{align*}
&C_2\,\Big(\|\bu_{p,\epsilon}(\ubv_1) - \bu_{p,\epsilon}(\ubv_2)\|^2_{\bX_p}
+ \|\bsi_{e,\epsilon}(\ubv_1) - \bsi_{e,\epsilon}(\ubv_2)\|^2_{\bSigma_e} \Big) \nonumber \\
%& \leq \cB'(\ubv_2-\ubv_1)(\0,\bu_{p,\epsilon}(\ubv_1) - \bu_{p,\epsilon}(\ubv_2),\bsi_{e,\epsilon}(\ubv_1) - \bsi_{e,\epsilon}(\ubv_2))\nonumber \\
&\quad \leq\, C_{\cB}\,\|\big(\0,\bu_{p,\epsilon}(\ubv_1) - \bu_{p,\epsilon}(\ubv_2),\bsi_{e,\epsilon}(\ubv_1) - \bsi_{e,\epsilon}(\ubv_2)\big)\|_{\bQ}\|\ubv_1 - \ubv_2\|_{\bS} \,, %\label{eq:continuity-biot-solution}
\end{align*} 
%
with $C_2$ as in \eqref{eq:E1-A-epsR coercive}, which implies
\begin{equation}\label{eq:bound-monotonicity-up-bsie}
\|\big(\0,\bu_{p,\epsilon}(\ubv_1) - \bu_{p,\epsilon}(\ubv_2),\bsi_{e,\epsilon}(\ubv_1) - \bsi_{e,\epsilon}(\ubv_2)\big)\|_{\bQ} \,\leq\, C\,\|\ubv_1 - \ubv_2\|_{\bS} \,.
\end{equation}
In turn, testing \eqref{eq:auxiliary-problem-first-row} now with $\ubtau = (\bsi_{f,\epsilon}(\ubv_1) - \bsi_{f,\epsilon}(\ubv_2), \0,\0)$, employing the monotonicity of $a_f$ (cf. \eqref{eq:coercivity-af}) and $\cR_f$ (cf. \eqref{eq:R_f monotone}), and the continuity of $\cB, \cK_{\bw_f}$ (cf. \eqref{eq:continuity-cA-cB-cC}, \eqref{eq:continuity-cK-wf}), we deduce that
\begin{align}
&\frac{1}{2\,\mu}\,\|\bsi^\rd_{f,\epsilon}(\ubv_1) - \bsi^\rd_{f,\epsilon}(\ubv_2)\|^2_{\bbL^2(\Omega_f)} + \epsilon\,C\,\frac{\|\bdiv(\bsi_{f,\epsilon}(\ubv_1)) - \bdiv(\bsi_{f,\epsilon}(\ubv_2))\|^2_{\bL^{4/3}(\Omega_f)}}{\|\bsi_{f,\epsilon}(\ubv_1)\|^{2/3}_{\bbX_f} + \|\bsi_{f,\epsilon}(\ubv_2)\|^{2/3}_{\bbX_f}} \nonumber\\ 
&\quad \, \leq \, \left(C_{\cB} + \frac{\rho_f\,n^{1/2}}{2\,\mu}\,\|\bw_f\|_{\bV_f}\right)\|\ubv_1 - \ubv_2\|_\bS\,\|\bsi_{f,\epsilon}(\ubv_1) - \bsi_{f,\epsilon}(\ubv_2)\|_{\bbX_f} \,,
\label{eq:bound-continuity-sigf}
\end{align} 
and, using \eqref{eq:tau-d-H0div-inequality}--\eqref{eq:tau-H0div-Xf-inequality}, with exponent $q=2$, to bound the left hand side of \eqref{eq:bound-continuity-sigf}, we derive
\begin{equation}\label{eq:bound-sigf-2}
\|\bsi_{f,\epsilon}(\ubv_1) - \bsi_{f,\epsilon}(\ubv_2)\|_{\bbX_f} 
\,\leq\, C\,(1 + \|\bw_f\|_{\bV_f})\,\|\ubv_1 - \ubv_2\|_\bS \,,
\end{equation}
with $C$ depending on $\epsilon$, $\bsi_{f,\epsilon}(\ubv_1)$ and $\bsi_{f,\epsilon}(\ubv_2)$, but remaining fixed.
Finally, combining \eqref{eq:bound-monotonicity-up-bsie} and \eqref{eq:bound-sigf-2}, we obtain \eqref{eq:auxiliary-bound-4} and conclude the proof.
\end{proof}

According to the above, problem \eqref{eq:regularization-T-auxiliary-problem-operator} is equivalent to:
Find $\ubu_\epsilon\in \bS$ such that
\begin{equation}\label{eq:operator-J}
\cJ_{\bw_f,\bzeta}(\ubu_\epsilon)(\ubv) 
\,:=\, -\cB(\ubsi_\epsilon(\ubu_\epsilon))(\ubv) + (\cE_2 + \cC + \cL_{\bzeta})(\ubu_\epsilon)(\ubv) 
\,=\, \wh{\bG}(\ubv) \quad \forall\,\ubv\in \bS.
\end{equation}
More precisely, given $\ubsi_\epsilon(\ubu_\epsilon)\in \bQ$, solution of \eqref{eq:auxiliary-problem-first-row}, with $\ubu_\epsilon\in \bS$, solution of \eqref{eq:operator-J}, the vector $(\ubsi_\epsilon,\ubu_\epsilon) = (\ubsi_\epsilon(\ubu_\epsilon),\ubu_\epsilon)\in \bQ\times \bS$ solves \eqref{eq:regularization-T-auxiliary-problem-operator}.
The converse is straightforward.
Hence, we now focus on proving that $\cJ_{\bw_f,\bzeta}$ is bijective or equivalently \eqref{eq:operator-J} is well-posed.
\begin{lem}\label{lem:J-bijective}
Let $\|\bw_f\|_{\bV_f}\leq r_1^0$ (cf. \eqref{eq:r_1^0 defn}) and $\|\bzeta\|_{\bLambda_f} \leq r_2^0$,  where
\begin{equation}\label{eq:r20-def}
r_2^0 :=\frac{\mu \beta^2}{6C_{\cL}},
\end{equation}
where $\beta$ is defined in Lemma~\ref{lem:auxiliary-problem-first-fow} and $C_{\cL}$ is the continuity constant for operator $\cL_{\bzeta}$. Then, the operator $\cJ_{\bw_f,\bzeta}$ is bounded, continuous, coercive and monotone.
\end{lem}
%
\begin{proof}
We first prove that $\cJ_{\bw_f,\bzeta}$ is bounded.
In fact, given $\ubv\in \bS$, from the definition of $\cJ_{\bw_f,\bzeta}$ (cf. \eqref{eq:operator-J}), the continuity of $\cE_2, \cB, \cC$ and $\cL_{\bzeta}$ (cf. \eqref{eq:continuity-cE1-cE2}, \eqref{eq:continuity-cA-cB-cC}, \eqref{eq:continuity-cL-zeta}), in combination with \eqref{eq:auxiliary-bound-1}, yields
\begin{equation*}
\|\cJ_{\bw_f,\bzeta}(\ubv)\|_{\bS'}
\, \leq \, C_{\cB}\,C \left(\|\wh{\bF}\|_{\bQ'} + (1 + \|\bw_f\|_{\bV_f})\,\|\ubv\|_\bS\right)^3 + \Big(C_{\cE_2} + C_f+ C_{\cL}\|\bzeta\|_{\bLambda_f}\Big)\,\|\ubv\|_\bS,
\end{equation*}
which implies that $\cJ_{\bw_f,\bzeta}$ is bounded.
%
Next, we provide the continuity of $\cJ_{\bw_f,\bzeta}$.
Let $\{\ubv_n\}\subseteq \bS$ and $\ubv\in \bS$ such that $\|\ubv_n - \ubv\|_\bS\to 0$ as $n\to \infty$. 
Thus, using again the definition of $\cJ_{\bw_f,\bzeta}$, the continuity of $\cB, \cE_2$ and $\cC$ (cf. \eqref{eq:continuity-cA-cB-cC}, \eqref{eq:continuity-cE1-cE2}), in  combination with \eqref{eq:auxiliary-bound-4}, it follows that
\begin{equation*}
\|\cJ_{\bw_f,\bzeta}(\ubv_n) - \cJ_{\bw_f,\bzeta}(\ubv)\|_{\bS'}
\,\leq \,  \Big( C\,C_{\cB}\,(1 + \|\bw_f\|_{\bV_f}) + C_{\cE_2} + C_f+C_{\cL}\|\bzeta\|_{\bLambda_f} \Big)\,\|\ubv_n - \ubv\|_\bS \,,
\end{equation*}
which prove the continuity of $\cJ_{\bw_f,\bzeta}$.

Now, for the coercivity of $\cJ_{\bw_f,\bzeta}$, we first notice from \eqref{eq:auxiliary-bound-2} that
\begin{equation}\label{eq:bount-to-coercivity}
\|\ubv\|_\bS \,\leq\, C\,\max\big\{\|\wh{\bF}\|_{\bQ'},1\big\}\,\Big( 1 + \|\ubsi_\epsilon(\ubv)\|_\bQ + \,\|\bdiv(\bsi_{f,\epsilon}(\ubv))\|^{1/3}_{\bL^{4/3}(\Omega_f)}+\, \|\div(\bu_{p,\epsilon}(\ubv))\|_{\bL^2(\Omega_p)}\Big)\,,
\end{equation}
and then, it is clear that $\|\ubsi_\epsilon(\ubv)\|_\bQ \to \infty$ as $\|\ubv\|_\bS \to \infty$.
Next, using again the definition of $\cJ_{\bw_f,\bzeta}$ (cf. \eqref{eq:operator-J}), the monotonicity of $\cE_2$ and  $\cC$ (cf. Lemma \ref{lem:coercivity-properties-A-E2}) and the identity \eqref{eq:auxiliary-problem-first-row} with $\ubtau = \ubsi_{\epsilon}(\ubv) := (\bsi_{f,\epsilon}(\ubv),\bu_{p,\epsilon}(\ubv),\bsi_{e,\epsilon}(\ubv))$, we find that
\begin{align*}
& \cJ_{\bw_f,\bzeta}(\ubv)(\ubv) \,=\, -\,\cB(\bsi_\epsilon(\ubv))(\ubv) + (\cE_2 + \cC+\cL_{\bzeta})(\ubv)(\ubv) \,\geq\, -\,\cB'(\ubv)(\bsi_\epsilon(\ubv))+\cL_{\bzeta}(\ubv)(\ubv) \nonumber\\ 
& \quad \,=\, (\cE_1 + \cA + \epsilon\,\cR)(\ubsi_\epsilon(\ubv))(\ubsi_\epsilon(\ubv)) + \cK_{\bw_f}(\ubv)(\ubsi_\epsilon(\ubv))+\cL_{\bzeta}(\ubv)(\ubv) - \wh{\bF}(\ubsi_\epsilon(\ubv))\,,
\end{align*}
which combined with the  non-negativity bounds of $\cE_1, \cA$ and $\cR$ (cf. \eqref{eq: operator E_1-monotone}, \eqref{eq: operator A-monotone}, \eqref{eq: R-coercive}), and the continuity of $\cK_{\bw_f},\cL_{\bzeta}$ (cf. \eqref{eq:continuity-cK-wf}, \eqref{eq:continuity-cL-zeta}) and $\wh{\bF}$, yields
\begin{align}
&\cJ_{\bw_f,\bzeta}(\ubv)(\ubv) 
\,\geq\, \frac{1}{2\,\mu} \|\bsi^\rd_{f,\epsilon}(\ubv)\|^2_{\bbL^2(\Omega_f)} 
+ \epsilon\,\|\bdiv(\bsi_{f,\epsilon}(\ubv))\|^{4/3}_{\bL^{4/3}(\Omega_f)}
+ C_2\,\Big( \|\bu_{p,\epsilon}(\ubv)\|^2_{\bX_p} 
+ \|\bsi_{e,\epsilon}(\ubv)\|^2_{\bSigma_e} \Big) \nonumber\\
& \qquad- \frac{\rho_f\,n^{1/2}}{2\,\mu}\,\|\bw_f\|_{\bV_f}\,\|\bv_f\|_{\bV_f}\,\|\bsi^\rd_{f,\epsilon}(\ubv)\|_{\bbL^2(\Omega_f)}
- C_{\cL}\|\bzeta\|_{\bLambda_f}\|\bpsi\|^2_{\bLambda_f}
- \|\wh{\bF}\|_{\bQ'}\,\|\ubsi_\epsilon(\ubv)\|_{\bQ} \,, \label{eq:J-coercivity1}
\end{align}
with $C_2$ as in \eqref{eq:E1-A-epsR coercive}.
To control the terms involving $\|\bw_f\|_{\bV_f}$ and $\|\bzeta\|_{\bLambda_f}$, and  similarly to \eqref{eq:inf-sup-B-1}, we use the inf-sup condition \eqref{eq:inf-sup-vf-chif} along with the equation in \eqref{eq:auxiliary-problem-first-row} with test function $\ubtau=(\btau_f,\0,\0)$, to obtain
\begin{align}
&\frac{5\beta}{6}\,\|(\bv_f, \bchi_f, \bpsi)\|_{\bV_f\times \bbQ_f\times \bLambda_f}  \,\leq\, \sup_{\0 \neq\btau_f\in \bbX_f}\frac{-a_f(\bsi_f(\ubv),\btau_f) - \epsilon\,R_f(\bsi_f(\ubv),\btau_f)}{\|\btau_f\|_{\bchi_f}}\nonumber\\
&\qquad \leq\, \frac{1}{2\,\mu} \|\bsi^\rd_f(\ubv)\|_{\bbL^2(\Omega_f)}
+ \epsilon\,\|\bdiv(\bsi_f(\ubv))\|^{1/3}_{\bL^{4/3}(\Omega_f)} \,, \label{eq:bount-to-coercivity1},
\end{align}
and deduce that both $\|\bv_f\|_{\bV_f}$ and $\|\bpsi\|_{\bLambda_f}$ are bounded by
\begin{equation*}
\frac{3}{5\mu\,\beta}\|\bsi^\rd_{f,\epsilon}(\ubv)\|_{\bbL^2(\Omega_f)} + \frac{6\,\epsilon}{5\beta}\,\|\bdiv(\bsi_{f,\epsilon}(\ubv))\|^{1/3}_{\bL^{4/3}(\Omega_f)}\,.
\end{equation*}
%
%\begin{equation*}%\label{eq:J-coercive-psi-bound}
%\mbox{and}\quad \|\bpsi\|_{\bLambda_f} \leq \frac{3}{5\mu\,\beta}\|\bsi^\rd_{f,\epsilon}(\ubv)\|_{\bbL^2(\Omega_f)} + \frac{6\,\epsilon}{5\beta}\,\|\bdiv(\bsi_{f,\epsilon}(\ubv))\|^{1/3}_{\bL^{4/3}(\Omega_f)}.
%\end{equation*}
Then, using the facts that $\|\bw_f\|_{\bV_f} \leq r_1^0$ (cf. \eqref{eq:r_1^0 defn}) and $\|\bzeta\|_{\bLambda_f} \leq r_2^0$ (cf. \eqref{eq:r20-def})  and applying Young's inequality, we get 
\begin{equation}\label{eq:kappa-continuity bound}
\frac{\rho_f\,n^{1/2}}{2\,\mu}\,\|\bw_f\|_{\bV_f}\,\|\bv_f\|_{\bV_f}\,\|\bsi^\rd_{f,\epsilon}(\ubv)\|_{\bbL^2(\Omega_f)} \leq \frac{1}{10}\left(\frac{1}{\mu}+\epsilon\right)\,\|\bsi^\rd_{f,\epsilon}(\ubv)\|^2_{\bbL^2(\Omega_f)} + \frac{\epsilon}{10}\|\bdiv(\bsi_{f,\epsilon}(\ubv))\|^{2/3}_{\bL^{4/3}(\Omega_f)}\,,
\end{equation}
%
\begin{equation}\label{eq:operator L continuity bound}
\mbox{and}\quad C_{\cL}\|\bzeta\|_{\bLambda_f}\|\bpsi\|^2_{\bLambda_f} \leq \frac {3}{25\mu}\,\|\bsi^\rd_{f,\epsilon}(\ubv)\|^2_{\bbL^2(\Omega_f)}+\frac{12\mu\epsilon^2}{25}\|\bdiv(\bsi_{f,\epsilon}(\ubv))\|^{2/3}_{\bL^{4/3}(\Omega_f)} \,.
\end{equation}
Replacing back \eqref{eq:kappa-continuity bound} and \eqref{eq:operator L continuity bound} into \eqref{eq:J-coercivity1}, using the definition of $r_2^0$ (cf. \eqref{eq:r20-def}), choosing
\begin{equation}\label{eq:epsilon-bound}
0<\epsilon \leq \frac{5}{24\mu} \,,
\end{equation} 
and after some algebraic manipulations, we deduce that
\begin{align}
&\cJ_{\bw_f,\bzeta}(\ubv)(\ubv) 
\,\geq \frac{311}{1200\mu}\|\bsi^\rd_{f,\epsilon}(\ubv)\|^2_{\bbL^2(\Omega_f)} + C_2\,\Big( \|\bu_{p,\epsilon}(\ubv)\|^2_{\bX_p} + \|\bsi_{e,\epsilon}(\ubv)\|^2_{\bSigma_e} \Big) \nonumber\\
&\qquad +\,\, \epsilon\,\|\bdiv(\bsi_{f,\epsilon}(\ubv))\|^{4/3}_{\bL^{4/3}(\Omega_f)} - \frac{\epsilon}{5}\|\bdiv(\bsi_{f,\epsilon}(\ubv))\|^{2/3}_{\bL^{4/3}(\Omega_f)} - \|\wh{\bF}\|_{\bQ'}\,\|\ubsi_\epsilon(\ubv)\|_{\bQ}.\label{eq:J-coercivity3a}
\end{align}
Now, we consider the following two cases:

\medskip
\noindent {\bf Case 1:} If $\|\bdiv(\bsi_{f,\epsilon}(\ubv))\|_{\bL^{4/3}(\Omega_f)} \geq 1$
then the terms multiplied by $\epsilon$ on right-hand side of \eqref{eq:J-coercivity3a} are replaced by  $\frac{4\,\epsilon}{5}\|\bdiv(\bsi_{f,\epsilon}(\ubv))\|^{4/3}_{\bL^{4/3}(\Omega_f)}$.
Using the latter in \eqref{eq:J-coercivity3a} in combination with \eqref{eq:tau-d-H0div-inequality}--\eqref{eq:tau-H0div-Xf-inequality} for $q=4/3$, similar arguments to \eqref{eq:coercivity-E1-A-R}, \eqref{eq:bount-to-coercivity}, and some algebraic computations, we deduce that
\begin{equation}\label{eq:J-coercivity-case-1}
\frac{\cJ_{\bw_f,\bzeta}(\ubv)(\ubv)}{\|\ubv\|_\bS} 
\,\geq\, C\,\bA(\ubsi_\epsilon(\ubv))\Big(\|\ubsi_\epsilon(\ubv)\|^{1/3}_\bQ \,-\, \|\wh{\bF}\|_{\bQ'} \Big)\,,
\end{equation}
where $\bA(\ubsi_\epsilon(\ubv)) = \frac{\|\ubsi_\epsilon(\ubv)\|_\bQ}{\bB(\ubsi_\epsilon(\ubv))}$ and $\bB(\ubsi_\epsilon(\ubv))= 1 + \|\ubsi_\epsilon(\ubv)\|_\bQ + \|\bdiv(\bsi_{f,\epsilon}(\ubv))\|^{1/3}_{\bL^{4/3}(\Omega_f)}+ \|\div(\bu_{p,\epsilon}(\ubv))\|_{\bL^2(\Omega_p)}$. 

\medskip
\noindent {\bf Case 2:} If $\|\bdiv(\bsi_{f,\epsilon}(\ubv))\|_{\bL^{4/3}(\Omega_f)} < 1$ then the negative term multiplied by $\epsilon$ on right-hand side of \eqref{eq:J-coercivity3a} is bounded. 
Using again inequalities \eqref{eq:tau-d-H0div-inequality}--\eqref{eq:tau-H0div-Xf-inequality} with exponent $q=4/3$, similar arguments to \eqref{eq:coercivity-E1-A-R},
\eqref{eq:bount-to-coercivity}, and after simple algebraic manipulation, we deduce that
\begin{equation}\label{eq:J-coercivity-case-2}
\frac{\cJ_{\bw_f,\bzeta}(\ubv)(\ubv)}{\|\ubv\|_\bS} 
\,\geq\, \wt{C}\,\bA(\ubsi_\epsilon(\ubv)) \Big(\|\ubsi_\epsilon(\ubv)\|^{1/3}_\bQ \,-\, \|\wh{\bF}\|_{\bQ'} \Big) 
- \wh{C}\,\frac{\|\bdiv(\bsi_{f,\epsilon}(\ubv))\|^{2/3}_{\bL^{4/3}(\Omega_f)}}{\bB(\ubsi_\epsilon(\ubv))}\,.
\end{equation}
Taking $\|\ubv\|_\bS\to \infty$ in both \eqref{eq:J-coercivity-case-1} and \eqref{eq:J-coercivity-case-2}, we conclude that $\cJ_{\bw_f,\bzeta}$ is coercive.


Finally, employing the identity \eqref{eq:identity-E1-A-R-2} with $\ubtau = \ubsi_\epsilon(\ubv_1) - \ubsi_\epsilon(\ubv_2)\in \bQ$ and the monotonicity of $\cE_2 + \cC$, we deduce
\begin{equation*}
\begin{array}{l}
\ds (\cJ_{\bw_f,\bzeta}(\ubv_1) - \cJ_{\bw_f,\bzeta}(\ubv_2))(\ubv_1 - \ubv_2) 
\,\geq\, \cB'(\ubv_2 - \ubv_1)(\ubsi_\epsilon(\ubv_1) - \ubsi_\epsilon(\ubv_2))+\cL_{\bzeta}(\ubv_1 - \ubv_2) (\ubv_1 - \ubv_2)  \\ [2ex]
\ds\quad \,=\, (\cE_1 + \cA)(\ubsi_\epsilon(\ubv_1) - \ubsi_\epsilon(\ubv_2))(\ubsi_\epsilon(\ubv_1) - \ubsi_\epsilon(\ubv_2)) + \cK_{\bw_f}(\ubv_1 - \ubv_2)(\ubsi_\epsilon(\ubv_1) - \ubsi_\epsilon(\ubv_2)) \\ [2ex]
\ds\quad \,+\,\,\epsilon\,(\cR(\ubsi_\epsilon(\ubv_1)) - \cR(\ubsi_\epsilon(\ubv_2)))(\ubsi_\epsilon(\ubv_1) - \ubsi_\epsilon(\ubv_2))+\cL_{\bzeta}(\ubv_1 - \ubv_2) (\ubv_1 - \ubv_2) ,
\end{array}
\end{equation*}
which together with the monotonicity of $\cE_1, \cA, \cR$ (cf. \eqref{eq: operator A-monotone}, \eqref{eq: operator E_1-monotone}, \eqref{eq:R_f monotone}, \eqref{eq:R_p monotone}), the continuity of $\cK_{\bw_f}$ and $\cL_{\bzeta}$ (cf. \eqref{eq:continuity-cK-wf}, \eqref{eq:continuity-cL-zeta}) with $\|\bw_f\|_{\bV_f} \leq r_1^0$ and $\|\bzeta\|_{\bLambda_f} \leq r_2^0$ (cf. \eqref{eq:r_1^0 defn}, \eqref{eq:r20-def}), and similar arguments to the coercivity bound, we conclude the monotonicity of $\cJ_{\bw_f,\bzeta}$ and complete the proof.
\end{proof}
\begin{lem}\label{thm:well-posedness-1}
For each $\wh{\f}_f\in \bL^2(\Omega_f), \wh{\f}_p\in \bL^2(\Omega_p), \wh{\f}_e \in \bbL^2(\Omega_p)$, and $\wh{q}_p\in \L^2(\Omega_p)$, the problem \eqref{eq:T-auxiliary-problem-operator} has a unique solution $(\ubsi,\ubu)\in \bQ\times \bS$ for each $(\bw_f,\bzeta)\in \bV_f\times \bLambda_f$ such that $\|\bw_f\|_{\bV_f} \leq r_1^0$ and $\|\bzeta\|_{\bLambda_f} \leq r_2^0$ (cf. \eqref{eq:r_1^0 defn}, \eqref{eq:r20-def}).
Moreover, there exists a constant $c_\bT > 0$, independent of $\bw_f,\bzeta$ and the data $\wh{\f}_f, \wh{\f}_p, \wh{\f}_e $, and $\wh{q}_p$, such that
\begin{equation}\label{eq:ubsi-ubu-bound-solution}
\, \|(\ubsi, \ubu)\|_{\bQ\times \bS} 
\,\leq\, {c_\bT\,\Big\{ \|\wh{\f}_f\|_{\bL^2(\Omega_f)} +  \|\wh{\f}_e\|_{\bbL^2(\Omega_p)}+ \|\wh{\f}_p\|_{\bL^2(\Omega_p)} + \| \wh{q}_p\|_{\L^2(\Omega_p)}\Big\}}.
\end{equation}
\end{lem}
%
\begin{proof}
We begin by noting that the well-posedness of the regularized problem \eqref{eq:regularization-T-auxiliary-problem-operator} follows straightforwardly from Lemmas \ref{lem:auxiliary-problem-first-fow} and \ref{lem:J-bijective} in combination with the Browder--Minty theorem \cite[Theorem 10.49]{Renardy-Rogers}, that is, we establish the existence of a solution $(\ubsi_\epsilon, \ubu_\epsilon)\in \bQ\times \bS$ of \eqref{eq:regularization-T-auxiliary-problem-operator}, where $\ubsi_\epsilon = (\bsi_{f,\epsilon}, \bu_{p,\epsilon}, \bsi_{e,\epsilon})$ and $\ubu_\epsilon = (\bu_{f,\epsilon}, p_{p,\epsilon}, \bvarphi_{\epsilon}, \bu_{s,\epsilon}, \bgamma_{f,\epsilon}, \lambda_{\epsilon})$.
Now, in order to bound $\|\ubsi_\epsilon\|_\bQ$ and $\|\ubu_\epsilon\|_\bS$ independently of $\epsilon$, we proceed similarly to \cite[Lemma~4.6]{aeny2019}.
To that end, we begin testing \eqref{eq:regularization-T-auxiliary-problem-operator} with $\ubtau = \ubsi_\epsilon$ and $\ubv = \ubu_\epsilon$, to obtain
\begin{equation*}%\label{eq:regularization-T-auxiliary-problem-operator-a}
(\cE_1 + \cA + \epsilon\,\cR)(\ubsi_\epsilon)(\ubsi_\epsilon) + \cK_{\bw_f}(\ubu_\epsilon)(\ubsi_\epsilon) + (\cE_2 + \cC +\cL_{\bzeta})(\ubu_\epsilon)(\ubu_\epsilon) = \wh{\bF}(\ubsi_\epsilon) + \wh{\bG}(\ubu_\epsilon)\,,
\end{equation*}
which, together with the non-negativity and coercivity estimates of the operators $\cE_1, \cA, \cR, \cE_2, \cC$ (cf. \eqref{eq: operator E_1-monotone}, \eqref{eq: operator A-monotone}, \eqref{eq: R-coercive}, \eqref{eq: operator E_2-monotone}, \eqref{eq: operator C-monotone}), and the continuity of $\cK_{\bw_f}$ and $\cL_{\bzeta}$ (cf. \eqref{eq:continuity-cK-wf}, \eqref{eq:continuity-cL-zeta}), yields
\begin{align}
& \frac{1}{2\,\mu} \|\bsi^\rd_{f,\epsilon}\|^2_{\bbL^2(\Omega_f)} 
+ \mu\,k_{\min} \|\bu_{p,\epsilon}\|^2_{\bL^2(\Omega_p)} 
+ C_A \|\bsi_{e,\epsilon}\|^2_{\bSigma_e} 
+ \epsilon\,\|\bdiv(\bsi_{f,\epsilon})\|^{4/3}_{\bL^{4/3}(\Omega_f)}  
+ \epsilon\,\|\div(\bu_{p,\epsilon})\|^2_{\L^2(\Omega_p)}
\nonumber \\
&\quad +\, \rho_f\|\bu_{f,\epsilon}\|^2_{\bL^2(\Omega_f)} + s_0 \|p_{p,\epsilon}\|^2_{\W_p}
+ \rho_p\|\bu_{s,\epsilon}\|^2_{\bL^2(\Omega_p)} 
+ c_I\,\sum^{n-1}_{j=1} \|( \bvarphi_{\epsilon}-\bu_{s,\epsilon})\cdot\bt_{f,j}\|^2_{\L^2(\Gamma_{fp})} \nonumber \\
&\quad -\, \frac{\rho_f\,n^{1/2}}{2\,\mu}\,\|\bw_f\|_{\bV_f}\,\|\bu_{f,\epsilon}\|_{\bV_f}\,\|\bsi^\rd_{f,\epsilon}\|_{\bbL^2(\Omega_f)}
- C_{\cL}\|\bzeta\|_{\bLambda_f}\|\bvarphi_{\epsilon}\|_{\bLambda_f}^2 \nonumber \\
& \leq\,   
\|\wh{\f}_f\|_{\bL^2(\Omega_f)} \,\|\bu_{f,\epsilon}\|_{\bL^2(\Omega_f)} + \|\wh{\f}_p\|_{\bL^2(\Omega_p)} \|\bu_{s,\epsilon}\|_{\bL^2(\Omega_p)}\, + \,\,\|\wh{q}_p\|_{\L^2(\Omega_p)}\|p_{p,\epsilon}\|_{\W_p}+\|\wh{\f}_e\|_{\bbL^2(\Omega_p)}\,\|\bsi_{e,\epsilon}\|_{\bSigma_e}. \label{eq:bound-solution-data-1}
\end{align}
%Applying \eqref{eq:epsilon-bound} in \eqref{eq:operator L continuity bound} and definition of $r_2^0$ (cf. \eqref{eq:r20-def}) we get, 
%\begin{equation}\label{eq:operator L continuity bound-1}
%C_{\cL}\|\bzeta\|_{\bLambda_f}\|\bvarphi_{\epsilon}\|^2_{\bLambda_f} 
%\,\leq\, \frac {3}{25\mu}\,\|\bsi^\rd_{f,\epsilon}\|^2_{\bbL^2(\Omega_f)}
%+ \frac{\epsilon}{10}\|\bdiv(\bsi_{f,\epsilon})\|^{2/3}_{\bL^{4/3}(\Omega_f)} \,.
%\end{equation}
%Applying \eqref{eq:kappa-continuity bound}, \eqref{eq:operator L continuity bound-1} and non-negativity estimate \eqref{eq:positivity-aBJS} in \eqref{eq:bound-solution-data-1}, and using \eqref{eq:epsilon-bound}  we get
Then, choosing $\epsilon$ as in \eqref{eq:epsilon-bound}, using \eqref{eq:kappa-continuity bound} and \eqref{eq:operator L continuity bound}, the definition of $r_2^0$ (cf. \eqref{eq:r20-def}), \eqref{eq:positivity-aBJS} to bound the ninth term in \eqref{eq:bound-solution-data-1} by zero, and some algebraic computations, we deduce that
\begin{align}
& \frac{311}{1200\mu}\|\bsi^\rd_{f,\epsilon}\|^2_{\bbL^2(\Omega_f)} 
+ \mu\,k_{\min} \|\bu_{p,\epsilon}\|^2_{\bL^2(\Omega_p)}
+ C_A\,\|\bsi_{e,\epsilon}\|^2_{\bSigma_e}
+ \epsilon\,\|\bdiv(\bsi_{f,\epsilon})\|^{4/3}_{\bL^{4/3}(\Omega_f)} 
- \frac{\epsilon}{5}\,\|\bdiv(\bsi_{f,\epsilon})\|^{2/3}_{\bL^{4/3}(\Omega_f)}
\nonumber \\
&\quad +\, \epsilon\,\|\div(\bu_{p,\epsilon})\|^2_{\L^2(\Omega_p)}
+ \rho_f \|\bu_{f,\epsilon}\|^2_{\bL^2(\Omega_f)} 
+ s_0\|p_{p,\epsilon}\|^2_{\W_p}
+ \rho_p\|\bu_{s,\epsilon}\|^2_{\bL^2(\Omega_p)} 
\nonumber \\
& \leq  \|\wh{\f}_f\|_{\bL^2(\Omega_f)} \|\bu_{f,\epsilon}\|_{\bL^2(\Omega_f)} 
+ \|\wh{\f}_p\|_{\bL^2(\Omega_p)} \|\bu_{s,\epsilon}\|_{\bL^2(\Omega_p)}
+ \|\wh{q}_p\|_{\L^2(\Omega_p)}\|p_{p,\epsilon}\|_{\W_p} 
+ \|\wh{\f}_e\|_{\bbL^2(\Omega_p)}\,\|\bsi_{e,\epsilon}\|_{\bSigma_e} \,.
\label{eq:bound-solution-data-2}
\end{align}

On the other hand, from the first row of \eqref{eq:regularization-T-auxiliary-problem-operator}, employing inf-sup condition of $\cB$ (cf. Lemma ~\ref{lem:inf-sup-conditions}), continuity of $\cK_{\bw_f}$ (cf. \eqref{eq:continuity-cK-wf}) along with $\|\bw_f\|_{\bV_f}\leq r_1^0$ (cf. \eqref{eq:r_1^0 defn}) and continuity of $\cR$ (cf. \eqref{eq: R-continuous}), we deduce that
\begin{align}
&\|\ubu_\epsilon\|_\bS
\, \leq \, C\,\Big( \|\bsi^\rd_{f,\epsilon}\|_{\bbL^2(\Omega_f)} + \|\bu_{p,\epsilon}\|_{\bL^2(\Omega_p)} + \|\bsi_{e,\epsilon}\|_{\bSigma_e} + \|\wh{\f}_e\|_{\bbL^2(\Omega_p)}  \nonumber \\ 
&\qquad +\, \epsilon\,\|\bdiv(\bsi_{f,\epsilon})\|^{1/3}_{\bL^{4/3}(\Omega_f)}
+ \epsilon\,\|\div(\bu_{p,\epsilon})\|_{\bL^2(\Omega_p)} \Big)\,.\label{eq:ubu-bound}
\end{align}
In turn, from the second row of \eqref{eq:regularization-T-auxiliary-problem-operator} and recalling that $\bdiv(\bbX_f) = (\bV_f)'$ and $\div(\bX_p) = (\W_p)'$, it follows that
\begin{align}
& \|\bdiv(\bsi_{f,\epsilon})\|_{\bL^{4/3}(\Omega_f)} + \|\div(\bu_{p,\epsilon})\|_{\L^2(\Omega_p)} \nonumber \\ 
&\quad \leq\, C\Big(\|\wh{\f}_f\|_{\bL^2(\Omega_f)} + \|\wh{q}_p\|_{\L^2(\Omega_p)} + \|\bu_{f,\epsilon}\|_{\bL^2(\Omega_f)} + s_0\|p_{p,\epsilon}\|_{\W_p} + \|\bu_{s,\epsilon}\|_{\bV_p} \Big).\label{eq:bound-solution-data-2a}
\end{align}
Next, as in Lemma \ref{lem:J-bijective}, we consider the following two cases:

\medskip
\noindent {\bf Case 1:} If $\|\bdiv(\bsi_{f,\epsilon}(\ubv))\|_{\bL^{4/3}(\Omega_f)} \geq 1$
then the fourth and fifth terms on right-hand side of \eqref{eq:bound-solution-data-2} are replaced by  $\frac{4\,\epsilon}{5}\|\bdiv(\bsi_{f,\epsilon}(\ubv))\|^{4/3}_{\bL^{4/3}(\Omega_f)}$.
Thus, using similar arguments to the ones employed in Lemma \ref{lem:J-bijective}, Young's inequality with appropriate weights on right hand side of \eqref{eq:bound-solution-data-2} in combination with \eqref{eq:ubu-bound} and \eqref{eq:bound-solution-data-2a}, and estimates \eqref{eq:tau-d-H0div-inequality}--\eqref{eq:tau-H0div-Xf-inequality}, we obtain
%\begin{align}
%& \|\bsi^\rd_{f,\epsilon}\|^2_{\bbL^2(\Omega_f)} + \|\bu_{p,\epsilon}\|^2_{\bL^2(\Omega_p)}
%+ \|\bsi_{e,\epsilon}\|^2_{\bSigma_e}
%+ \epsilon\,\|\bdiv(\bsi_{f,\epsilon})\|^{4/3}_{\bL^{4/3}(\Omega_f)} 
%+ \epsilon\,\|\div(\bu_{p,\epsilon})\|^2_{\L^2(\Omega_p)}+\|\bu_{f,\epsilon}\|^2_{\bL^2(\Omega_f)}  \nonumber\\ 
%&\quad +\, \|p_{p,\epsilon}\|^2_{\W_p} + \|\bu_{s,\epsilon}\|^2_{\bL^2(\Omega_p)} 
%\,\leq\, C\big( \|\wh{\f}_f\|^2_{\bL^2(\Omega_f)}  + \|\wh{\f}_p\|^2_{\bL^2(\Omega_p)} + \,\, \|\wh{q}_p\|^2_{\L^2(\Omega_p)}+ \|\wh{\f}_e\|^2_{\bbL^2(\Omega_p)}\big) \nonumber\\ 
%&  \quad+ \frac{1}{2}\,\big( \,\|\bu_{f,\epsilon}\|^2_{\bL^2(\Omega_f)}+\|\bu_{s,\epsilon}\|^2_{\bL^2(\Omega_p)}+ \|p_{p,\epsilon}\|^2_{\W_p}+\,\|\bsi_{e,\epsilon}\|^2_{\bSigma_e}\big) \,.
%\label{eq:bound-solution-data-3}
%\end{align}
%Applying  \eqref{eq:bound-solution-data-3} in \eqref{eq:ubu-bound}, we get
%\begin{align*}
%&\|\ubu_\epsilon\|^2_\bS
%\, \leq C\big(\|\wh{\f}_f\|^2_{\bL^2(\Omega_f)}  + \|\wh{\f}_p\|^2_{\bL^2(\Omega_p)} +\,\, \|\wh{q}_p\|^2_{\L^2(\Omega_p)}+ \|\wh{\f}_e\|^2_{\bbL^2(\Omega_p)}\big)\,, 
%%\label{eq:bound-solution-data-4}
%\end{align*}
%which together with \eqref{eq:bound-solution-data-3}, yields 
\begin{align}
& \|\bsi_{f,\epsilon}\|^2_{\bbX_f} 
+ \|\bu_{p,\epsilon}\|^2_{\bX_p}
+ \|\bsi_{e,\epsilon}\|^2_{\bSigma_e}
+ \|\ubu_\epsilon\|^2_\bS 
+ s_0\,\|p_{p,\epsilon}\|^2_{\W_p} 
+ \epsilon\,\|\bdiv(\bsi_{f,\epsilon})\|^{4/3}_{\bL^{4/3}(\Omega_f)}
+ \epsilon\,\|\div(\bu_{p,\epsilon})\|^2_{\L^2(\Omega_p)} 
\nonumber \\
&\quad \,\leq\,  C\big(\|\wh{\f}_f\|^2_{\bL^2(\Omega_f)}  + \|\wh{\f}_p\|^2_{\bL^2(\Omega_p)} + \,\, \|\wh{q}_p\|^2_{\L^2(\Omega_p)}+ \|\wh{\f}_e\|^2_{\bbL^2(\Omega_p)}\big) \,.
\label{eq:bound-solution-data-5}
\end{align}

\medskip
\noindent {\bf Case 2:} If $\|\bdiv(\bsi_{f,\epsilon})\|_{\bL^{4/3}(\Omega_f)} < 1$ then the negative term in \eqref{eq:bound-solution-data-2} is moved to the right-hand side and bounded by $\epsilon/5$.
Thus, analogously to \eqref{eq:bound-solution-data-5}, we use Young's inequality, \eqref{eq:ubu-bound} and \eqref{eq:bound-solution-data-2a}, and estimates \eqref{eq:tau-d-H0div-inequality}--\eqref{eq:tau-H0div-Xf-inequality}, to derive
%Applying Young's inequality with appropriate weights on right hand side of \eqref{eq:bound-solution-data-2}, it follows that
%\begin{align}
%& \|\bsi^\rd_{f,\epsilon}\|^2_{\bbL^2(\Omega_f)} + \|\bu_{p,\epsilon}\|^2_{\bL^2(\Omega_p)}+  \,\|\bsi_{e,\epsilon}\|^2_{\bSigma_e}+ \epsilon\,\|\bdiv(\bsi_{f,\epsilon})\|^{4/3}_{\bL^{4/3}(\Omega_f)} 
%+ \epsilon\,\|\div(\bu_{p,\epsilon})\|^2_{\L^2(\Omega_p)} + \|\bu_{f,\epsilon}\|^2_{\bL^2(\Omega_f)}   \nonumber \\
%&\quad +\, \|p_{p,\epsilon}\|^2_{\W_p}+ \|\bu_{s,\epsilon}\|^2_{\bL^2(\Omega_p)}
%\,\leq\,  C\big(\|\wh{\f}_f\|^2_{\bL^2(\Omega_f)}  + \|\wh{\f}_p\|^2_{\bL^2(\Omega_p)} + \,\, \|\wh{q}_p\|^2_{\L^2(\Omega_p)}+ \|\wh{\f}_e\|^2_{\bbL^2(\Omega_p)} + \epsilon\big).\label{eq:bound-solution-data-7}
%\end{align}
%Since $\|\bdiv(\bsi_{f,\epsilon}(\ubu_\epsilon))\|_{\bL^{4/3}(\Omega_f)} < 1$ , \eqref{eq:ubu-bound} results in
%\begin{equation}\label{eq:ubu-bound-a}
%\|\ubu_\epsilon\|_\bS
%\, \leq \, C\,\Big( \|\bsi^\rd_{f,\epsilon}\|_{\bbL^2(\Omega_f)} + \|\bu_{p,\epsilon}\|_{\bL^2(\Omega_p)} + \|\bsi_{e,\epsilon}\|_{\bSigma_e} + \|\wh{\f}_e\|_{\bbL^2(\Omega_p)}
%\,+\, \epsilon( 1 +\, \|\div(\bu_{p,\epsilon})\|_{\bL^2(\Omega_p)})\Big) \,.
%\end{equation}
%Applying  \eqref{eq:bound-solution-data-7} in \eqref{eq:ubu-bound-a} and using \eqref{eq:epsilon-bound}, we get
%\begin{align*}
%&\|\ubu_\epsilon\|^2_\bS
%\, \leq C\,\Big(\|\wh{\f}_f\|^2_{\bL^2(\Omega_f)}  + \|\wh{\f}_p\|^2_{\bL^2(\Omega_p)} + \,\, \|\wh{q}_p\|^2_{\L^2(\Omega_p)}+ \|\wh{\f}_e\|^2_{\bbL^2(\Omega_p)}+ \epsilon \Big)\,,
%%\label{eq:bound-solution-data-8}
%\end{align*}
%which together with \eqref{eq:bound-solution-data-7}, yields 
\begin{align}
&\|\bsi_{f,\epsilon}\|^2_{\bbX_f} 
+ \|\bu_{p,\epsilon}\|^2_{\bX_p}
+ \|\bsi_{e,\epsilon}\|^2_{\bSigma_e}
+ \|\ubu_\epsilon\|^2_\bS 
+ s_0\,\|p_{p,\epsilon}\|^2_{\W_p} 
+ \epsilon\,\|\bdiv(\bsi_{f,\epsilon})\|^{4/3}_{\bL^{4/3}(\Omega_f)} 
+ \epsilon\,\|\div(\bu_{p,\epsilon})\|^2_{\L^2(\Omega_p)}
\nonumber \\ 
&\quad \,\leq\,  C\,\Big(\|\wh{\f}_f\|^2_{\bL^2(\Omega_f)}  + \|\wh{\f}_p\|^2_{\bL^2(\Omega_p)} + \,\, \|\wh{q}_p\|^2_{\L^2(\Omega_p)}+ \|\wh{\f}_e\|^2_{\bbL^2(\Omega_p)}+ \epsilon \Big)\,.
\label{eq:bound-solution-data-9}
\end{align}
%Combining both the cases stated above (cf. \eqref{eq:bound-solution-data-5} and \eqref{eq:bound-solution-data-9}), we get
%\begin{equation}\label{eq:bound-solution-data-10}
%\|\ubsi_{\epsilon}\|^2_{\bQ} 
%+ \|\ubu_\epsilon\|^2_\bS 
%\,\leq\, C\,\Big(\|\wh{\f}_f\|^2_{\bL^2(\Omega_f)}  + \|\wh{\f}_p\|^2_{\bL^2(\Omega_p)} + \,\, \|\wh{q}_p\|^2_{\L^2(\Omega_p)}+ \|\wh{\f}_e\|^2_{\bbL^2(\Omega_p)}+ \epsilon \Big)\,.
%\end{equation}
%Therefore, from \eqref{eq:bound-solution-data-2a} and \eqref{eq:bound-solution-data-10}, in combination with \eqref{eq:tau-d-H0div-inequality}, \eqref{eq:tau-H0div-Xf-inequality} and Young inequalities, there exists $c_\bT > 0$ independent of $\epsilon$, such that
From \eqref{eq:bound-solution-data-5} and \eqref{eq:bound-solution-data-9}, we deduce that for both cases, there exists $\wt{c}_\bT > 0$ independent of $\epsilon$ and $s_0$, such that
\begin{equation}\label{eq:bound-solution-independently-epsilon-1}
\|(\ubsi_\epsilon,\ubu_\epsilon)\|_{\bQ\times \bS}
\,\leq\,\wt{c}_\bT\,\Big\{ \|\wh{\f}_f\|_{\bL^2(\Omega_f)} +  \|\wh{\f}_e\|_{\bbL^2(\Omega_p)} + \|\ \wh{\f}_p\|_{\bL^2(\Omega_p)} + \| \wh{q}_p\|_{\L^2(\Omega_p)}  + \epsilon^{1/2} \Big\},
\end{equation}
which prove that the solution $(\ubsi_\epsilon, \ubu_\epsilon)\in \bQ\times \bS$ of \eqref{eq:regularization-T-auxiliary-problem-operator} is bounded independently of $\epsilon$.

Given $\epsilon>0$, $\{(\ubsi_\epsilon, \ubu_\epsilon)\}$ solution of \eqref{eq:regularization-T-auxiliary-problem-operator} is a bounded sequence independently of $\epsilon$.
Then, since $\bQ$ and $\bS$ are reflexive Banach spaces as $\epsilon\to 0$ we can extract weakly convergent subsequences $\{\ubsi_{\epsilon,n}\}^\infty_{n=1}$, $\{\ubu_{\epsilon,n}\}^\infty_{n=1}$, such that $\ubsi_{\epsilon,n} \rightharpoonup \ubsi$ in $\bQ$ and $\ubu_{\epsilon,n}\rightharpoonup \ubu$ in $\bS$, which together with the fact that $\cE_1, \cE_2, \cA, \cB, \cK_{\bw_f}$, $\cL_\bzeta$, and $\cC$ are continuous, we deduce that $(\ubsi,\ubu)$ solve \eqref{eq:T-auxiliary-problem-operator}.
Moreover, proceeding analogously to \eqref{eq:bound-solution-independently-epsilon-1} we derive \eqref{eq:ubsi-ubu-bound-solution}, with $c_\bT$ independent of $\bw_f$ and $\bzeta$.

Finally, we prove that the solution of \eqref{eq:T-auxiliary-problem-operator} is unique. Since \eqref{eq:T-auxiliary-problem-operator} is linear, it is sufficient to prove that the problem with zero data has only the zero solution. 
Taking $(\wh{\bF},\wh{\bG}) = (\0,\0) $ in \eqref{eq:T-auxiliary-problem-operator}, testing it with solution $(\ubsi,\ubu)$ and employing non-negativity and coercivity estimates of $\cE_1, \cA,\cE_2, \cC$ (cf. Lemma \ref{lem:coercivity-properties-A-E2}) and continuity of $\cK_{\bw_f}$ and $\cL_{\bzeta}$ (cf. \eqref{eq:continuity-cK-wf}, \eqref{eq:continuity-cL-zeta}), yields 
\begin{align}
&\frac{1}{2\,\mu}\,\| \bsi^\rd_f\|^2_{\bbL^2(\Omega_f)} 
+ \mu\,k_{\min}\,\|\bu_p\|^2_{\bL^2(\Omega_p)} 
+ C_A\,\|\bsi_e\|^2_{\bSigma_e} 
+ \rho_f \|\bu_f\|^2_{\bL^2(\Omega_f)} 
+ s_0\|p_p\|^2_{\W_p}
+ \rho_p\|\bu_s\|^2_{\bL^2(\Omega_p)}
\nonumber \\
&\quad +\, c_I\,\sum^{n-1}_{j=1} \|( \bvarphi-\bu_s)\cdot\bt_{f,j}\|^2_{\L^2(\Gamma_{fp})}
- \frac{\rho_f\,n^{1/2}}{2\,\mu}\,\|\bw_f\|_{\bV_f}\|\bu_f\|_{\bV_f}\|\bsi^\rd_f\|_{\bbL^2(\Omega_f)}
- C_{\cL}\|\bzeta\|_{\bLambda_f}\|\bvarphi\|^2_{\bLambda_f} \,\leq\, 0 \,,
\label{eq:sol-uniqueness-4.8-1}
\end{align}
where, for the last two terms we proceed as in \eqref{eq:bount-to-coercivity1}, and test 
the first row of \eqref{eq:T-auxiliary-problem-operator} with $\ubtau=(\btau_f, \0, \0)$ in combination with the inf-sup condition of $B_f$ (cf. \eqref{eq:inf-sup-vf-chif}), the continuity of $a_f$ (cf. \eqref{bilinear-form-1}), and the fact that $\|\bw_f\|_{\bV_f}\leq r_1^0$ (cf. \eqref{eq:r_1^0 defn}), such that
\begin{equation}\label{eq:ubu-bound-1}
\frac{5\,\beta}{6}\,\|(\bu_f, \bgamma_f, \bvarphi)\|_{\bV_f\times \bbQ_f\times \bLambda_f}
\,\leq \frac{1}{2\,\mu}\,\| \bsi^\rd_f \|_{\bbL^2(\Omega_f)}\,.
\end{equation}
Thus, bounding the seventh term in \eqref{eq:sol-uniqueness-4.8-1} by zero, \eqref{eq:ubu-bound-1} to bound both 
$\|\bu_f\|_{\bV_f}$ and $\|\bvarphi\|_{\bLambda_f}$ by $\frac{3}{5\mu\beta}\,\| \bsi^\rd_f\|_{\bbL^2(\Omega_f)}$ in \eqref{eq:sol-uniqueness-4.8-1}, and recalling that $\|\bw_f\|_{\bV_f}\leq r_1^0$  (cf. \eqref{eq:r_1^0 defn}) and $\|\bzeta\|_{\bLambda_f} \leq r_2^0$ (cf. \eqref{eq:r20-def}), \eqref{eq:sol-uniqueness-4.8-1}, we deduce that
\begin{align*}
&\| \bsi^\rd_f\|^2_{\bbL^2(\Omega_f)} 
+ \|\bu_p\|^2_{\bL^2(\Omega_p)} 
+ \|\bsi_e\|^2_{\bSigma_e}
+ \|\bu_f\|^2_{\bL^2(\Omega_f)}
+ \|p_p\|^2_{\W_p}
+ \|\bu_s\|^2_{\bL^2(\Omega_p)} \,\leq\, 0\,, %\label{eq:sol-uniqueness-4.8-2}
\end{align*}
and then $\bsi^\rd_f = \0, \bu_p = \0, \bsi_e = \0, \bu_f = \0, p_p = 0$ and $\bu_s = \0$. 
On the other hand, we test the first row of \eqref{eq:T-auxiliary-problem-operator} with $\ubtau=(\btau_f,\bv_p,\0)$, employ the inf-sup conditions of $\cB$ (cf. Lemma \ref{lem:inf-sup-conditions}), and the continuity of $\kappa_{\bw_f}$ (cf. \eqref{eq:continuity-cK-wf}) with $\|\bw_f\|_{\bV_f}\leq r_1^0$ (cf. \eqref{eq:r_1^0 defn}), to obtain
\begin{equation}\label{eq:sol-uniqueness-4.8-3}
\frac{5\beta}{6}\,\|(\bu_f ,p_p, \bgamma_f, \bvarphi, \lambda)\|
\,\leq\, \sup_{\0\neq (\btau_f,\bv_p)\in \bbX_{f}\times \bX_p}  \frac{-a_f(\bsi_f,\btau_f) - a^d_p(\bu_p ,\bv_p)}{\|(\btau_f,\bv_p)\|_{\bbX_f\times \bX_p}} \,=\, 0\,.
\end{equation}
Therefore, $\bgamma_f = \0, \bvarphi = \0,$ and  $\lambda = 0 $. Finally, from the second row in \eqref{eq:T-auxiliary-problem-operator}, we have the identity
\begin{equation*}
b_f(\bsi_f ,\bv_f) 
\, = \,  \rho_f\,(\bu_f,\bv_f)_{\Omega_f} \,=\, 0 \quad \forall\,\bv_f\in \bV_f, 
\end{equation*}
which together with the property $\bdiv(\bbX_f) = (\bV_f)'$ allow us to deduce: $\bdiv(\bsi_f) = \0$, and employing again the inequalities \eqref{eq:tau-d-H0div-inequality} and \eqref{eq:tau-H0div-Xf-inequality}, we conclude that $\bsi_f = \0$, so then we can conclude that \eqref{eq:T-auxiliary-problem-operator} has a unique solution.
\end{proof}

As an immediate consequence we have the following corollary.
%
\begin{cor}\label{eq:T-well-delfined}
Assume that the conditions of Lemma ~\ref{thm:well-posedness-1} are satisfied. The operator $\bT$ defined in \eqref{eq:definition-operator-T} is well defined and it satisfies
\begin{equation}\label{eq:operator T-bound-solution}
\|\bT(\bw_f,\bzeta)\|_{\bV_f\times \bLambda_f} 
\,\leq\, {c_\bT\,\Big\{ \|\wh{\f}_f\|_{\bL^2(\Omega_f)} +  \|\wh{\f}_e\|_{\bbL^2(\Omega_p)}+ \|\wh{\f}_p\|_{\bL^2(\Omega_p)} +  \| \wh{q}_p\|_{\L^2(\Omega_p)} \Big\}}\,.
\end{equation}
\end{cor}

%**********************************************************************
%**********************************************************************

\subsubsection[]{Step 2: Resolvent system \eqref{eq:T-auxiliary-problem-operator-A} is well-posed}\label{sec:domain-D-nonempty}

In this section we proceed analogously to \cite{cgos2017} (see also \cite{cgo2021}) and establish the existence and uniqueness of a fixed-point of operator $\bT$ (cf. \eqref{eq:definition-operator-T}) by means of the well known Banach fixed-point theorem which implies that \eqref{eq:T-auxiliary-problem-operator-A} has unique solution.

\begin{lem}\label{lem:T-contraction-mapping}
Let $r_1\in (0,r_1^0]$ and $r_2\in (0,r_2^0]$, with $r_1^0$ and $r_2^0$ defined in \eqref{eq:r_1^0 defn} and \eqref{eq:r20-def}, respectively. Let $\bW_{r_1,r_2}$ be the closed set defined by
\begin{equation}\label{eq:Wr-definition}
\bW_{r_1,r_2} := \Big\{ (\bw_f,\bzeta)\in \bV_f\times \bLambda_f \,:\quad \| \bw_f\|_{\bV_f} \leq r_1,\quad \|\bzeta\|_{\bLambda_f} \leq r_2 \Big\}\,,
\end{equation}
and set $r_0:=\min\big\{r_1^0, r_2^0\big\}\,.$ Then, for all $(\bw_f,\bzeta), (\wt{\bw}_f,\wt{\bzeta}) \in \bW_{r_1,r_2}$, there holds
\begin{align}
&\|\bT(\bw_f,\bzeta) - \bT(\wt{\bw}_f,\wt{\bzeta}) \|_{\bV_f\times \bLambda_f} \nonumber \\
& \quad \leq\, \frac{c_\bT}{r_0}\,\Big\{ \|\wh{\f}_f\|_{\bL^2(\Omega_f)} +  \|\wh{\f}_e\|_{\bbL^2(\Omega_p)}+ \|\wh{\f}_p\|_{\bL^2(\Omega_p)} + \| \wh{q}_p\|_{\L^2(\Omega_p)} \Big\} \|(\bw_f,\bzeta) - (\wt{\bw}_f,\wt{\bzeta})\|_{\bV_f\times \bLambda_f}.\label{eq:Lipschitz-continuity}
\end{align}
\end{lem}
%
\begin{proof}
Given $(\bw_f,\bzeta), (\wt{\bw}_f,\wt{\bzeta}) \in \bW_{r_1,r_2}$, we let $(\bu_f,\bvarphi) := \bT(\bw_f,\bzeta)$ and $(\wt{\bu}_f,\wt{\bvarphi}) := \bT(\wt{\bw}_f,\wt{\bzeta})$. 
According to the definition of $\bT$ (cf. \eqref{eq:definition-operator-T}) and some algebraic computations, it follows that
\begin{align}
(\cE_1 + \cA)(\ubsi - \wt{\bsi})(\ubtau) + (\cB' + \cK_{\wt{\bw}_f})(\ubu - \wt{\ubu})(\ubtau) &= -\,\cK_{\bw_f - \wt{\bw}_f}(\ubu)(\ubtau) \quad \forall\,\ubtau\in \bQ\,, \nonumber \\[0.5ex]
\ds -\,\cB(\ubsi - \wt{\ubsi})(\ubv) + (\cE_2 + \cC +\cL_{\wt{\bzeta}})(\ubu - \wt{\ubu})(\ubv) &= -\cL_{\bzeta-\wt{\bzeta}}(\ubu )(\ubv) \quad \forall\,\ubv\in \bS\,. \label{eq:subtracting-auxiliary-problemas-T}
\end{align}
Testing with $\ubtau = \ubsi - \wt{\ubsi}$ and $\ubv = \ubu - \wt{\ubu}$ and employing the monotonicity of $a_f, a^d_p$ and $\cE_1$ (cf. \eqref{eq:coercivity-af}, \eqref{eq:coercivity-adp}, \eqref{eq: operator E_1-monotone}) and the continuity of  $\kappa_{\bw_f}$ and $l_{\bzeta}$ (cf. \eqref{eq:continuity-cK-wf}, \eqref{eq:continuity-cL-zeta}), we deduce that
\begin{align}
& \|\bsi^\rd_f - \wt{\bsi}^\rd_f\|^2_{\bbL^2(\Omega_f)}
\,\leq\, \rho_f\,n^{1/2}\,\Big( \|\bu_f\|_{\bV_f}\,\|\bw_f - \wt{\bw}_f\|_{\bV_f} + \|\wt{\bw}_f\|_{\bV_f}\,\|\bu_f - \wt{\bu}_f\|_{\bV_f} \Big)\|\bsi^\rd_f - \wt{\bsi}^\rd_f\|_{\bbL^2(\Omega_f)} \nonumber \\ 
& \qquad  +\, 2\mu C_{\cL}\Big(\|\wt{\bzeta}\|_{\bLambda_f}\|\bvarphi-\wt{\bvarphi}\|^2_{\bLambda_f}+\|\bvarphi\|_{\bLambda_f}\|\bzeta-\wt{\bzeta}\|_{\bLambda_f} \|\bvarphi-\wt{\bvarphi}\|_{\bLambda_f}\Big)\,, \label{eq:cont-bound-1a}
\end{align}
where, using again Young's inequality with $\epsilon = 8\,C_{\cL}/(\mu\beta^2)$ on the last term of right hand side of \eqref{eq:cont-bound-1a}, we obtain
\begin{align}
&\|\bsi^\rd_f - \wt{\bsi}^\rd_f\|^2_{\bbL^2(\Omega_f)}
\,\leq\, \rho_f^2\,n\,\Big(2 \|\bu_f\|^2_{\bV_f}\,\|\bw_f - \wt{\bw}_f\|^2_{\bV_f} + 2\|\wt{\bw}_f\|^2_{\bV_f}\,\|\bu_f - \wt{\bu}_f\|^2_{\bV_f} \Big) \nonumber \\ 
& \qquad +\, 4\mu C_{\cL}\|\wt{\bzeta}\|_{\bLambda_f}\|\bvarphi-\wt{\bvarphi}\|^2_{\bLambda_f}+\frac{16 C_{\cL}^2}{\beta^2}\|\bvarphi\|^2_{\bLambda_f}\|\bzeta-\wt{\bzeta}\|^2_{\bLambda_f} + \frac{\mu^2\beta^2}{4}\|\bvarphi-\wt{\bvarphi}\|^2_{\bLambda_f}.\label{eq:sigfbound-1}
\end{align}
In turn, rearranging the first row of \eqref{eq:subtracting-auxiliary-problemas-T}, we obtain
\begin{equation*}
\ds (\cB' + \cK_{\wt{\bw}_f})(\ubu - \wt{\ubu})(\ubtau) 
\,=\, -\,\cK_{\bw_f - \wt{\bw}_f}\,(\ubu)(\ubtau) - (\cE_1 + \cA)\,(\ubsi - \wt{\ubsi})(\ubtau) \quad \forall\,\ubtau\in \bQ.
\end{equation*}
Testing with $\ubtau=(\btau_f, \0, \0)$ and combining with the inf-sup condition of $B_f$ (cf. \eqref{eq:inf-sup-vf-chif}), and the continuity of $a_f$ and $\cK_{\wt{\bw}_f}$ (cf. \eqref{bilinear-form-1}, \eqref{eq:continuity-cK-wf}) along with the fact that $\|\wt{\bw}_f\|_{\bV_f} \leq r_1$, yields
%\begin{equation}\label{eq:Lipschitz-1}
%\frac{5\beta}{6}(\,\|\bu_f - \wt{\bu}_f\|^2_{\bV_f} +\|\bvarphi-\wt{\bvarphi}\|^2_{\bLambda_f})^{1/2}
%\,\leq\, \frac{\rho_f}{2\,\mu}\,\|\bu_f\|_{\bV_f}\,\|\bw_f - \wt{\bw}_f\|_{\bV_f} + \frac{1}{2\,\mu}\,\|\bsi^\rd_f - \wt{\bsi}^\rd_f\|_{\bbL^2(\Omega_f)}.
%\end{equation}
%Squaring on both the sides of above inequality and using $ (a+b)^2 \leq 2(a^2+b^2)$, \eqref{eq:Lipschitz-1} results in 
\begin{equation}\label{eq:Lipschitz-2}
\frac{25 \mu^2 \beta^2}{9}(\,\|\bu_f - \wt{\bu}_f\|^2_{\bV_f} +\|\bvarphi-\wt{\bvarphi}\|^2_{\bLambda_f})
\,\leq\, 2\,\rho_f^2\,n\,\|\bu_f\|^2_{\bV_f}\,\|\bw_f - \wt{\bw}_f\|^2_{\bV_f} + 2\,\|\bsi^\rd_f - \wt{\bsi}^\rd_f\|^2_{\bbL^2(\Omega_f)}.
\end{equation}
Combining \eqref{eq:sigfbound-1} and \eqref{eq:Lipschitz-2}, and using the bounds $\|\wt{\bw}_f\|_{\bV_f} \leq r_1$ and $\|\wt{\bzeta}\|_{\bLambda_f}\leq r_2$, we get
%\begin{align}
%&\frac{7}{3}\,\|\bu_f - \wt{\bu}_f\|^2_{\bV_f} +\frac{17}{18}\|\bvarphi-\wt{\bvarphi}\|^2_{\bLambda_f}
% \,\leq \frac{2}{3(r_1^0)^2}\|\bu_f\|^2_{\bV_f}\,\|\bw_f - \wt{\bw}_f\|^2_{\bV_f} + \frac{8}{9(r_2^0)^2}\|\bvarphi\|^2_{\bLambda_f}\|\bzeta-\wt{\bzeta}\|^2_{\bLambda_f},\label{eq:Lipschitz-3}
%\end{align}
%implying
\begin{equation*}%\label{eq:Lipschitz-3a}
\,\|\bu_f - \wt{\bu}_f\|^2_{\bV_f} +\|\bvarphi-\wt{\bvarphi}\|^2_{\bLambda_f}
\,\leq \frac{16}{17(r_1^0)^2}\|\bu_f\|^2_{\bV_f}\,\|\bw_f - \wt{\bw}_f\|^2_{\bV_f} + \frac{16}{17(r_2^0)^2}\|\bvarphi\|^2_{\bLambda_f}\|\bzeta-\wt{\bzeta}\|^2_{\bLambda_f}\,,
\end{equation*}
and applying simple algebraic computations, we obtain
\begin{align}
&\|\bT(\bw_f,\bzeta) - \bT(\wt{\bw}_f,\wt{\bzeta}) \|_{\bV_f\times \bLambda_f} \leq \frac{1}{r_0}\|(\bu_f,\bvarphi)\|_{\bV_f\times\bLambda_f}\|(\bw_f-\wt{\bw_f},\bzeta-\wt{\bzeta})\|_{\bV_f\times \bLambda_f}.\label{eq:Lipschitz-4}
\end{align}
Finally, using \eqref{eq:ubsi-ubu-bound-solution} in \eqref{eq:Lipschitz-4}, we obtain \eqref{eq:Lipschitz-continuity} and complete the proof.
%\begin{align}
%&\|\bT(\bw_f,\bzeta) - \bT(\wt{\bw}_f,\wt{\bzeta}) \|_{\bV_f\times \bLambda_f}\nonumber \\
%&\quad \leq \frac{c_\bT}{r_0}\,\Big\{ \|\wh{\f}_f\|_{\bL^2(\Omega_f)} +  \|\wh{\f}_e\|_{\bbL^2(\Omega_p)}+ \|\wh{\f}_p\|_{\bL^2(\Omega_p)} +  \| \wh{q}_p\|_{\L^2(\Omega_p)} \Big\}\|(\bw_f-\wt{\bw_f},\bzeta-\wt{\bzeta})\|_{\bV_f\times \bLambda_f},\label{eq:Lipschitz-5}
%\end{align}
%which completes the proof.
\end{proof}

\medskip

We are now in position of establishing the main result of this section.

\begin{thm}\label{thm:well-posed-domain-D}
Let $\bW_{r_1,r_2}$ be as in \eqref{eq:Wr-definition} and let $r:=\min\big\{r_1, r_2\big\}$. 
Assume that the data satisfy
%
\begin{equation}\label{eq:T-maps-Wr-into-Wr}
{c_\bT\,\Big\{ \|\wh{\f}_f\|_{\bL^2(\Omega_f)} +  \|\wh{\f}_e\|_{\bbL^2(\Omega_p)}+ \|\wh{\f}_p\|_{\bL^2(\Omega_p)} + \| \wh{q}_p\|_{\L^2(\Omega_p)} \Big\}} 
\,\leq\, r\,.
\end{equation}	
Then, the resolvent problem \eqref{eq:T-auxiliary-problem-operator-A} has a unique solution $(\ubsi,\ubu)\in \bQ\times \bS$ with $(\bu_f,\bvarphi)\in \bW_{r_1,r_2}$, and there holds
\begin{equation}\label{eq:bound-solution-steady-state}
\|(\ubsi,\ubu)\|_{\bQ\times \bS} 
\,\leq\, c_\bT\,\Big\{ \|\wh{\f}_f\|_{\bL^2(\Omega_f)} +  \|\wh{\f}_e\|_{\bbL^2(\Omega_p)}+ \|\wh{\f}_p\|_{\bL^2(\Omega_p)} + \|\wh{q}_p\|_{\L^2(\Omega_p)} \Big\} \,.
\end{equation} 
\end{thm}
%
\begin{proof}
Employing \eqref{eq:T-maps-Wr-into-Wr} in \eqref{eq:operator T-bound-solution} implies that the operator $\bT:\bW_{r_1,r_2}\to \bW_{r_1,r_2}$ is well-defined. In particular, for $(\bu_f,\bvarphi) = \bT(\bw_f,\bzeta)$ we have that
%\begin{equation*}
$\|\bu_f\|_{\bV_f} \leq r \leq r_1$ and $\|\bvarphi\|_{\bLambda_f} \leq r \leq r_2$.
%\end{equation*}
In turn, from \eqref{eq:Lipschitz-continuity} and assumption \eqref{eq:T-maps-Wr-into-Wr} we obtain that
\begin{equation*}%\label{eq:T-contraction-bound}
\|\bT(\bw_f,\bzeta) - \bT(\wt{\bw}_f,\wt{\bzeta}) \|_{\bV_f\times \bLambda_f}  \leq \frac{r}{r_0}\|(\bw_f,\bzeta) - (\wt{\bw}_f,\wt{\bzeta})\|_{\bV_f\times \bLambda_f}\,,
\end{equation*}
and then $\bT$ is a contraction mapping.
Therefore, from a direct application of the classical Banach fixed-point theorem we conclude that $\bT$ possesses a unique fixed-point $(\bu_f,\bvarphi)\in \bW_{r_1,r_2}$, or equivalently, problem \eqref{eq:T-auxiliary-problem-operator-A} is well-posed. In addition, \eqref{eq:bound-solution-steady-state} follows directly from \eqref{eq:ubsi-ubu-bound-solution}.
\end{proof}

%**********************************************************************
%**********************************************************************

\subsubsection[]{Step 3: Well posedness of the alternative formulation \eqref{eq:alternative-formulation-operator-form}} \label{sec:wellposedness-alternative-formulation}

In this section we establish the existence of a solution to \eqref{eq:alternative-formulation-operator-form}.
We begin by showing that $\cM$ defined by \eqref{eq:defn-E-N-M} is a monotone operator.

\begin{lem}\label{lem:M-monotone-operator}
Let $(\bu_f,\bvarphi)\in \bW_{r_1,r_2}$ (cf. \eqref{eq:Wr-definition}). Then, the operator $\cM$ defined by \eqref{eq:defn-E-N-M} is monotone.
\end{lem}
%
\begin{proof}
First, for each $u^i \in \cD$, $i\in \{1,2\}$, and using the definition of $\cM$ (cf. \eqref{eq:defn-E-N-M}), we have
\begin{align*}
&(\cM(u^1)-\cM(u^2))(u^1-u^2) =  \,  a_f(\bsi^1_f-\bsi^2_f,\bsi^1_f-\bsi^2_f) +  a^d_p(\bu_p^1-\bu_p^2,\bu_p^1-\bu_p^2) \nonumber\\ 
& \quad +\, \kappa_{\bu_f^1}(\bu_f^1, \bsi^1_f-\bsi^2_f)
- \kappa_{\bu_f^2}(\bu_f^2, \bsi^1_f-\bsi^2_f) + c_{\BJS}(\bu_s^1-\bu_s^2,\bvarphi^1-\bvarphi^2;\bu_s^1-\bu_s^2,\bvarphi^1-\bvarphi^2) \nonumber\\ 
& \quad +\, l_{\bvarphi^1}(\bvarphi^1,\bvarphi^1-\bvarphi^2)
- l_{\bvarphi^2}(\bvarphi^2,\bvarphi^1-\bvarphi^2)\,,
\end{align*}
which, together with the non-negativity bounds of $a_f, a^d_p$ and $c_\BJS$ (cf. \eqref{eq:coercivity-af}, \eqref{eq:coercivity-adp}, \eqref{eq:positivity-aBJS}), 
%\begin{align*}
%&(\cM(u^1)-\cM(u^2))(u^1-u^2) \geq \, \frac{1}{2 \mu} \|(\bsi^1_f - \bsi^2_f)^\rd\|^2_{\bbL^2(\Omega_f)} + \mu k_{\min} \|\bu_p^1 - \bu_p^2\|^2_{\bL^2(\Omega_p)} \nonumber\\ 
%& \quad+ c_I \sum^{n-1}_{j=1} \big\|(\bu_s^1 - \bu_s^2){\cdot}\bt_{f,j} + (\bvarphi^1 - \bvarphi^2){\cdot} \bt_{f,j}\big\|^2_{\L^2(\Gamma_{fp})}-C_{\cL}\big(\|\bvarphi^1\|_{\bLambda_f}+\|\bvarphi^2\|_{\bLambda_f}\big)\|\bvarphi^1-\bvarphi^2\|^2_{\bLambda_f} \nonumber\\ 
%& \quad-\,\,\frac{\rho_f}{2\,\mu}\,\big(\|\bu_f^1\|_{\bV_f} + \|\bu_f^2\|_{\bV_f}\big)\,\|\bu_f^1 - \bu_f^2\|_{\bV_f} \|(\bsi^1_f - \bsi^2_f)^\rd\|_{\bbL^2(\Omega_f)} \,,
%\end{align*}
the continuity of $\kappa_{\bw_f}$ and $l_{\bzeta}$ (cf. \eqref{eq:continuity-cK-wf}, \eqref{eq:continuity-cL-zeta}), and some algebraic computations, yields
\begin{align}
&(\cM(u^1)-\cM(u^2))(u^1-u^2)\geq \, \frac{1}{2 \mu} \|(\bsi^1_f -\bsi^2_f)^\rd\|^2_{\bbL^2(\Omega_f)}-C_{\cL}\big(\|\bvarphi^1\|_{\bLambda_f}+\|\bvarphi^2\|_{\bLambda_f}\big)\|\bvarphi^1-\bvarphi^2\|^2_{\bLambda_f} \nonumber\\ 
& \quad\,-\,\frac{\rho_f\,n^{1/2}}{2\,\mu}\,\big(\|\bu_f^1\|_{\bV_f} + \|\bu_f^2\|_{\bV_f}\big)\,\|\bu_f^1 - \bu_f^2\|_{\bV_f} \|(\bsi^1_f - \bsi^2_f)^\rd\|_{\bbL^2(\Omega_f)}\,.
\label{eq:mono f bound1}
\end{align}
%Consider \eqref{eq:continuous-weak-formulation-1a},
%\begin{equation}\label{eq:continuous-weak-formulation-1i} 
%- \pil\btau_f\bn_f,\bvarphi \pir_{\Gamma_{fp}} + (\bu_f,\bdiv(\btau_f))_{\Omega_f} + (\bgamma_f,\btau_f)_{\Omega_f} 
%+ \frac{\rho_f}{2\,\mu}\,((\bu_f\otimes\bu_f)^\rd,\btau_f)_{\Omega_f} = -\frac{1}{2\,\mu}\,(\bsi^\rd_f,\btau^\rd_f)_{\Omega_f}, \forall \btau_f \in  \bbX_f.
%\end{equation}
%For $(\bu_f^i, \bgamma_f^i, \bvarphi^i)\in \bV_f\times \bbQ_f\times \bLambda_f$ and $\bsi^i_f\in \bbX_f$  with $i\in \{1,2\}$ in \eqref{eq:continuous-weak-formulation-1i} and subtracting and using the inf-sup condition of $\cB$ (cf. Lemma ~\ref{lem:inf-sup-conditions}) and the continuity of $a_f(\bsi_f,\btau_f), \kappa_{\bw_f}(\bu_f,\btau_f)$ (cf. \eqref{eq:continuity-cA-cB-cC}, \eqref{eq:continuity-cK-wf}) as above, we get
In turn, proceeding as in \eqref{eq:bount-to-coercivity1}, that is, using the inf-sup condition of $B_f$ (cf. \eqref{eq:inf-sup-vf-chif}) and the continuity of $a_f, \kappa_{\bw_f}$ (cf. \eqref{bilinear-form-1}, \eqref{eq:continuity-cK-wf}), we are able to deduce that 
\begin{equation*}%\label{eq:inf-sup-B1}
\big(\beta - \frac{\rho_f\,n^{1/2}}{2\,\mu}(\|\bu^1_f\|_{\bV_f}+\|\bu^2_f\|_{\bV_f})\big)\,\|(\bu_f^1, \bgamma_f^1, \bvarphi^1)-(\bu_f^2, \bgamma_f^2, \bvarphi^2) \|_{\bV_f\times \bbQ_f\times \bLambda_f} \,\leq\, \frac{1}{2\,\mu} \|(\bsi^1_f - \bsi^2_f)^\rd\|_{\bbL^2(\Omega_f)}\,,
\end{equation*}
and recalling that both $\|\bu_f^1\|_{\bV_f} $ and $\|\bu^2_f\|_{\bV_f}$ are bounded by $r_1$, where $r_1\in (0,r_1^0]$ (cf. \eqref{eq:r_1^0 defn}), we obtain that both $\|\bu_f^1 - \bu_f^2\|_{\bV_f}$ and $\|\bvarphi^1-\bvarphi^2\|_{\bLambda_f}$ are bounded by $\frac{3}{4\beta\,\mu} \|(\bsi^1_f - \bsi^2_f)^\rd\|_{\bbL^2(\Omega_f)}$, and using again the estimates $\|\bu_f^1\|_{\bV_f}, \|\bu_f^2\|_{\bV_f} \leq r_1$, and $\|\bvarphi^1\|_{\bLambda_f}, \|\bvarphi^2\|_{\bLambda_f} \leq r_2$, where $r_2\in (0,r_2^0]$ (cf. \eqref{eq:r20-def}), give
\begin{equation}\label{eq:mono u bound2}
(\cM(u^1)-\cM(u^2))(u^1-u^2)
\,\geq \frac{1}{16\mu}\|(\bsi^1_f - \bsi^2_f)^\rd\|^2_{\bbL^2(\Omega_f)}\geq 0\,,
\end{equation}
which implies the monotonicity of $\cM$.
\end{proof}


Now, we establish a suitable initial condition result, which is necessary to apply Theorem \ref{thm:auxiliary-theorem} to the
context of \eqref{eq:alternative-formulation-operator-form}.
%
\begin{lem}\label{lem:sol0-in-M-operator}
Assume the initial data $(\bu_{f,0},p_{p,0},\bbeta_{p,0},\bu_{s,0}) \in \bH$, where
%
\begin{subequations}\label{eq:H-space-initial-condition}
\begin{align}
  & \ds \bH := \Big\{ (\bv_f,w_p,\bxi_p,\bv_s)\in \bH^1(\Omega_f) \times \H^1(\Omega_p) \times \bV_p\times \bV_p  \,:
  \nonumber \\[1ex] 
& \ds \qquad (\bv_f,\bv_f|_{\Gamma_{fp}}) \in \bW_{r_1,r_2},\quad 
\bdiv(\be(\bv_f)) \in \bL^2(\Omega_f),\quad 
\bdiv(\bv_f\otimes \bv_f)\in \bL^2(\Omega_f), \nonumber \\[1ex]
& \ds \qquad \div(\bv_f) = 0\,\mbox{ in }\, \Omega_f, \quad \big( 2\mu\be(\bv_f) - \rho_f(\bv_f\otimes\bv_f) \big)\bn_f = \0 \,\mbox{ on }\, \Gamma^{\rN}_f,\quad 
\bv_f = \0 \qon \Gamma^{\rD}_f, \label{bdycond-1}\\[1ex]
& \ds\qquad \bK\,\nabla\,w_p\in \bH^1(\Omega_p),\quad \bK\,\nabla\,w_p\cdot\bn_p = 0 \,\mbox{ on }\, \Gamma^{\rN}_p,\quad w_p = 0 \,\mbox{ on }\, \Gamma^{\rD}_p, \label{bdycond-2}\\[1ex]
& \ds \qquad  \ds \bv_f\cdot\bn_f + \left(\bv_s -\frac{1}{\mu}  \bK\,\nabla\,w_p\right)\cdot\bn_p \,=\, 0 \mbox{ on } \Gamma_{fp}, \label{bdycond-3} \\[1ex]
& \ds\qquad  2\mu\,\be(\bv_f)\bn_f + (A^{-1}(\, \be(\bxi_{p})) - \alpha_p\,w_{p}\,\bI)\bn_p \,=\, 0 \mbox{ on } \Gamma_{fp},\label{bdycond-4}\\
& \ds \qquad  2\mu\,\be(\bv_f)\bn_f+ \mu\,\alpha_{\BJS}\sum^{n-1}_{j=1}\,\sqrt{\bK^{-1}_j}\left\{\left(\bv_f - \bv_s\right)\cdot\bt_{f,j}\right\}\,\bt_{f,j} \,=\, -\,w_p\bn_f \mbox{ on } \Gamma_{fp} \label{bdycond-5}\Big\}.
\end{align}
\end{subequations}
Furthermore, assume that there holds
\begin{align}
&\|\bdiv( \be(\bu_{f,0}))\|_{\bL^{2}(\Omega_f)} 
+ \|\bdiv(\bu_{f,0}\otimes \bu_{f,0})\|_{\bL^{2}(\Omega_f)}
+ \|\bu_{s,0}\|_{\bH^1(\Omega_p)}  \nonumber \\
&\qquad +\, \|\bdiv(A^{-1}(\, \be(\bbeta_{p,0})))\|_{\bL^2(\Omega_p)}
+ \|p_{p,0}\|_{\H^1(\Omega_p)} 
+ \|\div(\bK\nabla p_{p,0})\|_{\L^2(\Omega_p)} 
\,<\, \frac{1}{C_0}\frac{r}{c_\bT} \,, \label{eq:extra-assumption}
\end{align}
with $C_0$ satisfying \eqref{eq: initial data bound} and
for $r\in (0,r_0)$, with $r_0$ and $c_\bT$ defined in Lemmas \ref{lem:T-contraction-mapping} and \ref{thm:well-posedness-1}, respectively. Then, there exists $\ubsi_0 := (\bsi_{f,0}, \bu_{p,0},\bsi_{e,0})\in \bQ$ and   $\ubu_0 := (\bu_{f,0},p_{p,0},\bgamma_{f,0},\bu_{s,0},\bvarphi_0, \lambda_0)\in \bS$  with $(\bu_{f,0},\bvarphi_0)\in \bW_{r_1,r_2}$ (cf. \eqref{eq:Wr-definition}) such that 
\begin{align}
\cA(\ubsi_0) + (\cB' + \cK_{\bu_{f,0}})(\ubu_0) &=  \wh{\bF}_0 \quad \mbox{ in }\quad  \bQ', \nonumber\\ 
-\,\cB(\ubsi_0) + ( \cC +\cL_{\bvarphi_0})(\ubu_0) &=  \wh{\bG}_0  \quad \mbox{ in } \quad \bS', \label{eq:initial-data-system}
\end{align}
where $\wh{\bF}_0(\ubtau) \,:=\, (\wh{\f}_{e,0},\btau_e)_{\Omega_p}$ and 
$\wh{\bG}_0(\ubv) \,:=\, (\wh{\f}_{f,0},\bv_f)_{\Omega_f} + (\wh{q}_{p,0},w_p)_{\Omega_p} + (\wh{\f}_{p,0},\bv_s)_{\Omega_p}\,\,  \forall  (\ubtau,\ubv)\in \bQ\times \bS$,
with some $(\wh{\f}_{e,0}, \wh{\f}_{f,0}, \wh{q}_{p,0},\wh{\f}_{p,0})\in \bE'_b$ satisfying
%
\begin{equation}\label{eq:initial-data-bound-1}
{c_\bT\,\Big\{ \|\wh{\f}_{f,0}\|_{\bL^2(\Omega_f)} +  \|\wh{\f}_{e,0}\|_{\bbL^2(\Omega_p)}+ \|\wh{\f}_{p,0}\|_{\bL^2(\Omega_p)} + \|\wh{q}_{p,0}\|_{\L^2(\Omega_p)} \Big\}} 
\,\leq\, r \,.
\end{equation}
\end{lem}
%
\begin{proof}
Starting with the given data $(\bu_{f,0},p_{p,0},\bu_{s,0},\bbeta_{p,0})\in \bH$, we take a sequence of steps associated with individual subproblems in order to obtain complete initial data that satisfy the coupled problem \eqref{eq:initial-data-system}. 

\noindent 1. Define $\bu_{p,0} := -\dfrac{1}{\mu}\,\bK\nabla p_{p,0}$, it follows that
%
\begin{equation}\label{eq:sol0-up0-pp0}
\mu\,\bK^{-1}\bu_{p,0} = -\nabla p_{p,0},\quad 
\div(\bu_{p,0}) = -\frac{1}{\mu}\,\div(\bK\nabla p_{p,0}) \qin \Omega_p,\quad
\bu_{p,0}\cdot\bn_p = 0 \qon \Gamma^{\rN}_p.
\end{equation}
%
Defining $\lambda_0 := p_{p,0}|_{\Gamma_{fp}}\in \Lambda_p$, integrating by parts the first equation in \eqref{eq:sol0-up0-pp0}, employing \eqref{bdycond-2}, and imposing in a weak sense the second equation of \eqref{eq:sol0-up0-pp0}, we obtain
\begin{equation}\label{eq:system-sol0-1}
\begin{array}{ll}
a^d_p(\bu_{p,0},\bv_p) + b_p(\bv_p,p_{p,0}) + b_{\bn_p}(\bv_p,\lambda_0) = 0 & \forall\,\bv_p\in \bV_p\,, \\[2ex]
-b_p(\bu_{p,0},w_p) = \ds -\frac{1}{\mu}\,(\div(\bK\nabla p_{p,0}),w_p)_{\Omega_p} & \forall\,w_p\in \W_p\,.
\end{array} 
\end{equation}

\noindent 2. Defining $\bsi_{f,0} := 2\mu\be(\bu_{f,0}) - \rho_f(\bu_{f,0}\otimes\bu_{f,0})$, $\bgamma_{f,0} := \dfrac{1}{2}\left( \nabla\bu_{f,0} - (\nabla\bu_{f,0})^\rt \right)$, and $\bvarphi_0 := \bu_{f,0}|_{\Gamma_{fp}}$, it follows that $\bsi_{f,0}\in \bbX_f$, and
\begin{align}
& \frac{1}{2\mu}\bsi^\rd_{f,0} = \nabla\bu_{f,0} - \bgamma_{f,0} - \frac{\rho_f}{2\mu}(\bu_{f,0}\otimes\bu_{f,0})^\rd\,,\quad
-\bdiv(\bsi_{f,0}) = -\bdiv(2\mu\be(\bu_{f,0}) - \rho_f(\bu_{f,0}\otimes \bu_{f,0}))\,. \label{eq:sol0-sigmaf0-uf0-gammaf0}
\end{align}
Then, integrating by parts the first equation in \eqref{eq:sol0-sigmaf0-uf0-gammaf0}, employing  \eqref{bdycond-1}, and imposing in a weak sense the second equation of \eqref{eq:sol0-sigmaf0-uf0-gammaf0}, we get
\begin{equation}\label{eq:system-sol0-2}
\begin{array}{ll}
a_f(\bsi_{f,0},\btau_f) + b_f(\btau_f,\bu_{f,0}) + b_{\sk}(\btau_f,\bgamma_{f,0}) + b_{\bn_f}(\btau_f,\bvarphi_0) + \kappa_{\bu_{f,0}}(\bu_{f,0},\btau_f) = 0 & \forall\,\btau_f\in \bbX_f\,, \\[2ex]
- b_f(\bsi_{f,0},\bv_f) = - (\bdiv(2\mu\be(\bu_{f,0}) - \rho_f(\bu_{f,0}\otimes\bu_{f,0})),\bv_f)_{\Omega_f} & \forall\,\bv_f\in \bV_f\,, \\[2ex]
-\,b_{\sk}(\bsi_{f,0},\bchi_f) = 0 & \forall\,\bchi_f\in \bbQ_f\,.
\end{array} 
\end{equation}

\noindent 3. From \eqref{bdycond-3} and \eqref{bdycond-5}, and using the data $\bu_{p,0}$, $\lambda_{0}$, $\bsi_{f,0}$, and $\bvarphi_0$ satisfying previous steps, we have
%
\begin{align*}
& \bu_{s,0}\cdot\bn_p \,=\, -\bvarphi_0\cdot\bn_f - \bu_{p,0}\cdot\bn_p \qon \Gamma_{fp}\,, \nonumber\\ 
& \mu\,\alpha_{\BJS}\sum^{n-1}_{j=1} \sqrt{\bK^{-1}_j}(\bu_{s,0}\cdot\bt_{f,j})\bt_{f,j} \nonumber\\ 
& \qquad
= \bsi_{f,0}\bn_f + \mu\alpha_{\BJS}\sum^{n-1}_{j=1} \sqrt{\bK^{-1}_j}(\bvarphi_0\cdot\bt_{f,j})\bt_{f,j}+ \rho_f(\bvarphi_0\otimes \bvarphi_0)\bn_f
+ p_{p,0}\bn_f \qon \Gamma_{fp}\,,
\end{align*}
%
which imply
\begin{equation}\label{eq:system-sol0-3}
\begin{array}{ll}
c_{\Gamma}(\bu_{s,0},\bvarphi_0;\xi) - b_{\bn_p}(\bu_{p,0},\xi) = 0 & \forall\,\xi\in \Lambda_p\,, \\[2ex]
c_{\BJS}(\bu_{s,0},\bvarphi_0;\0,\bpsi) - c_{\Gamma}(\0,\bpsi;\lambda_0) - b_{\bn_f}(\bsi_{f,0},\bpsi) +l_{\bvarphi_0}(\bvarphi_0,\bpsi)= 0 & \forall\,\bpsi\in \bLambda_{f} \,.
\end{array}
\end{equation}

\noindent 4. Define $\bsi_{e,0} \in \bSigma_e$ such that
\begin{equation}\label{eq:system-sol0-4}
\bsi_{e,0} := A^{-1}(\be(\bbeta_{p,0}))\,.
\end{equation}
%
It follows that $- \bdiv(\bsi_{e,0} - \alpha_p\,p_{p,0}\,\bI)
= - \bdiv(A^{-1}(\be(\bbeta_{p,0})) - \alpha_p\,p_{p,0}\,\bI)$.
Thus, multiplying this expression by $\bv_s \in \bV_p$, integrating by parts, and employing \eqref{bdycond-4} and \eqref{bdycond-5}, we obtain
\begin{equation}\label{eq:system-sol0-4a}
-b_s(\bsi_{e,0},\bv_s)  + c_{\BJS}(\bu_{s,0},\bvarphi_0;\bv_s,\0) - c_{\Gamma}(\bv_s,\0;\lambda_0) + \alpha_p\,b_p(\bv_s,p_{p,0}) = -(\bdiv(A^{-1}(\be(\bbeta_{p,0})) - \alpha_p\,p_{p,0}\,\bI),\bv_s)_{\Omega_p}\,.
\end{equation}

Now, using \eqref{bdycond-1}, \eqref{eq:system-sol0-1}, \eqref{eq:system-sol0-2}, \eqref{eq:system-sol0-3}, and \eqref{eq:system-sol0-4a}, we find
$(\bsi_{f,0}, \bu_{p,0}, \bsi_{e,0})\in \bQ$ and  $(\bu_{f,0},p_{p,0},\bgamma_{f,0},\newline\bu_{s,0},\bvarphi_0,\lambda_0)\in \bS$ satisfying \eqref{eq:initial-data-system} with data $(\wh{\f}_{f,0},\wh{\f}_{e,0},\wh{q}_{p,0},\wh{\f}_{p,0})\in 
\bL^2(\Omega_f)\times\bbL^2(\Omega_p)\times \L^2(\Omega_p) \times \bL^2(\Omega_p)$ defined as
%
\begin{align*}
& \wh{\f}_{f,0} := -\,\bdiv(2\mu\be(\bu_{f,0}) - \rho_f(\bu_{f,0}\otimes\bu_{f,0})) \,,\quad 
{\f}_{e,0} := -\be(\bu_{s,0}) \,,\\[0.5ex]
& \wh{\f}_{p,0} := -\,\bdiv(A^{-1}(\be(\bbeta_{p,0})) - \alpha_p\,p_{p,0}\,\bI) \,,\quad 
\wh{q}_{p,0} := \alpha_p\,\div(\bu_{s,0}) - \frac{1}{\mu}\,\div(\bK\nabla p_{p,0}) \,,
\end{align*}
%
which, satisfy that there exists $C_0 > 0$ such that
\begin{align}
&\|\wh{\f}_{f,0}\|_{\bL^2(\Omega_f)} + \|\wh{\f}_{e,0}\|_{\bbL^2(\Omega_p)} +  \|\wh{\f}_{p,0}\|_{\bL^2(\Omega_p)} + \|\wh{q}_{p,0}\|_{\L^2(\Omega_p)} \nonumber\\ 
&\quad \leq C_0 \Big(\|\bdiv(\be(\bu_{f,0}))\|_{\bL^{2}(\Omega_f)} 
+ \|\bdiv(\bu_{f,0}\otimes \bu_{f,0})\|_{\bL^2(\Omega_f)}
+ \|\bu_{s,0}\|_{\bH^1(\Omega_p)} \nonumber\\ 
&\qquad +\, \|\bdiv(A^{-1}(\be(\bbeta_{p,0})))\|_{\bbL^2(\Omega_p)} 
+ \|p_{p,0}\|_{\H^1(\Omega_p)}+\|\div(\bK\nabla p_{p,0})\|_{\L^2(\Omega_p)}\Big)\,, \label{eq: initial data bound}
\end{align}
and using the data assumption \eqref{eq:extra-assumption}, we deduce that \eqref{eq:initial-data-bound-1} holds,
completing the proof.
\end{proof}

We are in position to establish the well-posedness of the alternative formulation \eqref{eq:alternative-formulation-operator-form}.

\begin{lem}\label{lem:parabolic-solution}
For each 
$\f_f\in \W^{1,1}(0,T;\bL^2(\Omega_f))$, $q_p\in \W^{1,1}(0,T;\L^2(\Omega_p))$, and $\f_p\in \W^{1,1}(0,T;\bL^2(\Omega_p))$ satisfying for all $t \in [0,T]$
%
\begin{equation}\label{small-data}
\|\f_f(t)\|_{\bL^2(\Omega_f)} + \|\f_p(t)\|_{\bL^2(\Omega_p)} + \|q_p(t)\|_{\L^2(\Omega_p)} 
< \frac{r}{c_\bT}\,,
\end{equation}
%
and initial data $(\bu_{f,0},p_{p,0},\bbeta_{p,0},\bu_{s,0})$
satisfying the assumptions of Lemma \ref{lem:sol0-in-M-operator}, there exists a solution of \eqref{eq:alternative-formulation-operator-form}, $(\ubsi,\ubu): [0,T] \to \bQ\times\bS$ with $(\bu_f(t),\bvarphi(t))\in \bW_{r_1,r_2}$ (cf. \eqref{eq:Wr-definition}),
%
\begin{equation*}
(\bsi_e, \bu_f, p_p,\bu_s)\in \W^{1,\infty}(0,T;\bSigma_e)\times \W^{1,\infty}(0,T;\bL^2(\Omega_f))\times \W^{1,\infty}(0,T;\W_p)\times\W^{1,\infty}(0,T; \bL^2(\Omega_p))\,,
\end{equation*}
%
and $(\bsi_e(0),\bu_f(0),p_p(0),\bu_s(0)) = (\bsi_{e,0},\bu_{f,0},p_{p,0},\bu_{s,0})$, where $\bsi_{e,0}$ is constructed in Lemma \ref{lem:sol0-in-M-operator}.
\end{lem}
%
\begin{proof}
We recall that \eqref{eq:alternative-formulation-operator-form} fits in the framework of Theorem \ref{thm:auxiliary-theorem} with $E$, $u$, $\cN$, and $\cM$ defined in \eqref{eq:defn-E-N-M} and $E'_b$ defined in \eqref{eq:defn-E'_b-D}. In addition, we define the domain 
%
$
\cD := \{u \in E: (\bu_f,\bvarphi) \in \bW_{r_1,r_2}\}.
$
%
Next, we will apply Theorem \ref{thm:auxiliary-theorem} with the restricted range space 
%
$
\wt{E}'_b := \{(\0,\0,\wh{\f}_e,\wh{\f}_f,\wh{q}_p,\0,\wh{\f}_p,\0,0) \in E'_b: \mbox{\eqref{eq:T-maps-Wr-into-Wr} holds}\}.
$
%
Indeed, due to assumption \eqref{small-data}, in problem \eqref{eq:alternative-formulation-operator-form} we have $\cF = (\0,\0,\0,\f_f,q_p,\0,\f_p,\0,0) \in \wt{E}'_b$. 
From the definition of operators $\cE_1$ and $\cE_2$ (cf. \eqref{defn-E1-E2}), it is clear that $\cN$ is linear, symmetric and monotone. 
Furthermore, from Lemma \ref{lem:M-monotone-operator}, we obtain that $\cM$ is monotone in the domain $\cD$. 
The range condition $Rg(\cN+\cM) = \wt{E}'_b$ is established in Theorem \ref{thm:well-posed-domain-D}, where we show that for each $\wh{\cF} \in \wt{E}'_b$ there exists $\wh{u}\in \cD$ solving \eqref{eq:T-auxiliary-problem-operator-A}. 
Finally, initial data $u_0 \in \cD$ with $\cM(u_0) \in \wt{E}'_b$ is constructed in Lemma \ref{lem:sol0-in-M-operator}. 
The statement of the theorem now follows by applying Theorem \ref{thm:auxiliary-theorem} in our context.
\end{proof}

%**********************************************************************
%**********************************************************************

\subsection{Existence and uniqueness of solution of the original formulation}

In this section we discuss how the well-posedness of the original formulation \eqref{eq:continuous-weak-formulation-1} follows from the existence of a solution of the alternative formulation \eqref{eq:NS-Biot-formulation-2} (cf. \eqref{eq:alternative-formulation-operator-form}).
Recall that $\bu_s$ is the structure velocity, so the displacement solution can be recovered from
\begin{equation}\label{eq:etap-us-relation}
\bbeta_p(t) = \bbeta_{p,0} + \int^t_0 \bu_s(s)\,ds \quad \forall\, t\in (0,T],
\end{equation}
%Since $\bu_s(t)\in \L^\infty(0,T;\bV_p)$, then $\bbeta_p(t)\in \W^{1,\infty}(0,T;\bV_p)$ for any $\bbeta_{p,0}\in \bV_p$.
and then, by construction, $\bu_s = \partial_t\,\bbeta_p$ and $\bbeta_p(0) = \bbeta_{p,0}$.

We note that $a^e_p(\cdot,\cdot)$ satisfies the bounds, for some $c_e, C_e > 0$, for all $\bbeta_p, \bxi_p\in \bV_p$,
\begin{equation}\label{eq:aep-coercivity-bound}
c_e\,\|\bxi_p\|^2_{\bV_p} \,\leq\, a^e_p(\bxi_p, \bxi_p),\quad
a^e_p(\bbeta_p, \bxi_p) \,\leq\, C_e\,\|\bbeta_p\|_{\bV_p}\,\|\bxi_p\|_{\bV_p},
\end{equation}
where the coercivity bound follows from Korn's inequality.
We now state the aforementioned result.
%
\begin{thm}\label{thm:unique soln}
For each $(\bu_f(0), p_p(0), \bbeta_p(0), \partial_t\bbeta_p(0)) = (\bu_{f,0}, p_{p,0}, \bbeta_{p,0}, \bu_{s,0})\in \bV_f\times \W_p\times \bV_p\times \bV_p$, where $\bu_{f,0}, p_{p,0}, \bbeta_{p,0}$ and $\bu_{s,0}$ are compatible initial data satisfying Lemma \ref{lem:sol0-in-M-operator} (cf. \eqref{eq:extra-assumption}), and for each
\begin{equation*}
\f_f\in \W^{1,1}(0,T;\bL^2(\Omega_f)),\quad \f_p\in \W^{1,1}(0,T;\bL^2(\Omega_p)),\quad q_p\in \W^{1,1}(0,T;\L^2(\Omega_p))
\end{equation*}
under the assumptions of Lemma \ref{lem:parabolic-solution} (cf. \eqref{small-data}), there exists a unique solution of \eqref{eq:continuous-weak-formulation-1}$, (\bsi_f,\bu_p, \bbeta_p,$
$\bu_f, p_p, \bgamma_f, \bvarphi, \lambda): [0,T]\to \bbX_f\times \bX_p\times\bV_p\times\bV_f\times \W_p\times \bbQ_f\times \bLambda_f\times \Lambda_p$ with $(\bu_f(t),\bvarphi(t)):[0,T]\to \bW_{r_1,r_2}$. In addition, $\bsi_f(0) = \bsi_{f,0}, \bu_p(0) = \bu_{p,0},  \bgamma_f(0) = \bgamma_{f,0}, \bvarphi(0) = \bvarphi_{0}$ and $\lambda(0) = \lambda_{0}.$
\end{thm}
%
\begin{proof}
We begin by using the existence of a solution of the alternative formulation \eqref{eq:NS-Biot-formulation-2} to establish solvability of the original formulation \eqref{eq:continuous-weak-formulation-1}.
Let $(\bsi_f, \bu_p, \bsi_e, \bu_f, p_p, \bgamma_f, \bu_s, \bvarphi, \lambda)$ be a solution to \eqref{eq:NS-Biot-formulation-2}.
Let $\bbeta_p$ be defined in \eqref{eq:etap-us-relation}, so $\bu_s = \partial_t\,\bbeta_p$.
Then \eqref{eq:NS-Biot-formulation-2} with $\btau_e = \0$ and $\bv_s = \0$ implies \eqref{eq:continuous-weak-formulation-1} with $\bxi_p = \0$.
In turn, testing the first equation in \eqref{eq:NS-Biot-formulation-2} with $\btau_e\in \bSigma_e$ gives $\big(\partial_t\,(A(\bsi_e) - \be(\bbeta_p)), \btau_e\big)_{\Omega_p} = 0$, which, using that $\be(\bX_p) \subset \bSigma_e$, implies that $\partial_t\,(A(\bsi_e) - \be(\bbeta_p)) = \0$.
Integrating from $0$ to $t\in (0,T]$ and using that $\bsi_e(0) = A^{-1}(\be(\bbeta_p(0)))$ implies that $\bsi_e(t) = A^{-1}(\be(\bbeta_p(t)))$.
Therefore, with \eqref{eq:elasticity-stress-isotropic}, we deduce
\begin{equation*}
b_s(\bsi_e,\bv_s) \,=\, -\,(\bsi_e, \be(\bv_s))_{\Omega_p} 
\,=\, -\,(A^{-1}(\be(\bbeta_p)),\be(\bv_s))_{\Omega_p} 
\,=\, -\,a^e_p(\bbeta_p, \bv_s),
\end{equation*} 
and then replacing back into the second equation in \eqref{eq:NS-Biot-formulation-2} with $\bv_s\in \bV_p$, yields
\begin{equation*}
 \rho_p\,(\partial_{tt}\bbeta_p,\bv_s)_{\Omega_p}+\,\, c_{\BJS}(\partial_{t}\bbeta_p,\bvarphi;\bv_s,\0)- c_{\Gamma}(\bv_s,\0;\lambda) +a^e_p(\bbeta_p, \bv_s) + \alpha_p\,b_p(p_p, \bv_s) \,=\, (\f_p, \bv_s)_{\Omega_p}.
\end{equation*}
Therefore \eqref{eq:NS-Biot-formulation-2} implies \eqref{eq:continuous-weak-formulation-1}, which establishes that $(\bsi_f, \bu_p, \bbeta_{p,0} + \int^t_0 \bu_s(s)\,ds, \bu_f, p_p, \bgamma_f, \bvarphi, \lambda)$ is a solution to \eqref{eq:continuous-weak-formulation-1}.

Now, assume that the solution of \eqref{eq:continuous-weak-formulation-1} is not unique.
Let $(\bsi^i_f, \bu^i_p, \bbeta^i_p, \bu^i_f, p^i_p, \bgamma^i_f, \bvarphi^i, \lambda^i)$, with $i\in \{1,2\}$, be two solutions corresponding to the same data. Taking \eqref{eq:continuous-weak-formulation-1} with $\btau_f = \bsi^1_f - \bsi^2_f, \bv_p = \bu^1_p - \bu^2_p, \bxi_p = \partial_t\,\bbeta^1_p - \partial_t\,\bbeta^2_p, \bv_f = \bu^1_f - \bu^2_f, w_p = p^1_p - p^2_p, \bchi_f = \bgamma^1_f - \bgamma^2_f, \bpsi = \bvarphi^1 - \bvarphi^2$ and $\xi = \lambda^1 - \lambda^2$, making use of the inf-sup condition of $B_f$ (cf. \eqref{eq:inf-sup-vf-chif}), the continuity of  $\kappa_{\bw_f}$ and $l_{\bzeta}$ (cf. \eqref{eq:continuity-cK-wf}, \eqref{eq:continuity-cL-zeta}), the fact that $(\bu^i_f(t),\bvarphi^i(t))\in \bW_{r_1,r_2}$ (cf. \eqref{eq:Wr-definition}) and \eqref{eq:aep-coercivity-bound}, 
%\begin{equation*}
%\begin{array}{l}
%\ds \frac{1}{2}\,\partial_t\,\Big( c_e\|\bbeta^1_p - \bbeta^2_p\|^2_{\bV_p} + \rho_f\,\|\bu^1_f - \bu^2_f\|^2_{\bL^2(\Omega_f)} + s_0\,\|p^1_p - p^2_p\|^2_{\W_p} +\rho_p\,\|\partial_t\,\bbeta^1_p - \partial_t\,\bbeta^2_p\|^2_{\bL^2(\Omega_p)}\Big)  \\ [2ex]
%\ds\, \frac{1}{2\mu} \|(\bsi^1_f - \bsi^2_f)^\rd\|^2_{\bbL^2(\Omega_f)} 
%+ \mu k_{\min} \|\bu^1_p - \bu^2_p\|^2_{\bL^2(\Omega_p)} 
%+ c_I \sum^{n-1}_{j=1} \|  (\bvarphi^1 - \bvarphi^2)\cdot \bt_{f,j} - (\partial_t\,\bbeta^1_p - \partial_t\,\bbeta^2_p)\cdot\bt_{f,j}\|^2_{\L^2(\Gamma_{fp})} \\ [2ex]
%\ds -\,\, \frac{\rho_f}{2\,\mu}\,\Big(\|\bu^1_f\|_{\bV_f} + \|\bu^2_f\|_{\bV_f}\Big)\,\|\bu^1_f - \bu^2_f\|_{\bV_f}\|(\bsi^1_f - \bsi^2_f)^\rd\|_{\bbL^2(\Omega_f)}  \\ [2ex]
%\ds -C_{\cL}\big(\|\bvarphi^1\|_{\bLambda_f}+\|\bvarphi^2\|_{\bLambda_f}\big)\|\bvarphi^1-\bvarphi^2\|^2_{\bLambda_f} \,\leq\, 0.
%\end{array}
%\end{equation*}
%Employing similar arguments from \eqref{eq:mono f bound1}- \eqref{eq:mono u bound2} used in Lemma~\ref{lem:M-monotone-operator} and the non-negativity estimate \eqref{eq:positivity-aBJS}, we obtain
and similar arguments to the ones employed in \eqref{eq:mono f bound1}--\eqref{eq:mono u bound2}, we obtain
\begin{align}
&\frac{1}{2}\,\partial_t\,\Big( c_e\|\bbeta^1_p - \bbeta^2_p\|^2_{\bV_p}  + \rho_f\,\|\bu^1_f - \bu^2_f\|^2_{\bL^2(\Omega_f)} + s_0\,\|p^1_p - p^2_p\|^2_{\W_p} +\rho_p\,\|\partial_t\,\bbeta^1_p - \partial_t\,\bbeta^2_p\|^2_{\bL^2(\Omega_p)}\Big)  \nonumber\\ 
&\quad \frac{1}{16\,\mu}\,\|(\bsi^1_f - \bsi^2_f)^\rd\|^2_{\bbL^2(\Omega_f)} + \mu\,k_{\min}\,\|\bu^1_p - \bu^2_p\|^2_{\bL^2(\Omega_p)}\,\leq\, 0 \,,\label{eq:thm4.11-coercivitybounds}
\end{align}
and integrating in time from $0$ to $t\in (0,T]$,  and using $\bu^1_f(0) = \bu^2_f(0), p^1_p(0) = p^2_p(0), \bbeta^1_p(0) = \bbeta^2_p(0)$,
$\partial_t\,\bbeta^1_p(0) = \partial_t\,\bbeta^2_p(0)$, yields
\begin{align*}
&\frac{1}{2}\,\Big( c_e\|\bbeta^1_p - \bbeta^2_p\|^2_{\bV_p}  + \rho_f\,\|\bu^1_f - \bu^2_f\|^2_{\bL^2(\Omega_f)} + s_0\,\|p^1_p - p^2_p\|^2_{\W_p} +\rho_p\,\|\partial_t\,\bbeta^1_p - \partial_t\,\bbeta^2_p\|^2_{\bL^2(\Omega_p)}\Big)  \nonumber\\ 
& \quad +\,\, C\,\int^t_0 \Big(\|(\bsi^1_f - \bsi^2_f)^\rd\|^2_{\bbL^2(\Omega_f)} + \|\bu^1_p - \bu^2_p\|^2_{\bL^2(\Omega_p)} \Big)\, ds \,\leq\, 0 \,.%\label{eq:aep-integral-inequality}
\end{align*}
%
Thus, $(\bsi^1_f(t))^\rd = (\bsi^2_f(t))^\rd, \bu^1_p(t) = \bu^2_p(t), \bbeta^1_p(t) = \bbeta^2_p(t), \bu_f^1(t) = \bu_f^2(t), p^1_p(t) = p^2_p(t)$, and $\partial_t\,\bbeta^1_p(t) = \partial_t\,\bbeta^2_p(t)$ for all $t\in (0,T]$.

On the other hand, for $(\bu^i_f, p^i_p, \bgamma^i_f, \bvarphi^i, \lambda^i) \in \bV_f\times \W_p\times \bbQ_f\times \bLambda_f\times \Lambda_p$ and $(\bsi^i_f, \bu^i_p) \in \bbX_f\times \bX_p$  with $i\in \{1,2\}$ satisfying \eqref{eq:continuous-weak-formulation-1a} and \eqref{eq:continuous-weak-formulation-1e}, using the inf-sup conditions of $\cB$ (cf. Lemma \ref{lem:inf-sup-conditions}) and continuity of $\kappa_{\bw_f}$ (cf. \eqref{eq:continuity-cK-wf}) along with $(\bu^1_f,\bvarphi^1), (\bu^2_f,\bvarphi^2)\in \bW_{r_1,r_2}$ (cf. \eqref{eq:Wr-definition}), we obtain
%\begin{align}
%& \ds  - \pil\btau_f\bn_f,\bvarphi \pir_{\Gamma_{fp}} + (\bu_f,\bdiv(\btau_f))_{\Omega_f} + (\bgamma_f,\btau_f)_{\Omega_f} 
%+ \frac{\rho_f}{2\,\mu}\,((\bu_f\otimes\bu_f)^\rd,\btau_f)_{\Omega_f}- (p_p,\div(\bv_p))_{\Omega_p} \nonumber \\
%&\ds  + \pil\bv_p\cdot\bn_p,\lambda\pir_{\Gamma_{fp}}= -\frac{1}{2\,\mu}\,(\bsi^\rd_f,\btau^\rd_f)_{\Omega_f} -  \mu\,(\bK^{-1}\bu_p,\bv_p)_{\Omega_p} \quad \forall (\btau_f, \bv_p) \in \bbX_f\times \bX_p. \label{eq:continuous-weak-formulation-1a+1e}
%\end{align}
%For $(\bu^i_f, p^i_p, \bgamma^i_f, \bvarphi^i, \lambda^i) \in \bV_f\times \W_p\times \bbQ_f\times \bLambda_f\times \Lambda_p$ and $(\bsi^i_f, \bu^i_p) \in \bbX_f\times \bX_p$  with $i\in \{1,2\}$ in \eqref{eq:continuous-weak-formulation-1a+1e} and subtracting and employing the inf-sup conditions of $\cB$ (cf. Lemma~\ref{lem:inf-sup-conditions}) and continuity of $\kappa_{\bw_f}$ (cf. \eqref{eq:continuity-cK-wf}) along with $(\bu^1_f,\bvarphi^1), (\bu^2_f,\bvarphi^2)\in \bW_{r_1,r_2}$ (cf. \eqref{eq:Wr-definition}), we obtain
\begin{align*}
& \frac{2\beta}{3}\,\|(\bu^1_f - \bu^2_f, p^1_p - p^2_p, \bgamma^1_f - \bgamma^2_f, \bvarphi^1 - \bvarphi^2, \lambda^1 - \lambda^2)\| \nonumber\\
& \quad \,\leq \sup_{\0\neq (\btau_f,\bv_p)\in \bbX_{f}\times \bX_p}   \frac{-a_f(\bsi^2_f - \bsi^1_f,\btau_f) - a^d_p(\bu^2_p - \bu^1_p,\bv_p)}{\|(\btau_f,\bv_p)\|_{\bbX_f\times \bX_p}} \,=\, 0 \,.
\end{align*}
Therefore, $\bgamma^1_f(t) = \bgamma^2_f(t), \bvarphi^1(t) = \bvarphi^2(t)$, and $\lambda^1(t) = \lambda^2(t)$ for all $t\in (0,T]$.
In turn, from \eqref{eq:continuous-weak-formulation-1b}, we have the identity
\begin{equation*}
b_f(\bsi^1_f - \bsi^2_f,\bv_f)  
\,=\, \rho_f\,(\partial_t\,(\bu^1_f - \bu^2_f),\bv_f)_{\Omega_f} 
\,=\, 0 \quad \forall\,\bv_f\in \bV_f, 
\end{equation*}
which together with the property $\bdiv(\bbX_f) = (\bV_f)'$ allow us to deduce: $\bdiv(\bsi^1_f(t)) = \bdiv(\bsi^2_f(t))$ for all $t\in (0,T]$, and employing again the inequalities \eqref{eq:tau-d-H0div-inequality} and \eqref{eq:tau-H0div-Xf-inequality}, we conclude that $\bsi^1_f(t) = \bsi^2_f(t)$ for all $t\in (0,T]$, so then we can conclude that \eqref{eq:continuous-weak-formulation-1} has a unique solution.

Finally, let $\overline{\bsi}_f =\bsi_f(0) - \bsi_{f,0}$, with a similar definition and notation for the rest of the variables, except for $\overline{\partial_t\,\bbeta}_{p} = \partial_t\,\bbeta_{p}(0) - \bu_{s,0}$. 
We can take $t \to 0$ in \eqref{eq:continuous-weak-formulation-1}. 
Using that the initial data $(\ubsi_0,\ubu_0)$ constructed in Lemma \ref{lem:sol0-in-M-operator} satisfies \eqref{eq:initial-data-system}, which corresponds to \eqref{eq:T-auxiliary-problem-operator-B} at $t=0$ without the operator $\cE_1$, and that $\overline{\bu}_{f} = \0, \overline{p}_{p} = 0, \overline{\bbeta}_{p} = \0 $ and $\overline{\partial_t\,\bbeta}_{p} = \0,$ we obtain
\begin{subequations}\label{eq:init-0}
\begin{align}
&\ds  \frac{1}{2\,\mu}\,(\overline{\bsi}^\rd_f,\btau^\rd_f)_{\Omega_f} - \pil\btau_f\bn_f,\overline{\bvarphi} \pir_{\Gamma_{fp}} + (\overline{\bgamma}_f,\btau_f)_{\Omega_f} =0, \label{init-1} \\[1ex]
& \ds -\,(\overline{\bsi}_f,\bchi_f)_{\Omega_f} = 0, \label{init-2} \\[1ex]
& \ds \mu\,(\bK^{-1}\overline{\bu}_p,\bv_p)_{\Omega_p} + \pil\bv_p\cdot\bn_p,\overline{\lambda}\pir_{\Gamma_{fp}} = 0, \label{init-3} \\[1ex]
& \ds - \pil\overline{\bvarphi}\cdot\bn_f + \overline{\bu}_p\cdot\bn_p,\xi\pir_{\Gamma_{fp}} = 0, \label{init-4} \\[1ex]
& \ds \pil\overline{\bsi}_f\bn_f,\bpsi\pir_{\Gamma_{fp}} + \mu\,\alpha_{\BJS}\,\sum_{j=1}^{n-1} \pil\sqrt{\bK^{-1}_j} \overline{\bvarphi} \cdot\bt_{f,j},\bpsi\cdot\bt_{f,j} \pir_{\Gamma_{fp}}   +\,\, \rho_f\pil \overline{\bvarphi}\cdot\bn_f, \bvarphi(0)\cdot\bpsi \pir_{\Gamma_{fp}} \nonumber \\[2ex]
& \ds\quad +\, \rho_f\pil \bvarphi_0\cdot\bn_f, \overline{\bvarphi}\cdot\bpsi \pir_{\Gamma_{fp}}+ \pil\bpsi\cdot\bn_f,\overline{\lambda}\pir_{\Gamma_{fp}} = 0.\label{init-5} 
\end{align}
\end{subequations}
Taking $(\btau_f,\bv_p,\bchi_f,\bpsi, \xi) = (\overline{\bsi}_f,\overline{\bu}_p,\overline{\bgamma}_f,\overline{\bvarphi},\overline{\lambda})$ in \eqref{eq:init-0}, using that $(\bu_{f,0},\bvarphi_0), (\bu_{f}(0),\bvarphi(0))\in \bW_{r_1,r_2}$ (cf. \eqref{eq:Wr-definition}), combining the equations and proceeding as in  \eqref{eq:thm4.11-coercivitybounds}, we get
\begin{equation*}
\,\|\overline{\bsi}^\rd_f \|^2_{\bbL^2(\Omega_f)} + \,\|\overline{\bu}_p\|^2_{\bL^2(\Omega_p)} + \,\sum^{n-1}_{j=1} \|\overline{\bvarphi}\cdot \bt_{f,j} \|^2_{\L^2(\Gamma_{fp})}\,\leq\, 0,  
\end{equation*}
which implies that $\overline{\bsi}^\rd_f = \0, \overline{\bu}_p=\0$ and $\overline{\bvarphi}\cdot \bt_{f,j} = 0.$ In addition, \eqref{init-4} implies that $\pil\overline{\bvarphi}\cdot\bn_f,\xi\pir_{\Gamma_{fp}} = 0$ for all $\xi \in \H^{1/2}(\Gamma_{fp}).$  Since $\H^{1/2}(\Gamma_{fp})$ 
is dense in $\L^2(\Gamma_{fp})$, it follows that $\overline{\bvarphi}\cdot\bn_f= 0;$ hence $\overline{\bvarphi}=\0$. Again, combining \eqref{init-1} and \eqref{init-3}, employing the inf-sup conditions of $\cB$ (cf. Lemma~\ref{lem:inf-sup-conditions}), together with  $\overline{\bsi}^\rd_f = \0$, $\overline{\bvarphi} = \0$ and $\overline{\bu}_p=\0$, implies that $\overline{\bgamma}_f = \0$ and $\overline{\lambda} =0$. Finally, from \eqref{eq:continuous-weak-formulation-1b} at t = 0, we obtain
\begin{equation}\label{div-sigf-init}
  b_f(\overline\bsi_{f},\bv_f) =  \rho_f\, (\partial_t\,\bu_f(0),\bv_f)_{\Omega_f} - (\f_f(0),\bv_f)_{\Omega_f} - b_f(\bsi_{f,0},\bv_f) = 0,
\end{equation}
where we have chosen $\f_f(0)= \rho_f\,\partial_t\,\bu_f(0) - \bdiv(\bsi_{f,0})$. Applying the property $\bdiv(\bbX_f) = (\bV_f)'$ allow us to deduce: $\bdiv(\overline{\bsi}_f) = \0$, which, combined with the inequalities \eqref{eq:tau-d-H0div-inequality} and \eqref{eq:tau-H0div-Xf-inequality}, yields $\overline{\bsi}_f = \0$, completing the proof.
\end{proof}

We next recall from \cite[Lemma 3.3]{cwy2025} a result that will be employed to derive the stability bound for the solution of \eqref{eq:continuous-weak-formulation-1} without relying on Gr\"onwall's inequality.
%
\begin{lem}\label{xing-lemma}
Suppose that for all $t \in (0,T]$,
\begin{equation*}%\label{xing-assump}
H^2(t) + R(t) \leq A(t) + 2 \int_0^t B(s) H(s) \, ds,
\end{equation*}
where $H, R, A$ and $B$ are non-negative functions. Then
\begin{equation*}%\label{xing-conc}
\sqrt{H^2(T) + R(T)} \leq \sup_{0\leq t\leq T} \sqrt{A(t)} + \int_0^T B(t) dt.
\end{equation*}
\end{lem}
%
%\begin{proof}
%We refer the reader to \cite[Lemma 3.3]{cwy2025}. Further details are omitted.
%\end{proof}
%

We conclude with the aforementioned stability bound.
%
\begin{thm}\label{thm: continuous stability}
Under the assumptions of Theorem \ref{thm:unique soln}, and assuming that $\f_f \in \H^1(0,T;\bL^2(\Omega_f))$, $\f_p \in \H^1(0,T;\bL^2(\Omega_p))$, and $q_p \in \H^1(0,T;\L^2(\Omega_p))$, 
%then the solution of \eqref{eq:continuous-weak-formulation-1} has regularity
%$\bsi_f \in \L^2(0,T;\bbX_f),\bu_p \in \L^2(0,T;\bX_p)$,
%$\bbeta_p \in \W^{1,\infty}(0,T;\bV_p),\bu_f \in \W^{1,\infty}(0,T;\bL^2(\Omega_f))\cap \L^{\infty}(0,T;\bV_f), p_p \in \W^{1,\infty}(0,T;\W_p),\bgamma_f \in \H^1(0,T;\bbQ_f)$,
%$\bvarphi \in \H^1(0,T;\bLambda_f)$ and $ \lambda \in \H^1(0,T;\Lambda_p)$.
%In addition, there exists a positive constant $C$, independent of $s_0$, such that
there exists a positive constant $C$, independent of $s_0$, such that
\begin{align}
&\|\bsi_f\|_{\L^2(0,T;\bbX_f)} 
+ \|\partial_t\bsi^\rd_f\|_{\L^2(0,T;\bbL^2(\Omega_f))} 
+ \|\bu_p\|_{\L^2(0,T;\bX_p)} 
+ \| \partial_t\bu_p\|_{\L^2(0,T;\bL^2(\Omega_p))}
+ \|\bbeta_p\|_{\L^\infty(0,T;\bV_p)} 
\nonumber \\
&\quad +\, \| \partial_t\bbeta_p\|_{\L^\infty(0,T;\bL^2(\Omega_p))}
+ \sum^{n-1}_{j=1} \|(\bvarphi-\partial_t\,\bbeta_p)\cdot\bt_{f,j}\|_{\H^1(0,T;\L^2(\Gamma_{fp}))}
+\, \| \partial_{tt}\bbeta_p\|_{\L^\infty(0,T;\bL^2(\Omega_p))}
\nonumber \\
&\quad +\, \|\bu_f\|_{\H^1(0,T;\bV_f)}
+ \|\bu_f\|_{\W^{1,\infty}(0,T;\bL^2(\Omega_f))}
+ \sqrt{s_0}\,\|p_p\|_{\W^{1,\infty}(0,T;\W_p)}
+ \|p_p\|_{\H^1(0,T;\W_p)}
\nonumber \\[1ex]
&\quad 
+\, \|\bgamma_f\|_{\H^1(0,T;\bbQ_f)} 
+ \|\bvarphi\|_{\H^1(0,T;\bLambda_f)} 
+ \|\lambda\|_{\H^1(0,T;\Lambda_p)} 
\nonumber \\
&\leq\, C \sqrt{T}\,\Bigg( \|\f_f\|_{\H^1(0,T;\bL^2(\Omega_f))}  
+ \|\f_p\|_{\H^1(0,T;\bL^2(\Omega_p))}
+ \|q_p\|_{\H^1(0,T;\L^2(\Omega_p))}
+ \frac{1}{\sqrt{s_0}}\|q_p(0)\|_{\L^2(\Omega_p)}
\nonumber \\
& \quad +\, \|\bu_{f,0}\|_{\bL^2(\Omega_f)}
+ \|\bdiv( \be(\bu_{f,0}))\|_{\bL^{2}(\Omega_f)}
+ \|\bdiv(\bu_{f,0}\otimes \bu_{f,0})\|_{\bL^2(\Omega_f)} 
+ \sqrt{s_0}\,\|p_{p,0}\|_{\W_p}
\nonumber \\
& \quad +\, \|p_{p,0}\|_{\H^1(\Omega_p)}
+ \frac{1}{\sqrt{s_0}}\|\div(\bK\nabla p_{p,0})\|_{\bL^2(\Omega_p)}
+ \|\bbeta_{p,0}\|_{\bV_p} 
+ \|\bdiv(A^{-1}(\, \be(\bbeta_{p,0})))\|_{\bL^2(\Omega_p)} 
\nonumber \\
&\quad 
%+\, \|\bu_{s,0}\|_{\bL^2(\Omega_p)}
+\, \left(1+\frac{1}{\sqrt{s_0}}\right)\|\bu_{s,0}\|_{\bV_p}  
\Bigg) \,.
\label{eq:continuous-stability}
\end{align}
\end{thm}
%
\begin{proof}
We begin by choosing $(\btau_f, \bv_p, \bxi_p, \bv_f, w_p, \bchi_f, \bpsi, \xi) = (\bsi_f, \bu_p, \partial_t\,\bbeta_p, \bu_f, p_p, \bgamma_f, \bvarphi, \lambda)$ in \eqref{eq:continuous-weak-formulation-1} to get
\begin{align}
&\frac{1}{2}\,\partial_t\,\Big( \rho_f\,\|\bu_f\|^2_{\bL^2(\Omega_f)} 
+ s_0\,\|p_p\|^2_{\W_p} 
+ a^e_p(\bbeta_p, \bbeta_p)  
+ \rho_p\,\|\partial_t\bbeta_p\|^2_{\bL^2(\Omega_p)}\Big) 
+ a_f(\bsi_f, \bsi_f) 
+ a^d_p(\bu_p,\bu_p) \nonumber \\
&\quad +\, c_{\BJS}(\partial_t\,\bbeta_p, \bvarphi;\partial_t\,\bbeta_p, \bvarphi) 
+ \kappa_{\bu_f}(\bu_f,\bsi_f) 
+ l_{\bvarphi}(\bvarphi,\bvarphi)
\,=\, (\f_p,\partial_t\,\bbeta_p)_{\Omega_p} 
+ (\f_f,\bu_f)_{\Omega_f} 
+ (q_p,p_p)_{\Omega_p} \,.\label{eq:stability-bound-1}
\end{align}
Next, we integrate \eqref{eq:stability-bound-1} from $0$ to $t\in (0,T]$, use the coercivity bound of $a^e_p$ (cf. \eqref{eq:aep-coercivity-bound}), the non-negativity bounds of $a_f, a^d_p, \cE_2$ and $c_\BJS$ (cf. \eqref{eq:coercivity-af}, \eqref{eq:coercivity-adp}, \eqref{eq: operator E_2-monotone}, \eqref{eq: operator C-monotone}) in combination with the inf-sup condition of $B_f$ (cf. \eqref{eq:inf-sup-vf-chif}), the continuity of $\kappa_{\bw_f}$ and $l_{\bzeta}$ (cf. \eqref{eq:continuity-cK-wf}, \eqref{eq:continuity-cL-zeta}), the fact that $(\bu_f(t),\bvarphi(t)):[0,T]\to \bW_{r_1,r_2}$ (cf. \eqref{eq:Wr-definition}), the identity
\begin{equation*}
\int_0^t (\f_p,\partial_t\,\bbeta_p)_{\Omega_p}\, ds = (\f_p,\,\bbeta_p)_{\Omega_p}|_0^t - \int_0^t (\partial_t \f_p,\\,\bbeta_p)_{\Omega_p}\, ds,
\end{equation*}
and similar arguments to the ones used in \eqref{eq:mono f bound1}--\eqref{eq:mono u bound2} to deal with the terms $\kappa_{\bu_f}$ and $l_\bvarphi$, to obtain 
\begin{align}\label{eq:bound-unsteady-state-solution-1}
& \ds \frac{\rho_f}{2}\,\|\bu_f(t)\|^2_{\bL^2(\Omega_f)} + \frac{s_0}{2}\,\|p_p(t)\|^2_{\W_p} + \frac{c_e}{2}\|\bbeta_p(t)\|^2_{\bV_p} + \frac{\rho_p}{2}\,\| \partial_t\bbeta_p(t)\|^2_{\bL^2(\Omega_p)}\nonumber \\[1ex]
&\ds\quad +\, \min\left\{\frac{17}{50\mu},\mu\,k_{\min},c_I\right\}\, \int^t_0 \Big( \| \bsi^\rd_f\|^2_{\bbL^2(\Omega_f)} + \|\bu_p\|^2_{\bL^2(\Omega_p)} + \sum^{n-1}_{j=1} \|( \bvarphi - \partial_t\,\bbeta_p)\cdot\bt_{f,j}\|^2_{\L^2(\Gamma_{fp})} \Big)\, ds \nonumber \\[1ex]
& \ds \leq\, \frac{\rho_f}{2}\,\|\bu_f(0)\|^2_{\bL^2(\Omega_f)} +  \frac{s_0}{2}\,\|p_p(0)\|^2_{\W_p} + \frac{c_e}{2}\|\bbeta_p(0)\|^2_{\bV_p} + \frac{\rho_p}{2}\,\| \partial_t\bbeta_p(0)\|^2_{\bL^2(\Omega_p)} + (\f_p(t),\bbeta_p(t))_{\Omega_p}  \nonumber \\[1ex]
&\ds\quad -\, (\f_p(0),\bbeta_p(0))_{\Omega_p}\,-\,\, \int^t_0 \Big( (\partial_t\,\f_p,\bbeta_p)_{\Omega_p} -  (\f_f,\bu_f)_{\Omega_f}  - (q_p,p_p)_{\Omega_p} \Big)\, ds \,.
\end{align}
%Thus, applying Cauchy-Schwarz and Young's inequalities on the right-hand side of \eqref{eq:bound-unsteady-state-solution-1}, we get
%\begin{align}\label{eq:bound-unsteady-state-solution-2}
%& \ds \,\|\bu_f(t)\|^2_{\bL^2(\Omega_f)} + s_0\,\|p_p(t)\|^2_{\W_p} + \|\bbeta_p(t)\|^2_{\bV_p} +  \,\| \partial_t\bbeta_p(t)\|^2_{\bL^2(\Omega_p)}\nonumber \\[1ex]
%& \ds\,+ \, \int^t_0 \Big( \| \bsi^\rd_f\|^2_{\bbL^2(\Omega_f)} + \|\bu_p\|^2_{\bL^2(\Omega_p)} + \sum^{n-1}_{j=1} \|( \bvarphi - \partial_t\,\bbeta_p)\cdot\bt_{f,j}\|^2_{\L^2(\Gamma_{fp})} \Big)\, ds \nonumber \\[1ex]
%& \ds \,\leq C\,\Big( \,\|\bu_f(0)\|^2_{\bL^2(\Omega_f)} +  s_0\,\|p_p(0)\|^2_{\W_p} + \|\bbeta_p(0)\|^2_{\bV_p} +  \,\| \partial_t\bbeta_p(0)\|^2_{\bL^2(\Omega_p)}\, + \|\f_p(0)\|^2_{\bL^2(\Omega_p)}  \nonumber \\[1ex]
%& \ds\quad + \|\bbeta_p(0)\|^2_{\bV_p} + \|\f_p(t)\|^2_{\bL^2(\Omega_p)} \Big) \,+\,\, C\,\int^t_0 \Big(\|\partial_t\,\f_p\|^2_{\bL^2(\Omega_p)} + \|\bbeta_p\|^2_{\bV_p} + \|\f_f\|^2_{\bL^2(\Omega_f)} + \|q_p\|^2_{\L^2(\Omega_p)} \Big)\, ds  \nonumber \\[1ex]
%& \ds\quad +\,\, \delta\,\|\bbeta_p(t)\|^2_{\bV_p} + \delta\,\int^t_0 \Big( \|\bu_f\|^2_{\bV_f} + \|p_p\|^2_{\W_p} \Big)\, ds, 
%\end{align}
%where $\delta > 0$ is suitably chosen.

On the other hand, similar to \eqref{eq:sol-uniqueness-4.8-3}, adding \eqref{eq:continuous-weak-formulation-1a} and \eqref{eq:continuous-weak-formulation-1e}, applying the inf-sup condition of $\cB$ in Lemma \ref{lem:inf-sup-conditions} for $(\bu_f, p_p, \bgamma_f, \bvarphi, \lambda)$, the continuity of $\kappa_{\bw_f}$ (cf. \eqref{eq:continuity-cK-wf}), with $(\bu_f(t),\bvarphi(t)):[0,T]\to \bW_{r_1,r_2}$ (cf. \eqref{eq:Wr-definition}), and the continuity bounds of $a_f, a^d_p$ (cf. \eqref{defn-A}, \eqref{eq:continuity-cA-cB-cC}), we deduce that
%\begin{equation*}
%\begin{array}{l}
%\ds \frac{5\beta}{6}\,\|(\bu_f, p_p, \bgamma_f, \bvarphi, \lambda)\| \leq\, \sup_{(\btau_f,\bv_p)\in \bbX_f\times \bX_p} \frac{|-\,a_f(\bsi_f,\btau_f) - a^d_p(\bu_p,\bv_p)|}{\|(\btau_f,\bv_p)\|_{\bbX_f\times \bX_p}}, 
%\end{array}
%\end{equation*}
%which combined with the continuity bounds in \eqref{eq:continuity-cA-cB-cC} allow us to deduce
\begin{equation}\label{eq:bound-unsteady-state-solution-3}
\int^t_0 \|(\bu_f, p_p, \bgamma_f, \bvarphi, \lambda)\|^2 \, ds 
\,\leq\, C\,\int^t_0 \Big( \| \bsi^\rd_f\|^2_{\bbL^2(\Omega_f)} + \|\bu_p\|^2_{\bL^2(\Omega_p)} \Big)\, ds.
\end{equation}
%
%\begin{align}
%&\,\|\bu_f(t)\|^2_{\bL^2(\Omega_f)} + s_0\,\|p_p(t)\|^2_{\W_p} + \|\bbeta_p(t)\|^2_{\bV_p} + \,\| \partial_t\bbeta_p(t)\|^2_{\bL^2(\Omega_p)}  \nonumber \\
%& \,+\,\, \int^t_0 \Bigg( \|\bsi^\rd_f\|^2_{\bbL^2(\Omega_f)} + \|\bu_p\|^2_{\bL^2(\Omega_p)} + \sum^{n-1}_{j=1} \|( \bvarphi - \partial_t\,\bbeta_p )\cdot\bt_{f,j}\|^2_{\L^2(\Gamma_{fp})} + \|(\bu_f, p_p, \bgamma_f, \bvarphi, \lambda)\|^2 \Bigg)\, ds \nonumber \\
%&\,\leq\, C\,\Bigg( \int^t_0 \|\bbeta_p\|^2_{\bV_p}\, ds + \|\f_p(t)\|^2_{\bL^2(\Omega_p)} + \|\f_p(0)\|^2_{\bL^2(\Omega_p)} + \,\|\bu_f(0)\|^2_{\bL^2(\Omega_f)} + s_0\,\|p_p(0)\|^2_{\W_p}  \nonumber \\
%&\quad \,+\,\, \|\bbeta_p(0)\|^2_{\bV_p}  + \,\| \partial_t\bbeta_p(0)\|^2_{\bL^2(\Omega_p)} + \int^t_0 \Big( \|\partial_t\,\f_p\|^2_{\bL^2(\Omega_p)} + \|\f_f\|^2_{\bL^2(\Omega_f)} + \|q_p\|^2_{\L^2(\Omega_p)} \Big)\, ds \Bigg).\label{eq:bound-unsteady-state-solution-4}
%\end{align}

Thus, applying Cauchy--Schwarz and Young's inequalities on the right-hand side of \eqref{eq:bound-unsteady-state-solution-1} in combination with \eqref{eq:bound-unsteady-state-solution-3}, we obtain
\begin{align}
&\|\bu_f(t)\|^2_{\bL^2(\Omega_f)} + s_0\,\|p_p(t)\|^2_{\W_p} + \|\bbeta_p(t)\|^2_{\bV_p}+ \,\| \partial_t\bbeta_p(t)\|^2_{\bL^2(\Omega_p)} \nonumber \\
&\quad +\, \int^t_0 \Bigg(  \| \bsi^\rd_f\|^2_{\bbL^2(\Omega_f)} + \|\bu_p\|^2_{\bL^2(\Omega_p)} + \sum^{n-1}_{j=1} \|( \bvarphi - \partial_t\,\bbeta_p )\cdot\bt_{f,j}\|^2_{\L^2(\Gamma_{fp})} + \|(\bu_f, p_p, \bgamma_f, \bvarphi, \lambda)\|^2 \Bigg)\, ds \nonumber \\
&\leq\, C\,\Bigg( \int^t_0 \Big( \|\f_f\|^2_{\bL^2(\Omega_f)} + \|q_p\|^2_{\L^2(\Omega_p)} \Big)\,ds 
+ \|\f_p(t)\|^2_{\bL^2(\Omega_p)} + \|\f_p(0)\|^2_{\bL^2(\Omega_p)} +\, \|\bu_f(0)\|^2_{\bL^2(\Omega_f)} 
\nonumber \\
&\quad 
+ s_0\,\|p_p(0)\|^2_{\W_p} + \|\bbeta_p(0)\|^2_{\bV_p} 
+ \| \partial_t\bbeta_p(0)\|^2_{\bL^2(\Omega_p)}  +\, \int^t_0 \|\partial_t\,\f_p\|_{\bL^2(\Omega_p)}\|\bbeta_p\|_{\bV_p} \, ds\Bigg)\,.
\label{eq:bound-unsteady-state-solution-6}
\end{align}

Next, we obtain a stability bound for $\|\bdiv(\bsi_f)\|_{\bL^{4/3}(\Omega_f)}$ and $\|\div(\bu_{p})\|_{\L^2(\Omega_p)}$. From \eqref{eq:continuous-weak-formulation-1b}, \eqref{eq:continuous-weak-formulation-1f}, and recalling that $\bdiv(\bbX_f) = (\bV_f)'$ and $\div(\bX_p) = (\W_p)'$, it follows that
\begin{align}
&\|\bdiv(\bsi_f)\|_{\bL^{4/3}(\Omega_f)} 
\,\leq\, |\Omega_f|^{1/4}\,\big(\|\f_f \|_{\bL^2(\Omega_f)} + \rho_f\,\|\partial_t\,\bu_f \|_{\bL^2(\Omega_f)}\big), \nonumber \\[1ex]
&\mbox{and}\quad \|\div(\bu_{p})\|_{\L^2(\Omega_p)} 
\,\leq\, \|q_p\|_{\L^2(\Omega_p)} + s_0\,\|\partial_t p_{p}\|_{\W_p} + \alpha_p\,n^{1/2}\,\|\partial_t\bbeta_{p}\|_{\bV_p} .\label{eq:bound-unsteady-state-solution-5}
\end{align}
\noindent{\bf Bounds on time derivatives on the right-hand side of \eqref{eq:bound-unsteady-state-solution-5}.}

\noindent In order to bound the time derivative terms in \eqref{eq:bound-unsteady-state-solution-5}, we take a finite difference in time of the whole system \eqref{eq:continuous-weak-formulation-1}. 
In particular, given $t \in [0,T)$ and $s > 0$ with $t+s \le T$, let $\partial_t^s\phi := \frac{\phi(t+s) - \phi(t)}{s}$. 
Thus, applying the operator to \eqref{eq:continuous-weak-formulation-1}, and testing with
$(\btau_f, \bv_p, \bxi_p, \bv_f, w_p, \bchi_f, \bpsi, \xi) = (\partial_t^s\bsi_f, \partial_t^s\bu_p, \partial_t^s\partial_t^s\,\bbeta_p, \partial_t^s\bu_f, \partial_t^s p_p, \partial_t^s\bgamma_f$, $\partial_t^s\bvarphi, \partial_t^s\lambda)$, similarly to \eqref{eq:stability-bound-1}, we get 
\begin{align}
& \frac{1}{2}\,\partial_t\,\Big( \rho_f\,\|\partial_t^s\,\bu_f\|^2_{\bL^2(\Omega_f)} 
+ s_0\,\|\partial_t^s\,p_p\|^2_{\W_p} 
+ a^e_p(\partial_t^s\bbeta_p, \partial_t^s\bbeta_p) 
+ \rho_p\,\|\partial_t^s\partial_t^s\bbeta_p\|^2_{\bL^2(\Omega_p)}\Big) 
\nonumber \\
&\quad +\, a_f(\partial_t^s\bsi_f, \partial_t^s\bsi_f)
+ a^d_p(\partial_t^s\bu_p,\partial_t^s\bu_p) 
+ c_{\BJS}(\partial_t^s\partial_t^s\,\bbeta_p,\partial_t^s \bvarphi;\partial_t^s\partial_t^s\,\bbeta_p, \partial_t^s\bvarphi) 
+ \kappa_{\partial_t^s\bu_f}(\bu_f(t),\partial_t^s\bsi_f)  
\nonumber \\
&\quad +\, \kappa_{\bu_f(t+s)}(\partial_t^s\bu_f,\partial_t^s\bsi_f)
+ l_{\partial_t^s\bvarphi}(\bvarphi(t),\partial_t^s\bvarphi) 
+ l_{\bvarphi(t+s)}(\partial_t^s\bvarphi,\partial_t^s\bvarphi)
\nonumber \\
& =\, (\partial_t^s\f_p,\partial_t^s\partial_t^s\,\bbeta_p)_{\Omega_p} 
+ (\partial_t^s\f_f,\partial_t^s\bu_f)_{\Omega_f} 
+ (\partial_t^s q_p,\partial_t^s p_p)_{\Omega_p}\,.
\label{eq:stability-bound-7}
\end{align}
Next, we integrate from 0 to t $\in (0,T)$, use the coercivity bound in \eqref{eq:aep-coercivity-bound}, the non-negativity bounds of $a_f, a^d_p$ and $c_{\BJS}$ in Lemma \ref{lem:coercivity-properties-A-E2} in combination with the inf-sup condition of $B_f$ (cf. \eqref{eq:inf-sup-vf-chif}), the continuity of  $\kappa_{\bw_f}$ and $l_{\bzeta}$ (cf. \eqref{eq:continuity-cK-wf}, \eqref{eq:continuity-cL-zeta}), the fact that $(\bu_f(t),\bvarphi(t)):[0,T]\to \bW_{r_1,r_2}$ (cf. \eqref{eq:Wr-definition}), and take $s\to 0$ to obtain
\begin{align}\label{eq:bound-unsteady-state-solution-8}
&\ds \frac{\rho_f}{2}\,\|\partial_t \bu_f(t)\|^2_{\bL^2(\Omega_f)} 
+ \frac{s_0}{2}\,\| \partial_t p_p(t)\|^2_{\W_p} 
+ \frac{c_e}{2}\| \partial_t\bbeta_p(t)\|^2_{\bV_p} 
+ \frac{\rho_p}{2}\,\| \partial_{tt}\bbeta_p(t)\|^2_{\bL^2(\Omega_p)}
\nonumber \\[1ex]
&\ds\quad +\, \min\left\{\frac{1}{16\mu},\mu\,k_{\min},c_I\right\}\, \int^t_0 \Big( \| \partial_t\bsi^\rd_f\|^2_{\bbL^2(\Omega_f)} 
+ \| \partial_t\bu_p\|^2_{\bL^2(\Omega_p)} 
+ \sum^{n-1}_{j=1} \|( \partial_t\bvarphi - \partial_{tt}\,\bbeta_p)\cdot\bt_{f,j}\|^2_{\L^2(\Gamma_{fp})} \Big)\, ds 
\nonumber \\[1ex]
&\ds\leq\, \frac{\rho_f}{2}\,\|\partial_t\bu_f(0)\|^2_{\bL^2(\Omega_f)} 
+ \frac{s_0}{2}\,\|\partial_t p_p(0)\|^2_{\W_p} 
+ \frac{c_e}{2}\|\partial_t\bbeta_p(0)\|^2_{\bV_p} 
+ \frac{\rho_p}{2}\,\|\partial_{tt}\bbeta_p(0)\|^2_{\bL^2(\Omega_p)}  
\nonumber \\[1ex]
& \ds\quad +\, \int^t_0 \Big( (\partial_t\f_p,\partial_{tt}\,\bbeta_p)_{\Omega_p} 
+ (\partial_t\f_f,\partial_t\bu_f)_{\Omega_f} 
+ (\partial_t q_p,\partial_t p_p)_{\Omega_p} \Big)\, ds \,. 
\end{align}
%\begin{align}\label{eq:bound-unsteady-state-solution-9}
%& \ds \,\| \partial_t\bu_f(t)\|^2_{\bL^2(\Omega_f)} + s_0\,\| \partial_t p_p(t)\|^2_{\W_p} + \| \partial_t \bbeta_p(t)\|^2_{\bV_p} +  \,\| \partial_{tt}\bbeta_p(t)\|^2_{\bL^2(\Omega_p)}\nonumber \\[1ex]
%& \ds \,+ \, \int^t_0 \Big( \| \partial_t\bsi^\rd_f\|^2_{\bbL^2(\Omega_f)} + \| \partial_t\bu_p\|^2_{\bL^2(\Omega_p)} + \sum^{n-1}_{j=1} \|( \partial_t\bvarphi - \partial_{tt}\,\bbeta_p)\cdot\bt_{f,j}\|^2_{\L^2(\Gamma_{fp})} \Big)\, ds \nonumber \\[1ex]
%& \ds \,\leq C \Bigg( \,\| \partial_t\bu_f(0)\|^2_{\bL^2(\Omega_f)} +  s_0\,\|\partial_t p_p(0)\|^2_{\W_p} + \| \partial_t\bbeta_p(0)\|^2_{\bV_p} +  \,\| \partial_{tt}\bbeta_p(0)\|^2_{\bL^2(\Omega_p)}\Bigg)\, \nonumber \\[1ex]
%& \ds\quad    \,+\,\, C\,\int^t_0 \Big(\| \partial_{tt}\bbeta_p\|^2_{\bL^2(\Omega_p)} + \|\partial_{t}\,\f_p\|^2_{\bL^2(\Omega_p)}  + \| \partial_t\f_f\|^2_{\bL^2(\Omega_f)} + \|\partial_t q_p\|^2_{\L^2(\Omega_p)} \Big)\, ds \nonumber \\[1ex]
%& \ds\quad + \delta\,\int^t_0 \Big( \| \partial_t\bu_f\|^2_{\bV_f} + \|\partial_t p_p\|^2_{\W_p} \Big)\, ds. 
%\end{align}
In turn, differentiating in time \eqref{eq:continuous-weak-formulation-1a} and \eqref{eq:continuous-weak-formulation-1e} and applying the inf-sup condition of $\cB$ in Lemma \ref{lem:inf-sup-conditions} (cf. \eqref{eq:inf-sup-qp-xi}, \eqref{eq:inf-sup-vf-chif}) for $(\partial_t\bu_f, \partial_t p_p, \partial_t\bgamma_f, \partial_t\bvarphi, \partial_t\lambda)$, and the continuity of $\kappa_{\bw_f}$ (cf. \eqref{eq:continuity-cK-wf}), with $(\bu_f(t),\bvarphi(t)):[0,T]\to \bW_{r_1,r_2}$  (cf. \eqref{eq:Wr-definition}) as in \eqref{eq:bound-unsteady-state-solution-3}, to derive
\begin{equation}\label{eq:bound-unsteady-state-solution-14}
\int^t_0 \|(\partial_t\bu_f, \partial_t p_p, \partial_t\bgamma_f, \partial_t\bvarphi, \partial_t\lambda)\|^2 \, ds 
\,\leq\, C\,\int^t_0 \Big( \| \partial_t\bsi^\rd_f\|^2_{\bbL^2(\Omega_f)} + \|\partial_t\bu_p\|^2_{\bL^2(\Omega_p)} \Big)\,ds \,.
\end{equation}
Thus, using Cauchy--Schwarz and Young's inequalities on the right-hand side of \eqref{eq:bound-unsteady-state-solution-8} in combination with \eqref{eq:bound-unsteady-state-solution-14}, we get
\begin{align}\label{eq:bound-unsteady-state-solution-10}
&\ds \|\partial_t\bu_f(t)\|^2_{\bL^2(\Omega_f)} 
+ s_0\,\| \partial_t p_p(t)\|^2_{\W_p} 
+ \| \partial_t \bbeta_p(t)\|^2_{\bV_p} 
+ \| \partial_{tt}\bbeta_p(t)\|^2_{\bL^2(\Omega_p)}
+ \int^t_0 \Big( \| \partial_t\bsi^\rd_f\|^2_{\bbL^2(\Omega_f)} 
\nonumber \\[1ex]
&\ds\quad +\, \| \partial_t\bu_p\|^2_{\bL^2(\Omega_p)} + \sum^{n-1}_{j=1} \|( \partial_t\bvarphi - \partial_{tt}\,\bbeta_p)\cdot\bt_{f,j}\|^2_{\L^2(\Gamma_{fp})} + \|(\partial_t\bu_f, \partial_t p_p, \partial_t\bgamma_f, \partial_t\bvarphi, \partial_t\lambda)\|^2 \Big)\, ds \nonumber \\[1ex]
&\ds\leq\, C\,\Bigg( \,\int^t_0 \big(   \|\partial_t\f_f\|^2_{\bL^2(\Omega_f)} 
+ \|\partial_t q_p\|^2_{\L^2(\Omega_p)}\big)\,ds
+ \| \partial_t\bu_f(0)\|^2_{\bL^2(\Omega_f)} 
+ s_0\,\|\partial_t p_p(0)\|^2_{\W_p} 
\nonumber \\[1ex]
&\ds\quad +\, \|\partial_t\bbeta_p(0)\|^2_{\bV_p}
+ \|\partial_{tt}\bbeta_p(0)\|^2_{\bL^2(\Omega_p)} 
+ \int^t_0 \|\partial_{t}\,\f_p\|_{\bL^2(\Omega_p)}\|\partial_{tt}\bbeta_p\|_{\bL^2(\Omega_p)}\, ds  \Bigg)\,.
\end{align}

\noindent{\bf Bound on initial data.}

\noindent Recall that $(\bu_f(0), p_p(0), \bbeta_p(0), \partial_t\bbeta_p(0)) = (\bu_{f,0}, p_{p,0}, \bbeta_{p,0}, \bu_{s,0})$ is the initial data  given to us.
From \eqref{eq:continuous-weak-formulation-1b}, \eqref{eq:continuous-weak-formulation-1d} and \eqref{eq:continuous-weak-formulation-1f} at time t=0, integrating by parts backwardly \eqref{eq:continuous-weak-formulation-1d} in combination with the fact that the initial data satisfy \eqref{bdycond-4}--\eqref{bdycond-5} to cancel the terms on the interface, and choosing $(\bv_f,\bxi_p,w_p) = (\partial_t \bu_f(0), \partial_{tt}\bbeta_p(0), \partial_t p_p(0))$, we get
\begin{align}\label{eq:bound-unsteady-state-solution-11}
&\ds \rho_f\,\| \partial_t\bu_f(0)\|^2_{\bL^2(\Omega_f)} 
+ s_0\,\| \partial_t p_p(0)\|^2_{\W_p} 
+ \rho_p\,\| \partial_{tt}\bbeta_p(0)\|^2_{\bL^2(\Omega_p)} 
\,=\, (\partial_t\bu_f(0),\bdiv(\bsi_f)(0))_{\Omega_f} 
\nonumber \\[1ex]
&\ds\quad +\, ( \bdiv(A^{-1}(\be(\bbeta_p)) - \alpha_p\,p_p\,\bI)(0),\partial_{tt}\bbeta_p(0))_{\Omega_p}
-\alpha_p\,(\div(\partial_t\,\bbeta_p)(0), \partial_t p_p(0))_{\Omega_p}   
\nonumber \\[1ex]
&\ds\quad -\, (\partial_t p_p(0),\div(\bu_p)(0))_{\Omega_p}
+ (\f_f(0),\partial_t\bu_f(0))_{\Omega_f}
+ (\f_p(0), \partial_{tt}\bbeta_p(0))_{\Omega_p} 
+ (q_p(0),\partial_t p_p(0))_{\Omega_p} \,.
\end{align}
Then, applying Cauchy--Schwarz and Young's inequalities with appropriate weights on the right-hand side of \eqref{eq:bound-unsteady-state-solution-11}, we obtain
\begin{align*}%\label{eq:bound-unsteady-state-solution-12}
& \ds \|\partial_t\bu_f(0)\|^2_{\bL^2(\Omega_f)} 
+ s_0\,\| \partial_t p_p(0)\|^2_{\W_p} 
+ \| \partial_{tt}\bbeta_p(0)\|^2_{\bL^2(\Omega_p)}
\,\leq\, C\bigg( \|\bdiv(A^{-1}(\, \be(\bbeta_{p})))(0)\|^2_{\bL^2(\Omega_p)} 
\nonumber \\
&\ds\quad +\, \|p_{p}(0)\|^2_{\H^1(\Omega_p)}
+ \|\bdiv(\bsi_{f}(0))\|^2_{\bL^2(\Omega_f)}
+ \frac{1}{s_0}\,\|\partial_t\,\bbeta_p(0)\|^2_{\bV_p} 
+ \frac{1}{s_0}\,\|\div(\bu_p)(0)\|^2_{\bL^2(\Omega_p)}   
+ \|\f_f(0) \|^2_{\bL^2(\Omega_f)}
\nonumber \\
&\ds\quad +\, \|\f_p(0)\|^2_{\bL^2(\Omega_p)} 
+ \frac{1}{s_0}\|q_p(0)\|^2_{\L^2(\Omega_p)}\Bigg)
+ \delta\,\Bigg( \|\partial_t\bu_f(0)\|^2_{\bL^2(\Omega_f)} 
+ s_0\,\| \partial_t p_p(0)\|^2_{\W_p} 
+ \|\partial_{tt}\bbeta_p(0)\|^2_{\bL^2(\Omega_p)}\Bigg)\,,
\end{align*}
and taking $\delta$ small enough, using \eqref{eq:sol0-up0-pp0} and \eqref{eq:sol0-sigmaf0-uf0-gammaf0} to bound $\div(\bu_p)(0)$ and $\bdiv(\bsi_{f})(0)$, respectively, implies 
\begin{align}\label{eq:bound-unsteady-state-solution-13}
&\ds \|\partial_t\bu_f(0)\|^2_{\bL^2(\Omega_f)} 
+ s_0\,\|\partial_t p_p(0)\|^2_{\W_p} 
+ \| \partial_{tt}\bbeta_p(0)\|^2_{\bL^2(\Omega_p)} 
\nonumber \\
&\ds\quad \leq\, C\Bigg( \|\bdiv(\be(\bu_{f,0}))\|^2_{\bL^{2}(\Omega_f)}
+ \|\bdiv(\bu_{f,0}\otimes \bu_{f,0})\|^2_{\bL^2(\Omega_f)}
+ \|\bdiv(A^{-1}(\, \be(\bbeta_{p,0})))\|^2_{\bL^2(\Omega_p)} 
+ \|p_{p,0}\|^2_{\H^1(\Omega_p)}
\nonumber \\ 
&\ds\quad \quad+\, \frac{1}{s_0}\|\bu_{s,0}\|^2_{\bV_p} 
+ \frac{1}{s_0}\|\div(\bK\nabla p_{p,0})\|^2_{\bL^2(\Omega_p)} 
+ \|\f_f(0) \|^2_{\bL^2(\Omega_f)} 
+ \|\f_p(0)\|^2_{\bL^2(\Omega_p)} 
+ \frac{1}{s_0}\|q_p(0)\|^2_{\L^2(\Omega_p)}\Bigg)\,.
\end{align}

Finally, combining \eqref{eq:bound-unsteady-state-solution-6}, \eqref{eq:bound-unsteady-state-solution-10}, and \eqref{eq:bound-unsteady-state-solution-13}, using the Sobolev embedding of $\H^1(0,T)$ into $\L^\infty(0,T)$, and applying Lemma \ref{xing-lemma} in the context of the non-negative functions $H=\big( \|\bbeta_p\|^2_{\bV_p} + \|\partial_{tt}\bbeta_p\|^2_{\bL^2(\Omega_p)}\big)^{1/2}$ and $B=\|\partial_{t}\,\f_p\|_{\bL^2(\Omega_p)}$, with $R$ and $A$ representing the remaining terms, to control $\int^t_0 B(s)\,H(s)\,ds$, along with \eqref{eq:bound-unsteady-state-solution-5} and the Sobolev embedding of $\L^\infty(0,T)$ into $\L^2(0,T)$,
\begin{equation}\label{sobolev-L-infty-L2}
\|\Phi\|_{\L^2(0,T)} \,\leq\, \sqrt{T}\,\|\Phi\|_{\L^\infty(0,T)} \,,
\end{equation}
for any scalar or vector-valued function $\Phi$ to bound the terms $\|\bdiv(\bsi_f)\|_{\bL^{4/3}(\Omega_f)}$ and $\|\div(\bu_{p})\|_{\L^2(\Omega_p)}$ in $\L^2(0,T)$, along with their corresponding right-hand side in $\L^\infty(0,T)$. Then, applying \eqref{eq:tau-d-H0div-inequality}--\eqref{eq:tau-H0div-Xf-inequality} and performing some algebraic manipulations, we obtain \eqref{eq:continuous-stability}, completing the proof.
\end{proof}

%**********************************************************************
%**********************************************************************

\section{Semidiscrete continuous-in-time approximation}\label{sec:Semidiscrete continuous-in-time approximation}

In this section we introduce and analyze the semidiscrete continuous-in-time approximation of \eqref{eq:continuous-weak-formulation-1}. 
We analyze its solvability by employing the strategy developed in Section \ref{sec:well-posedness-model}. 
In addition, we derive error estimates with rates of convergence. 

Let $\cT_h^f$ and $\cT_h^p$ be shape-regular and quasi-uniform affine finite element partitions of $\Omega_f$ and $\Omega_p$, respectively, where $h$ is the maximum element diameter. The two partitions may be non-matching along the interface $\Gamma_{fp}$. For the discretization, we consider the following conforming finite element spaces:
\begin{equation*}
\bbX_{fh}\times \bV_{fh}\times \bbQ_{fh}\subset \bbX_f \times \bV_f\times \bbQ_{f}, \quad
\bV_{ph}\subset\bV_p, \quad \bX_{ph}\times\W_{ph} \subset \bX_p\times\W_p \,.
\end{equation*}
%
We choose $(\bbX_{fh}, \bV_{fh}, \bbQ_{fh})$ to be any stable finite element spaces for mixed elasticity with weakly imposed stress symmetry, such as the
Amara--Thomas \cite{at1979}, PEERS \cite{abd1984}, Stenberg
\cite{stenberg1988}, Arnold--Falk--Winther \cite{afw2007,awanou2013},
or Cockburn--Gopalakrishnan--Guzman \cite{cgg2010} families of spaces. 
We take $(\bX_{ph},\W_{ph})$ to be any stable mixed finite element Darcy spaces, such as the Raviart--Thomas (RT) or Brezzi--Douglas--Marini (BDM) spaces \cite{Brezzi-Fortin}. 
%We employ a conforming finite element space $\bV_{ph} \subset \bV_p$ to approximate the structure displacement. 
We note that these spaces satisfy
\begin{equation}\label{eq: div-prop} 
\div(\bX_{ph})=\W_{ph}, \quad
\bdiv(\bbX_{fh}) = \bV_{fh}.
\end{equation}
%
For the Lagrange multipliers, we choose the conforming approximations
%
\begin{equation}\label{defn-Lambda-h}
\Lambda_{ph} \subset \Lambda_{p}\,,\quad 
\bLambda_{fh} \subset \bLambda_{f}\,,
\end{equation}
%
equipped with $\H^{1/2}$-norms as in \eqref{eq:H1/2-norms}. If the normal traces of the spaces $\bX_{ph},$ or $\bbX_{fh}$ contain piecewise polynomials in $\cP_k$ on simplices or $\cQ_k$ on cube with $k \geq 1$, where $\cP_k$ denotes polynomials of total degree $k$ and $\cQ_k$ stands for polynomials of degree $k$ in each variable, we take the Lagrange multiplier spaces to be continuous piecewise polynomials in $\cP_k$ or $\cQ_k$ on the traces of the corresponding subdomain grids. In the case $k = 0$, we take the Lagrange multiplier spaces to be continuous piecewise polynomials in $\cP_1$ or $\cQ_1$ on grids obtained by coarsening by two traces of the subdomain grids. Note that these choices guarantee the inf-sup conditions given below in Lemma ~\ref{lem: discrete inf-sup}.


The semidiscrete continuous-in-time approximation to \eqref{eq:continuous-weak-formulation-1a} is: find $(\bsi_{fh},\bu_{ph}, \bbeta_{ph}, \bu_{fh}, p_{ph}, \bgamma_{fh}, \bvarphi_{h},$ $\lambda_{h}) : [0,T]\to \bbX_{fh}\times \bX_{ph}\times \bV_{ph}\times \bV_{fh}\times \W_{ph}\times \bbQ_{fh}\times \bLambda_{fh}\times \Lambda_{ph}$, such that for a.e. $t\in (0,T)$:
\begin{align}
& \rho_f (\partial_t\,\bu_{fh},\bv_{fh})_{\Omega_f}+ a_f(\bsi_{fh},\btau_{fh})+b_{\bn_f}(\btau_{fh},\bvarphi_{h}) + b_f(\btau_{fh},\bu_{fh}) + b_\sk(\bgamma_{fh},\btau_{fh})+\kappa_{\bu_{fh}}(\bu_{fh}, \btau_{fh}) \nonumber \\ 
&\quad -\, b_f(\bsi_{fh},\bv_{fh}) - b_\sk(\bsi_{fh},\bchi_{fh})  \, = \, (\f_{f},\bv_{fh})_{\Omega_f}\,, \nonumber \\ 
& \rho_p(\partial_{tt}\bbeta_{ph},\bxi_{ph})_{\Omega_p} + a^e_p(\bbeta_{ph},\bxi_{ph})+ \alpha_p\,b_p(\bxi_{ph},p_{ph}) +c_{\BJS}(\partial_t\,\bbeta_{ph}, \bvarphi_{h};\bxi_{ph}, \bpsi_{h})- c_{\Gamma}(\bxi_{ph},\bpsi_{h};\lambda_{h}) \nonumber \\ 
&\quad -\,b_{\bn_f}(\bsi_{fh},\bpsi_{h})+l_{\bvarphi_{h}}(\bvarphi_{h},\bpsi_{h}) \, = \,(\f_{p},\bxi_{ph})_{\Omega_p}\,, \nonumber \\ 
& s_0 (\partial_t\,p_{ph},w_{ph})_{\Omega_p} +a^d_p(\bu_{ph},\bv_{ph}) +b_p(\bv_{ph},p_{ph})+b_{\bn_p}(\bv_{ph},\lambda_{h})  - \alpha_p\,b_p(\partial_t\,\bbeta_{ph},w_{ph})- b_p(\bu_{ph},w_{ph}) \nonumber \\ 
&\quad = \,(q_{p},w_{ph})_{\Omega_p}\,, \nonumber \\ 
& c_{\Gamma}(\partial_t\,\bbeta_{ph},\bvarphi_{h};\xi_{h})-b_{\bn_p}(\bu_{ph},\xi_{h})\, = \,0\,,  
\label{eq:NS-Biot-semiformulation-1}
\end{align}
for all $(\btau_{fh}, \bv_{ph}, \bxi_{ph}, \bv_{fh}, w_{ph}, \bchi_{fh}, \bpsi_{h}, \xi_{h})\in \bbX_{fh}\times \bX_{ph}\times \bV_{ph}\times \bV_{fh}\times \W_{ph}\times \bbQ_{fh}\times \bLambda_{fh}\times \Lambda_{ph}$. The initial conditions $\bu_{fh}(0),p_{ph}(0),\bbeta_{ph}(0)$ and $\partial_t\,\bbeta_{ph}(0)$ are chosen as suitable approximations of $\bu_{f,0},p_{p,0},\bbeta_{p,0}$ and $\bu_{s,0}$ such that $(\bu_{fh}(0),p_{ph}(0),\bbeta_{ph}(0),\partial_t\,\bbeta_{ph}(0))$ are compatible initial data. 

To establish the well-posedness of the semi-discrete formulation \eqref{eq:NS-Biot-semiformulation-1}, we follow the same strategy employed for its continuous counterpart.
For analytical purposes only, we consider a discretization of the weak formulation \eqref{eq:NS-Biot-formulation-2}.
Let $\bV_{ph}$ consist of polynomials of degree at most $s_{\bbeta_p}\geq 1$. We introduce the stress finite element space $\bSigma_{eh} \subset \bSigma_e$, as symmetric tensors with elements that are discontinuous polynomials of degree at most $s_{\bbeta_p}-1$:
\begin{equation*}
\bSigma_{eh} \,:=\, \Big\{ \bsi_e\in \bSigma_{e} :\quad\bsi_e|_{T\in \cT_h^p}\in \cP^{sym}_{s_{\bbeta_p}-1}(T)^{d\times d} \Big\}.
\end{equation*}
Then the corresponding semi-discrete formulation is: find $(\bsi_{fh}, \bu_{ph}, \bsi_{eh}, \bu_{fh}, p_{ph}, \bgamma_{fh}, \bu_{sh}, \bvarphi_{h}, \lambda_{h}) : [0,T]\to \bbX_{fh}\times \bX_{ph}\times \bSigma_{eh}\times \bV_{fh}\times \W_{ph}\times \bbQ_{fh}\times \bV_{ph}\times \bLambda_{fh}\times \Lambda_{ph}$, such that for a.e. $t\in (0,T)$:
\begin{align}
& a_f(\bsi_{fh},\btau_{fh}) + \kappa_{\bu_{fh}}(\bu_{fh}, \btau_{fh}) + a^d_p(\bu_{ph},\bv_{ph}) + a^s_p(\partial_t\,\bsi_{eh},\btau_{eh}) \nonumber\\ 
&\quad +\, b_f(\btau_{fh},\bu_{fh}) + b_p(\bv_{ph},p_{ph}) + b_\sk(\btau_{fh},\bgamma_{fh})
+ b_s(\btau_{eh},\bu_{sh}) + b_{\bn_f}(\btau_{fh},\bvarphi_{h}) + b_{\bn_p}(\bv_{ph},\lambda_{h})  
\,=\, 0\,, \nonumber\\ 
& \rho_f\,(\partial_t\,\bu_{fh},\bv_{fh})_{\Omega_f} + s_0\,(\partial_t\,p_{ph},w_{ph})_{\Omega_p}
+ \rho_p\,(\partial_t\bu_{sh},\bv_{sh})_{\Omega_p} \nonumber\\ 
&\quad +\, c_{\BJS}(\bu_{sh},\bvarphi_{h};\bv_{sh},\bpsi_{h}) + c_{\Gamma}(\bu_{sh},\bvarphi_{h};\xi_{h}) - c_{\Gamma}(\bv_{sh},\bpsi_{h};\lambda_{h}) 
+ \alpha_p\,b_p(\bv_{sh},p_{ph}) - \alpha_p\,b_p(\bu_{sh},w_{ph}) \nonumber\\  
&\quad -\, b_f(\bsi_{fh},\bv_{fh}) - b_p(\bu_{ph},w_{ph}) - b_\sk(\bsi_{fh},\bchi_{fh})  
- b_s(\bsi_{eh},\bv_{sh}) - b_{\bn_f}(\bsi_{fh},\bpsi_{h}) - b_{\bn_p}(\bu_{ph},\xi_{h}) \nonumber\\  
&\quad +\, l_{\bvarphi_{h}}(\bvarphi_{h},\bpsi_{h}) =\, (\f_f,\bv_{fh})_{\Omega_f} + (q_p,w_{ph})_{\Omega_p} + (\f_p,\bv_{sh})_{\Omega_p}\,, 
\label{eq:NS-Biot-semi-formulation-2}
\end{align}
for all
$(\btau_{fh}, \bv_{ph}, \btau_{eh}, \bv_{fh}, w_{ph}, \bchi_{fh}, \bv_{sh}, \bpsi_{h}, \xi_{h})\in \bbX_{fh}\times \bX_{ph}\times \bSigma_{eh}\times \bV_{fh}\times \W_{ph}\times \bbQ_{fh}\times \bV_{ph}\times \bLambda_{fh}\times \Lambda_{ph}$. 
The initial conditions $\bu_{fh}(0)$, $p_{ph}(0)$, $\bu_{sh}(0)$, and $\bsi_{eh}(0)$ are approximations of $\bu_{f,0}$, $p_{p,0}$, $\bu_{s,0}$, and $\bsi_{e,0}$, respectively, chosen to ensure compatibility as initial data.

Now, we group the spaces, unknowns and test functions similarly to the continuous case:
\begin{equation*}
\begin{array}{c}
\ds \bQ_h := \bbX_{fh}\times \bX_{ph}\times \bSigma_{eh},\quad
\bS_h := \bV_{fh}\times \W_{ph}\times \bbQ_{fh}\times \bV_{ph}\times \bLambda_{fh}\times \Lambda_{ph}, \\ [1.5ex]
\ds \ubsi_h := (\bsi_{fh}, \bu_{ph}, \bsi_{eh})\in \bQ_h,\quad 
\ubu_h := (\bu_{fh}, p_{ph}, \bgamma_{fh}, \bu_{sh}, \bvarphi_{h}, \lambda_{h})\in \bS_h, \\[1ex]
\ds \ubtau_h := (\btau_{fh}, \bv_{ph}, \btau_{eh})\in \bQ_h,\quad 
\ubv_h := (\bv_{fh}, w_{ph}, \bchi_{fh}, \bv_{sh}, \bpsi_{h}, \xi_{h})\in \bS_h,
\end{array}
\end{equation*}
%
where the spaces $\bQ_h$ and $\bS_h$ are respectively endowed with the norms
\begin{equation*}
\begin{array}{ll}
\|\ubtau_h\|^2_{\bQ_h} & = \, \|\btau_{fh}\|^2_{\bbX_{f}} + \|\bv_{ph}\|^2_{\bX_{p}}+ \|\btau_{eh}\|^2_{\bSigma_{e}}\,, \\[1ex]
\|\ubv_h\|^2_{\bS_h} & = \, \|\bv_{fh}\|^2_{\bV_{f}} + \|w_{ph}\|^2_{\W_{p}} + \|\bchi_{fh}\|^2_{\bbQ_{f}} + \|\bv_{sh}\|^2_{\bV_{p}} + \|\bpsi_{h}\|^2_{\bLambda_{f}} + \|\xi_{h}\|^2_{\Lambda_{p}}\,.
\end{array}
\end{equation*}
Thus, the semidiscrete continuous-in-time approximation to \eqref{eq:alternative-formulation-operator-form} (cf. \eqref{eq:NS-Biot-semi-formulation-2}) reads: find $(\ubsi_{h},\ubu_{h}): [0,T]\to \bQ_{h}\times \bS_{h}$ such that for all $(\ubtau_h,\ubv_{h}) \in \bQ_{h}\times \bS_{h}$, and for a.e. $t\in (0,T),$
\begin{align}
\frac{\partial}{\partial\,t}\,\cE_1(\ubsi_{h})(\ubtau_{h}) + \cA(\ubsi_{h})(\ubtau_{h}) + \cB'(\ubu_{h})(\ubtau_{h}) + \cK_{\bu_{fh}}(\ubu_{h})(\ubtau_{h}) &= \bF(\ubtau_{h})\,, \nonumber\\ 
\frac{\partial}{\partial\,t}\,\cE_2(\ubu_{h})(\ubv_{h}) - \cB(\ubsi_{h})(\ubv_{h}) + \cC(\ubu_{h})(\ubv_{h})+\cL_{\bvarphi_{h}}(\ubu_{h})(\ubv_{h}) &= \bG(\ubv_{h})\,.
\label{eq:alternative-discrete-formulation-operator-form}
\end{align}

We observe that the well-posedness of \eqref{eq:alternative-discrete-formulation-operator-form} (cf. \eqref{eq:NS-Biot-semi-formulation-2}) follows by using Theorem \ref{thm:auxiliary-theorem} and similar arguments to the ones employed in Section \ref{sec:well-posedness-model}.
Indeed, we first need to solve the resolvent system: find $(\ubsi_{h},\ubu_{h})\in \bQ_{h}\times \bS_{h}$, such that
\begin{align}
\cE_1(\ubsi_{h})(\ubtau_{h}) + \cA(\ubsi_{h})(\ubtau_{h}) + \cB'(\ubu_{h})(\ubtau_{h}) + \cK_{\bu_{fh}}(\ubu_{h})(\ubtau_{h}) &= \wh{\bF}(\ubtau_{h}), \nonumber\\ 
\cE_2(\ubu_{h})(\ubv_{h}) - \cB(\ubsi_{h})(\ubv_{h}) + \cC(\ubu_{h})(\ubv_{h})+\cL_{\bvarphi_{h}}(\ubu_{h})(\ubv_{h}) &=  \wh{\bG}(\ubv_{h}),\label{eq:T-auxiliary-discrete-problem-operator-A}
\end{align}
for all $(\ubtau_{h},\ubv_{h})\in \bQ_{h}\times \bS_{h}$ where $\wh{\bF} \in \bQ_{h}'$ and $\wh{\bG} \in \bS_{h}'$ are such that
%
\begin{align*}
& \wh{\bF}(\ubtau_{h}) \,:=\, (\wh{\f}_{eh},\btau_{eh})_{\Omega_p} \quad \forall\,\ubtau_{h} \in \bQ_{h}\,, \nonumber\\ 
& \wh{\bG}(\ubv_{h}) \,:=\, (\wh{\f}_{fh},\bv_{fh})_{\Omega_f} + (\wh{q}_{ph},w_{ph})_{\Omega_p} + (\wh{\f}_{ph},\bv_{sh})_{\Omega_p} \quad \forall\,\ubv_{h} \in \bS_{h}\,.
\nonumber
\end{align*}
To that end, we let $\bT_{\tt d}$ be discrete version of the operator $\bT$ defined in \eqref{eq:definition-operator-T} and proceed as in Sections \ref{sec:fixed-point-approach}--\ref{sec:domain-D-nonempty}.
We observe that the continuity of all bilinear forms in the discrete case follows directly from their continuous counterparts (cf. Lemma \ref{lem:cont}).
In addition, the discrete inf-sup conditions satisfied by the finite element spaces are established in the following lemma.
%
\begin{lem}\label{lem: discrete inf-sup}
There exist constants $\wt{\beta}_1, \wt{\beta}_2, \wt{\beta}_3 > 0$ such that
\begin{equation}\label{eq:discrete inf-sup-vs}
\sup_{\0\neq \btau_{eh}\in \bSigma_{eh}} \frac{b_s(\btau_{eh},\bv_{sh})}{\|\btau_{eh}\|_{\bSigma_{e}}} 
\geq \wt{\beta}_1\,\|\bv_{sh}\|_{\bV_{p}} \quad \forall\,\bv_{sh}\in \bV_{ph} \,,
\end{equation}
%
\begin{equation}\label{eq:discrete inf-sup-qp-xi}
\sup_{\0\neq \bv_{ph}\in \bX_{ph}} \frac{b_p(\bv_{ph},w_{ph}) + b_{\bn_p}(\bv_{ph},\xi_{h})}{\|\bv_{ph}\|_{\bX_{p}}} 
\geq \wt{\beta}_2\,\|(w_{ph},\xi_{h})\|_{\W_{p}\times \Lambda_{p}} \quad \forall\,(w_{ph},\xi_{h})\in \W_{ph}\times \Lambda_{ph} \,,
\end{equation}
and 
\begin{equation}\label{eq:discrete inf-sup-vf-chif}
\sup_{\0\neq \btau_{fh}\in \bbX_{fh}} \frac{B_f(\btau_{fh},(\bv_{fh},\bchi_{fh},\bpsi_{h}))}{\|\btau_{fh}\|_{\bbX_{f}}} 
\geq \wt{\beta}_3\,\|(\bv_{fh}, \bchi_{fh}, \bpsi_{h})\|_{\bV_{f}\times \bbQ_{f}\times \bLambda_{f}}\,,
\end{equation}
for all $(\bv_{fh},\bchi_{fh},\bpsi_{h})\in \bV_{fh}\times \bbQ_{fh}\times \bLambda_{fh}$ and $B_f$ defined in \eqref{eq:inf-sup-vf-chif}.
\end{lem}
%
\begin{proof}
For the proof of \eqref{eq:discrete inf-sup-vs} we refer the reader to \cite[eq. (5.18) in Theorem 5.1]{aeny2019}.
By combining a slight adaptation of \cite[eqs. (4.13) and (4.22)]{gos2011}, we deduce \eqref{eq:discrete inf-sup-qp-xi}. 
Finally, for the proof of \eqref{eq:discrete inf-sup-vf-chif}, we again utilize the diagonal character of $B_f$ and show individual discrete inf-sup conditions for $b_f, b_\sk$ and $b_{\bn_f}$. 
Specifically, the inf-sup conditions for $b_f$ and $b_\sk$ follow by combining \cite[eqs. (4.28) and (4.29) in Section 4.4.2]{gobs2021}, while observing that $\|\btau_{fh}\|_{\bbX_{f}} \leq C\,\|\btau_{fh}\|_{\bbH(\bdiv;\Omega_{f})}$, we have
\begin{equation*}
\sup_{\0\neq \btau_{fh}\in \bbX_{fh}} \frac{b_{\bn_f}(\btau_{fh},\bpsi_h)}{\|\btau_{fh}\|_{\bbX_{f}}}
\geq \sup_{\0\neq \btau_{fh}\in \bbX_{fh}} \frac{b_{\bn_f}(\btau_{fh},\bpsi_h)}{\|\btau_{fh}\|_{\bbH(\bdiv;\Omega_{f})}} \,,
\end{equation*}
which, together with \cite[eqs. (5.26)--(5.29)]{gmor2014}, yields the inf-sup condition for $b_{\bn_f}$ and completes the proof.
\end{proof}

Thus, defining $\beta_{d}:= \min\left\{ \wt{\beta}_1, \wt{\beta}_2, \wt{\beta}_3\right\}$ (cf. \eqref{eq:discrete inf-sup-vs}, \eqref{eq:discrete inf-sup-qp-xi}, \eqref{eq:discrete inf-sup-vf-chif}), 
and the radii
\begin{equation}\label{eq:discrete r_1^0 r_2^0 defn}
r_{1\ttd}^0:= \frac{\mu\,\beta_{\ttd}}{3\,\rho_f\,n^{1/2}} \qan 
r_{2\ttd}^0 := \frac{\mu \beta_{\ttd}^2}{6\,C_{\cL}}\,,
\end{equation}
and proceeding as in its discrete counterparts Lemma \ref{thm:well-posedness-1},
we are able to obtain the uniqueness of solution $(\ubsi_{h},\ubu_{h})\in \bQ_{h}\times \bS_{h}$ of the problem \eqref{eq:T-auxiliary-discrete-problem-operator-A}.
The aforementioned result is stated next.
%
\begin{lem}\label{thm:discrete-well-posedness-1}
For each $\wh{\f}_{fh}\in \bL^2(\Omega_f), \wh{\f}_{ph}\in \bL^2(\Omega_p), \wh{\f}_{eh} \in \bbL^2(\Omega_p)$, and $\wh{q}_{ph}\in \L^2(\Omega_p)$, the problem \eqref{eq:T-auxiliary-discrete-problem-operator-A} has a unique solution $(\ubsi_{h},\ubu_{h})\in \bQ_{h}\times \bS_{h}$ for each $(\bw_{fh},\bzeta_{h})\in \bV_{fh}\times \bLambda_{fh}$ such that $\|\bw_{fh}\|_{\bV_f} \leq r_{1\ttd}^0$ and $\|\bzeta_{h}\|_{\bLambda_f} \leq r_{2\ttd}^0$ (cf. \eqref{eq:discrete r_1^0 r_2^0 defn}).
Moreover, there exists a constant $c_{\bT_{\ttd}} > 0$, independent of $\bw_{fh},\bzeta_{h}$ and the data $\wh{\f}_{fh}, \wh{\f}_{ph}, \wh{\f}_{eh} $, and $\wh{q}_{ph}$, such that
\begin{equation*}%\label{eq:discrete-ubsi-ubu-bound-solution}
\, \|(\ubsi_{h}, \ubu_{h})\|_{\bQ\times \bS} 
\,\leq\, {c_{\bT_{\ttd}}\,\Big\{ \|\wh{\f}_{fh}\|_{\bL^2(\Omega_f)} +  \|\wh{\f}_{eh}\|_{\bbL^2(\Omega_p)}+ \|\wh{\f}_{ph}\|_{\bL^2(\Omega_p)} + \| \wh{q}_{ph}\|_{\L^2(\Omega_p)}\Big\}}.
\end{equation*}
\end{lem}
As an immediate consequence we have the following corollary.
\begin{cor}\label{eq:T-discrete-well-delfined}
Assume that the conditions of Lemma ~\ref{thm:discrete-well-posedness-1} are satisfied. The operator $\bT_{\ttd}$ is well defined and it satisfies
\begin{equation*}%\label{eq:operator T-discrete-bound-solution}
\|\bT_{\tt d}(\bw_{fh},\bzeta_{h})\|_{\bV_f\times \bLambda_f} 
\,\leq\, {c_{\bT_{\tt d}}\,\Big\{ \|\wh{\f}_{fh}\|_{\bL^2(\Omega_f)} +  \|\wh{\f}_{eh}\|_{\bbL^2(\Omega_p)}+ \|\wh{\f}_{ph}\|_{\bL^2(\Omega_p)} + \| \wh{q}_{ph}\|_{\L^2(\Omega_p)} \Big\}}.
\end{equation*}
\end{cor}

We continue by establishing the existence and uniqueness of a fixed-point of operator $\bT_{\tt d}$ by means of the well known Banach fixed-point theorem which implies that \eqref{eq:T-auxiliary-discrete-problem-operator-A} has unique solution. We state the following lemma for $\bT_{\tt d}$ to be a continuous operator. 
The proof employ similar arguments to the ones used in the proof of Lemma \ref{lem:T-contraction-mapping}.
%
\begin{lem}\label{lem:T-discrete-contraction-mapping}
Let $r_{1\ttd}\in (0,r_{1\ttd}^0]$ and $r_{2\ttd}\in (0,r_{2\ttd}^0]$, where $r_{1\ttd}^0$ and $r_{2\ttd}^0$ are defined in \eqref{eq:discrete r_1^0 r_2^0 defn}. Let $\bW_{r_{1\ttd},r_{2\ttd}}$ be the closed set defined by
\begin{equation}\label{eq:Wr-discrete-definition}
\bW_{r_{1\ttd},r_{2\ttd}} := \Big\{ (\bw_{fh},\bzeta_{h})\in \bV_{fh}\times \bLambda_{fh} \,:\quad \| \bw_{fh}\|_{\bV_f} \leq r_{1\ttd},\quad \|\bzeta_{h}\|_{\bLambda_f} \leq r_{2\ttd} \Big\} \,,
\end{equation}
and define $r_\ttd^0:= \min\big\{r_{1\ttd}^0, r_{2\ttd}^0\big\}$. 
Then, for all $(\bw_{fh},\bzeta_{h}), (\wt{\bw}_{fh},\wt{\bzeta}_{h}) \in \bW_{r_{1\ttd},r_{2\ttd}}$ there holds 
\begin{align*}
&\|\bT_{\tt d}(\bw_{fh},\bzeta_{h}) - \bT_{\tt d}(\wt{\bw}_{fh},\wt{\bzeta}_{h}) \|_{\bV_f\times \bLambda_f}  \nonumber \\[1ex]
&\quad \leq\, \frac{c_{\bT_{\tt d}}}{r_\ttd^0}\,\Big\{ \|\wh{\f}_{fh}\|_{\bL^2(\Omega_f)} +  \|\wh{\f}_{eh}\|_{\bbL^2(\Omega_p)}+ \|\wh{\f}_{ph}\|_{\bL^2(\Omega_p)} + \| \wh{q}_{ph}\|_{\L^2(\Omega_p)} \Big\}\|(\bw_{fh},\bzeta_{h}) - (\wt{\bw}_{fh},\wt{\bzeta}_{h})\|_{\bV_f\times \bLambda_f}\,. %\label{eq:discrete-Lipschitz-continuity}
\end{align*}
\end{lem}

According to the above and mimicking the arguments employed in Theorem 
\ref{thm:well-posed-domain-D} we are able to obtain the following well-posedness result for the discrete resolvent problem \eqref{eq:T-auxiliary-discrete-problem-operator-A}. 
%
\begin{thm}\label{thm:well-posed-discrete}
Let $\bW_{r_{1\ttd},r_{2\ttd}}$ be as in \eqref{eq:Wr-discrete-definition} and let $r_\ttd:=\min\big\{r_{1\ttd}, r_{2\ttd}\big\}$. 
Assume that the data satisfy
\begin{equation*}%\label{eq:discrete-T-maps-Wr-into-Wr}
c_{\bT_{\tt d}}\,\Big\{ \|\wh{\f}_{fh}\|_{\bL^2(\Omega_f)} 
+ \|\wh{\f}_{eh}\|_{\bbL^2(\Omega_p)}
+ \|\wh{\f}_{ph}\|_{\bL^2(\Omega_p)} 
+ \|\wh{q}_{ph}\|_{\L^2(\Omega_p)} \Big\} \,\leq\, r_\ttd\,,
\end{equation*}	
with $c_{\bT_{\tt d}}>0$, independent of $\bw_{fh},\bzeta_{h}$.
Then, the discrete resolvent problem \eqref{eq:T-auxiliary-discrete-problem-operator-A} has a unique solution $(\ubsi_{h},\ubu_{h})\in \bQ_{h}\times \bS_{h}$ with $(\bu_{fh},\bvarphi_{h})\in \bW_{r_{1\ttd},r_{2\ttd}}$, and there holds
%
\begin{equation*}%\label{eq:discrete-bound-solution-steady-state}
\|(\ubsi_{h},\ubu_{h})\|_{\bQ\times \bS} 
\,\leq\, c_{\bT_{\tt d}}\,\Big\{ \|\wh{\f}_{fh}\|_{\bL^2(\Omega_f)} 
+  \|\wh{\f}_{eh}\|_{\bbL^2(\Omega_p)}
+ \|\wh{\f}_{ph}\|_{\bL^2(\Omega_p)} 
+ \|\wh{q}_{ph}\|_{\L^2(\Omega_p)}\Big\}\,.
\end{equation*} 
\end{thm}

Next, we continue with the construction of the discrete initial data.
To that end, we first consider the following auxiliary resolvent system: find $(\ubsi_{h},\ubu_{h})\in \bQ_{h}\times \bS_{h}$, such that
\begin{align}
\cE_1(\ubsi_{h})(\ubtau_{h}) + \cA(\ubsi_{h})(\ubtau_{h}) + \cB'(\ubu_{h})(\ubtau_{h}) + \cK_{\bu_{fh}}(\ubu_{h})(\ubtau_{h}) 
&= \wh{\bF}(\ubtau_{h}), \nonumber\\ 
- \cB(\ubsi_{h})(\ubv_{h}) + \cC(\ubu_{h})(\ubv_{h})+\cL_{\bvarphi_{h}}(\ubu_{h})(\ubv_{h}) 
&= \wh{\bG}(\ubv_{h}),\label{eq:T-auxiliary-discrete-problem-operator-A-1}
\end{align}
for all $(\ubtau_{h},\ubv_{h})\in \bQ_{h}\times \bS_{h}$ where $\wh{\bF} \in \bQ_{h}'$ and $\wh{\bG} \in \bS_{h}'$ are such that
%
\begin{align*}
& \wh{\bF}(\ubtau_{h}) \,:=\, (\wh{\f}_{eh},\btau_{eh})_{\Omega_p} \quad \forall \ubtau_{h} \in \bQ_{h}, \nonumber\\ 
& \wh{\bG}(\ubv_{h}) \,:=\, (\wh{\f}_{fh},\bv_{fh})_{\Omega_f} + (\wh{q}_{ph},w_{ph})_{\Omega_p} + (\wh{\f}_{ph},\bv_{sh})_{\Omega_p} \quad \forall \ubv_{h} \in \bS_{h} \,.
\end{align*}
Then, as in Theorem \ref{thm:well-posed-discrete} we obtain the well-posedness result for the discrete resolvent problem \eqref{eq:T-auxiliary-discrete-problem-operator-A-1}.
%
\begin{thm}\label{thm:well-posed-discrete-1}
Let $\bW_{r_{1\ttd},r_{2\ttd}}$ be as in \eqref{eq:Wr-discrete-definition} and let $r_\ttd:=\min\big\{r_{1\ttd}, r_{2\ttd}\big\}$. 
Assume that the data satisfy
%
\begin{equation}\label{eq:discrete-T-maps-Wr-into-Wr-1}
\wt{c}_{\bT_{\ttd}}\,\Big\{ \|\wh{\f}_{fh}\|_{\bL^2(\Omega_f)} 
+ \|\wh{\f}_{eh}\|_{\bbL^2(\Omega_p)}
+ \|\wh{\f}_{ph}\|_{\bL^2(\Omega_p)} 
+ \|\wh{q}_{ph}\|_{\L^2(\Omega_p)} \Big\} \,\leq\, r_\ttd\,,
\end{equation}	
with $\wt{c}_{\bT_{\tt d}}>0$.
Then, the discrete resolvent problem \eqref{eq:T-auxiliary-discrete-problem-operator-A-1} has a unique solution $(\ubsi_{h},\ubu_{h})\in \bQ_{h}\times \bS_{h}$ with $(\bu_{fh},\bvarphi_{h})\in \bW_{r_{1\ttd},r_{2\ttd}}$, and there holds
%
\begin{equation*}%\label{eq:discrete-bound-solution-steady-state-1}
\|(\ubsi_{h},\ubu_{h})\|_{\bQ\times \bS} 
\,\leq\, \widetilde{c}_{\bT_{\ttd}}\,\Big\{ \|\wh{\f}_{fh}\|_{\bL^2(\Omega_f)} +  \|\wh{\f}_{eh}\|_{\bbL^2(\Omega_p)}+ \|\wh{\f}_{ph}\|_{\bL^2(\Omega_p)} + \|\wh{q}_{ph}\|_{\L^2(\Omega_p)}\Big\} \,.
\end{equation*} 
\end{thm}
%
The aforementioned compatible discrete initial data $(\ubsi_{h,0},\ubu_{h,0})$ is stated in the next lemma.
%
\begin{lem}\label{lem: discrete initial condition}
Assume that the conditions of Lemma \ref{lem:sol0-in-M-operator} are satisfied.
Assume in addition that there exist constants $\widetilde{C}_0, \wh{C}_0>0$ such that the data satisfy
\begin{align}
&\|\bdiv(\be(\bu_{f,0}))\|_{\bL^{2}(\Omega_f)} 
+ \|\bdiv(\bu_{f,0}\otimes \bu_{f,0})\|_{\bL^{2}(\Omega_f)} 
+ \|\bbeta_{p,0}\|_{\bV_p} 
+ \|\bdiv(A^{-1}(\be(\bbeta_{p,0})))\|_{\bL^2(\Omega_p)}    \nonumber \\
&\quad +\,\|p_{p,0}\|_{\H^1(\Omega_p)} 
+ \|\bu_{s,0}\|_{\bV_p}
+ \|\div(\bK\nabla p_{p,0})\|_{\bL^2(\Omega_p)}
\,\leq\, \frac{r_\ttd}{\wt{c}_{\bT_{\tt d}}}\,\min\left\{ \frac{1}{\wt{C}_0}, \,\frac{1}{\wh{C}_0} \right\}\,,
\label{eq:extra-assumption-discrete}
\end{align}
where $r_\ttd$ and $\widetilde{c}_{\bT_{\tt d}}$ are defined in Theorems \ref{thm:well-posed-discrete} and \ref{thm:well-posed-discrete-1}, respectively.
Then, there exist $\ubsi_{h,0} := (\bsi_{fh,0}, \bu_{ph,0},$ $\bsi_{eh,0})\in \bQ_h$ and   $\ubu_{h,0} := (\bu_{fh,0},p_{ph,0},\bgamma_{fh,0},\bu_{sh,0},\bvarphi_{h,0}, \lambda_{h,0})\in \bS_h$  with $(\bu_{fh,0},\bvarphi_{h,0})\in \bW_{r_{1\ttd},r_{2\ttd}}$ (cf. \eqref{eq:Wr-discrete-definition}) such that 
\begin{align}
\cA(\ubsi_{h,0}) + (\cB' + \cK_{\bu_{fh,0}})(\ubu_{h,0}) &= \wh\bF_{h,0} \qin \bQ_{h}'\,, \nonumber\\ 
-\,\cB(\ubsi_{h,0}) + ( \cC +\cL_{\bvarphi_{h,0}})(\ubu_{h,0}) &= \wh\bG_{h,0} \qin \bS_{h}'\,, \label{eq:initial-discrete-data-system}
\end{align}
where $\wh\bF_{h,0}(\ubtau_h) := (\wh{\f}_{eh,0},\btau_{eh})_{\Omega_p}$ and  $\wh\bG_{h,0}(\ubv_h) := (\wh{\f}_{fh,0},\bv_{fh})_{\Omega_f} + (\wh{q}_{ph,0},w_{ph})_{\Omega_p} + (\wh{\f}_{ph,0},\bv_{sh})_{\Omega_p} \,\,  \forall\,(\ubtau_{h},\ubv_{h})\in \bQ_h\times \bS_h $
with some $(\wh{\f}_{eh,0}, \wh{\f}_{fh,0}, \wh{q}_{ph,0},\wh{\f}_{ph,0})\in \bE'_b$ satisfying
%
\begin{equation}\label{eq:initial-discrete-data-bound-1}
\wt{c}_{\bT_{\tt d}}\,\Big\{ \|\wh{\f}_{fh,0}\|_{\bL^2(\Omega_f)} 
+ \|\wh{\f}_{eh,0}\|_{\bbL^2(\Omega_p)}
+ \|\wh{\f}_{ph,0}\|_{\bL^2(\Omega_p)} 
+ \|\wh{q}_{ph,0}\|_{\L^2(\Omega_p)}\Big\} \,\leq\, r_{\ttd}\,.
\end{equation}
\end{lem}
%
\begin{proof}
The discrete initial data is defined as a suitable elliptic projection of the continuous initial data $(\ubsi_0,\ubu_0)$ constructed in Lemma \ref{lem:sol0-in-M-operator}. 
First, let $(\bsi_{fh,0}, \bu_{ph,0}, \widetilde\bsi_{eh,0}, \bu_{fh,0}, p_{ph,0}, \bgamma_{fh,0}, \bu_{sh,0}, \bvarphi_{h,0}, \lambda_{h,0}) \in \bbX_{fh}\times \bX_{ph}\times \bSigma_{eh}\times \bV_{fh}\times \W_{ph}\times \bbQ_{fh}\times \bV_{ph}\times \bLambda_{fh}\times \Lambda_{ph}$ be such that, for all $(\btau_{fh}, \bv_{ph}, \btau_{eh}, \bv_{fh}, w_{ph}, \bchi_{fh}, \linebreak \bv_{sh}, \bpsi_{h}, \xi_{h})\in \bbX_{fh}\times \bX_{ph}\times \bSigma_{eh}\times \bV_{fh}\times \W_{ph}\times \bbQ_{fh}\times \bV_{ph}\times \bLambda_{fh}\times \Lambda_{ph}$,
%
\begin{subequations}\label{eq:system-discrete-sol0-1}
\begin{align}
& \ds a_f(\bsi_{fh,0},\btau_{fh}) + b_f(\btau_{fh},\bu_{fh,0}) + b_{\sk}(\btau_{fh},\bgamma_{fh,0}) + b_{\bn_f}(\btau_{fh},\bvarphi_{h,0}) + \kappa_{\bu_{fh,0}}(\bu_{fh,0},\btau_{fh})\nonumber \\[1ex] 
& \ds \quad = a_f(\bsi_{f,0},\btau_{fh}) + b_f(\btau_{fh},\bu_{f,0}) + b_{\sk}(\btau_{fh},\bgamma_{f,0}) + b_{\bn_f}(\btau_{fh},\bvarphi_{0}) + \kappa_{\bu_{f,0}}(\bu_{f,0},\btau_{fh}) = 0\,, \label{eq:semi-discrete-weak-formulation-1a}  \\[1ex] 
& \ds - b_f(\bsi_{fh,0},\bv_{fh})
= - b_f(\bsi_{f,0},\bv_{fh}) =  -(\bdiv(2\mu\be(\bu_{f,0}) - \rho_f(\bu_{f,0}\otimes \bu_{f,0})),\bv_{fh})_{\Omega_f}\,, \label{eq:semi-discrete-weak-formulation-1b}  \\[1ex] 
& \ds -\,b_{\sk}(\bsi_{fh,0},\bchi_{fh}) =  -\,b_{\sk}(\bsi_{f,0},\bchi_{fh}) = 0 \,, \label{eq:semi-discrete-weak-formulation-1c} \\[1ex]
& \ds  a^s_p(\widetilde\bsi_{eh,0},\btau_{eh}) + b_s(\btau_{eh},\bu_{sh,0})= a^s_p(\bsi_{e,0},\btau_{eh}) + b_s(\btau_{eh},\bu_{s,0})
= ( \be(\bbeta_{p,0}) - \be(\bu_{s,0}),\btau_{eh})_{\Omega_p}\,,
\label{eq:semi-discrete-weak-formulation-1d} \\[1ex] 
& \ds - b_s(\widetilde\bsi_{eh,0},\bv_{sh})  + c_{\BJS}(\bu_{sh,0},\bvarphi_{h,0};\bv_{sh},\0) - c_{\Gamma}(\bv_{sh},\0;\lambda_{h,0}) + \alpha_p\,b_p(\bv_{sh},p_{ph,0}) \nonumber \\[1ex] 
& \ds \quad =
-b_s(\bsi_{e,0},\bv_{sh})  + c_{\BJS}(\bu_{s,0},\bvarphi_{0};\bv_{sh},\0) - c_{\Gamma}(\bv_{sh},\0;\lambda_{0}) + \alpha_p\,b_p(\bv_{sh},p_{p,0}) \nonumber \\[1ex]
&\ds \quad = (- \bdiv(A^{-1}(\be(\bbeta_{p,0})) - \alpha_p\,p_{p,0}\,\bI),\bv_{sh})_{\Omega_p}\,, \label{eq:semi-discrete-weak-formulation-1e} \\[1ex] 
& \ds a^d_p(\bu_{ph,0},\bv_{ph}) + b_p(\bv_{ph},p_{ph,0}) + b_{\bn_p}(\bv_{ph},\lambda_{h,0}) \nonumber\\[1ex] 
& \ds \quad = a^d_p(\bu_{p,0},\bv_{ph}) + b_p(\bv_{ph},p_{p,0}) + b_{\bn_p}(\bv_{ph},\lambda_{0}) = 0\,, \label{eq:semi-discrete-weak-formulation-1f} \\[1ex] 
& \ds - \alpha_p b_p(\bu_{sh,0},w_{ph})  - b_p(\bu_{ph,0},w_{ph}) =
- \alpha_p b_p(\bu_{s,0},w_{ph})
- b_p(\bu_{p,0},w_{ph}) \nonumber \\[1ex] 
& \ds \quad = (\alpha_p \div (\,\bu_{s,0})
- \frac{1}{\mu}\,\div(\bK\nabla p_{p,0}),w_{ph})_{\Omega_p}\,, \label{eq:semi-discrete-weak-formulation-1g} \\[1ex] 
& \ds c_{\Gamma}(\bu_{sh,0},\bvarphi_{h,0};\xi_h) - b_{\bn_p}(\bu_{ph,0},\xi_h) = c_{\Gamma}(\bu_{s,0},\bvarphi_{0};\xi_h) - b_{\bn_p}(\bu_{p,0},\xi_h) = 0\,, \label{eq:semi-discrete-weak-formulation-1h} \\[1ex] 
& \ds c_{\BJS}(\bu_{sh,0},\bvarphi_{h,0};\0,\bpsi_h) - c_{\Gamma}(\0,\bpsi_h;\lambda_{h,0}) - b_{\bn_f}(\bsi_{fh,0},\bpsi_h) +l_{\bvarphi_{h,0}}(\bvarphi_{h,0},\bpsi_h) \nonumber \\[1ex]
& \ds \quad =  c_{\BJS}(\bu_{s,0},\bvarphi_{0};\0,\bpsi_h) - c_{\Gamma}(\0,\bpsi_h;\lambda_{0}) - b_{\bn_f}(\bsi_{f,0},\bpsi_h) +l_{\bvarphi_{0}}(\bvarphi_{0},\bpsi_h) = 0\,. \label{eq:semi-discrete-weak-formulation-1i} 
\end{align}
\end{subequations}
%
We observe that \eqref{eq:system-discrete-sol0-1} fit in the form of the auxiliary resolvent system \eqref{eq:T-auxiliary-discrete-problem-operator-A-1} with source terms
%
\begin{align*}
& \wh{\f}_{fh} =-\bdiv(2\mu\be(\bu_{f,0}) - \rho_f(\bu_{f,0}\otimes \bu_{f,0})) \,,\quad \wh{\f}_{eh} =\be(\bbeta_{p,0}) - \be(\bu_{s,0}) \,, \\ \nonumber
& \wh{\f}_{ph} =- \bdiv(A^{-1}(\be(\bbeta_{p,0})) - \alpha_p\,p_{p,0}\,\bI)  \,,\quad \wh{q}_{ph} = \alpha_p \div (\bu_{s,0})
- \frac{1}{\mu}\,\div(\bK\nabla p_{p,0}) \,,
\end{align*}
%
%which satisfy
%%
%\begin{subequations}\label{eq:discrete source term bounds}
%\begin{align}
%& \|\wh{\f}_{fh}\|_{\bL^2(\Omega_f)} 
%\,\leq\, C\,\big(\|\bdiv( \be(\bu_{f,0}))\|_{\bL^{2}(\Omega_f)} 
%+ \|\bdiv(\bu_{f,0}\otimes \bu_{f,0})\|_{\bL^2(\Omega_f)} \big), \label{eq:f_fh,0 bound} \\
%& \|\wh{\f}_{eh}\|_{\bbL^2(\Omega_p)}   
%\,\leq\, C\,\big( \|\bbeta_{p,0}\|_{\bV_p} + \|\bu_{s,0}\|_{\bV_p} \big), \label{eq:f_eh,0 bound} \\
%& \|\wh{\f}_{ph}\|_{\bL^2(\Omega_p)} 
%\,\leq\, C\,\big( \|\bdiv(A^{-1}(\be(\bbeta_{p,0})))\|_{\bL^2(\Omega_p)} 
%+ \|p_{p,0}\|_{\H^1(\Omega_p)} \big),\label{eq:f_ph,0 bound} \\
%& \|\wh{q}_{ph}\|_{\L^2(\Omega_p)}  
%\,\leq\, C\,\big( \|\div(\bK\nabla p_{p,0})\|_{\L^2(\Omega_p)}
%+ \|\bu_{s,0}\|_{\bV_p}\big) \,. \label{eq: q_ph,0 bound}
%\end{align}
%\end{subequations}
%%
%Thus, combining \eqref{eq:f_fh,0 bound}--\eqref{eq: q_ph,0 bound}, there exists $\wt{C}_0 > 0$ such that
satisfying that there exists $\wt{C}_0 > 0$ such that
\begin{align}
& \|\wh{\f}_{fh}\|_{\bL^2(\Omega_f)} + \|\wh{\f}_{eh}\|_{\bbL^2(\Omega_p)} +  \|\wh{\f}_{ph}\|_{\bL^2(\Omega_p)} + \|\wh{q}_{ph}\|_{\L^2(\Omega_p)}  \nonumber \\
&\quad \leq\, \wt{C}_0 \Big(\|\bdiv(\be(\bu_{f,0}))\|_{\bL^{2}(\Omega_f)} 
+ \|\bdiv(\bu_{f,0}\otimes\bu_{f,0})\|_{\bL^{2}(\Omega_f)} 
+ \|\bbeta_{p,0}\|_{\bV_p} \nonumber \\
&\qquad +\, \|\bdiv(A^{-1}(\be(\bbeta_{p,0})))\|_{\bL^2(\Omega_p)}
+ \|p_{p,0}\|_{\H^1(\Omega_p)} 
+ \|\bu_{s,0}\|_{\bV_p} 
+ \|\div(\bK\nabla p_{p,0})\|_{\bL^2(\Omega_p)}\Big) \,. \label{eq: discrete initial data bound}
\end{align}
Thus, \eqref{eq:extra-assumption-discrete} guarantee \eqref{eq:discrete-T-maps-Wr-into-Wr-1} and hence Theorem \ref{thm:well-posed-discrete-1} yields the well-posedness of 
\eqref{eq:system-discrete-sol0-1} and 
\begin{equation}\label{eqn:discrete-soln-1-bound}
\|(\widetilde{\ubsi}_{h,0}, \ubu_{h,0})\|_{\bQ\times \bS} 
\,\leq \widetilde{c}_{\bT_{\ttd}}\,\Big\{ \|\wh{\f}_{fh}\|_{\bL^2(\Omega_f)} 
+ \|\wh{\f}_{eh}\|_{\bbL^2(\Omega_p)}
+ \|\wh{\f}_{ph}\|_{\bL^2(\Omega_p)} + \| \wh{q}_{ph}\|_{\L^2(\Omega_p)}\Big\} \,,
\end{equation}
which together with \eqref{eq: discrete initial data bound} and \eqref{eq:extra-assumption-discrete}, implies that  $(\bu_{fh,0},\bvarphi_{h,0})\in \bW_{r_{1\ttd},r_{2\ttd}}$.

Now, let $(\bsi_{eh,0},\widetilde\bbeta_{ph,0}) \in \bSigma_{eh}\times\bV_{ph}$ be the unique solution to the mixed elasticity problem
%
\begin{subequations}\label{eq:mixed-elasticity-discrete-problem}
\begin{align}
&  a^s_p(\bsi_{eh,0},\btau_{eh}) + b_s(\btau_{eh},\widetilde\bbeta_{ph,0})= a^s_p(\bsi_{e,0},\btau_{eh}) + b_s(\btau_{eh},\bbeta_{p,0}) = 0\,, \label{eq:semi-discrete-weak-formulation-1k}  \\
& \ds - b_s(\bsi_{eh,0},\bv_{sh})  + c_{\BJS}(\bu_{sh,0},\bvarphi_{h,0};\bv_{sh},\0) - c_{\Gamma}(\bv_{sh},\0;\lambda_{h,0}) + \alpha_p\,b_p(\bv_{sh},p_{ph,0}) \nonumber \\[1ex] 
& \ds \quad = 
-b_s(\bsi_{e,0},\bv_{sh})  + c_{\BJS}(\bu_{s,0},\bvarphi_{0};\bv_{sh},\0) - c_{\Gamma}(\bv_{sh},\0;\lambda_{0}) + \alpha_p\,b_p(\bv_{sh},p_{p,0}) \nonumber \\[1ex]
& \ds \quad = -(\bdiv(A^{-1}(\be(\bbeta_{p,0})) - \alpha_p\,p_{p,0}\,\bI),\bv_{sh})_{\Omega_p}\,, \label{eq:semi-discrete-weak-formulation-1l} 
\end{align}
\end{subequations}
%
which is well-posed as a direct application of the classical Babu{\v s}ka--Brezzi theory. 
Note that $\bu_{sh,0}, \bvarphi_{h,0}$, $\lambda_{h,0}$ and $p_{ph,0}$ are data for this problem, and using \eqref{eqn:discrete-soln-1-bound} in combination with \eqref{eq: discrete initial data bound}, there holds
\begin{align}
&\|\bsi_{eh,0}\|_{\bSigma_{e}} 
+ \|\widetilde\bbeta_{ph,0}\|_{\bV_p}  \nonumber \\
&\quad \leq\, C\,\Big( \|\bu_{sh,0}\|_{\bV_p} 
+ \|\bvarphi_{h,0}\|_{\bLambda_{f}} 
+ \|\lambda_{h,0}\|_{\Lambda_{p}} 
+ \|p_{ph,0}\|_{\W_{p}} 
+ \|\bdiv(A^{-1}(\be(\bbeta_{p,0})))\|_{\bL^2(\Omega_p)}
+ \|p_{p,0}\|_{\H^1(\Omega_p)} \Big)\nonumber \\
&\quad \leq\, C\,\Big( \|\bdiv(\be(\bu_{f,0}))\|_{\bL^{2}(\Omega_f)} 
+ \|\bdiv(\bu_{f,0}\otimes \bu_{f,0})\|_{\bL^{2}(\Omega_f)} 
+ \|\bbeta_{p,0}\|_{\bV_p}  \nonumber \\
&\qquad +\, \|\bdiv(A^{-1}(\be(\bbeta_{p,0})))\|_{\bL^2(\Omega_p)}  
+ \|p_{p,0}\|_{\H^1(\Omega_p)}
+ \|\bu_{s,0}\|_{\bV_p}
+ \|\div(\bK\nabla p_{p,0})\|_{\bL^2(\Omega_p)}\Big) \,.\label{eq: discrete sige-etap-bound}
\end{align}

Finally, combining \eqref{eq:semi-discrete-weak-formulation-1a}--\eqref{eq:semi-discrete-weak-formulation-1d}, \eqref{eq:semi-discrete-weak-formulation-1f}--\eqref{eq:semi-discrete-weak-formulation-1i} and \eqref{eq:semi-discrete-weak-formulation-1l}, we obtain \eqref{eq:initial-discrete-data-system} with source terms
%
\begin{align*}
& \wh{\f}_{fh,0} = -\bdiv(2\mu\be(\bu_{f,0}) - \rho_f(\bu_{f,0}\otimes \bu_{f,0})) \,,\quad 
\wh{\f}_{eh,0} =\be(\bbeta_{p,0}) - \be(\bu_{s,0}) - A(\widetilde\bsi_{eh,0}) \,, \\ \nonumber
& \wh{\f}_{ph,0} = - \bdiv(A^{-1}(\be(\bbeta_{p,0})) - \alpha_p\,p_{p,0}\,\bI) \,,\quad 
\wh{q}_{ph,0} = \alpha_p \div (\bu_{s,0}) - \frac{1}{\mu}\,\div(\bK\nabla p_{p,0}) \,,
\end{align*}
%
%which satisfy
%%
%\begin{subequations}\label{eq:discrete source term bounds-1}
%\begin{align}
%&  \|\wh{\f}_{fh,0}\|_{\bL^2(\Omega_f)} \leq C \big(\|\bdiv( \be(\bu_{f,0}))\|_{\bL^{2}(\Omega_f)} + \|\bdiv(\bu_{f,0}\otimes\bu_{f,0})\|_{\bL^2(\Omega_f)} \big), \label{eq:f_fh,0 bound-1} \\
%& \|\wh{\f}_{eh,0}\|_{\bbL^2(\Omega_p)}   \leq C \big( \|\bbeta_{p,0}\|_{\bH^1(\Omega_p)} + \|\bu_{s,0}\|_{\bH^1(\Omega_p)} + \|\widetilde\bsi_{eh,0}\|_{\bSigma_{e}}\big), \label{eq:f_eh,0 bound-1} \\
%& \|\wh{\f}_{ph,0}\|_{\bL^2(\Omega_p)} \leq C\big( \|\bdiv(A^{-1}(\be(\bbeta_{p,0})))\|_{\bL^2(\Omega_p)} + \|p_{p,0}\|_{\bH^1(\Omega_p)} \big),\label{eq:f_ph,0 bound-1} \\
%& \|\wh{q}_{ph,0}\|_{\L^2(\Omega_p)}  \leq C\big( \|\,\div(\bK\nabla p_{p,0})\|_{\bL^2(\Omega_p)}+\|\bu_{s,0}\|_{\bH^1(\Omega_p)}\big). \label{eq: q_ph,0 bound-1}
%\end{align}
%\end{subequations}
%%
%Combining \eqref{eq:f_fh,0 bound-1}--\eqref{eq: q_ph,0 bound-1}, and using \eqref{eqn:discrete-soln-1-bound}, we obtain that there exists $\wh{C}_0 > 0$ such that
satisfying that there exists $\wh{C}_0 > 0$ such that
\begin{align}
&\|\wh{\f}_{fh,0}\|_{\bL^2(\Omega_f)} 
+ \|\wh{\f}_{eh,0}\|_{\bbL^2(\Omega_p)} 
+ \|\wh{\f}_{ph,0}\|_{\bL^2(\Omega_p)} 
+ \|\wh{q}_{ph,0}\|_{\L^2(\Omega_p)} \nonumber \\
&\ds\quad \leq\,
\wh{C}_0\,\Big(\|\bdiv(\be(\bu_{f,0}))\|_{\bL^{2}(\Omega_f)} 
+ \|\bdiv(\bu_{f,0}\otimes\bu_{f,0})\|_{\bL^{2}(\Omega_f)} 
+ \|\bbeta_{p,0}\|_{\bV_p} \nonumber \\
&\qquad +\, \|\bdiv(A^{-1}(\be(\bbeta_{p,0})))\|_{\bL^2(\Omega_p)}
+ \|p_{p,0}\|_{\H^1(\Omega_p)} 
+ \|\bu_{s,0}\|_{\bV_p}
+ \|\div(\bK\nabla p_{p,0})\|_{\bL^2(\Omega_p)}\Big) \,, 
\label{eq: discrete initial data bound-1}
\end{align}
and using the data assumption \eqref{eq:extra-assumption-discrete}, we deduce that \eqref{eq:initial-discrete-data-bound-1} holds, completing the proof.
\end{proof}

%**********************************************************************
%**********************************************************************

\subsection{Existence and uniqueness of a solution}
We define $\bbeta_{ph,0} \in \bV_{ph}$ as the unique solution to the problem
\begin{equation}\label{discrete etap0-bound}
a^e_p(\bbeta_{ph,0},\bxi_{ph}) = - b_s(\bsi_{eh,0},\bxi_{ph})\quad \forall\, \bxi_{ph} \in \bV_{ph} \,,
\end{equation}
where $\bsi_{eh,0}$ satisfy the problem \eqref{eq:mixed-elasticity-discrete-problem}. Notice that \eqref{discrete etap0-bound} is well-posed by a direct application of Lax-Milgram theorem and satisfies
\begin{equation}\label{discrete etap0-bound-1}
\|\bbeta_{ph,0}\|_{\bV_{p}} \,\leq\, C\,\|\bsi_{eh,0}\|_{\bSigma_{e}} \,.
\end{equation}
Thus, as in the continuous case \eqref{eq:etap-us-relation}, we can recover the displacement solution from 
\begin{equation*}
\bbeta_{ph}(t) = \bbeta_{ph,0} + \int_0^t \bu_{sh}(s) \, ds, \quad \forall \, t \in [0,T] \,.
\end{equation*}

Now, we establish the well-posedness of problem \eqref{eq:alternative-discrete-formulation-operator-form} and the corresponding stability bound.
%
\begin{thm}\label{thm: well-posedness main result semi}
For each $(\bu_{f,0}, p_{p,0}, \bbeta_{p,0}, \bu_{s,0}) \in \bH$ (cf. \eqref{eq:H-space-initial-condition}) and compatible discrete initial data satisfying Lemma \ref{lem: discrete initial condition}, and

\begin{equation*}
\f_f\in \W^{1,1}(0,T;\bL^2(\Omega_f)),\quad \f_p\in \W^{1,1}(0,T;\bL^2(\Omega_p)),\quad q_p\in \W^{1,1}(0,T;\L^2(\Omega_p))
\end{equation*}
under the assumptions of Theorem \ref{thm:well-posed-domain-D} (cf. \eqref{eq:T-maps-Wr-into-Wr}), there exists a unique solution of \eqref{eq:NS-Biot-semiformulation-1}, $(\bsi_{fh},\bu_{ph},\bbeta_{ph},$ 
$\bu_{fh}, p_{ph}, \bgamma_{fh}, \bvarphi_h, \lambda_h) : [0,T]\to \bbX_{fh}\times \bX_{ph}\times\bV_{ph}\times\bV_{fh}\times \W_{ph}\times \bbQ_{fh}\times \bLambda_{fh}\times \Lambda_{ph}$  with $(\bu_{fh}(t),\bvarphi_{h}(t)):[0,T]\to \bW_{r_{1\ttd},r_{2\ttd}}$. 
In addition, $\bsi_{fh}(0) = \bsi_{fh,0}, \bu_{ph}(0) = \bu_{ph,0},  \bgamma_{fh}(0) = \bgamma_{fh,0}, \bvarphi_{h}(0) = \bvarphi_{h,0}$ and $\lambda_{h}(0) = \lambda_{h,0}$. 
Moreover, assuming sufficient regularity of the data, there exists a positive constant $C$, independent of $h$ and $s_0$, such that
\begin{align}
&\|\bsi_{fh}\|_{\L^2(0,T;\bbX_f)} 
+ \|\partial_t\bsi^\rd_{fh}\|_{\L^2(0,T;\bbL^2(\Omega_f))}
+ \|\bu_{ph}\|_{\L^2(0,T;\bX_p)} 
+ \| \partial_t\bu_{ph}\|_{\L^2(0,T;\bL^2(\Omega_p))}
+ \|\bbeta_{ph}\|_{\L^\infty(0,T;\bV_p)} 
\nonumber \\
&\quad +\, \|\partial_t\bbeta_{ph}\|_{\L^\infty(0,T;\bL^2(\Omega_p))}
+ \sum^{n-1}_{j=1} \|(\bvarphi_h-\partial_t\,\bbeta_{ph})\cdot\bt_{f,j}\|_{\H^1(0,T;\L^2(\Gamma_{fp}))}
+ \|\partial_{tt}\bbeta_{ph}\|_{\L^\infty(0,T;\bL^2(\Omega_p))}
\nonumber \\[1ex]
&\quad +\, \|\bu_{fh}\|_{\H^1(0,T;\bV_f)}
+ \|\bu_{fh}\|_{\W^{1,\infty}(0,T;\bL^2(\Omega_f))}
+ \sqrt{s_0}\,\|p_{ph}\|_{\W^{1,\infty}(0,T;\W_p)}
+ \|p_{ph}\|_{\H^1(0,T;\W_p)}
\nonumber \\[1ex]
&\quad +\, \|\bgamma_{fh}\|_{\H^1(0,T;\bbQ_f)} 
+ \|\bvarphi_h\|_{\H^1(0,T;\bLambda_f)} 
+ \|\lambda_h\|_{\H^1(0,T;\Lambda_p)} 
\nonumber \\
&\leq\, C \sqrt{T}\,\Bigg( \|\f_f\|_{\H^1(0,T;\bL^2(\Omega_f))}  
+ \|\f_p\|_{\H^1(0,T;\bL^2(\Omega_p))}
+ \|q_p\|_{\H^1(0,T;\L^2(\Omega_p))}
+ \frac{1}{\sqrt{s_0}}\|q_p(0)\|_{\L^2(\Omega_p)}
\nonumber \\
& \quad +\, \|\bu_{f,0}\|_{\bL^2(\Omega_f)}
+ \|\bdiv( \be(\bu_{f,0}))\|_{\bL^{2}(\Omega_f)}
+ \|\bdiv(\bu_{f,0}\otimes \bu_{f,0})\|_{\bL^2(\Omega_f)} 
+ \sqrt{s_0}\,\|p_{p,0}\|_{\W_p}
\nonumber \\
& \quad +\, \|p_{p,0}\|_{\H^1(\Omega_p)}
+ \frac{1}{\sqrt{s_0}}\|\div(\bK\nabla p_{p,0})\|_{\bL^2(\Omega_p)}
+ \|\bbeta_{p,0}\|_{\bV_p} 
+ \|\bdiv(A^{-1}(\, \be(\bbeta_{p,0})))\|_{\bL^2(\Omega_p)} 
\nonumber \\
&\quad 
%+\, \|\bu_{s,0}\|_{\bL^2(\Omega_p)}
+\, \left(1+\frac{1}{\sqrt{s_0}}\right)\|\bu_{s,0}\|_{\bV_p}  
\Bigg) \,.
\label{eq:discrete-stability}
\end{align} 
\end{thm}
%
\begin{proof}
From the fact that $\bQ_h \subset \bQ$, $\bS_h \subset \bS,$ and $\div(\bX_{ph})=\W_{ph}, 
\bdiv(\bbX_{fh}) = \bV_{fh}$, considering $(\ubsi_{h,0},\ubu_{h,0})$ satisfying \eqref{eq:initial-discrete-data-system}, and employing the continuity properties of $\cK_{\bw_{fh}}, \cL_{\bzeta_{h}}$ (cf. \eqref{eq:continuity-cK-wf}, \eqref{eq:continuity-cL-zeta}) with $(\bu_{fh}(t),\bvarphi_{h}(t)):[0,T]\to \bW_{r_{1\ttd},r_{2\ttd}}$ (cf. \eqref{eq:Wr-discrete-definition}) and monotonicity properties of  $\cA, \cE_1, \cE_2$ and $\cC$ (cf. Lemma \ref{lem:coercivity-properties-A-E2}) and coercivity bounds in \eqref{eq:aep-coercivity-bound}, as well as discrete inf-sup conditions \eqref{eq:discrete inf-sup-vs}, \eqref{eq:discrete inf-sup-qp-xi}, and \eqref{eq:discrete inf-sup-vf-chif}, the proof is identical to the proofs of Theorems \ref{thm:unique soln} and \ref{thm: continuous stability}.
We note that the proof of Theorem \ref{thm:unique soln} works at a discrete level due to the choice of the discrete initial data as the elliptic projection of the continuous initial data (cf. \eqref{eq:system-discrete-sol0-1}, \eqref{eq:semi-discrete-weak-formulation-1l}).
Note also that the discrete version of the stability bounds \eqref{eq:bound-unsteady-state-solution-6} and \eqref{eq:bound-unsteady-state-solution-10} can be derived following the proof of Theorem \ref{thm: continuous stability}, but with the corresponding discrete initial data on the right-hand side. Thus, we need to bound
\begin{equation}\label{eq:bound-discrete-initial-solution-1}
\|\bu_{fh}(0)\|^2_{\bL^2(\Omega_f)} 
+ s_0\,\|p_{ph}(0)\|^2_{\W_p} 
+ \|\bbeta_{ph}(0)\|^2_{\bV_p} 
+ \| \partial_t\bbeta_{ph}(0)\|^2_{\bL^2(\Omega_p)}  
\end{equation}
and
\begin{equation}\label{eq:bound-discrete-initial-solution-2}
\| \partial_t\bu_{fh}(0)\|^2_{\bL^2(\Omega_f)} 
+ s_0\,\|\partial_t p_{ph}(0)\|^2_{\W_p} 
+ \|\partial_t\bbeta_{ph}(0)\|^2_{\bV_p}
+ \|\partial_{tt}\bbeta_{ph}(0)\|^2_{\bL^2(\Omega_p)} \,.
\end{equation}
in terms of the corresponding continuous initial data.
In particular, the bound for $\|\bbeta_{ph}(0)\|^2_{\bV_p}$ follows by combining \eqref{discrete etap0-bound-1} with \eqref{eq: discrete sige-etap-bound}.
Noting that $\partial_t\bbeta_{ph}(0) = \bu_{sh,0}$, the remaining terms in \eqref{eq:bound-discrete-initial-solution-1} and the third one in \eqref{eq:bound-discrete-initial-solution-2} can be bounded by \eqref{eq: discrete initial data bound}--\eqref{eqn:discrete-soln-1-bound},
whereas to bound $\|\partial_t\bu_{fh}(0)\|^2_{\bL^2(\Omega_f)}$ and  
$s_0\,\|\partial_t p_{ph}(0)\|^2_{\W_p}$ in \eqref{eq:bound-discrete-initial-solution-2}, we consider \eqref{eq:NS-Biot-semiformulation-1} for the test function $(\bv_{fh}, w_{ph})$ at $t=0$, and use \eqref{eq:semi-discrete-weak-formulation-1b} and \eqref{eq:semi-discrete-weak-formulation-1g}, to get 
%
\begin{align*}%\label{eq: discrete initial 1}
&  \rho_f (\partial_t\,\bu_{fh}(0),\bv_{fh})_{\Omega_f} +  s_0 (\partial_t\,p_{ph}(0),w_{ph})_{\Omega_p} 
\,=\, (\bdiv(2\mu\be(\bu_{f,0}) - \rho_f(\bu_{f,0}\otimes \bu_{f,0})),\bv_{fh})_{\Omega_f} \nonumber \\[1ex]
&\qquad +\, (\f_{f}(0),\bv_{fh})_{\Omega_f} - (\alpha_p \div (\,\bu_{s,0})
- \frac{1}{\mu}\,\div(\bK\nabla p_{p,0}),w_{ph})_{\Omega_p} + (q_{p}(0),w_{ph})_{\Omega_p} \,,\nonumber
\end{align*}
%
and taking $(\bv_{fh}, w_{ph}) = (\partial_t\bu_{fh}(0), \partial_t p_{ph}(0))$, applying the Cauchy--Schwarz and Young inequalities, and simple computations, we deduce 
%
\begin{align}\label{eq: discrete initial 2}
&  \|\partial_t\,\bu_{fh}(0)\|^2_{\bL^2(\Omega_f)} +  s_0 \|\partial_t\,p_{ph}(0)\|^2_{\L^2(\Omega_p)}  
\,\leq\, C\,\Big(\|\bdiv(\be(\bu_{f,0}))\|^2_{\bL^{2}(\Omega_f)} 
+ \|\bdiv(\bu_{f,0}\otimes\bu_{f,0})\|^2_{\bL^{2}(\Omega_f)} \nonumber \\[1ex]
&\quad + \frac{1}{s_0} \|\bu_{s,0}\|^2_{\bV_p}
+ \frac{1}{s_0}\|\div(\bK\nabla p_{p,0})\|^2_{\bL^2(\Omega_p)}
+ \|\f_{f}(0)\|^2_{\bL^2(\Omega_f)} + \frac{1}{s_0}\|q_p(0)\|^2_{\L^2(\Omega_p)} \Big) \,.
\end{align}
%
In turn, to bound $\|\partial_{tt}\bbeta_{ph}(0)\|^2_{\bL^2(\Omega_p)}$, we consider \eqref{eq:NS-Biot-semiformulation-1} at $t=0$ for the test function $\bxi_{ph}$ and use \eqref{discrete etap0-bound} and \eqref{eq:semi-discrete-weak-formulation-1l}, to obtain
%
\begin{align*}%\label{eq: discrete initial 4}
&  \rho_p(\partial_{tt}\bbeta_{ph}(0),\bxi_{ph})_{\Omega_p} =  (\bdiv(A^{-1}(\be(\bbeta_{p,0})) - \alpha_p\,p_{p,0}\,\bI),\bxi_{ph})_{\Omega_p} + (\f_{p}(0),\bxi_{ph})_{\Omega_p} \,,
\end{align*}
and taking $\bxi_{ph} = \partial_{tt}\bbeta_{ph}(0)$, applying Cauchy--Schwarz and Young's inequalities, and some algebraic manipulations, we derive
%
\begin{align}\label{eq: discrete initial 6}
&  \|\partial_{tt}\bbeta_{ph}(0)\|^2_{\bL^2(\Omega_p)} 
\,\leq\, C\,\Big(\, \|\bdiv(A^{-1}(\be(\bbeta_{p,0})))\|^2_{\bL^2(\Omega_p)} + \|p_{p,0}\|^2_{\H^1(\Omega_p)} + \|\f_{p}(0)\|^2_{\bL^2(\Omega_p)} \Big) \,.
\end{align}

Finally, \eqref{eq:discrete-stability} follows by combining the discrete versions of \eqref{eq:bound-unsteady-state-solution-6} and \eqref{eq:bound-unsteady-state-solution-10} with \eqref{eq: discrete initial 2} and \eqref{eq: discrete initial 6}, using the Sobolev embedding of $\H^1(0,T)$ into $\L^\infty(0,T)$, applying Lemma \ref{xing-lemma} in the context of the non-negative functions $H=\big( \|\bbeta_{ph}\|^2_{\bV_p} + \|\partial_{tt}\bbeta_{ph}\|^2_{\bL^2(\Omega_p)}\big)^{1/2}$ and $B=\|\partial_{t}\,\f_p\|_{\bL^2(\Omega_p)}$, with $R$ and $A$ representing the remaining terms, to control $\int^t_0 B(s)\,H(s)\,ds$, \eqref{eq:tau-d-H0div-inequality}--\eqref{eq:tau-H0div-Xf-inequality} and the same arguments described at the end of the proof of Theorem \ref{thm: continuous stability}.
\end{proof}



%**********************************************************************
%**********************************************************************

\subsection{Error analysis}

We proceed with establishing rates of convergence.
To that end, let us set $\V\in \big\{ \W_p, \bV_f, \bbQ_f, \bSigma_{e} \big\}$, 
$\Lambda\in \big\{ \bLambda_f, \Lambda_p \big\}$ and let
$\V_h, \Lambda_h$ be the discrete counterparts. Let $P_h^{\V}: \V\to \V_h$ and 
$P_h^{\Lambda}: \Lambda\to \Lambda_h$ be the $\L^2$-projection operators, satisfying
\begin{equation}\label{eq:projection1}
( u - P_h^{\V}(u), v_{h} )_{\Omega_\star} = 0 \quad \forall\,v_h\in \V_{h}\,,\quad
\langle \varphi - P_h^{\Lambda}(\varphi), \psi_h \rangle_{\Gamma_{fp}} = 0  \quad\forall\, \psi_h\in \Lambda_{h}\,,
\end{equation}
where $\star\in \{f,p\}$, $u\in \big\{ p_p, \bu_f, \bgamma_f, \bsi_e \big\}$, $\varphi\in \big\{ \bvarphi, \lambda
\big\}$, and $v_h, \psi_h$ are the corresponding discrete test
functions. We have the approximation properties \cite{ciarlet1978}:
\begin{equation}\label{eq:approx-property1}
\|u - P^{\V}_h(u)\|_{\L^p(\Omega_\star)} 
\,\leq\, C\,h^{s_{u} + 1}\, \|u\|_{\W^{s_{u} + 1,p}(\Omega_\star)}\,,\quad
\|\varphi - P^{\Lambda}_h(\varphi)\|_{\Lambda} 
\,\leq\, C\,h^{s_{\varphi} + 1/2}\,
\|\varphi\|_{\H^{s_{\varphi} + 1}(\Gamma_{fp})} \,,
\end{equation}
where $p \in \{2,4\}$ and $s_{u}\in \big\{ s_{p_p}, s_{\bu_f}, s_{\bgamma_f}, s_{\bsi_e} \big\}$, $s_{\varphi}\in \big\{ s_{\bvarphi}, s_{\lambda} \big\}$ are the degrees of polynomials in the
spaces $\V_h$ and $\Lambda_h$, respectively. 

Next, denote $\X\in \big\{ \bbX_f, \bX_p \big\}$, $\sigma\in
\big\{ \bsi_f, \bu_p \big\}\in \X$ and let $\X_h$ and $\tau_h$ be
their discrete counterparts. 
Let $I^{\X}_h : \X \cap \H^{1}(\Omega_{\star})\to \X_{h}$ be the mixed finite element
projection operator \cite{Brezzi-Fortin} satisfying
\begin{equation}\label{eq:projection2}
(\div(I^{\X}_h(\sigma)), w_h)_{\Omega_\star} = (\div(\sigma), w_h)_{\Omega_\star}\,,\quad 
\pil I^{\X}_h(\sigma)\bn_{\star}, \tau_h\bn_{\star} \pir_{\Gamma_{fp}} = \pil \sigma \bn_{\star}, \tau_{h}\bn_{\star} \pir_{\Gamma_{fp}} \,, 
\end{equation}
for all $w_h\in \W_h$ and $\tau_h\in \X_h$, and
\begin{align}
&\|\sigma - I^{\X}_h(\sigma)\|_{\L^2(\Omega_{\star})} \,\leq\, C\,h^{s_{\sigma} + 1} \| \sigma \|_{\H^{s_{\sigma} + 1}(\Omega_{\star})},  \nonumber \\
&\|\div(\sigma - I^{\X}_h(\sigma))\|_{\L^q(\Omega_{\star})} \,\leq\, C\,h^{s_{\sigma} + 1} \|\div(\sigma)\|_{\W^{s_{\sigma} + 1,q}(\Omega_{\star})},\label{eq:approx-property2}
\end{align}
where $q \in \{2,4/3\}$,  $w_h\in \big\{ \bv_{fh}, w_{ph} \big\}$, $\W_h\in
\big\{ \bV_f, \W_p \big\}$, and $s_{\sigma}\in \big\{
s_{\bsi_f}, s_{\bu_p} \big\}$ -- the degrees of polynomials
in the spaces $\X_h$.

Finally, let $S_h^{\bV_p}$ be the Scott--Zhang interpolation operators onto $\bV_{ph}$, satisfying \cite{sz1990}\,:
\begin{equation}\label{eq: approx property 3}
\|\bbeta_p - S^{\bV_p}_h(\bbeta_p)\|_{\bH^1(\Omega_p)} 
\,\leq\, C\,h^{s_{\bbeta_p}} \| \bbeta_p \|_{\bH^{s_{\bbeta_p} + 1}(\Omega_p)} \,.
\end{equation}

Now, let $(\bsi_f,\bu_p, \bbeta_p, \bu_f, p_p, \bgamma_f, \bvarphi, \lambda)$ and $(\bsi_{fh},\bu_{ph}, \bbeta_{ph}, \bu_{fh}, p_{ph}, \bgamma_{fh}, \bvarphi_{h}, \lambda_{h})$ be the solutions of \eqref{eq:continuous-weak-formulation-1} and \eqref{eq:NS-Biot-semiformulation-1}, respectively.   We
introduce the error terms as the differences of these two solutions and
decompose them into approximation and discretization errors
using the interpolation operators:
\begin{align}
& \be_{\sigma} \,:=\, \sigma - \sigma_{h} \,=\,
(\sigma - I^{\X}_h(\sigma)) + (I^{\X}_h(\sigma) - \sigma_{h})
\,:=\, \be^I_{\sigma} + \be^h_{\sigma}\,,
\quad \sigma\in \big\{ \bsi_f, \bu_p \big\}\,, \nonumber \\
& \be_{\varphi} \,:=\, \varphi - \varphi_h
\,=\, (\varphi - P^{\Lambda}_h(\varphi)) + (P^{\Lambda}_h(\varphi) - \varphi_h)
\,:=\, \be^I_{\varphi} + \be^h_{\varphi}\,,
\quad \varphi\in \big\{ \bvarphi, \lambda \big\}\,,\nonumber \\
&\be_{u} \,:=\, u - u_{h} \,=\, (u - P^{\V}_h(u)) + (P^{\V}_h(u) - u_{h})
\,:=\, \be^I_{u} + \be^h_{u}\,, \quad
u\in \big\{ p_p, \bu_f, \bgamma_f, \bsi_e \big\}\,,\nonumber \\
& \be_{\bbeta_p} \,:=\, \bbeta_p - \bbeta_{ph} \,=\, (\bbeta_p - S^{\bV_p}_h(\bbeta_p)) + (S^{\bV_p}_h(\bbeta_p) - \bbeta_{ph})
\,:=\, \be^I_{\bbeta_p} + \be^h_{\bbeta_p} \,.
\label{eq:error-decomposition}
\end{align}
Then, we set the global errors endowed with the above decomposition:
%
\begin{equation*}
\be_{\ubsi} := (\be_{\bsi_f}, \be_{\bu_p}),\qan
\be_{\ubu} := (\be_{\bbeta_p}, \be_{\bu_f}, \be_{p_p}, \be_{\bgamma_f}, \be_{\bvarphi},  \be_{\lambda})\,.
\end{equation*}
%
We form the error equation by subtracting the discrete equations \eqref{eq:NS-Biot-semiformulation-1} from the continuous one  \eqref{eq:continuous-weak-formulation-1}:
\begin{align}
&\rho_f (\partial_t\be_{\bu_f},\bv_{fh})_{\Omega_f}+ a_f(\be_{\bsi_f},\btau_{fh})+b_{\bn_f}(\btau_{fh},\be_{\bvarphi}) + b_f(\btau_{fh},\be_{\bu_f}) + b_\sk(\be_{\bgamma_f},\btau_{fh}) \nonumber\\ 
&\quad +\, \kappa_{\bu_{f}}(\bu_{f}, \btau_{fh}) - \kappa_{\bu_{fh}}(\bu_{fh}, \btau_{fh})-b_f(\be_{\bsi_f},\bv_{fh}) - b_\sk(\be_{\bsi_f},\bchi_{fh})  \, = \,0,\nonumber\\ 
& \rho_p(\partial_{tt}\be_{\bbeta_p},\bxi_{ph})_{\Omega_p} + a^e_p(\be_{\bbeta_p},\bxi_{ph})+ \alpha_p\,b_p(\bxi_{ph},\be_{p_p}) +c_{\BJS}(\partial_t\be_{\bbeta_p}, \be_{\bvarphi};\bxi_{ph}, \bpsi_{h})- c_{\Gamma}(\bxi_{ph},\bpsi_{h};\be_{\lambda}) \nonumber\\ 
&\quad -\, b_{\bn_f}(\be_{\bsi_f},\bpsi_{h})+l_{\bvarphi}(\bvarphi,\bpsi_{h}) - l_{\bvarphi_{h}}(\bvarphi_{h},\bpsi_{h})\, = \,0,\nonumber\\ 
& s_0 (\partial_t\be_{p_p},w_{ph})_{\Omega_p} +a^d_p(\be_{\bu_p},\bv_{ph}) +b_p(\bv_{ph},\be_{p_p})+b_{\bn_p}(\bv_{ph},\be_{\lambda})  - \alpha_p\,b_p(\partial_t\be_{\bbeta_p},w_{ph}) \nonumber\\ 
&\quad -\, b_p(\be_{\bu_p},w_{ph}) \,=\, 0,\nonumber\\ 
&c_{\Gamma}(\partial_t\be_{\bbeta_p},\be_{\bvarphi};\xi_{h})-b_{\bn_p}(\be_{\bu_p},\xi_{h})\, = \,0,  \label{eq:NS-Biot-errorformulation-1}
\end{align}
for all $(\btau_{fh}, \bv_{ph}, \bxi_{ph}, \bv_{fh}, w_{ph}, \bchi_{fh}, \bpsi_{h}, \xi_{h})\in \bbX_{fh}\times \bX_{ph}\times \bV_{ph}\times \bV_{fh}\times \W_{ph}\times \bbQ_{fh}\times \bLambda_{fh}\times \Lambda_{ph}$.

\begin{rem}
	
Let $\wh{\beta}:= \min\left\{ \wh{\beta}_1, \wh{\beta}_2, \wh{\beta}_3\right\}$, with $\wh{\beta}_i = \min\big\{\beta_i, \wt{\beta}_i\big\}, i\in \{1,2,3\}$, satisfying the inf-sup conditions in both Lemmas \ref{lem:inf-sup-conditions} and \ref{lem: discrete inf-sup}. 
It follows that $(\bu_f(t),\bvarphi(t)), (\bu_{fh}(t),\bvarphi_{h}(t)) \in \wh{\bW}_{r_{1},r_{2}} \quad \forall t\in [0,T]$, with $\wh{\bW}_{r_{1},r_{2}}$ defined by
\begin{equation}\label{eq:wh-Wr-definition}
\wh{\bW}_{r_{1},r_{2}} := \Big\{ (\bw_{f},\bzeta)\in \bV_{f}\times \bLambda_{f} \,:\quad \| \bw_{f}\|_{\bV_f} \leq \wh{r}_{1},\quad \|\bzeta\|_{\bLambda_f} \leq\wh{r}_{2} \Big\}\,,
\end{equation}
where $\wh{r}_1 \in (0,\wh{r}_{1}^0]$, $\wh{r}_{2}\in (0,\wh{r}_{2}^0]$, and $\wh{r}_1^0, \wh{r}_{2}^0$ are defined as 
$\wh{r}_{1}^0 := \mu\,\wh{\beta}/(3\,\rho_f\,n^{1/2})$ and  
$\wh{r}_{2}^0 := \mu \wh{\beta}^2/(6C_{\cL})$.
\end{rem}

We now establish the main result of this section.
%
\begin{thm}\label{thm: error analysis}
Let the assumptions in Theorem \ref{thm: well-posedness main result semi} holds. For the solutions of the continuous and semidiscrete problems \eqref{eq:continuous-weak-formulation-1} and \eqref{eq:NS-Biot-semiformulation-1}, respectively, assuming  sufficient regularity of the true solution according to \eqref{eq:approx-property1} and \eqref{eq:approx-property2}, 
there exists a positive constant $C$ depending on the solution regularity, but independent of $h$, such that
%
\begin{align}\label{eq:errror-rate-of-convergence}
&\|\be_{\bsi_f}\|_{\L^2(0,T;\bbX_f)} 
+ \| \partial_t(\be_{\bsi_f})^{\rd}\|_{\L^2(0,T;\bbL^2(\Omega_f))} 
+ \|\be_{\bu_p}\|_{\L^2(0,T;\bX_p)} 
+ \| \partial_t\be_{\bu_p}\|_{\L^2(0,T;\bL^2(\Omega_p))}
+ \|\be_{\bbeta_p}\|_{\L^\infty(0,T;\bV_p)} \nonumber \\
&\quad +\, \|\partial_t\be_{\bbeta_p}\|_{\L^\infty(0,T;\bL^2(\Omega_p))} 
+ \sum^{n-1}_{j=1} \|( \be_{\bvarphi}-\partial_t\be_{\bbeta_p})\cdot\bt_{f,j}\|_{\H^1(0,T;\L^2(\Gamma_{fp}))}
+ \|\partial_{tt}\be_{\bbeta_p}\|_{\L^\infty(0,T;\bL^2(\Omega_p))}
\nonumber \\
&\quad +\, \|\be_{\bu_f}\|_{\H^1(0,T;\bV_f)}
+ \|\be_{\bu_f}\|_{\W^{1,\infty}(0,T;\bL^2(\Omega_f))}
+ \sqrt{s_0}\,\|\be_{p_p}\|_{\W^{1,\infty}(0,T;\W_p)} 
+ \|\be_{p_p}\|_{\H^1(0,T;\W_p)} 
\nonumber \\[0.5ex]
&\quad +\, \|\be_{\bgamma_f}\|_{\H^1(0,T;\bbQ_f)} 
+ \|\be_{\bvarphi}\|_{\H^1(0,T;\bLambda_f)} 
+ \|\be_{\lambda}\|_{\H^1(0,T;\Lambda_p)}
\nonumber \\[0.5ex]
&\leq\, C\,\sqrt{T}\,\left( h^{s_{\underline{\sigma}}+1} + h^{s_{\bbeta_p}} 
+ h^{s_{\underline{u}}+1} +  h^{s_{\underline{\varphi}}+  1/2} \right) \,,
\end{align}
%
where $s_{\underline{\sigma}}= \min\{ s_{\bsi_f}, s_{\bu_p}\}$, $s_{\underline{u}}= \min\{ s_{p_p}, s_{\bu_f}, s_{\bgamma_f}, s_{\bsi_e} \}$, and  $s_{\underline{\varphi}}= \min\{ s_{\varphi}, s_{\lambda}\}$.
\end{thm}
%
\begin{proof}
We start by taking $(\btau_{fh}, \bv_{ph}, \bxi_{ph}, \bv_{fh}, w_{ph}, \bchi_{fh}, \bpsi_{h}, \xi_{h}) = (\be_{\bsi_f}^h, \be_{\bu_p}^h, \partial_t\be_{\bbeta_p}^h, \be_{\bu_f}^h, \be_{p_p}^h, \be_{\bgamma_f}^h, \be_{\bvarphi}^h, \be_{\lambda}^h)$ in \eqref{eq:NS-Biot-errorformulation-1}, to obtain
\begin{align}\label{eq: error equation 1}
&\ds\frac{1}{2}\,\partial_t\,\Big( \rho_f\,\|\be_{\bu_f}^h\|^2_{\bL^2(\Omega_f)} 
+ s_0\,\|\be_{p_p}^h\|^2_{\W_p} 
+ a^e_p(\be_{\bbeta_p}^h,\be_{\bbeta_p}^h) 
+ \rho_p\,\|\partial_{t}\be_{\bbeta_p}^h\|^2_{\bL^2(\Omega_p)} \Big) 
+ a_f(\be_{\bsi_f}^h,\be_{\bsi_f}^h) 
+ a^d_p(\be_{\bu_p}^h,\be_{\bu_p}^h) \nonumber \\[1ex]
&\ds\quad +\, c_{\BJS}(\partial_t\be_{\bbeta_p}^h, \be_{\bvarphi}^h;\partial_t\be_{\bbeta_p}^h, \be_{\bvarphi}^h)
+ \kappa_{\bu_{fh}}(\be_{\bu_{f}}^h, \be_{\bsi_f}^h)
+ \kappa_{\be_{\bu_{f}}^h}(\bu_{f}, \be_{\bsi_f}^h)
+ l_{\bvarphi_h}(\be_{\bvarphi}^h,\be_{\bvarphi}^h) 
+ l_{\be_{\bvarphi}^h}(\bvarphi,\be_{\bvarphi}^h)\nonumber \\[1ex]
&\ds = -\,  a^e_p(\be_{\bbeta_p}^I,\partial_t\be_{\bbeta_p}^h) 
-\rho_p(\partial_{tt}\be_{\bbeta_p}^I,\partial_{t}\be_{\bbeta_p}^h)_{\Omega_p}
- a_f(\be_{\bsi_f}^I,\be_{\bsi_f}^h) 
- a^d_p(\be_{\bu_p}^I,\be_{\bu_p}^h) 
- c_{\BJS}(\partial_t\be_{\bbeta_p}^I, \be_{\bvarphi}^I;\partial_t\be_{\bbeta_p}^h, \be_{\bvarphi}^h) \nonumber \\[1ex]
&\ds\quad- \kappa_{\bu_{fh}}(\be_{\bu_{f}}^I, \be_{\bsi_f}^h)
\, -\, \kappa_{\be_{\bu_{f}}^I}(\bu_{f}, \be_{\bsi_f}^h) 
- l_{\bvarphi_h}(\be_{\bvarphi}^I,\be_{\bvarphi}^h) 
- l_{\be_{\bvarphi}^I}(\bvarphi,\be_{\bvarphi}^h) 
- b_\sk(\be_{\bgamma_f}^I,\be_{\bsi_f}^h) 
+ b_\sk(\be_{\bsi_f}^I,\be_{\bgamma_f}^h) \nonumber \\[1ex]
&\ds\quad - b_{\bn_f}(\be_{\bsi_f}^h,\be_{\bvarphi}^I)
\, +\, b_{\bn_f}(\be_{\bsi_f}^I,\be_{\bvarphi}^h)
- \alpha_p\,b_p(\partial_t\be_{\bbeta_p}^h,\be_{p_p}^I) 
+ \alpha_p\,b_p(\partial_t\be_{\bbeta_p}^I,\be_{p_p}^h) 
- b_{\bn_p}(\be_{\bu_p}^h, \be_{\lambda}^I) 
+ b_{\bn_p}(\be_{\bu_p}^I, \be_{\lambda}^h) \nonumber \\[1ex]
&\ds \quad - c_{\Gamma}(\partial_t\be_{\bbeta_p}^I,\be_{\bvarphi}^I;\be_{\lambda}^h) 
+ c_{\Gamma}(\partial_t\be_{\bbeta_p}^h,\be_{\bvarphi}^h;\be_{\lambda}^I)\,,
\end{align}
where, the right-hand side of \eqref{eq: error equation 1} has been simplified, 
since the projection properties \eqref{eq:projection1} and 
\eqref{eq:projection2}, and the fact that 
$\div(\bX_{ph})=\W_{ph}$,
$\bdiv(\bbX_{fh}) = \bV_{fh}$ (cf. \eqref{eq: div-prop}), 
imply that the following terms are zero:
\begin{equation*}
s_0 (\partial_t\be_{p_p}^I,\be_{p_p}^h)_{\Omega_p}, \,\, \rho_f (\partial_t\be_{\bu_f}^I,\be_{\bu_f}^h)_{\Omega_f}, \,\,  b_f(\be_{\bsi_f}^h,\be_{\bu_f}^I), \,\, b_f(\be_{\bsi_f}^I,\be_{\bu_f}^h),\,\,
b_p(\be_{\bu_p}^h,\be_{p_p}^I), \,\, b_p(\be_{\bu_p}^I,\be_{p_p}^h)\,.
\end{equation*}
%
In turn, testing \eqref{eq:NS-Biot-errorformulation-1} with $(\btau_{fh}, \bv_{ph}) =(\be_{\bsi_f}^h, \be_{\bu_p}^h)$ and using discrete inf-sup conditions \eqref{eq:discrete inf-sup-qp-xi}--\eqref{eq:discrete inf-sup-vf-chif} in Lemma \ref{lem: discrete inf-sup} along with the continuity of $\kappa_{\bw_f}$ (cf. \eqref{eq:continuity-cK-wf}) and the fact that $(\bu_f(t),\bvarphi(t)),(\bu_{fh}(t),\bvarphi_{h}(t)):[0,T]\to \wh{\bW}_{r_1,r_2}$ (cf. \eqref{eq:wh-Wr-definition}), we get
\begin{align}
&\|(\be_{\bu_{f}}^h, \be_{p_{p}}^h, \be_{\bgamma_{f}}^h, \be_{\bvarphi}^h, \be_{\lambda}^h)\|
\,\leq\, C\,\Big( \|(\be_{\bsi_f}^h)^\rd\|_{\bbL^2(\Omega_f)} + \|\be_{\bu_p}^h\|_{\bL^2(\Omega_p)} +  \|(\be_{\bsi_f}^I)^\rd\|_{\bbL^2(\Omega_f)} + \|\be_{\bu_p}^I\|_{\bL^2(\Omega_p)}\nonumber \\
& \qquad  +\, \|\be_{\bu_f}^I\|_{\bV_f} + \|\be_{\bgamma_f}^I\|_{\bbQ_f} + \|\be_{\bvarphi}^I\|_{\bLambda_f} + \|\be_{p_p}^I\|_{\W_p} + \|\be_{\lambda}^I\|_{\Lambda_p} \Big)\,.
\label{eq: error equation 3}
\end{align}	
Then, proceeding as in \eqref{eq:bound-unsteady-state-solution-1}--\eqref{eq:bound-unsteady-state-solution-6}, integrating \eqref{eq: error equation 1} from $0$ to $t\in (0,T]$, using \eqref{eq:aep-coercivity-bound}, the non-negativity bounds in Lemma \ref{lem:coercivity-properties-A-E2} in combination with the discrete inf-sup condition of $B_f$ (cf. \eqref{eq:discrete inf-sup-vf-chif}), the continuity of $\kappa_{\bu_f}$ and $l_{\bvarphi}$ (cf. \eqref{eq:continuity-cK-wf}, \eqref{eq:continuity-cL-zeta}), the fact that $(\bu_f(t),\bvarphi(t)),(\bu_{fh}(t),\bvarphi_{h}(t)):[0,T]\to \wh{\bW}_{r_1,r_2}$ (cf. \eqref{eq:wh-Wr-definition}), the Cauchy--Schwarz and Young's inequalities, and \eqref{eq: error equation 3}, we obtain
%
\begin{align*}%\label{eq: error equation 4}
&\ds \|\be_{\bu_f}^h(t)\|^2_{\bL^2(\Omega_f)} 
+ s_0 \|\be_{p_p}^h(t)\|^2_{\W_p} 
+ \|\be_{\bbeta_p}^h(t)\|^2_{\bV_p} 
+ \|\partial_{t}\be_{\bbeta_p}^h(t)\|^2_{\bL^2(\Omega_p)} 
+\, \int_0^t \Big(\|(\be_{\bsi_f}^h)^{\rd}\|^2_{\bbL^2(\Omega_f)}   + \|\be_{\bu_p}^h\|^2_{\bL^2(\Omega_p)} \nonumber \\[1ex]
&\ds\quad 
+ \sum^{n-1}_{j=1} \|( \be_{\bvarphi}^h-\partial_t\be_{\bbeta_p}^h)\cdot\bt_{f,j}\|^2_{\L^2(\Gamma_{fp})} 
+ \|(\be_{\bu_{f}}^h, \be_{p_{p}}^h, \be_{\bgamma_{f}}^h, \be_{\bvarphi}^h, \be_{\lambda}^h)\|^2\Big)\,ds
\nonumber \\[1ex]
&\ds\leq\, C\int_0^t \Big(\|\be_{\ubsi}^I\|^2
+ \|\be_{\ubu}^I\|^2 
+ \sum^{n-1}_{j=1} \|( \be_{\bvarphi}^I-\partial_t\be_{\bbeta_p}^I)\cdot\bt_{f,j}\|^2_{\L^2(\Gamma_{fp})}  
+ \,\|\partial_{t}\be_{\bbeta_p}^I\|^2_{\bV_p} + \,\|\partial_{tt}\be_{\bbeta_p}^I\|^2_{\bL^2(\Omega_p)} \Big)\,ds  
\nonumber \\[1ex]
&\ds\quad + \delta_1 \int_0^t \Big(  \sum^{n-1}_{j=1} \|( \be_{\bvarphi}^h-\partial_t\be_{\bbeta_p}^h)\cdot\bt_{f,j}\|^2_{\L^2(\Gamma_{fp})} 
+ \|\be_{p_{p}}^h\|^2_{\W_p} 
+ \|\be_{\bgamma_{f}}^h\|^2_{\bbQ_f} 
+ \|\be_{\bvarphi}^h\|^2_{\bLambda_f} 
+ \|\be_{\lambda}^h\|^2_{\Lambda_p}  \Big)\,ds  
\nonumber \\[1ex]
&\ds\quad + \delta_2 \int_0^t \Big( \|\be_{\bsi_f}^h\|^2_{\bbX_f}   + \|\be_{\bu_p}^h\|^2_{\bX_p}  \Big)\,ds + C\,\Bigg( \|\be_{\bu_f}^h(0)\|^2_{\bL^2(\Omega_f)}
+ s_0\,\|\be_{p_p}^h(0)\|^2_{\W_p} 
+ \|\be_{\bbeta_p}^h(0)\|^2_{\bV_p}
\nonumber \\[1ex]
&\ds\quad +\,\|\partial_{t}\be_{\bbeta_p}^h(0)\|^2_{\bL^2(\Omega_p)}
+ \int_0^t \big(\|\partial_{tt}\be_{\bbeta_p}^I\|^2_{\bL^2(\Omega_p)}   
+ \|\be_{\bbeta_p}^I\|^2_{\bV_p}  
+  \|\be_{p_{p}}^I\|^2_{\W_p}  
+ \|\be_{\lambda}^I\|^2_{\Lambda_p} \big)^{1/2}\|\partial_{t}\be_{\bbeta_p}^h\|_{\bV_p}\,ds   
\Bigg) \,,
\end{align*}
and taking $\delta_1$ small enough, we get
%
\begin{align}\label{eq: error equation 4a}
&\ds \|\be_{\bu_f}^h(t)\|^2_{\bL^2(\Omega_f)} 
+ s_0 \|\be_{p_p}^h(t)\|^2_{\W_p} 
+ \|\be_{\bbeta_p}^h(t)\|^2_{\bV_p} 
+ \|\partial_{t}\be_{\bbeta_p}^h(t)\|^2_{\bL^2(\Omega_p)} 
+\, \int_0^t \Big(\|(\be_{\bsi_f}^h)^{\rd}\|^2_{\bbL^2(\Omega_f)}   + \|\be_{\bu_p}^h\|^2_{\bL^2(\Omega_p)} \nonumber \\[1ex]
&\ds\quad 
+ \sum^{n-1}_{j=1} \|( \be_{\bvarphi}^h-\partial_t\be_{\bbeta_p}^h)\cdot\bt_{f,j}\|^2_{\L^2(\Gamma_{fp})} 
+ \|(\be_{\bu_{f}}^h, \be_{p_{p}}^h, \be_{\bgamma_{f}}^h, \be_{\bvarphi}^h, \be_{\lambda}^h)\|^2\Big)\,ds
\nonumber \\[1ex]
&\ds\leq\, C\int_0^t \Big(\|\be_{\ubsi}^I\|^2
+ \|\be_{\ubu}^I\|^2 
+ \sum^{n-1}_{j=1} \|( \be_{\bvarphi}^I-\partial_t\be_{\bbeta_p}^I)\cdot\bt_{f,j}\|^2_{\L^2(\Gamma_{fp})}  
+ \,\|\partial_{t}\be_{\bbeta_p}^I\|^2_{\bV_p} + \,\|\partial_{tt}\be_{\bbeta_p}^I\|^2_{\bL^2(\Omega_p)} \Big)\,ds  
\nonumber \\[1ex]
&\ds\quad + \delta_2 \int_0^t \Big( \|\be_{\bsi_f}^h\|^2_{\bbX_f}   + \|\be_{\bu_p}^h\|^2_{\bX_p}  \Big)\,ds \,+\, C\,\Bigg( \|\be_{\bu_f}^h(0)\|^2_{\bL^2(\Omega_f)}
+ s_0\,\|\be_{p_p}^h(0)\|^2_{\W_p} 
+ \|\be_{\bbeta_p}^h(0)\|^2_{\bV_p}
\nonumber \\[1ex]
&\ds\quad +\,\|\partial_{t}\be_{\bbeta_p}^h(0)\|^2_{\bL^2(\Omega_p)}
+ \int_0^t \big(\|\partial_{tt}\be_{\bbeta_p}^I\|^2_{\bL^2(\Omega_p)}   
+ \|\be_{\bbeta_p}^I\|^2_{\bV_p}  
+  \|\be_{p_{p}}^I\|^2_{\W_p}  
+ \|\be_{\lambda}^I\|^2_{\Lambda_p} \big)^{1/2}\|\partial_{t}\be_{\bbeta_p}^h\|_{\bV_p}\,ds   
\Bigg) \,.
\end{align}
In addition, from the equations in \eqref{eq:NS-Biot-errorformulation-1} corresponding to
test functions $\bv_{fh}$ and $w_{ph}$,
using the projection properties \eqref{eq:projection1} and \eqref{eq:projection2}, we find
that
%
\begin{equation*}
\begin{array}{c}
\ds b_f(\be^h_{\bsi_f}, \bv_{fh}) =  \rho_f (\partial_t\be_{\bu_f}^h,\bv_{fh})_{\Omega_f} \quad \forall\,\bv_{fh}\in \bV_{fh}\,, \\ [2ex]
\ds b_p(\be^h_{\bu_p}, w_{ph}) = (s_0\,\partial_t\,\be^h_{p_p}, w_{ph})_{\Omega_p}  - \alpha_p\,b_p(\partial_t\be_{\bbeta_p}^h,w_{ph})- \alpha_p\,b_p(\partial_t\be_{\bbeta_p}^I,w_{ph}) \quad \forall\, w_{ph}\in \W_{ph}\,,
\end{array}
\end{equation*}
%
which, together with the fact that $\bdiv(\bbX_f) = (\bV_f)'$ and $\div(\bX_p) = (\W_p)'$, implies that
\begin{align}
&\|\bdiv(\be_{\bsi_f}^h)\|_{\bL^{4/3}(\Omega_f)} 
\,\leq\, \rho_f\,|\Omega_f|^{1/4}\,\|\partial_t\be_{\bu_f}^h\|_{\bL^2(\Omega_f)}\,, \nonumber \\[1ex]
& \|\div(\be^h_{\bu_p})\|_{\L^2(\Omega_p)} 
\,\leq\, s_0\,\|\partial_t\,\be^h_{p_p} \|_{\W_p} 
+ \alpha_p\,n^{1/2}\,\|\partial_t\be_{\bbeta_p}^h\|_{\bV_p}
+ \alpha_p\,n^{1/2}\,\|\partial_t\be_{\bbeta_p}^I\|_{\bV_p}\,.
\label{eq:error-analysis7}
\end{align}
\medskip

\noindent{\bf{Bounds on time derivatives on the right-hand side of \eqref{eq:error-analysis7}.}}

\noindent In order to bound the time derivative terms $\,\|\partial_t\be_{\bu_f}^h\|_{\bL^2(\Omega_f)}, s_0\,\|\partial_t\,\be^h_{p_p} \|_{\W_p}$, and $\|\partial_t\be_{\bbeta_p}^h\|_{\bV_p}$ in \eqref{eq:error-analysis7}, we differentiate in time the whole system \eqref{eq:NS-Biot-errorformulation-1} and testing with $(\btau_{fh}, \bv_{ph}, \bxi_{ph}, \bv_{fh}, w_{ph}, \bchi_{fh}, \bpsi_{h}, \xi_{h}) = (\partial_t\be_{\bsi_f}^h, \partial_t\be_{\bu_p}^h,
\partial_{tt}\be_{\bbeta_p}^h, \partial_t\be_{\bu_f}^h,$ 
$\partial_t\be_{p_p}^h, \partial_t\be_{\bgamma_f}^h, \partial_t\be_{\bvarphi}^h, \partial_t\be_{\lambda}^h)$, similarly to \eqref{eq: error equation 1}, we get 
%
\begin{align}\label{eq: error equation 5}
&\ds\frac{1}{2}\,\partial_t\,\Big( \rho_f\|\partial_t\be_{\bu_f}^h\|^2_{\bL^2(\Omega_f)} 
+ s_0\,\|\partial_t\be_{p_p}^h\|^2_{\W_p} 
+ a^e_p(\partial_t\be_{\bbeta_p}^h,\partial_t\be_{\bbeta_p}^h) 
+ \rho_p\,\|\partial_{tt}\be_{\bbeta_p}^h\|^2_{\bL^2(\Omega_p)} \Big) 
\nonumber \\[1ex]
&\ds\quad +\, a_f(\partial_t\be_{\bsi_f}^h,\partial_t\be_{\bsi_f}^h) 
+ a^d_p(\partial_t\be_{\bu_p}^h,\partial_t\be_{\bu_p}^h) 
+ c_{\BJS}(\partial_{tt}\be_{\bbeta_p}^h, \partial_t\be_{\bvarphi}^h;\partial_{tt}\be_{\bbeta_p}^h, \partial_t\be_{\bvarphi}^h)
+ \kappa_{\bu_{fh}}(\partial_t\be_{\bu_{f}}^h, \partial_t\be_{\bsi_f}^h)
\nonumber \\[1ex]
&\ds\quad +\, \kappa_{\partial_t\be_{\bu_{f}}^h}(\bu_{f}, \partial_t\be_{\bsi_f}^h)
+ l_{\bvarphi_h}(\partial_t\be_{\bvarphi}^h,\partial_t\be_{\bvarphi}^h) 
+ l_{\partial_t\be_{\bvarphi}^h}(\bvarphi,\partial_t\be_{\bvarphi}^h)
\nonumber \\[1ex]
&\ds = -\, a^e_p(\partial_t\be_{\bbeta_p}^I,\partial_{tt}\be_{\bbeta_p}^h) 
-\rho_p(\partial_{ttt}\be_{\bbeta_p}^I,\partial_{tt}\be_{\bbeta_p}^h)_{\Omega_p}
- a_f(\partial_t\be_{\bsi_f}^I,\partial_t\be_{\bsi_f}^h)
-  a^d_p(\partial_t\be_{\bu_p}^I,\partial_t\be_{\bu_p}^h) 
\nonumber \\[1ex]
&\ds\quad - c_{\BJS}(\partial_{tt}\be_{\bbeta_p}^I, \partial_t\be_{\bvarphi}^I;\partial_{tt}\be_{\bbeta_p}^h, \partial_t\be_{\bvarphi}^h) 
-\, \kappa_{\partial_t\bu_{fh}}(\be_{\bu_{f}}^h, \partial_t\be_{\bsi_f}^h)
- \kappa_{\be_{\bu_{f}}^h}(\partial_t\bu_{f}, \partial_t\be_{\bsi_f}^h)
-l_{\partial_t\bvarphi_h}(\be_{\bvarphi}^h,\partial_t\be_{\bvarphi}^h)
\nonumber \\[1ex]
&\ds\quad -l_{\be_{\bvarphi}^h}(\partial_t\bvarphi,\partial_t\be_{\bvarphi}^h)  
- \kappa_{\partial_t\bu_{fh}}(\be_{\bu_{f}}^I, \partial_t\be_{\bsi_f}^h)  
-\, \kappa_{\partial_t\be_{\bu_{f}}^I}(\bu_{f}, \partial_t\be_{\bsi_f}^h)  
- \kappa_{\bu_{fh}}(\partial_t\be_{\bu_{f}}^I, \partial_t\be_{\bsi_f}^h) 
- \kappa_{\be_{\bu_{f}}^I}(\partial_t\bu_{f}, \partial_t\be_{\bsi_f}^h)
\nonumber \\[1ex]
&\ds\quad- l_{\partial_t\bvarphi_h}(\be_{\bvarphi}^I,\partial_t\be_{\bvarphi}^h) 
- l_{\partial_t\be_{\bvarphi}^I}(\bvarphi,\partial_t\be_{\bvarphi}^h) 
-\, l_{\bvarphi_h}(\partial_t\be_{\bvarphi}^I,\partial_t\be_{\bvarphi}^h) 
- l_{\be_{\bvarphi}^I}(\partial_t\bvarphi,\partial_t\be_{\bvarphi}^h)
- b_\sk(\partial_t\be_{\bsi_f}^h,\partial_t\be_{\bgamma_f}^I) 
\nonumber \\[1ex]
&\ds\quad + b_\sk(\partial_t\be_{\bsi_f}^I,\partial_t\be_{\bgamma_f}^h)  
- b_{\bn_f}(\partial_t\be_{\bsi_f}^h,\partial_t\be_{\bvarphi}^I) 
+\, b_{\bn_f}(\partial_t\be_{\bsi_f}^I,\partial_t\be_{\bvarphi}^h)
- \alpha_p\,b_p(\partial_{tt}\be_{\bbeta_p}^h,\partial_t\be_{p_p}^I)
+ \alpha_p\,b_p(\partial_{tt}\be_{\bbeta_p}^I,\partial_t\be_{p_p}^h) 
\nonumber \\[1ex]
&\ds\quad - b_{\bn_p}(\partial_t\be_{\bu_p}^h, \partial_t\be_{\lambda}^I)
+\, b_{\bn_p}(\partial_t\be_{\bu_p}^I, \partial_t\be_{\lambda}^h) 
- c_{\Gamma}(\partial_{tt}\be_{\bbeta_p}^I,\partial_t\be_{\bvarphi}^I;\partial_t\be_{\lambda}^h)
+ c_{\Gamma}(\partial_{tt}\be_{\bbeta_p}^h,\partial_t\be_{\bvarphi}^h;\partial_t\be_{\lambda}^I) \,,
\end{align}
where, using again the projection properties \eqref{eq:projection1} and 
\eqref{eq:projection2}, and the fact that 
$\div(\bX_{ph})=\W_{ph}$,
\newline
$\bdiv(\bbX_{fh}) = \bV_{fh}$ (cf. \eqref{eq: div-prop}), we have dropped from \eqref{eq: error equation 5} the following terms:
\begin{equation*}
\begin{array}{c}
s_0 (\partial_{tt}\be_{p_p}^I,\partial_{t}\be_{p_p}^h)_{\Omega_p}, \,\, \rho_f (\partial_{tt}\be_{\bu_f}^I,\partial_{t}\be_{\bu_f}^h)_{\Omega_f}, \,\, b_f(\partial_{t}\be_{\bsi_f}^h,\partial_{t}\be_{\bu_f}^I), \,\, b_f(\partial_{t}\be_{\bsi_f}^I,\partial_{t}\be_{\bu_f}^h)\,,  \\[2ex]
b_p(\partial_{t}\be_{\bu_p}^h,\partial_{t}\be_{p_p}^I), \,\, b_p(\partial_{t}\be_{\bu_p}^I,\partial_{t}\be_{p_p}^h)\,.
\end{array}
\end{equation*}
%
We now comment on the control of the terms $\kappa_{\bw_f}$ and $l_{\bzeta}$ in \eqref{eq: error equation 5}.
For the terms on the left-hand side of \eqref{eq: error equation 5}, we proceed as in \eqref{eq: error equation 4a}, using the non-negativity bound for $a_f(\partial_t\be_{\bsi_f}^h,\partial_t\be_{\bsi_f}^h)$ (cf. \eqref{eq:coercivity-af}), combined with the discrete inf-sup condition of $B_f$ (cf. \eqref{eq:discrete inf-sup-vf-chif}), the continuity of $\kappa_{\bw_f}$ and $l_{\bzeta}$ (cf. \eqref{eq:continuity-cK-wf}, \eqref{eq:continuity-cL-zeta}), and the fact that $(\bu_f(t),\bvarphi(t))$, $(\bu_{fh}(t),\bvarphi_{h}(t)):[0,T] \to \wh{\bW}_{r_1,r_2}$ (cf. \eqref{eq:wh-Wr-definition}).
For the terms on the right-hand side of \eqref{eq: error equation 5}, we first use the bound
\begin{equation*}
\begin{array}{l}
\ds \kappa_{\partial_t\be_{\bu_{f}}^I}(\bu_{f}, \partial_t\be_{\bsi_f}^h)  
+ \kappa_{\bu_{fh}}(\partial_t\be_{\bu_{f}}^I, \partial_t\be_{\bsi_f}^h)
+ l_{\bvarphi_h}(\partial_t\be_{\bvarphi}^I,\partial_t\be_{\bvarphi}^h)
+ l_{\partial_t\be_{\bvarphi}^I}(\bvarphi,\partial_t\be_{\bvarphi}^h) \\[2ex] 
\ds\quad \leq\, \delta_3\,\Big(\|\partial_t(\be_{\bsi_f}^h)^\rd\|^2_{\bbL^2(\Omega_f)} + \|\partial_t\be_{\bvarphi}^h\|^2_{\bLambda_f} \Big) 
+ C\,\Big(\|\partial_{t}\be_{\bu_f}^I\|^2_{\bV_f} + \|\partial_t\be_{\bvarphi}^I\|^2_{\bLambda_f}\Big) \,,
\end{array}
\end{equation*}
which follows from Cauchy--Schwarz and Young's inequalities, while the remaining eight terms can be bounded by
\begin{equation*}
\begin{array}{l}
\ds
\delta_3\,\Big(\|\partial_t(\be_{\bsi_f}^h)^\rd\|^2_{\bbL^2(\Omega_f)} + \|\partial_t\be_{\bvarphi}^h\|^2_{\bLambda_f} \Big) 
+ C\,\Big\{ ( \|\partial_{t}\bu_{fh}\|^2_{\bV_f} + \|\partial_{t}\bu_{f}\|^2_{\bV_f})(\|\be_{\bu_f}^I\|^2_{\bV_f} + \|\be_{\bu_f}^h\|^2_{\bV_f} ) \\[2ex]
\ds\quad +\, ( \|\partial_t\bvarphi_{h}\|^2_{\bLambda_f} + \|\partial_t\bvarphi\|^2_{\bLambda_f})(\|\be_{\bvarphi}^I\|^2_{\bLambda_f} + \|\be_{\bvarphi}^h\|^2_{\bLambda_f} ) \Big\} \,.
\end{array}
\end{equation*} 
In turn, from the rows of $\btau_{fh}$ and $\bv_{ph}$ in \eqref{eq:NS-Biot-errorformulation-1}, differentiating in time and using discrete inf-sup conditions \eqref{eq:discrete inf-sup-qp-xi}--\eqref{eq:discrete inf-sup-vf-chif} in Lemma \ref{lem: discrete inf-sup} for $(\partial_t\be_{\bu_{f}}^h, \partial_t\be_{p_{p}}^h, \partial_t\be_{\bgamma_{f}}^h, \partial_t\be_{\bvarphi}^h, \partial_t\be_{\lambda}^h)$ along with the continuity of $\kappa_{\bw_f}$ (cf. \eqref{eq:continuity-cK-wf}) and the fact that $(\bu_f(t),\bvarphi(t)),(\bu_{fh}(t),\bvarphi_{h}(t)):[0,T]\to \wh{\bW}_{r_1,r_2}$ (cf. \eqref{eq:wh-Wr-definition}), we get
\begin{align}
&\|(\partial_t\be_{\bu_{f}}^h, \partial_t\be_{p_{p}}^h, \partial_t\be_{\bgamma_{f}}^h, \partial_t\be_{\bvarphi}^h, \partial_t\be_{\lambda}^h)\|
\,\leq\, C\,\Big( \|(\partial_t\be_{\bsi_f}^h)^\rd\|_{\bbL^2(\Omega_f)} 
+ \|\partial_t\be_{\bu_p}^h\|_{\bL^2(\Omega_p)} 
+ \|\partial_{t}\bu_{fh}\|_{\bV_f}\,\|\be_{\bu_f}^h\|_{\bV_f} \nonumber \\
& \quad  +\, \|\partial_{t}\bu_{fh}\|_{\bV_f}\,\|\be_{\bu_f}^I\|_{\bV_f}
+ \|(\partial_t\be_{\bsi_f}^I)^\rd\|_{\bbL^2(\Omega_f)}
+ \|\partial_t\be_{\bu_p}^I\|_{\bL^2(\Omega_p)}
+ \|\partial_t\be_{\bu_f}^I\|_{\bV_f}
\nonumber \\
& \quad  +\,\|\partial_t\be_{\bgamma_f}^I\|_{\bbQ_f} 
+ \|\partial_t\be_{\bvarphi}^I\|_{\bLambda_f} + \|\partial_t\be_{p_p}^I\|_{\W_p} + \|\partial_t\be_{\lambda}^I\|_{\Lambda_p} \Big)\,.
\label{eq: error equation 5a}
\end{align}	
Then integrating \eqref{eq: error equation 5} from 0 to t $\in (0,T]$, using the identities:
\begin{align*}
& \int^t_0 a^e_p(\partial_t\be_{\bbeta_p}^I,\partial_{tt}\be_{\bbeta_p}^h)\,ds 
\,=\, a^e_p(\partial_t\be_{\bbeta_p}^I,\partial_{t}\be_{\bbeta_p}^h)\Big|_0^t 
- \int^t_0 a^e_p(\partial_{tt}\be_{\bbeta_p}^I,\partial_{t}\be_{\bbeta_p}^h)\,ds,\nonumber \\
& \int^t_0 \rho_p(\partial_{ttt}\be_{\bbeta_p}^I,\partial_{tt}\be_{\bbeta_p}^h)_{\Omega_p} \,ds \,=\, \rho_p(\partial_{ttt}\be_{\bbeta_p}^I,\partial_{t}\be_{\bbeta_p}^h)_{\Omega_p}\Big|_0^t - \int^t_0 \rho_p(\partial_{tttt}\be_{\bbeta_p}^I,\partial_{t}\be_{\bbeta_p}^h)_{\Omega_p}  \,ds,\nonumber \\
&  \int^t_0  b_\sk(\partial_t\be_{\bsi_f}^h,\partial_t\be_{\bgamma_f}^I)\,ds 
\,=\,  b_\sk(\be_{\bsi_f}^h,\partial_t\be_{\bgamma_f}^I)\Big|_0^t 
- \int^t_0  b_\sk(\be_{\bsi_f}^h,\partial_{tt}\be_{\bgamma_f}^I)\,ds,\nonumber \\
&  \int^t_0 b_{\bn_f}(\partial_t\be_{\bsi_f}^h,\partial_t\be_{\bvarphi}^I)\,ds 
\,=\, b_{\bn_f}(\be_{\bsi_f}^h,\partial_t\be_{\bvarphi}^I)\Big|_0^t 
- \int^t_0 b_{\bn_f}(\be_{\bsi_f}^h,\partial_{tt}\be_{\bvarphi}^I)\,ds,\nonumber \\
&  \int^t_0 \,b_p(\partial_{tt}\be_{\bbeta_p}^h,\partial_t\be_{p_p}^I) ds 
\,=\, \,b_p(\partial_{t}\be_{\bbeta_p}^h,\partial_t\be_{p_p}^I) \Big|_0^t 
- \int^t_0 \,b_p(\partial_{t}\be_{\bbeta_p}^h,\partial_{tt}\be_{p_p}^I) ds ,\nonumber \\
&  \int^t_0 b_{\bn_p}(\partial_t\be_{\bu_p}^h, \partial_t\be_{\lambda}^I)  ds 
\,=\, \,b_{\bn_p}(\be_{\bu_p}^h, \partial_t\be_{\lambda}^I)  \Big|_0^t 
- \int^t_0 \,b_{\bn_p}(\be_{\bu_p}^h, \partial_{tt}\be_{\lambda}^I)  ds,\nonumber \\
&  \int^t_0  c_{\Gamma}(\partial_{tt}\be_{\bbeta_p}^h,\partial_t\be_{\bvarphi}^h;\partial_t\be_{\lambda}^I)  ds 
\,=\,  c_{\Gamma}(\partial_{t}\be_{\bbeta_p}^h,\be_{\bvarphi}^h;\partial_t\be_{\lambda}^I)\Big|_0^t 
- \int^t_0  c_{\Gamma}(\partial_{t}\be_{\bbeta_p}^h,\be_{\bvarphi}^h;\partial_{tt}\be_{\lambda}^I) ds, %\label{eq:integration-by-parts-time-identity}
\end{align*}
%
and applying the non-negativity and continuity bounds of the bilinear forms involved (cf. Lemmas \ref{lem:cont} and \ref{lem:coercivity-properties-A-E2}), \eqref{eq: error equation 5a}, Cauchy--Schwarz and Young's inequalities, and by grouping them conveniently, we obtain
%
\begin{align}\label{eq: error equation 6}
&\ds \|\partial_t\be_{\bu_f}^h(t)\|^2_{\bL^2(\Omega_f)} 
+ s_0\,\|\partial_t\be_{p_p}^h(t)\|^2_{\W_p} 
+ \|\partial_t\be_{\bbeta_p}^h(t)\|^2_{\bV_p} 
+ \,\|\partial_{tt}\be_{\bbeta_p}^h(t)\|^2_{\bL^2(\Omega_p)}  
+\, \int^t_0 \Big(\|\partial_t(\be_{\bsi_f}^h)^\rd\|^2_{\bbL^2(\Omega_f)} 
\nonumber \\[1ex]
&\ds\quad + \|\partial_t\be_{\bu_p}^h\|^2_{\bL^2(\Omega_p)} 
+ \sum^{n-1}_{j=1} \|( \partial_t\be_{\bvarphi}^h-\partial_{tt}\be_{\bbeta_p}^h)\cdot\bt_{f,j}\|^2_{\L^2(\Gamma_{fp})} + \|(\partial_t\be_{\bu_{f}}^h, \partial_t\be_{p_{p}}^h, \partial_t\be_{\bgamma_{f}}^h, \partial_t\be_{\bvarphi}^h, \partial_t\be_{\lambda}^h)\|^2\Big)\,ds
\nonumber \\[1ex]
&\ds \leq\, C\,\Bigg(\|\partial_{ttt}\be_{\bbeta_p}^I(t)\|^2_{\bL^2(\Omega_p)} 
+ \|\partial_t\be_{\bbeta_p}^I(t)\|^2_{\bV_p} 
+ \|\partial_t\be_{p_p}^I(t)\|^2_{\W_p} 
+ \|\partial_t\be_{\bgamma_f}^I(t)\|^2_{\bbQ_f} 
+\, \|\partial_t\be_{\bvarphi}^I(t)\|^2_{\bLambda_f}
+\, \|\partial_t\be_{\lambda}^I(t)\|^2_{\Lambda_p}
\nonumber \\[1ex]
&\ds\quad + \int_0^t \Big(\|\partial_{t}\be_{\bsi_f}^I\|^2_{\bbX_f} 
+ \|\partial_{tt}\be_{p_p}^I\|^2_{\W_p} 
+ \|\partial_t(\be_{\bsi_f}^I)^\rd\|^2_{\bbL^2(\Omega_f)}
+ \|\partial_t\be_{\bu_p}^I\|^2_{\bX_p}
+\, \|\partial_{tt}\be_{\bbeta_p}^I\|^2_{\bV_p}
+ \|\partial_{t}\be_{\bvarphi}^I\|^2_{\bLambda_f}  
\nonumber \\[1ex]
&\ds\quad +\, \|\partial_{tt}\be_{\bvarphi}^I\|^2_{\bLambda_f}    
+ \sum^{n-1}_{j=1} \|( \partial_t\be_{\bvarphi}^I-\partial_{tt}\be_{\bbeta_p}^I)\cdot\bt_{f,j}\|^2_{\L^2(\Gamma_{fp})} 
+ \|\partial_{tt}\be_{\lambda}^I\|^2_{\Lambda_p}
+ \|\partial_{tt}\be_{\bgamma_f}^I\|^2_{\bbQ_f} \Big) ds
+ \|\partial_{ttt}\be_{\bbeta_p}^I(0)\|^2_{\bL^2(\Omega_p)} 
\nonumber \\[1ex]
&\ds\quad +\, \|\partial_t\be_{\bbeta_p}^I(0)\|^2_{\bV_p} 
+ \|\partial_t\be_{\bvarphi}^I(0)\|^2_{\bLambda_f}
+ \|\partial_t\be_{p_p}^I(0)\|^2_{\W_p}  
+ \|\partial_t\be_{\bgamma_f}^I(0)\|^2_{\bbQ_f} 
+ \|\partial_t\be_{\lambda}^I(0)\|^2_{\Lambda_p}\Bigg) 
\nonumber \\[1ex]
&\ds\quad  +\, \delta_3\,\Bigg( \int_0^t \Big(\|\partial_t(\be_{\bsi_f}^h)^\rd\|^2_{\bbL^2(\Omega_f)}
+ \|\partial_t\be_{\bu_p}^h\|^2_{\bL^2(\Omega_p)} 
+ \sum^{n-1}_{j=1} \|(\partial_t\be_{\bvarphi}^h-\partial_{tt}\be_{\bbeta_p}^h)\cdot\bt_{f,j}\|^2_{\L^2(\Gamma_{fp})} 
+ \|\partial_{t}\be^h_{p_p}\|^2_{\W_p}
\nonumber \\[1ex] 
&\ds\quad +\, \|\be_{\bsi_f}^h\|^2_{\bbX_f} 
+ \|\be_{\bu_p}^h\|^2_{\bX_p} 
+ \|\be_{\bvarphi}^h\|^2_{\bLambda_f}
+ \|\partial_{t}\be_{\bvarphi}^h\|^2_{\bLambda_f}                       
+ \|\partial_{t}\be_{\bgamma_f}^h\|^2_{\bbQ_f} 
+ \|\partial_{t}\be_{\lambda}^h\|^2_{\Lambda_p}\Big)\,ds 
+ \|\be_{\ubsi}^h(t)\|^2
+ \|\be_{\bvarphi}^h(t)\|^2_{\bLambda_f}
\nonumber \\[1ex] 
&\ds\quad +\, \|\partial_{t}\be_{\bbeta_p}^h(t)\|^2_{\bV_p} \Bigg) 
+ C \Bigg( \|\partial_t\be_{\bu_f}^h(0)\|^2_{\bL^2(\Omega_f)}  
+ \|\be_{\bvarphi}^h(0)\|^2_{\bLambda_f}     
+ s_0\|\partial_t\be_{p_p}^h(0)\|^2_{\W_p}
+ \|\be_{\ubsi}^h(0)\|^2 
+ \|\partial_t\be_{\bbeta_p}^h(0)\|^2_{\bV_p} 
\nonumber \\[1ex]
&\ds\quad +\, \|\partial_{tt}\be_{\bbeta_p}^h(0)\|^2_{\bL^2(\Omega_p)}
+ \int_0^t \big(\|\partial_{tttt}\be_{\bbeta_p}^I\|^2_{\bL^2(\Omega_p)}   
+ \|\partial_{tt}\be_{\bbeta_p}^I\|^2_{\bV_p}  
+ \|\partial_{tt}\be^I_{p_p} \|^2_{\W_p} 
+ \|\partial_{tt}\be_{\lambda}^I\|^2_{\Lambda_p} \big)^{1/2} 
\|\partial_{t}\be_{\bbeta_p}^h\|_{\bV_p}\,ds
\nonumber \\[1ex]
&\ds\quad 
+ \int_0^t (\|\partial_{t}\bu_{fh}\|^2_{\bV_f} 
+ \|\partial_{t}\bu_{f}\|^2_{\bV_f})(\|\be_{\bu_f}^I\|^2_{\bV_f} + \|\be_{\bu_f}^h\|^2_{\bV_f})\,ds 
\nonumber \\[1ex]
&\ds\quad +\, \int_0^t (\|\partial_t\bvarphi_{h}\|^2_{\bLambda_f} 
+ \|\partial_t\bvarphi\|^2_{\bLambda_f})(\|\be_{\bvarphi}^I\|^2_{\bLambda_f} + \|\be_{\bvarphi}^h\|^2_{\bLambda_f})\,ds \Bigg) \,.
\end{align}
%
Note that for the last two terms we can use the fact that $\|\partial_t \bu_f\|_{\L^2(0,T;\bV_f)},\|\partial_t \bu_{fh}\|_{\L^2(0,T;\bV_f)}$, $\|\partial_t \bvarphi\|_{\L^2(0,T;\bLambda_f)}$,
and
$\|\partial_t\bvarphi_{h}\|_{\L^2(0,T;\bLambda_f)}$ are bounded by data (cf. \eqref{eq:continuous-stability}, \eqref{eq:discrete-stability}), to obtain
\begin{align}\label{eq:eh-uf-varphi-bound}
&\ds \int_0^t \left\{ \big( \|\partial_t \bu_f\|_{\bV_f}^2 + \|\partial_t \bu_{fh}\|_{\bV_f}^2 \big) 
\big( \|\be_{\bu_f}^I\|_{\bV_f}^2 + \|\be_{\bu_f}^h\|_{\bV_f}^2 \big) 
+ \big(\|\partial_t\bvarphi_{h}\|^2_{\bLambda_f} + \|\partial_t\bvarphi\|^2_{\bLambda_f}\big)\big(\|\be_{\bvarphi}^I\|^2_{\bLambda_f} + \|\be_{\bvarphi}^h\|^2_{\bLambda_f} \big) \right\} ds \nonumber \\[3ex]
&\ds\qquad \leq\, C\,\left(  
\|\be_{\bu_f}^h\|_{\L^{\infty}(0,t;\bV_f)}^2
+ \|\be_{\bvarphi}^h\|_{\L^{\infty}(0,t;\bLambda_f)}^2
+ \|\be_{\bu_f}^I\|_{\L^{\infty}(0,t;\bV_f)}^2
+ \|\be_{\bvarphi}^I\|_{\L^{\infty}(0,t;\bLambda_f)}^2 \right) \,.
\end{align}
%
In turn, testing \eqref{eq:NS-Biot-errorformulation-1} with $\btau_{fh} = \be_{\bsi_f}^h$, using again the discrete inf-sup condition of $B_f$ (cf. \eqref{eq:discrete inf-sup-vf-chif}) along with the continuity of  $\kappa_{\bw_f}$ (cf. \eqref{eq:continuity-cK-wf}) and the fact that $(\bu_f(t),\bvarphi(t)),(\bu_{fh}(t),\bvarphi_{h}(t)):[0,T]\to \wh{\bW}_{r_1,r_2}$ (cf. \eqref{eq:wh-Wr-definition}), we deduce
\begin{align}
&\|(\be_{\bu_{f}}^h, \be_{\bgamma_{f}}^h, \be_{\bvarphi}^h)\|_{\L^\infty(0,t;\bV_f\times \bbQ_f\times\bLambda_f)} \nonumber \\
&\quad \leq\, C\,\Big( \|(\be_{\bsi_f}^h)^\rd(t)\|_{\bbL^2(\Omega_f)} +  \|(\be_{\bsi_f}^I)^\rd(t)\|_{\bbL^2(\Omega_f)} + \|\be_{\bu_f}^I(t)\|_{\bV_f}
+ \|\be_{\bgamma_f}^I(t)\|_{\bbQ_f} + \|\be_{\bvarphi}^I(t)\|_{\bLambda_f} \Big) \,. \label{eq: error equation 7}
\end{align}	

Next, we focus on bound the terms $\|\be_{\ubsi}^h(t)\|^2:=\|\be_{\bsi_f}^h(t)\|^2_{\bbX_f} + \|\be_{\bu_p}^h(t)\|^2_{\bX_p}$ and $\|\be_{\bvarphi}^h(t)\|^2_{\bLambda_f}$ in \eqref{eq: error equation 6} and \eqref{eq:eh-uf-varphi-bound}.
In fact, recalling that we have control on $\|(\be_{\bsi_f}^h)^\rd\|^2_{\L^2(0,t;\bbL^2(\Omega_f))}$ and $\|\partial_t(\be_{\bsi_f}^h)^\rd\|^2_{\L^2(0,t;\bbL^2(\Omega_f))}$ by \eqref{eq: error equation 4a} and \eqref{eq: error equation 6}, respectively, and using the Sobolev embedding of $\H^1(0,t)$ into $\L^\infty(0,t)$, we get
%
\begin{equation}\label{eq: error equation 8}
\|(\be_{\bsi_f}^h)^\rd(t)\|^2_{\bbL^2(\Omega_f)} 
\,\leq\, C\,\int_0^t \Big(\|(\be_{\bsi_f}^h)^\rd\|^2_{\bbL^2(\Omega_f)} 
+ \|\partial_t(\be_{\bsi_f}^h)^\rd\|^2_{\bbL^2(\Omega_f)}\Big)\,ds \,,
\end{equation}
%
which together with the first equation of \eqref{eq:error-analysis7} and using \eqref{eq:tau-d-H0div-inequality}--\eqref{eq:tau-H0div-Xf-inequality}, give a control on $\|\be_{\bsi_f}^h(t)\|^2_{\bbX_f}$. 
%
Similarly, by \eqref{eq: error equation 4a} and \eqref{eq: error equation 6}, we have control on $\|\be_{\bu_p}^h\|^2_{\L^2(0,t;\bL^2(\Omega_p))}$ and $\|\partial_t\be_{\bu_p}^h\|^2_{\L^2(0,t;\bL^2(\Omega_p))}$, respectively. 
Thus, using again the Sobolev embedding of $\H^1(0,t)$ into $\L^\infty(0,t)$, we obtain
%
\begin{equation}\label{eq: error equation 9}
\|\be_{\bu_p}^h(t)\|^2_{\bL^2(\Omega_p)}
\,\leq\, C\,\int_0^t \Big(\|\be_{\bu_p}^h\|^2_{\bL^2(\Omega_p)} 
+ \|\partial_t\be_{\bu_p}^h\|^2_{\bL^2(\Omega_p)}\Big)\,ds \,,
\end{equation}
%
which, combined with the second equation of \eqref{eq:error-analysis7}, gives a control on $\|\be_{\bu_p}^h(t)\|^2_{\bX_p}$, whereas to control $\|\be_{\bvarphi}^h(t)\|^2_{\bLambda_f}$ in \eqref{eq: error equation 6}, we can use \eqref{eq: error equation 7} in combination with \eqref{eq: error equation 8}.

Thus, we combine \eqref{eq: error equation 4a} and \eqref{eq: error equation 6}--\eqref{eq: error equation 9}, and apply Lemma \ref{xing-lemma} in the context of the non-negative functions $H= \|\partial_t\bbeta_p\|_{\bV_p}$ and 
$B=\big(
\|\be_{\bbeta_p}^I\|^2_{\bV_p}  
+ \|\partial_{tt}\be_{\bbeta_p}^I\|^2_{\bV_p} 
+ \|\partial_{tttt}\be_{\bbeta_p}^I\|^2_{\bL^2(\Omega_p)}   
+  \|\be_{p_{p}}^I\|^2_{\W_p}  
+ \|\partial_{tt}\be^I_{p_p} \|^2_{\W_p} 
+ \|\be_{\lambda}^I\|^2_{\Lambda_p} 
+ \|\partial_{tt}\be_{\lambda}^I\|^2_{\Lambda_p}\big)^{1/2}$, with $R$ and $A$ representing the remaining terms, to control $\int^t_0 B(s)\,H(s)\,ds$, 
along with \eqref{eq:error-analysis7} and the Sobolev embedding \eqref{sobolev-L-infty-L2} to bound the terms $\|\bdiv(\be_{\bsi_f}^h)\|_{\bL^{4/3}(\Omega_f)}$ and $\|\div(\be^h_{\bu_p})\|_{\L^2(\Omega_p)}$ in $\L^2(0,T)$, and their corresponding right-hand side in $\L^\infty(0,T)$. 
Then, employing \eqref{eq:tau-d-H0div-inequality}--\eqref{eq:tau-H0div-Xf-inequality}, choosing $\delta_2$ and $\delta_3$ small enough, and performing some algebraic computations, we obtain
%
\begin{align}
&\ds \|\be_{\bu_f}^h(t)\|^2_{\bL^2(\Omega_f)} 
+ s_0\,\|\be_{p_p}^h(t)\|^2_{\W_p} + \|\be_{\bbeta_p}^h(t)\|^2_{\bV_p} 
+ \|\partial_{t}\be_{\bbeta_p}^h(t)\|^2_{\bL^2(\Omega_p)} 
+ \|\partial_t\be_{\bu_f}^h(t)\|^2_{\bL^2(\Omega_f)}  
+ s_0\,\|\partial_t\be_{p_p}^h(t)\|^2_{\W_p} \nonumber \\[1ex]
&\ds\quad +\, \|\partial_t\be_{\bbeta_p}^h(t)\|^2_{\bV_p} 
+ \|\partial_{tt}\be_{\bbeta_p}^h(t)\|^2_{\bL^2(\Omega_p)} 
+ \int^t_0 \Big(\|\be_{\ubsi}^h\|^2  
+ \|\partial_t(\be_{\bsi_f}^h)^\rd\|^2_{\bbL^2(\Omega_f)}  
+ \|\partial_t\be_{\bu_p}^h\|^2_{\bL^2(\Omega_p)}
\nonumber \\[1ex]
&\ds\quad +\, \sum^{n-1}_{j=1} \|( \be_{\bvarphi}^h-\partial_t\be_{\bbeta_p}^h)\cdot\bt_{f,j}\|^2_{\L^2(\Gamma_{fp})} 
+ \sum^{n-1}_{j=1} \|( \partial_t\be_{\bvarphi}^h-\partial_{tt}\be_{\bbeta_p}^h)\cdot\bt_{f,j}\|^2_{\L^2(\Gamma_{fp})}
+ \|(\be_{\bu_{f}}^h, \be_{p_{p}}^h, \be_{\bgamma_{f}}^h, \be_{\bvarphi}^h, \be_{\lambda}^h)\|^2  
\nonumber \\[1ex]
&\ds\quad + \|(\partial_t\be_{\bu_{f}}^h, \partial_t\be_{p_{p}}^h, \partial_t\be_{\bgamma_{f}}^h, \partial_t\be_{\bvarphi}^h, \partial_t\be_{\lambda}^h)\|^2\Big)\,ds
\nonumber \\[1ex]
&\ds \leq\, C\,T\,\Bigg( \int_0^t \Big(\|\be_{\ubsi}^I\|^2
+ \|\be_{\ubu}^I\|^2 
+ \sum^{n-1}_{j=1} \|( \be_{\bvarphi}^I-\partial_t\be_{\bbeta_p}^I)\cdot\bt_{f,j}\|^2_{\L^2(\Gamma_{fp})} 
+ \|\partial_{t}\be_{\bbeta_p}^I\|^2_{\bV_p} 
+ \|\partial_{tt}\be_{\bbeta_p}^I\|^2_{\bV_p}
\nonumber \\[1ex]
&\ds\quad 
+\, \|\partial_{t}\be_{\bsi_f}^I\|^2_{\bbX_f} 
+ \|\partial_{tt}\be_{p_p}^I\|^2_{\W_p} 
+ \|\partial_t(\be_{\bsi_f}^I)^\rd\|^2_{\bbL^2(\Omega_f)} 
+ \|\partial_t\be_{\bu_p}^I\|^2_{\bX_p}
+ \|\partial_{t}\be_{\bvarphi}^I\|^2_{\bLambda_f}  
+ \|\partial_{tt}\be_{\bgamma_f}^I\|^2_{\bbQ_f} 
+ \|\partial_{tt}\be_{\bvarphi}^I\|^2_{\bLambda_f}  
\nonumber \\[1ex]
&\ds\quad 
+\, \sum^{n-1}_{j=1} \|( \partial_t\be_{\bvarphi}^I-\partial_{tt}\be_{\bbeta_p}^I)\cdot\bt_{f,j}\|^2_{\L^2(\Gamma_{fp})}   
+ \|\partial_{tt}\be_{\lambda}^I\|^2_{\Lambda_p} 
+ \|\partial_{tttt}\be_{\bbeta_p}^I\|^2_{\bL^2(\Omega_p)}\Big)\,ds
+ \|\partial_{ttt}\be_{\bbeta_p}^I(t)\|^2_{\bL^2(\Omega_p)}  
\nonumber \\[1ex]
&\ds\quad 
+ \|\partial_t\be_{\bbeta_p}^I(t)\|^2_{\bV_p} 
+ \|\partial_t\be_{p_p}^I(t)\|^2_{\W_p} 
+ \|\partial_t\be_{\bvarphi}^I(t)\|^2_{\bLambda_f}
+ \|\partial_t\be_{\bgamma_f}^I(t)\|^2_{\bbQ_f} 
+ \|\partial_t\be_{\lambda}^I(t)\|^2_{\Lambda_p} 
+ \|\partial_{ttt}\be_{\bbeta_p}^I(0)\|^2_{\bL^2(\Omega_p)}  
\nonumber \\[1ex]
&\ds\quad 
+ \|\partial_t\be_{\bbeta_p}^I(0)\|^2_{\bV_p} 
+ \|\partial_t\be_{p_p}^I(0)\|^2_{\W_p}  
+ \|\partial_t\be_{\bvarphi}^I(0)\|^2_{\bLambda_f}
+ \|\partial_t\be_{\bgamma_f}^I(0)\|^2_{\bbQ_f} 
+ \|\partial_t\be_{\lambda}^I(0)\|^2_{\Lambda_p}
+ \|\be_{\bu_f}^h(0)\|^2_{\bL^2(\Omega_f)} 
\nonumber \\[1ex]
&\ds\quad 
+ \|\be_{\bbeta_p}^h(0)\|^2_{\bV_p} 
+\, s_0\,\|\be^h_{p_p}(0)\|^2_{\W_p} 
+ \|\partial_t\be_{\bu_f}^h(0)\|^2_{\bL^2(\Omega_f)} 
+ s_0\,\|\partial_t\be_{p_p}^h(0)\|^2_{\W_p}
+ \|\partial_t\be_{\bbeta_p}^h(0)\|^2_{\bV_p} \nonumber \\[1ex]
&\ds\quad  
+ \,\|\partial_{tt}\be_{\bbeta_p}^h(0)\|^2_{\bL^2(\Omega_p)}
+ \|\be_{\ubsi}^h(0)\|^2 
+ \|\be_{\bvarphi}^h(0)\|^2_{\bLambda_f}  \Bigg)\,. 
\label{eq: error equation 10}
\end{align}
%
\noindent{\bf{Bounds on initial data.}}

\noindent Finally, to bound the initial data terms in \eqref{eq: error equation 10}, we first recall from Theorems \ref{thm:unique soln} and
\ref{thm: well-posedness main result semi} that 
$(\bsi_f(0),\bu_p(0), \bbeta_p(0), \bu_f(0), p_p(0), \bgamma_f(0), \bvarphi(0), \lambda(0)) = (\bsi_{f,0},\bu_{p,0}, \bbeta_{p,0}, \bu_{f,0}, p_{p,0}, \bgamma_{f,0}, \bvarphi_{0}, \lambda_{0})$ 
and 
$(\bsi_{fh}(0),$ $\bu_{ph}(0),\bbeta_{ph}(0),
\bu_{fh}(0), p_{ph}(0), \bgamma_{fh}(0), \bvarphi_{h}(0), \lambda_{h}(0)) = (\bsi_{fh,0},\bu_{ph,0}, \bbeta_{ph,0}, \bu_{fh,0}, p_{ph,0}, \bgamma_{fh,0}, \bvarphi_{h,0}, \lambda_{h,0})$, respectively.
Recall also that discrete initial data satisfy \eqref{eq:system-discrete-sol0-1} and \eqref{eq:mixed-elasticity-discrete-problem}. Then, in a way similar to \eqref{eqn:discrete-soln-1-bound} and \eqref{eq: discrete sige-etap-bound}, we obtain
%
\begin{align}\label{eq: error initial 1}
& \|\be_{\bsi_f}^h(0)\|_{\bbX_f} + \|\be_{\bu_p}^h(0)\|_{\bX_p} + \|P_h^{\V}(\bsi_{e,0})-\widetilde\bsi_{eh,0}\|_{\bSigma_{e}} +\|\be_{\bu_f}^h(0)\|_{\bL^2(\Omega_f)} + \|\be^h_{p_p}(0)\|_{\W_p} + \|\be_{\bgamma_f}^h(0)\|_{\bbQ_f}  \nonumber \\
&\quad +\, \|\partial_t\be_{\bbeta_p}^h(0)\|_{\bV_p} +  \|\be_{\bvarphi}^h(0)\|_{\bLambda_f} + \|\be_{\lambda}^h(0)\|_{\Lambda_p} 
\,\leq\, C\,\Big( \|\be_{\bsi_f}^I(0)\|_{\bbX_f} + \|\be_{\bu_p}^I(0)\|_{\bX_p} +\|\be_{\bu_f}^I(0)\|_{\bV_f} \nonumber \\
&\quad +\, \|\partial_t\be_{\bbeta_p}^I(0)\|_{\bV_p}  +  \|\be_{p_p}^I(0)\|_{\W_p}  +  \|\be_{\bvarphi}^I(0)\|_{\bLambda_f} + \|\be_{\lambda}^I(0)\|_{\Lambda_p}
+ \|\bsi_{e}(0) - P_h^{\V}(\bsi_{e,0})\|_{\bSigma_{e}}\Big) 
\end{align}
%
and
%
\begin{align}\label{eq: error initial 2a}
& \|P_h^{\V}(\bsi_{e,0})-\bsi_{eh,0}\|_{\bSigma_{e}} + \|S^{\bV_p}_h(\bbeta_{p,0})-\widetilde\bbeta_{ph,0}\|_{\bV_p} 
\,\leq\, C\,\Big( \|\partial_t\be_{\bbeta_p}^h(0)\|_{\bV_p} +  \|\be_{\bvarphi}^h(0)\|_{\bLambda_f} + \|\be_{\lambda}^h(0)\|_{\Lambda_p} \nonumber \\
& \quad + \|\be^h_{p_p}(0)\|_{\W_p} +\, \|\bsi_{e}(0) - P_h^{\V}(\bsi_{e,0})\|_{\bSigma_{e}}
+ \|\partial_t\be_{\bbeta_p}^I(0)\|_{\bV_p} +  \|\bbeta_p(0)-S^{\bV_p}_h(\bbeta_{p,0})\|_{\bV_p} + \|\be_{\bvarphi}^I(0)\|_{\bLambda_f} \nonumber \\
& \quad + \|\be_{\lambda}^I(0)\|_{\Lambda_p} + \|\be^I_{p_p}(0)\|_{\W_p} \Big) \,.
\end{align}
%
In addition, to bound $\|\be_{\bbeta_p}^h(0)\|_{\bV_p}$, we first multiply the equation \eqref{eq:system-sol0-4} by $\be(\bxi_{ph})$ to obtain
\begin{equation}\label{etap0-cont}
a^e_p(\bbeta_{p,0},\bxi_{ph}) 
= - b_s(\bsi_{e,0},\bxi_{ph})\quad \forall\,\bxi_{ph} \in \bV_{ph} \,,
\end{equation}
and then, subtracting \eqref{discrete etap0-bound} from \eqref{etap0-cont}, using coercivity and continuity of the bilinear forms involved, we deduce
%
\begin{align}\label{eq: error initial 2b}
\|\be^h_{\bbeta_p}(0)\|_{\bV_p} \,\leq\, C\,\Big( \|\be^I_{\bbeta_p}(0)\|_{\bV_p} + \|P_h^{\V}(\bsi_{e,0})-\bsi_{eh,0}\|_{\bSigma_{e}} + \, \|\bsi_{e}(0) - P_h^{\V}(\bsi_{e,0})\|_{\bSigma_{e}} \Big)\,.
\end{align}
%
In turn, for the terms $\|\partial_t\be_{\bu_f}^h(0)\|^2_{\bL^2(\Omega_f)}$ and 
$s_0\,\|\partial_t\be_{p_p}^h(0)\|^2_{\W_p}$ in \eqref{eq: error equation 10}, we consider the error equation \eqref{eq:NS-Biot-errorformulation-1} at $t=0$, 
use \eqref{eq:semi-discrete-weak-formulation-1a}--\eqref{eq:semi-discrete-weak-formulation-1c} and \eqref{eq:semi-discrete-weak-formulation-1f}--\eqref{eq:semi-discrete-weak-formulation-1i} and take $(\bv_{fh}, w_{ph}) = (\partial_t\be_{\bu_f}^h(0), \partial_t\be_{p_p}^h(0))$, to get
%
\begin{align}\label{eq: error initial 3a}
&\rho_f\|\partial_t\be_{\bu_f}^h(0)\|^2_{\bL^2(\Omega_f)} 
+ s_0\,\|\partial_t\be_{p_p}^h(0)\|^2_{\W_p}  = 0 \,,
\end{align}
%
where, the right-hand side of \eqref{eq: error initial 3a} has been simplified, since the projection property \eqref{eq:projection1}, 
imply that the terms :
$s_0 (\partial_t\be_{p_p}^I(0),\partial_t\be_{p_p}^h(0))_{\Omega_p}$ and $\rho_f (\partial_t\be_{\bu_f}^I(0),\partial_t\be_{\bu_f}^h(0))_{\Omega_f}$ are zero.


Finally, to bound $\|\partial_{tt}\be_{\bbeta_p}^h(0)\|^2_{\bL^2(\Omega_p)}$, we consider the error equation \eqref{eq:NS-Biot-errorformulation-1} at $t=0$ for the test function $\bxi_{ph}$, such that
\begin{equation}\label{error-etap0-2}
\begin{array}{l}
\ds\rho_p(\partial_{tt}\be_{\bbeta_p}(0),\bxi_{ph})_{\Omega_p} \\[1ex] 
\ds\quad \,=\, - a^e_p(\be_{\bbeta_p}(0),\bxi_{ph})
- \alpha_p\,b_p(\bxi_{ph},\be_{p_p}(0))  
- c_{\BJS}(\partial_t\be_{\bbeta_p}(0), \be_{\bvarphi}(0);\bxi_{ph}, \0)
+ c_{\Gamma}(\bxi_{ph},\0;\be_{\lambda}(0)) \,, 
\end{array}
\end{equation}
where, using \eqref{discrete etap0-bound} and \eqref{etap0-cont}, we deduce that $a^e_p(\be_{\bbeta_p}(0),\bxi_{ph}) = -b_s(\be_{\bsi_e}(0),\bxi_{ph})$, which, together with \eqref{eq:semi-discrete-weak-formulation-1l}, implies that the right-hand side of \eqref{error-etap0-2} is zero. Then, using the decomposition of the error $\partial_{tt}\be_{\bbeta_p}(0) = \partial_{tt}\be^h_{\bbeta_p}(0) + \partial_{tt}\be^I_{\bbeta_p}(0)$, and applying Cauchy–Schwarz and Young's inequalities, we obtain
%
%\begin{align*}
%&\|\partial_{tt}\be_{\bbeta_p}^h(0)\|^2_{\bL^2(\Omega_p)}  \leq C \|\partial_{tt}\be_{\bbeta_p}^I(0)\|^2_{\bL^2(\Omega_p)} + \delta \|\partial_{tt}\be_{\bbeta_p}^h(0)\|^2_{\bL^2(\Omega_p)},
%\end{align*}
%%
%and taking $\delta$ small enough, implies
%
\begin{align}\label{eq: error initial 3b}
& \|\partial_{tt}\be_{\bbeta_p}^h(0)\|^2_{\bL^2(\Omega_p)}  
\leq C\,\|\partial_{tt}\be_{\bbeta_p}^I(0)\|^2_{\bL^2(\Omega_p)} \,.
\end{align}
%
Hence, combining \eqref{eq: error equation 10} with \eqref{eq: error initial 1}, \eqref{eq: error initial 2a}, \eqref{eq: error initial 2b}, \eqref{eq: error initial 3a} and \eqref{eq: error initial 3b}, making use of triangle inequality and the approximation properties \eqref{eq:approx-property1}, \eqref{eq:approx-property2}, and \eqref{eq: approx property 3}, we obtain \eqref{eq:errror-rate-of-convergence} and complete the proof.
\end{proof}

%**********************************************************************
%**********************************************************************

\section{Numerical results}\label{sec:numerical}

For the fully discrete scheme utilized in the numerical tests we employ the backward Euler method for the time discretization. 
Let $\Delta t$ be the time step, $T=N \Delta t$, $t_m = m\,\Delta t$, $m=0,\dots, N$. 
Let $d_t\, u^m := (\Delta t)^{-1}(u^m - u^{m-1})$ be the first order (backward) discrete 
time derivative and  $d_{tt} \, u^m := (\Delta t)^{-2}(u^m - 2u^{m-1} + u^{m-2})$ be the second order (backward) discrete 
time derivative, where $u^m := u(t_m)$. Then the fully discrete model reads: 
Given $(\bsi_{fh}^0,\bu_{ph}^0, \bbeta_{ph}^0, \bu_{fh}^0, p_{ph}^0, \bgamma_{fh}^0, \bvarphi_{h}^0, \lambda_{h}^0)=(\bsi_{fh,0},\bu_{ph,0}, \bbeta_{ph,0}, \bu_{fh,0}, p_{ph,0},$
$ \bgamma_{fh,0}, \bvarphi_{h,0}, \lambda_{h,0})$, find $(\bsi_{fh}^m,\bu_{ph}^m, \bbeta_{ph}^m, \bu_{fh}^m, p_{ph}^m, \bgamma_{fh}^m, \bvarphi_{h}^m, \lambda_{h}^m)\in\bbX_{fh}\times \bX_{ph}\times \bV_{ph}\times \bV_{fh}\times \W_{ph}\times \bbQ_{fh}\times \bLambda_{fh}\times \Lambda_{ph}$, $m=1, \dots, N$, such that 
\begin{align}
& \rho_f (d_t\,\bu_{fh}^m,\bv_{fh})_{\Omega_f}
+ a_f(\bsi_{fh}^m,\btau_{fh})
+ b_{\bn_f}(\btau_{fh},\bvarphi_{h}^m) 
+ b_f(\btau_{fh},\bu_{fh}^m) 
+ b_\sk(\bgamma_{fh}^m,\btau_{fh})
\nonumber\\ 
&\ds\quad +\, \kappa_{\bu_{fh}^m}(\bu_{fh}^m, \btau_{fh}) 
- b_f(\bsi_{fh}^m,\bv_{fh}) 
- b_\sk(\bsi_{fh}^m,\bchi_{fh})  
\,=\, (\f_{f}^m,\bv_{fh})_{\Omega_f}\,,  
\nonumber\\ 
& \rho_p(d_{tt}\bbeta_{ph}^m,\bxi_{ph})_{\Omega_p} 
+ a^e_p(\bbeta_{ph}^m,\bxi_{ph})
+ \alpha_p\,b_p(\bxi_{ph}^m,p_{ph}) 
+ c_{\BJS}(d_t\,\bbeta_{ph}^m, \bvarphi_{h}^m;\bxi_{ph}, \bpsi_{h}) 
\nonumber\\ 
&\ds\quad -\, c_{\Gamma}(\bxi_{ph},\bpsi_{h};\lambda_{h}^m) 
- b_{\bn_f}(\bsi_{fh}^m,\bpsi_{h})
+ l_{\bvarphi_{h}^m}(\bvarphi_{h}^m,\bpsi_{h}) 
\,=\, (\f_{p}^m,\bxi_{ph})_{\Omega_p}\,, 
\nonumber\\ 
& s_0\,(d_t\,p_{ph}^m,w_{ph})_{\Omega_p} 
+ a^d_p(\bu_{ph}^m,\bv_{ph}) 
+ b_p(\bv_{ph},p_{ph}^m)
+ b_{\bn_p}(\bv_{ph},\lambda_{h}^m) 
- \alpha_p\,b_p(d_t\,\bbeta_{ph}^m,w_{ph})
\nonumber\\ 
&\ds\quad -\, b_p(\bu_{ph}^m,w_{ph}) 
\,=\, (q_{p}^m,w_{ph})_{\Omega_p},  
\nonumber\\ 
& c_{\Gamma}(d_t\,\bbeta_{ph}^m,\bvarphi_{h}^m;\xi_{h})
- b_{\bn_p}(\bu_{ph}^m,\xi_{h})
\,=\,0,  
\label{eq: discrete formulation}
\end{align}
for all $(\btau_{fh}, \bv_{ph}, \bxi_{ph}, \bv_{fh}, w_{ph}, \bchi_{fh}, \bpsi_{h}, \xi_{h})\in \bbX_{fh}\times \bX_{ph}\times \bV_{ph}\times \bV_{fh}\times \W_{ph}\times \bbQ_{fh}\times \bLambda_{fh}\times \Lambda_{ph}$.
The fully discrete method results in the solution of a nonlinear algebraic system at each time step. The well posedness and error analysis of the fully discrete scheme is beyond the scope of the paper.

Next, we present numerical results that illustrate the behavior of the fully discrete method \eqref{eq: discrete formulation}. We use the Newton--Rhapson method to solve this nonlinear algebraic system at each time step. 
The implementation is based on a {\tt FreeFEM} code \cite{Hecht} on triangular grids. 
For spatial discretization we use the following finite element spaces: $\bbBDM_1-\bP_0-\bbP_0$ for stress-velocity-vorticity in Navier--Stokes, 
$\bP_1$ for displacement in elasticity,
$\bBDM_1-\rP_0$ for Darcy velocity-pressure, and $\bP_1 - \rP_1$ for the the traces of fluid velocity and Darcy pressure. 

The examples considered in this section are described next.
Example 1 is used to corroborate the rates of convergence. 
In Example 2 we present a simulation of air flow through a filter. 

%**********************************************************************
%**********************************************************************

\subsection{Example 1: convergence test}

In this test we study the convergence for the space discretization using an analytical solution.
The domain is $\Omega = \Omega_f\cup\Gamma_{fp}\cup\Omega_p$, with $\Omega_f = (0,1)\times (0,1), \Gamma_{fp} = (0,1)\times\{0\}$, and $\Omega_p = (0,1)\times (-1,0)$; i.e.,
the upper half is associated with the Navier--Stokes flow, while the lower half represents the poroelastic medium governed by the Biot system, see Figure~\ref{fig:example1} (left). The analytical solution is given in Figure~\ref{fig:example1} (right). It satisfies the appropriate interface conditions along the interface $\Gamma_{fp}$.
%
\begin{figure}[htb!]
\begin{minipage}{.49\textwidth}
\centering\includegraphics[scale=1]{domain-ex1.png}
\end{minipage}
%
\hfill
\begin{minipage}{.5\textwidth}

Solution in the Navier-Stokes region:

  \bigskip
$ \ds \bu_f = \pi \cos(\pi t)
\begin{pmatrix}
\ds -3x + \cos(y) \\[1ex] \ds y+1
\end{pmatrix}$

\smallskip
$ \ds p_f = \exp(t)\,\sin(\pi x)\cos\Big(\frac{\pi y}{2}\Big) + 2\pi \cos(\pi t)$

\bigskip
\bigskip
Solution in the Biot region:

\bigskip
$\ds p_p = \exp(t)\,\sin(\pi x)\cos\Big(\frac{\pi y}{2}\Big)$

$\ds \bu_p = -\frac{1}{\mu} \bK \nabla p_p $

$ \ds \bbeta_p = \sin(\pi t) \begin{pmatrix} \ds -3x+\cos(y) \\[1ex] \ds y+1 \end{pmatrix}$
\end{minipage}
\caption{[Example 1] Left: computational domain at the coarsest mesh level. Right: analytical solution.}
\label{fig:example1}
\end{figure}

The model parameters are
%
$$
\mu = 1, \quad \rho_f = 1,  \quad \rho_p = 1, \quad \lambda_p = 1, \quad \mu_p = 1, \quad s_0 = 1, \quad \bK = \bI, \quad \alpha_p = 1, \quad \alpha_{\BJS} = 1\,.
$$
%
The right-hand side functions $\f_f, \f_p$ and $q_p$ are computed from \eqref{eq:Navier-Stokes-1}--\eqref{eq:Biot-model} using the analytical solution. 
Note that, we do not take the analytical solution of fluid velocity to be divergence-free and then we define $q_f:= \div(\bu_f)$ and modify accordingly \eqref{eq: discrete formulation} to account this data. 
%Therefore, \eqref{eq:continuous-weak-formulation-1a}--\eqref{eq:continuous-weak-formulation-1b} are now implementd as:
%%\begin{subequations*}%\label{eq:discrete-navier-sweak-formulation-1}
%\begin{align*}
%&\ds  a_f(\bsi_{fh}^m,\btau_{fh})+b_{\bn_f}(\btau_{fh},\bvarphi_{h}^m) + b_f(\btau_{fh},\bu_{fh}^m) + b_\sk(\bgamma_{fh}^m,\btau_{fh})+\kappa_{\bu_{fh}^m}(\bu_{fh}^m, \btau_{fh}) = -\frac{1}{n} (q_f^m, \tr(\btau_{fh}))_{\Omega_f}, %\label{eq:discrete-weak-formulation-1a}
%\\[1ex]
%&\ds \rho_f (d_t\,\bu_{fh}^m,\bv_{fh})_{\Omega_f} -\rho_f q_f(\bu_{fh}^m,\bv_{fh})_{\Omega_f} -b_f(\bsi_{fh}^m,\bv_{fh}) \, = \, (\f_{f}^m,\bv_{fh})_{\Omega_f}, %\label{eq:discrete-weak-formulation-1b}
%\end{align*}
%%\end{subequations*}
%for all $(\btau_{fh},\bv_{fh})\in \bbX_{fh}\times \bV_{fh}$.
The model problem is then complemented with the appropriate boundary conditions, as shown in Figure~\ref{fig:example1} (left), and initial data.
Notice that the boundary conditions are not homogeneous and therefore the right-hand side of the resulting system must be modified accordingly.
The total simulation time for this test case is $T=0.01$ and the time step is $\Delta t=10^{-3}$. The time step is sufficiently small, so that the time discretization error does not affect the spatial convergence rates.
Table~\ref{table1-example1} shows the convergence history for a sequence of quasi-uniform mesh refinements with non-matching grids along the interface employing conforming spaces for the Lagrange multipliers \eqref{defn-Lambda-h}. The grids on the coarsest level are shown in Figure~\ref{fig:example1} (left). In the table, $h_f$ and $h_p$ denote the mesh sizes in $\Omega_f$ and $\Omega_p$, respectively, while the mesh sizes for their traces on $\Gamma_{fp}$ are $h_{tf}$ and $h_{tp}$. 
We note that the Navier--Stokes pressure at $t_m$ is recovered by the post-processed formulae $\ds p_{fh}^m = -\frac{1}{n} \left( \tr(\bsi_{fh}^m)  + \rho_f\,\tr(\bu_{fh}^m\otimes\bu_{fh}^m) - 2\mu q_f^m \right)$ (cf. \eqref{eq:pseudostress-pressure-formulae}).
The results illustrate that at least the optimal spatial rate of convergence $\cO(h)$ established in Theorem~\ref{thm: error analysis} is attained for all subdomain variables.
The Lagrange multiplier variables, which are approximated in $\bP_1-\rP_1$, exhibit a rate of convergence $\cO(h^{3/2})$ in the $\H^{1/2}$-norm on $\Gamma_{fp}$, which is consistent with the order of approximation.
%%%%%%%%%%% Tables %%%%%%%%%%%%%%%%%%%%%
\begin{table}[ht]
\begin{center}
\small{			
\begin{tabular}{|c||cc|cc|cc|cc||c||cc|}
\hline
& \multicolumn{2}{|c|}{$\|\be_{\bsi_{f}}\|_{\ell^2(0,T;\bbX_f)}$}  
& \multicolumn{2}{|c|}{$\|\be_{\bu_f}\|_{\ell^2(0,T;\bV_f)}$} 
& \multicolumn{2}{|c|}{$\|\be_{\bgamma_f}\|_{\ell^2(0,T;\bbQ_f)}$}  
& \multicolumn{2}{|c||}{$\|\be_{p_f}\|_{\ell^2(0,T;\L^2(\Omega_f))}$} 
&
& \multicolumn{2}{|c|}{$\|\be_{\bvarphi}\|_{\ell^2(0,T;\bLambda_f)}$} \\
$h_f$  & error & rate & error & rate & error & rate & error & rate & $h_{tf}$ & & \\  \hline
0.354 & 7.5E-01 &   --  & 2.4E-02 &   --  & 6.3E-02 &   --  & 1.4E-01 &   --  & 0.500 & 5.0E-03 &   -- \\ 
0.177 & 2.4E-01 & 1.656 & 1.2E-02 & 0.999 & 3.3E-02 & 0.927 & 6.1E-02 & 1.164 & 0.250 & 1.5E-03 & 1.722 \\
0.088 & 8.7E-02 & 1.453 & 6.1E-03 & 1.000 & 1.7E-02 & 0.978 & 3.0E-02 & 1.025 & 0.125 & 5.7E-04 & 1.424 \\
0.044 & 3.7E-02 & 1.233 & 3.1E-03 & 1.000 & 8.4E-03 & 0.995 & 1.5E-02 & 1.004 & 0.063 & 2.1E-04 & 1.419 \\
0.022 & 1.8E-02 & 1.085 & 1.5E-03 & 1.000 & 4.2E-03 & 0.999 & 7.5E-03 & 1.001 & 0.031 & 7.4E-05 & 1.521 \\
0.011 & 8.6E-03 & 1.027 & 7.7E-04 & 1.000 & 2.1E-03 & 1.000 & 3.7E-03 & 1.000 & 0.016 & 2.4E-05 & 1.608 \\
\hline 
\end{tabular}
		
\medskip

\begin{tabular}{|c||cc|cc|cc||c||cc||c|}
\hline
& \multicolumn{2}{|c|}{$\|\be_{\bu_p}\|_{\ell^2(0,T;\bX_p)}$} 
& \multicolumn{2}{|c|}{$\|\be_{p_p}\|_{\ell^\infty(0,T;\W_p)}$}   
& \multicolumn{2}{|c||}{$\|\be_{\bbeta_p}\|_{\ell^\infty(0,T;\bV_p)}$}
&
& \multicolumn{2}{|c||}{$\|\be_{\lambda}\|_{\ell^2(0,T;\Lambda_p)}$}  & \\ 
$h_p$  & error  & rate & error & rate & error & rate & $h_{tp}$ & error & rate & iter  \\  \hline
0.236 & 5.5E-02 &   --  & 6.9E-02 &   --  & 1.3E-04 &   --  & 0.333 & 4.6E-03 &   --  & 2.2 \\
0.118 & 2.0E-02 & 1.454 & 3.5E-02 & 0.998 & 6.5E-05 & 1.001 & 0.167 & 1.6E-03 & 1.532 & 2.2 \\
0.059 & 7.9E-03 & 1.347 & 1.7E-02 & 0.999 & 3.2E-05 & 0.999 & 0.083 & 5.6E-04 & 1.515 & 2.2 \\
0.029 & 3.5E-03 & 1.167 & 8.6E-03 & 1.000 & 1.6E-05 & 0.998 & 0.042 & 2.0E-04 & 1.504 & 2.2 \\
0.015 & 1.7E-03 & 1.046 & 4.3E-03 & 1.000 & 8.1E-06 & 0.996 & 0.021 & 7.0E-05 & 1.501 & 2.2 \\
0.007 & 8.4E-04 & 1.008 & 2.2E-03 & 1.000 & 4.1E-06 & 0.995 & 0.010 & 2.5E-05 & 1.500 & 2.2 \\
\hline 
\end{tabular}
%
\caption{[Example 1] Mesh sizes, errors, rates of convergences and average Newton iterations for the fully discrete system $(\bbBDM_1-\bP_0-\bbP_0)-\bP_1-(\bBDM_1-\rP_0) - (\bP_1 - \rP_1)$ approximation for the Navier--Stokes--Biot model in non-matching grids.}\label{table1-example1}
}
\end{center}
\end{table}

%**********************************************************************
%**********************************************************************

\subsection{Example 2: Air flow through a filter}

The setting in this example is similar to the 
one presented in \cite{swgbh2020}, where the Navier--Stokes--Darcy model is considered. The domain is a two-dimensional rectangular channel with length $2.5$m and width $0.25$m, which on the bottom center is partially blocked by a square poroelastic filter of length and width $0.2$m, see Fig. \ref{fig:filter} (top).
%
\begin{figure}%[ht]
\includegraphics[width=\textwidth]{Domain1.png}
%
\begin{equation*}
\begin{array}{c}
\ds \bT_f \bn_f = -p_{in} \bn_f \qon \Gamma_f^{in},\quad \bT_f \bn_f = -p_{out} \bn_f \qon \Gamma_f^{out} \\[0.5ex]
\ds p_{in}= p_{ref} + 2\times 10^{-6} \text{kPa},\quad p_{out} = p_{ref},\quad \bu_f = 0 \qon \Gamma_f^{top} \cup \Gamma_f^{bottom} \\[0.5ex]
\ds \bbeta_p = \0 \qan \bu_p \cdot \bn_p = 0 \qon \Gamma_p^{bottom}
\end{array}
\end{equation*}
\vspace{-0.5cm}
\caption{[Example 2] Top: computational domain and boundaries; channel $\Omega_{f}$ in gray, filter $\Omega_{p}$ in brown. Bottom: boundary conditions with $p_{ref}=100\,\text{kPa}$.}
\label{fig:filter}
\end{figure}
%
The model parameters are set as
%
\begin{equation*}
\begin{array}{c}
\mu=1.81 \times 10^{-8} \text{ kPa s}, \quad \rho_f =1.225 \times 10^{-3} \text{ Mg}/\text{m}^3, \quad s_0=7 \times 10^{-2} \text{ kPa}^{-1}, \\[1ex]
\bK= [ 0.505, -0.495; -0.495,0.505] \times 10^{-6} \text{ m}^2,  \quad \rho_p =1.601 \times 10^{-2} \text{ Mg}/\text{m}^3, \quad \alpha_{\BJS}=1.0, \quad \alpha_p = 1.0.
\end{array}
\end{equation*}
%
Note that $\mu$ and $\rho_f$ are parameters for air. The permeability tensor $\bK$ is obtained by rotating the identity tensor by a $45^\circ$ rotation angle in order to consider the effect of material anisotropy on the flow. We further consider a stiff material in the poroelastic region with parameters:
$\lambda_p=1{\times}10^4\,\text{kPa}$ and $\mu_p = 1{\times}10^5\,\text{kPa}$. 
%
The top and bottom of the domain are rigid, impermeable walls. The flow is driven by a pressure difference $\Delta p = 2\times 10^{-6}\,\text{kPa}$ between the left and right boundary, see Figure \ref{fig:filter} (bottom) for the boundary conditions. The body force terms $\f_f$ and $\f_p$ and external source $q_p$ are set to zero. For the initial conditions, we consider
$$ p_{p,0}=100  \ \text{kPa}, \quad \bbeta_{p,0} = \0  \ \text{m},  \quad \bu_{s,0} = \0 \ \text{m/s},\quad \bu_{f,0} = \0  \ \text{m/s}. $$
%
The computational matching grid along $\Gamma_{fp}$ has a characteristic parameter $h=\max\{ h_f, h_p \} = 0.018$. The total simulation time is $T=400$s with $\Delta t=1$s.

The computed magnitude of the velocities and pressures are displayed in Figures \ref{fig:filter-velocity} and \ref{fig:filter-pressure}, respectively. We observe high velocity in the narrow open channel above the filter. Vortices develop behind the obstacle, which travel with the fluid and are smoothed out at later times. A sharp pressure gradient is observed in the region above the filter, as well as within the filter, where the permeability anisotropy affects both the pressure and velocity fields. This example illustrates the ability of the mixed method to produce oscillation-free solution in a regime of challenging physical parameters, including small viscosity and permeability and large Lam\'e coefficients. 
\begin{figure}[ht]
\includegraphics[width=\textwidth]{uh-magnitude-timestep-20.png}
\includegraphics[width=\textwidth]{uh-magnitude-timestep-80.png}
\includegraphics[width=\textwidth]{uh-magnitude-timestep-200.png}
\includegraphics[width=\textwidth]{uh-magnitude-timestep-400.png}
\vspace{-0.4cm}
\caption{[Example 2] Computed velocities $\bu_{fh}$ and $\bu_{ph}$ (arrows not scaled) and their magnitudes at times $t\in \{ 20, 80, 200, 400 \}$ (from top to bottom).} \label{fig:filter-velocity}
\end{figure}

\begin{figure}[ht]
\includegraphics[width=\textwidth]{ph-magnitude-timestep-20.png}
\includegraphics[width=\textwidth]{ph-magnitude-timestep-80.png}
\includegraphics[width=\textwidth]{ph-magnitude-timestep-200.png}	
\includegraphics[width=\textwidth]{ph-magnitude-timestep-400.png}
\vspace{-0.4cm}
\caption{[Example 2] Computed pressures $p_{fh} - p_{ref}$ and $p_{ph} - p_{ref}$ at times $t\in \{ 20, 80, 200, 400 \}$ (from top to bottom).} \label{fig:filter-pressure}
\end{figure}

%**********************************************************************
%**********************************************************************

\bibliographystyle{abbrv}
\bibliography{caucao-dalal-yotov}

\end{document}